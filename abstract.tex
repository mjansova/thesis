\chapternonum{Appendix: Résumé en français de mes travaux de thèse -- résultats principaux et discussion}

Pendant ma thèse, je travaille dans un équipe qui fait partie de la collaboration CMS. Le détecteur ``Compact Muon Solenoid''~(CMS) montré dans la Fig.~\ref{fig:figures/cmsdetector2} est un détecteur de physique des particules situé au CERN en Suisse, auprès du Large Hadron Collider~(LHC), un accélérateur circulaire de particules de 27 km de long qui fournit des collisions de protons à une énergie dans le centre de masse de 13 TeV depuis 2015 (Run II).

    \insertFigure{figures/cmsdetector2} % Filename = label
                 {0.99}       % Width, in fraction of the whole page width
                 { Schéma du détecteur de CMS~\cite{website:CMSdet}. }


Le programme de physique de CMS couvre principalement des mesures de précision du Modèle Standard~(MS) avec un énorme effort dédié aux études du boson de Higgs, ainsi que la recherche de phénomènes au-delà du MS aussi appelée recherche de ``nouvelle physique''.

\section{Introduction}

%This thesis started in October 2015, the same year as the beginning of the Run~2 data-taking period (2015-2018) during which  the center-of-mass energy, the instantaneous luminosity and the bunch crossing frequency was increased compared to the Run~1 (2008-2013). Consequently, the possibilities for physics analyses were largely extended with a price of a larger fluence on the detector side, especially on the tracker which is the inner subdetector. With this increasing fluence, the detector suffers from a larger irradiation which could be a cause of performance issues. Moreover, the detector is also ageing with irradiation, leading to a change of some of its characteristics. Therefore it is very important to monitor closely the detector, to keep its stable performance and consequently the potential physics reach.

%Particle physics is described by the standard model whose  last piece, the Higgs boson, was discovered in 2012 by the CMS and ATLAS collaborations. Although the standard model is now complete and in general describes excellently the physics phenomena, it suffers from several shortcomings. This issue makes us believe that the standard model is an effective theory at low energy of a more fundamental theory which is to be determined. Over years, many theories  were proposed and one, referred to as supersymmetry, became of a special interest due to its capability to address many of the standard model shortcomings. 

%Supersymmetry introduces a new partner to each standard model particle and therefore extensive searches for these particles have been performed by the CMS collaboration as well as other collaborations. One of these particles is the supersymmetric partner of the top quark, the stop, which is expected to have a mass around 1~TeV in natural supersymmetry and therefore be accessible at the LHC energies. No evidence for the stops was  found in Run~1, but the increase in luminosity as well as the center-of-mass energy in Run~2 allows us to probe the stop masses beyond the Run~1 exclusion. 

Cette thèse a commencé en octobre 2015, la même année que le début de la période de prise de données appelée Run~2 (2015-2018) au cours de laquelle l’énergie dans le centre de masse, la luminosité instantanée et la fréquence des collisions ont augmenté par rapport au Run~1 (2008-2013). Par conséquent, les possibilités d'analyses de processus physiques ont été largement étendues au prix d'une plus grande fluence du côté du détecteur, en particulier sur le trajectographe qui est le détecteur interne. Avec cette fluence croissante, le détecteur souffre d’une irradiation plus importante qui pourrait être à l’origine de problèmes de performance. De plus, le détecteur vieillit avec l'irradiation, entraînant une modification de certaines de ses caractéristiques. Par conséquent, il est très important de surveiller de près le détecteur, afin de conserver des performances stables et d'éviter tout impact négatif sur les analyses de physique. Dans les sections suivantes, je traiterai de deux groupes d’études sur lesquelles j’ai travaillé dans le cadre du trajectographe.

La physique des particules est décrite par le modèle standard dont la dernière pièce, le boson de Higgs, a été découverte en 2012 par les collaborations CMS et ATLAS. Bien que le modèle standard soit maintenant complet et décrive en général très bien les phénomènes physiques, il souffre de plusieurs défauts. Ce problème nous pousse à croire que le modèle standard est une théorie effective à basse énergie d’une théorie plus fondamentale à déterminer. Au fil des ans, de nombreuses théories ont été proposées et l'une d'entre elles, appelée supersymétrie, a suscité un intérêt particulier en raison de sa capacité à répondre à de nombreuses lacunes du modèle standard.

La supersymétrie introduit un nouveau partenaire pour chaque particule de modèle standard et, par conséquent, des recherches approfondies sur ces particules ont été et son encore actuellement effectuées par la collaboration CMS, ainsi que par d’autres collaborations. La particules qui nous intéresse dans le cadre de cette thése est le partenaire supersymétrique du quark top, le stop, qui devrait avoir une masse d'environ 1~TeV en supersymétrie naturelle~\cite{Martin:1997ns, Barbieri:1987fn, Papucci:2011wy} et donc être accessible aux énergies du LHC. Aucun signe de l'exsistence de stop n'a été observé au Run~1~\cite{website:SUSYresRunI}, mais l'augmentation de la luminosité ainsi que l'énergie dans le centre de masse au Run~2 nous permettent de sonder des masses de stop au-delà de l'exclusion de Run~1.

\section{Dispositif expérimental}

%The description of the Compact Muon Solenoid~(CMS) detector is given in the first chapter, Chapter~\ref{sec:detch}, together with a brief introduction of the Large Hadron Collider. This chapter focuses on the silicon strip tracker, whose deeper understanding is required for the following chapters. This chapter also presents the reconstruction of the physics objects corresponding to particles passing through the detector. In following Chapter~\ref{sec:HIPch}, I was interested in the highly ionizing particles in the silicon strip tracker. Already before the beginning of the Large Hadron Collider~(LHC) operation, the highly ionizing particles were studied and identified to cause dead-time in the front-end electronics leading to a hit inefficiency in the silicon strip tracker. In the beam tests of the tracker modules, it was found that by decreasing the inverter resistor value of the front-end electronics,  the dead-time can be shortened as shown in Table~\ref{tab:tableDeadtimes2} and therefore the design of the electronics was adjusted to mitigate this issue. Later no analysis with LHC collisions had been performed yet. So when an increase of the hit inefficiency in the strip tracker was observed in 2015-2016, a HIP was immediately identified as a possible explanation of it. I studied the HIP effect with collision events recorded in 2016 using a special data format in which no subtraction and suppression were applied at the back-end electronics level and therefore full information about all channels was available. They allowed me to  confirm that the HIP induces a large charge on few channels and drives the remaining channels belonging to the same front-end electronics, the APV chip, towards lower charges up to the point when the signal is too small to be converted into light what can be observed in Fig.~\ref{fig:figures/peakinmodule}. Reaching this zero light level also induces a decreased spread of the charges on all channels within one APV, when excluding the channels reading a large charge. With these data I was able to measure the rate of the HIP events, i.e. the number of events with a HIP interaction to all events, to be around $4 \times 10^{-3}$ for the first layer of Tracker Outer Barrel~(TOB). I also evaluated the dead-time for this layer to be up to around 250~ns. It is of the same order as the dead-times evaluated in  previous studies performed at the PSI beam test, namely a  mean and maximal dead-times for the TOB modules of 100~ns and 275~ns, respectively.  

La description du  détecteur Compact Muon Solenoid est donnée dans le premier chapitre, Chapitre~\ref{sec:detch}, avec une brève introduction du Large Hadron Collider~(LHC). Ce chapitre se concentre sur le trajectographe à pistes de silicium montré dans la Fig.~\ref{fig:figures/cmsTracker2}, dont la compréhension approfondie est requise pour les chapitres suivants. Ce chapitre présente également la reconstruction des objets physiques correspondant aux particules traversant le détecteur. 

    \insertFigure{figures/cmsTracker2}
                 {0.9}
                 { Schéma de la vue longitudinale de la moitié supérieure du trajectographe à pistes de silicium CMS et de la disposition de ses partitions. L'étoile représente le point de interaction. Les modules en bleu sont en stéréo, alors que les modules mono sont en noir~\cite{Chatrchyan:2014fea}.}
                 %{A schema of the longitudinal view of the upper half of the CMS silicon strip tracker and layout of its partitions. The star represents the IP. The modules in blue are stereo, while the mono ones are shown in black~\cite{Chatrchyan:2014fea}. }

\section{Étude des particules hautemant ionistantes dans le trajectographe de CMS}

Dans le Chapitre~\ref{sec:HIPch}, je me suis intéressée aux particules hautement ionisantes~(HIP) dans le trajectographe à pistes de silicium. Déjà avant le démarrage du Large Hadron Collider, les particules hautement ionisantes ont été étudiées et identifiées comme pouvant causer un temps mort dans l'électronique frontale, entraînant une inefficacité dans le trajectographe à pistes de silicium. Lors les tests en faisceau des modules du trajectographe, il avait été constaté qu'en diminuant la valeur de la résistance du convertisseur de l’électronique frontale, le temps mort pouvait être diminué~\cite{Bainbridge:2004jc} comme indiqué dans le Tableau~\ref{tab:tableDeadtimes2}, et le design de l'électronique avait donc été ajustée pour atténuer ce problème. En 2015-2016, lorsqu'une augmentation de l'inefficacité de reconstruction des hits dans le trajectographe a été observée pouvant aller jusqu'à 7\%, les HIP ont immédiatement été identifiés comme une explication possible. 

J'ai étudié l'effet des HIP pour la première fois avec des données de collisions LHC. Les  événements de collision enregistrés en 2016 dans un format de données spécial dans lequel aucune soustraction ni suppression n'ont été appliquées au niveau de l'électronique arrière, donnant accès aux informations complètes sur tous les canaux. Cette étude m'a permis de confirmer que le HIP induit une charge importante sur quelques canaux et a forcé les canaux restants connectés à la même puce électronique frontale, la puce APV, vers des charges plus faibles jusqu'au moment où le signal est trop petit pour être converti en lumière comme on peut l'observer sur la Fig.~\ref{fig:figures/peakinmodule}. Lorsqu'on atteint ce niveau sans lumière émise, on observe également une réduction des différences entre les charges de tous les canaux d'un APV, en excluant les canaux lisant une charge associée du HIP. A partir de ces données, j'ai pu mesurer le taux d'événements HIP, c'est-à-dire le nombre d'événements ayant une interaction HIP et  tous les événements. Ce taux est de 4 $\times 10^{-3} $ pour la première couche de Tracker Outer Barrel~(TOB), trop bas pour être responsable des inefficacités observées en 2015 et 2016. J'ai également évalué le temps mort pour cette couche à environ 250~ns. Il est du même ordre que les temps morts évalués dans des études antérieures effectuées lors de tests en faisceau à PSI, à savoir un temps mort moyen de 100~ns et maximal de 275~ns un temps mort pour les modules TOB, respectivement.

\begin{table}[h]
\begin{center}
\begin{tabular}{|l|l|l|}
\hline
Type de capteur et $R_{inv}$~[$\Omega$] & $\Gamma_{mean}$~[ns]  & $\Gamma_{max}$~[ns] \\
\hline
\hline
TIB 100 $\Omega$ & 99.5 $\pm$ 12.0 & 200 $\pm$ 25 \\
TIB 50  $\Omega$ & 69.6 $\pm$ 9.4 & 250 $\pm$ 25 \\
TOB 100  $\Omega$ & 122.5 $\pm$ 12.6 & 275 $\pm$ 25 \\
TOB 50 $\Omega$  & 100.5 $\pm$ 3.6 & 275 $\pm$ 25 \\
\hline
TIB $\Gamma_{mean}(50~\Omega )/\Gamma_{mean}(100~\Omega)$ &  0.70 $\pm$ 0.13  & \\
TOB $\Gamma_{mean}(50~\Omega )/\Gamma_{mean}(100~\Omega)$ &  0.82 $\pm$ 0.09 & \\
\hline
\end{tabular}
%\caption[Table caption text]{The mean~($\Gamma_{mean}$) and maximum~($\Gamma_{max}$) dead-time of the APV chip induced by the HIP events for fully suppressed~($\sigma_{raw}<1$~ADC) baseline events. The dead-times were evaluated for two different module geometries~(TIB or TOB) as well as for two inverter resistor values~(100 or 50~$\Omega$). These results were obtained with PSI beam test data~\cite{Bainbridge:2004jc}. }
\caption[Table caption text]{Le temps mort moyen~($\Gamma_{mean} $) et le temps mort maximal~($\Gamma_{max}$) de la puce APV induits par les événements HIP pour les lignes de base~(i.e. le median du ensemble de 128 pistes qui sont lu par une puce APV) qui sont entièrement supprimées~($\sigma_{raw} <1 $~ADC). Les temps morts ont été évalués pour deux géométries de module différentes~(TIB et TOB) ainsi que pour deux valeurs de résistance de convertisseurs~(100 et 50~$\Omega$). Ces résultats ont été obtenus avec des données de tests en faisceau à PSI avant le début de l'exploitation du LHC~\cite{Bainbridge:2004jc}.}
\label{tab:tableDeadtimes2}
\end{center}
\end{table}


    \insertFigure{figures/peakinmodule}
                 {0.47}
                 {Exemple de distributions du signal vu par les 512 pistes (strip) d'un module, sous format raw digis~(en rose), après soustraction du piédestal soustrait digis~(en bleu), après soustraction de la ligne de base~(en rouge) et après soustraction des amas~(en vert). Chaque ensemble de 128 pistes est lu par une puce APV. Le troisième APV dans le module montre un comportement induit par un événement HIP: une faible variation de charge pour les raw digis et un grand signal observé pour quelques canaux.}       % Width, in fraction of the whole page width
                 %{Example of distribution of raw digis~(pink), pedestal subtracted digis~(blue), baselines~(red) and clusters~(green) as a function of the strip number in one module. The third APV in the module shows a behavior induced by a HIP event: a low charge variation for the suppressed raw digis and a large observed signal for few channels.} 

%However the HIP effect was not responsible for the observed inefficiencies and after the largest source of inefficiency was found and fixed, a new set of data provided an opportunity to study the HIP effect in a greater detail disentangled of the previous issue. Because of the data-taking conditions, these new data permitted to perform a  measurement of the HIP probability in a cleaner environment. The HIP probability was computed per pp interaction, i.e. rescaled by the peak pileup of the fill, in order to have the possibility to apply this probability to any LHC fill. The HIP probability in the silicon strip tracker, shown in Fig.~\ref{fig:figures/probPerPU2}, was measured to be of the order of~$10^{-4}-10^{-3}$\% depending on the tracker layer/wheel/ring. The HIP probabilities, already evaluated in the past from simulations and using beam test data, were derived per particle and therefore they cannot easily be compared with the current measurement. Moreover during the beam test, a different particle composition and energy spectra were present compared to the CMS environment.

Au cours de l'été 2016, la plus grande source d'inefficacité de reconstruction des hits dans le trajectograph a été identifiée et corrigée en modifiant la configuration de la puce APV. Selon les conditions de run, l'inefficacité de le reconstruction des hits après correction de la configuration APV est d'environ moins de 1\%. Suite a cela un nouvel lot de données a été enregistré afin d'étudier l'effet HIP de manière plus détaillée. En raison des conditions de prise de données, ces nouvelles données ont permis de mesurer la probabilité d'avoir un HIP dans un environnement plus propre. La probabilité d'avoir un HIP ($p_{HIP}$) a été calculée par interaction proton-proton, c'est-à-dire que la fraction d'évenement avec un HIP ($f_{HIP}$) été divisée par le taux maximal d'interactions simultanées (peak pileup: PU), $p_{HIP} = f_{HIP}/PU$ , afin de pouvoir appliquer cette probabilité à toute configuration de structure des faisceaux du LHC. La probabilité d'avoir un HIP dans le trajectographe à pistes de silicium est, représentée sur la Fig.~\ref{fig:figures/probPerPU2}, est les mesures varient entre ~10$^{-4} $ et $ 10^{-3}$\% selon sur la position dans les différentes positions du trajectographe. Les probabilités d'avoir un HIP, déjà évaluées par le passé à partir de simulations et utilisant des données de tests en faisceau, ont été dérivées par particule et ne peuvent donc pas être facilement comparées à la mesure actuelle. De plus, lors des tests en faisceau, la composition des particules et leurs spectres en énergie étaient différents par rapport à l'environnement CMS.

    \insertTwoFigures{figures/probPerPU2}
                 {figures/probFinalLayerPU} % Filename = label
                 {figures/probFinalPURings} % Filename = label
                 {0.47}       % Width, in fraction of the whole page width
                 { Probabilité moyenne d'avoir un événement HIP par unité du pileup~(PU) pour les couches du TIB et du TOB (aussi bien sur la figure  de gauche que de droite) et pour les roues (à gauche) ou les anneaux (à droite) des partitions TID et TEC du trajectographe de silicium. Ces probabilités ont été calculées à partir du run 281604 enregistré en 2016 (2$^e$ étude des HIP).}
                 %{The average probability of a HIP event per PU for layers (left, right) of TIB, TOB, and wheels (left) or rings (right) of the TID and TEC partitions of the silicon strip tracker, computed from the data run 281604.}

%The HIP probability itself cannot be decreased but we might investigate if the dead-time, and consequently the hit inefficiency, could be reduced. An option how to decrease the dead-time was identified and studied in the past~\cite{website:hitLoss}. This option is to maximize the pedestal subtracted data in order to decrease the chance that the channel charges are shifted beyond the measurable range. The pedestal is the mean strip activity when no particle is present and in the standard data-taking mode, not pedestals themselves are subtracted by the back-end electronics, but the pedestals minus a fixed offset, resulting in the positive mean channel charges after this subtraction. This offset can be further maximized, but it was found out that this option only increases the fake cluster multiplicity without improving the hit efficiency. Moreover this option reduces the dynamic range, resulting in larger charges to be truncated faster. However this test was performed before the fix of the largest source of inefficiency and it could be worth  to repeat it  now.

La probabilité d'avoir un HIP ne peut pas être diminuée, mais nous pourrions chercher si le temps mort et par conséquent l’inefficacité de reconstruction du hits des traces pourraient eux être réduits. Une option permettant de réduire le temps mort a été identifiée et étudiée dans le passé ~\cite{website:hitLoss}. Cette option permet de maximiser les données après soustraction du piédestal où le piédestal est l’activité moyenne sur ue piste quand aucune particule n’est présente afin de réduire le risque que les charges des canaux soient décalées trop bas, au-delà de la plage mesurable. En réalité dans le mode de prise de données standard, c'est le piédestal moins un certain décalage fixe qui est soustrait par l’électronique back-end. Le choix de ce décalage pourrait être optimisé, mais il a été constaté que cette option ne fait qu’accroître la multiplicité des ``faux'' amas  sans améliorer l’efficacité de reconstruction des hits. De plus, cette option réduit la plage dynamique, ce qui se traduit par des charges plus importantes à tronquer plus rapidement. Cependant, ce test a été effectué avec l'ancienne configuration de la puce APV et il pourrait être utile de le répéter maintenant.

%A measured HIP probability of $10^{-4}-10^{-3}$\% means that in case of a dead-time of 250~ns obtained from the first data and bunch spacing of 25~ns, an APV chip would never be fully efficient if the pileup were of the order of $10^4-10^5$. The current pileup during the summer 2018 is of the order of 40 interactions per bunch crossing and therefore far from the fully inefficient chip scenario. The inefficiency resulting from the HIP was recently calculated, using the dead-time of 250~ns and the measured HIP probability, to be of the order of 0.1-1\%, for a pileup of 30 as in the representative run used for the 2018 hit efficiency measurement. This is of similar order as the measured 2018 hit inefficiency. Therefore the main source of the measured inefficiency now seems to come from the HIP effect. This inefficiency is not considered during the event simulation. 


Une probabilité d'avoir un HIP mesurée de $10^{-4}-10^{-3}$ signifie que dans le cas d'un temps mort de 250~ns (obtenu grâce à la première étude de cette thèse) d'un espacement de 25~ns entre les paquets des particules du faisceau, la puce APV ne serait jamais pleinement efficace uniquement si le pileup était de l'ordre de $10^4 - 10^5$. Le pileup actuel au cours de l'été 2018 est de l'ordre de 40 interactions par croissement de faisceau et donc loin du scénario d'une puce totalement inefficace. L'inefficacité de reconstruction des hits résultant des HIP a récemment été calculée en utilisant un temps mort de 250~ns et la probabilité d'avoir un HIP mesurée, elle est de l'ordre de 0.1-1\% pour un pileup de 30 comme pour le run représentatif utilisé pour la mesure de l'efficacité de reconstruction des hits en 2018. Cet ordre de grandeur est similaire à l'inefficacité mesurée en 2018. Par conséquent, la principale source de l’inefficacité mesurée semble désormais provenir de l’effet HIP. Cette inefficacité n'est pas prise en compte lors de la simulation des événements.

%The second set of data also permitted to study the cluster charge and multiplicity, presented in Fig.~\ref{fig:figures/avClusterMultiplicitySecondT2}, in presence of a HIP event or right after. I found out that the fake cluster multiplicity is increased after the HIP event compared to a standard event not influenced by a HIP. However the HIP probability is low and therefore in average over the runs the fake cluster multiplicity resulting from the HIP is of the same order of magnitude as the fake cluster multiplicity when no HIP had occurred. I observed that fake clusters induced by the HIP originate from the baseline distortions. For the first layer of TOB, I also estimated a lower limit on the probability that a track is reconstructed with at least one fake cluster originating from a HIP event. This probability was computed to be 0.002\% and therefore is negligible. These fake clusters are not included in the simulation either, but as discussed, their impact is negligible and therefore it is reasonable to omit them from simulation.

Le deuxième ensemble de données m'a également permis d'étudier la charge et la multiplicité des amas, présentées à la Fig.~\ref{fig:figures/avClusterMultiplicitySecondT2}, en présence d'un événement HIP ou juste après. J'ai découvert que la multiplicité de faux amas augmente après l'événement HIP par rapport à un événement standard non influencé par un HIP. Cependant, la probabilité d'avoir un HIP est faible et, par conséquent, en moyenne sur le run, la multiplicité de faux amas résultant du HIP est du même ordre de grandeur que la multiplicité de faux amas en l'absence de HIP, le nombre  étant deux ordres de grandeur inférieur au cas du out-of-time pileup. J'ai observé que les faux amas induits par le HIP proviennent de distorsions de la ligne de base. Pour la première couche du TOB, j'ai également estimé une limite inférieure sur la probabilité qu'une trace soit reconstruite avec au moins un faux amas provenant d'un événement HIP. Cette probabilité est de 0.002 \% et donc négligeable. Ces faux amas ne sont pas non plus inclus dans la simulation. Mais comme on l’a vu, leur impact est négligeable, il est donc raisonnable de les omettre de la simulation.


    \insertTwoFigures{figures/avClusterMultiplicitySecondT2}
                 {figures/avClusterMultiplicitySecond} % Filename = label
                 {figures/avClusterChargeSecond} % Filename = label
                 {0.45}       % Width, in fraction of the whole page width
                 {Nombre d'amas moyenne (à gauche) et charge moyenne portée par les amas (à droite) en fonction du numéro d'identification du croissement de faisceau pour le run 281604. Le premier triangle rose représente l’événement HIP, les triangles suivants représentes les événements successifs (post HIP), tandis que les carrés bleus sont pour les événements non affectés par un HIP. }
                 %{ The average cluster multiplicity (left) and average cluster charge (right) as a function of the bunch crossing number for the run 281604. In pink triangles is the average cluster multiplicity of the HIP event and events later while the blue squares are for the non-HIP events.  } % Caption



%Unfortunately, because of the data-taking conditions of the second data-taking period, it was impossible to measure the dead-time in them. The first data provided a better opportunity to measure the dead-time, but with the  difficulty to precisely fix the time of the HIP occurrence. There are not any simple options of data-taking conditions suitable to  measure the dead-time in future, in this paragraph I propose two more complex ideas of  data-takings which could be considered. The first option would be to arrange data-taking in which events would be recorded every 25~ns during at least 15 bunch crossings not respecting any trigger rule, in order to have the possibility to track a given HIP in time. On top of it, it would be needed to reconsider the analysis strategy. For example, as we would not be interested in the HIP probability anymore, we could spot the HIP event by presence of a large peak, which appears only for few ns and therefore should better fix the time of the HIP event. But this approach could overload the system and therefore it would be needed to evaluate if such data-taking would be reasonable and how we could possibly avoid the overload. The second option would be to have a special fill and partially violate the trigger rules as well. In such fill we would need trains of two bunches and in each train we would need different time gap between the two bunches: in the first one 25~ns, the second one 50~ns, the third one 75~ns, etc. In these data we would have the time of HIP event fixed by the first bunch crossing in train and in the second one we could measure the hit efficiency. The dead-time would be the time gap between two bunch crossings in the same train in which the same hit efficiency is measured. In both these proposed data-takings, it would be needed to make sure, that the fake clusters from baseline distortions are not considered as real clusters by tracking and therefore that the fake clusters do not artificially increase the hit efficiency. 

Malheureusement, en raison des conditions de prise de données utilisée pour la 2$^{eme}$ étude, il était impossible de mesurer du temps mort. Les premières données ont fourni une meilleure occasion de mesurer le temps mort, mais avec la difficulté de déterminer précisément le temps de l'occurrence du HIP. Il n’existe pas d’options simples de conditions de prise de données permettant de mesurer le temps mort à l’avenir. Dans ce paragraphe, je propose deux idées plus complexes de prises de données qui pourraient être envisagées. La première option consisterait à organiser la prise de données dans laquelle les événements seraient enregistrés tous les 25~ns pendant au moins 15 croisements de paquets, sans respecter aucune règle de déclenchement, afin de pouvoir suivre un HIP donné dans le temps. De plus, il serait nécessaire de reconsidérer la stratégie d’analyse. Par exemple, comme nous ne serions plus intéressés par la mesure de la probabilité d'avoir un HIP, nous pourrions repérer l’événement HIP en présence d’un grand pic, qui n’apparaît que pour quelques instants et devrait donc mieux déterminer le temps de l’événement HIP. Mais cette approche pourrait surcharger le système d'acquisition de CMS et il faudrait donc évaluer si une telle prise de données serait raisonnable et comment nous pourrions éventuellement éviter la surcharge. La deuxième option serait d'avoir un structure spéciale de remplissage du fascieaux du LHC et de violer partiellement les règles de déclenchement de CMS. Dans un tel scenario, nous aurions besoin de trains contenant uniquement deux paquets et pour chaque train, nous aurions besoin d'un intervalle de temps différent entre les deux paquets: pour le premier train les deux paquets seraient séparés par 25~ns, pour le second par 50~ns, pour le troisième par 75~ns, etc. Grâce à de telles données, nous aurions le temps de l'événement HIP fixé par le premier croisement du train et le second, nous permettrait de mesurer l'efficacité de reconstruction des hits. Le temps mort serait obtenu  comme l'intervalle de temps entre deux croisements de paquets d'un  même train pour lesquels la même efficacité de hit est mesurée. Pour ces deux propositions, il serait nécessaire de s'assurer que les faux amas issus des distorsions de la ligne de base ne sont pas considérées comme de bon amas lors de la reconstruction  des traces et que, par conséquent, les faux amas n'augmentent pas artificiellement l'efficacité.

%In the future upgrade of the silicon strip tracker for the High Luminosity LHC project (HL-LHC), we must pay attention to the HIP effect already at the design level. Ideally, it would be desired that the electronics is designed in a way that if a large charge is read by one channel, it does not affect the other channels belonging to the same chip as it is the case now. The HIP probability computed in this thesis is related to the module geometry and its distance from the interaction point and therefore cannot be easily extrapolated to the future detector. Assuming that the tracker would remain the same and for a pileup of around 200 interactions per bunch crossing, the HIP rate per chip could be expected to be of the order of $10^{-2}-10^{-1}$ \%. In the reality, the detector will be upgraded and therefore the HIP probability and induced dead-time will change depending on the design of the tracker modules and the readout electronics.

Pour future mise à niveau du trajectographe à pistes de silicium dans le cadre du projet High Luminosity LHC (HL-LHC), nous devons prêter attention à l’effet HIP dès la conception. Idéalement, il serait souhaitable que l’électronique soit conçue de telle manière que si une charge importante est lue par un canal, cela n’affecte pas les autres canaux appartenant à la même puce comme c’est le cas actuellement. La probabilité d'avoir un HIP calculée dans cette thèse est liée à la géométrie du module et à sa distance par rapport au point d'interaction et ne peut donc pas être facilement extrapolée à  un futur détecteur. En supposant que le trajectographe resterait le même et pour un pileup d'environ 200 interactions par croisement de paquets, le taux de HIP par puce pourrait être de l'ordre de 10$^{-2}-10 ^{-1}$\% . En réalité, le détecteur sera mis à niveau et, par conséquent, la probabilité d'avoir un HIP et le temps mort induit changeront en fonction de la conception des modules du trajectographe et de l’électronique de lecture.

\section{Amelioration de la simulation du trajectographe et mesure de la diaphonie}
%SIMU


%The third chapter, Chapter~\ref{ch:simu} focuses on the  simulation of the CMS silicon strip tracker and the tracker conditions used in simulation in order to provide realistic results. These conditions change with the tracker operating conditions, e.g. the temperature, but also evolve with tracker ageing resulting from the radiation damage, for example the cross talk, i.e. the charge shared with the neighboring strips. As the majority of these conditions was never updated since the beginning of the Run~1, we decided therefore to revise the simulation and study the impact of the outdated conditions on the simulation of the clusters in the tracker. I found out that the change of conditions (except noise and gains) has a negligible impact on the cluster properties in many cases, but the change of the cross talk parameters largely impacts the cluster width and seed charge.

Le troisième chapitre, Chapitre~\ref{ch:simu}, se concentre sur la simulation du trajectographe à  pistes en silicium de CMS et sur les paramètres de trajectographe utilisés dans la simulation afin de fournir des résultats réalistes. Ces paramètres changent avec les conditions de fonctionnement du trajectographe, par ex. la température, mais évoluent également avec le vieillissement résultant des dommages causés par le rayonnement, par exemple la diaphonie (``cross talk''), c'est-à-dire la charge partagée avec les pistes voisines comme le montre la Fig.~\ref{fig:figures/crossTalk2}. La majorité de ces conditions n’ayant jamais été mise à jour depuis le début du Run~1, nous avons donc décidé de réviser la simulation et d’étudier l’impact des paramètres obsolètes sur la simulation des amas dans le trajectographe. J'ai découvert que le changement de conditions (à l'exception du bruit et des gains) a un impact négligeable sur les propriétés des amas dans de nombreux cas, mais le changement des paramètres de la diaphonie affecte largement la largeur due l'amas et la charge de la piste collectant le signal maximal servant de grain à la formation de l'amas.

    \insertFigure{figures/crossTalk2} % Filename = label
                 {0.5}       % Width, in fraction of the whole page width
                 {Un schéma de partage de charge entre pistes voisines. }
                 %{ A schema of charge sharing between neighboring strips. }

%After having identified that the cross talk parameters cause large discrepancies in cluster shape distributions between data and simulation as shown in Fig.~\ref{fig:figures/seedwidthTOBXT2}, a cosmic data-taking period was arranged in absence of magnetic field, dedicated to re-measure the cross talk parameters. As for the HIP studies, no subtraction and suppression of channel charges were applied on the back-end electronics level. In these data I spotted that the cross talk parameters evolve as a function of the time when the particle arrives to a given module. Therefore the cross talk needs to be measured at the correct timing, i.e. as when a particle originating from the pp interaction would reach a given module. Due to the trigger conditions, there was only sufficient statistics to measure the cross talk parameters in the barrel. The newly measured parameters together with the current parameters used in simualtion are shown in Table~\ref{tab:measuredXtalk2}. I measured that the cross talk decreased for all barrel module geometries compared to the previous measurement. The sharing to the first neighboring strips in barrel decreased by around 18-27\%, the change being larger for geometries closer to the interaction point i.e. with larger fluence and consequently larger radiation damage. The sharing to the second neighboring strips diminished by around 24\%. To determine the cross talk parameters in disks and endcaps, I corrected the old cross talk parameters based on the change of the cross talk parameters in barrel and data to simulation comparisons of the mean cluster seed charge in collision events. The results of cross talk measurement in disks and endcaps are summarized in Table~\ref{tab:measuredXtalkTODTEC2}. Updating the cross talk parameters in simulation by newly measured values, we found out that the newly measured parameters improved largely description of cluster shape in data by simulation as shown on example in Fig.~\ref{fig:figures/widthTOBXTm2}.

Après avoir identifié que les paramètres de la diaphonie provoquent des écarts importants dans les distributions de forme des amas entre les données et la simulation, comme le montre la Fig.~\ref{fig:figures/seedwidthTOBXT2}, nous avons utiliser des données correspondant au passage de muons cosmiques, dans le détecteur  pour mesurer les paramètres de la diaphonie. Comme pour les études des HIP, aucune soustraction ni suppression de charge de canal n’ont été appliquées au niveau de l’électronique back-end. Dans ces données cosmiques, j'ai remarqué que les paramètres de la diaphonie évoluent en fonction du moment où la particule arrive dans un module donné. Par conséquent, la diaphonie doit être mesuré au bon moment, c'est-à-dire comme si une particule provenant de l'interaction proton-proton atteint un module donné. Ceci est particulièrement difficile pour des muons cosmiques arrivant de façon aléatoir.

J'ai mesuré que la diaphonie diminuait pour toutes les géométries de module dans la partice centrale de CMS (ou tonneau) par rapport à la mesure précédente. Les paramètres nouvellement mesurés ainsi que les paramètres actuels utilisés dans la simulation sont indiqués dans le Tableau~\ref{tab:measuredXtalk2}. Le partage de charge avec les premières pistes voisines a diminué d'environ 18-27\%, le changement étant plus important pour les géométries plus proches du point d'interaction, c'est-à-dire avec une plus grande fluence et, par conséquent, des dommages plus importants. Le partage de charge avec les deuxièmes pistes voisines a diminué d'environ 24\%. En raison des conditions de déclenchement, il n'y avait que suffisamment de statistiques pour mesurer les paramètres de la diaphonie dans les disques~(TID) et les bouchons du trajectographe~(TEC). Pour déterminer les paramètres de la diaphonie dans ces partitions, j'ai corrigé les anciens paramètres de la diaphonie en fonction du changement observé des paramètres dans le tonneau et tenant compte des comparaisons entre  données et simulation de la charge moyenne des amas dans les événements de collision. Les résultats des mesures de la diaphonie sur les disques et les bouchons sont résumés dans le Tableau~\ref{tab:measuredXtalkTODTEC2}. En mettant à jour les paramètres de la diaphonie dans la simulation, nous avons découvert que les nouvells mesures améliorent largement la description de la forme des amas dans les données par simulation, comme illustré à la Fig.~\ref{fig:figures/widthTOBXTm2}.

    \insertTwoFigures{figures/seedwidthTOBXT2} % Filename = label %TDO continue here
                 {figures/clusterwidthTOBl1to4XT}
                 {figures/clusterseedchargeRescaledTOBl1to4XT} % Filename = label
                 {0.45}       % Width, in fraction of the whole page width
                 {Distributions de la largeur des amas (à gauche) et de la charge portée par la piste collectant le plus de signal dans l'amas (à droite) pour la géométrie OB2. Les données de collision sont représentées par les points noirs, différentes valeurs de la diaphonie $XT$ sont utilisées dans la simulation (chaque set correspondant à une couleur différente). Le premier des trois nombres $XT$ correspond à la fraction de charge induite sur la piste principale, les deuxième et troisième nombres représentent les fractions de charge induites sur les premièrs et seconds pistes voisins, respectivement. Les distributions simulées sont renormalisés au nombre d'amas dans les données. Les graphique inférieurs représentent le rapports des distributions des données par la simulation.}
                 %{ Distribution of the on-track cluster width (left) and cluster seed charge (right) in data and simulation for the OB2 geometry and for different values of the cross talk $XT$. The first of the three XT numbers corresponds to the fraction of charge induced on the seed strip, the second and third number represent the fractions of charge induced on each first and second neighboring strips, respectively. The simulated distributions are rescaled to the number of clusters in data. The bottom plot represents the data to simulation ratios. }

\begin{table}[h]
\begin{center}
\begin{tabular}{|l|l|l|l|l|}
\hline
Géométrie & Type & $x_{0}$ & $x_{1}$ & $x_{2}$ \\
\hline
\hline
IB1 & nouvelle mesure & $ 0.836 \pm 0.009 $ & $0.070 \pm 0.004 $ & $0.012 \pm 0.002 $ \\
IB1 & simulation & $ 0.775 $ & $ 0.096 $ & $0.017 $  \\
\hline
IB2 &  nouvelle mesure & $0.862 \pm 0.008 $ & $0.059 \pm 0.003 $ & $0.010 \pm  0.002 $  \\
IB2 &  simulation &  $0.830 $ & $0.076 $ & $ 0.009$   \\
\hline
OB2 &  nouvelle mesure & $0.792 \pm 0.009 $ & $0.083 \pm 0.003 $ & $0.020 \pm 0.002$  \\
OB2 &  simulation &   $0.725 $ & $0.110 $ & $ 0.027 $  \\
\hline
OB1 &  nouvelle mesure &  $0.746 \pm 0.009 $ & $0.100 \pm 0.003 $ & $0.027 \pm 0.002 $  \\
OB1 &  simulation &  $0.687 $ & $0.122 $ & $ 0.034 $ \\
\hline
\end{tabular}
\caption[Table caption text]{Valeurs de la diaphonie mesurées en 2018 comparées aux anciennes mesures de diaphonie utilisées actuellement pour la simulation des géométries du tonneau. }
%\caption[Table caption text]{The cross talk measured in 2018 CRUZET VR data and the cross talk values used currently in simulation for barrel geometries. }
\label{tab:measuredXtalk2}
\end{center}
\end{table}

\begin{table}[h]
\begin{center}
\begin{tabular}{|l|l|l|l|l|}
\hline
Géométrie  & Type & $x_{0}$ & $x_{1}$ & $x_{2}$ \\
\hline
\hline
W1a &  nouvelle mesure & 0.8571 & 0.0608 & 0.0106 \\
W1a &  simulation & 0.786 & 0.093 & 0.014 \\
\hline
W2a &  nouvelle mesure & 0.8861 & 0.049 & 0.008 \\
W2a &  simulation & 0.7964 & 0.0914 & 0.0104 \\
\hline
W3a &  nouvelle mesure & 0.8984 & 0.0494 & 0.0014 \\
W3a &  simulation & 0.8164 & 0.09 & 0.0018 \\
\hline
W1b &  nouvelle mesure & 0.8827 & 0.0518 & 0.0068 \\
W1b &  simulation & 0.822 & 0.08 & 0.009 \\
\hline
W2b &  nouvelle mesure & 0.8943 & 0.0483 & 0.0046 \\
W2b &  simulation & 0.888 & 0.05 & 0.006 \\
\hline
W3b &  nouvelle mesure & 0.8611 & 0.0573 & 0.0121 \\
W3b &  simulation & 0.848 & 0.06 & 0.016 \\
\hline
W4 &  nouvelle mesure & 0.8881 & 0.0544 & 0.0015 \\
W4 &  simulation & 0.876 & 0.06 & 0.002 \\
\hline
W5 &  nouvelle mesure & 0.7997 & 0.077 & 0.0231 \\
W5 &  simulation & 0.7566 & 0.0913 & 0.0304 \\
\hline
W6 &  nouvelle mesure & 0.8067 & 0.0769 & 0.0198 \\
W6 &  simulation & 0.762 & 0.093 & 0.026 \\
\hline
W7 &  nouvelle mesure & 0.7883 & 0.0888 & 0.0171 \\
W7 &  simulation & 0.7828 & 0.0862 & 0.0224 \\
\hline
\end{tabular}
\caption[Table caption text]{ Valeurs de la diaphonie actualisées pour les anneaux du TID~(W1a, W2a, W3a) et du TEC (restant), comparées aux anciennes mesures utilisés dans la simulation. }
\label{tab:measuredXtalkTODTEC2}
\end{center}
\end{table}


    \insertTwoFigures{figures/widthTOBXTm2} % Filename = label %TDO continue here
                 {figures/clusterwidthTOBl1to4XTm}
                 {figures/clusterseedchargeRescaledTOBl1to4XTm} % Filename = label
                 {0.45}       % Width, in fraction of the whole page width
                 {Distribution de la largeur de l'amas (à gauche) et de la charge portée par la piste collectant le plus de signal dans l'amas (à droite) dans les données et la simulation pour la géométrie OB2, pour les paramètres de la diaphonie ($XT$) actuels (défaut, en rouge) et nouvellement mesurés (updated, en bleu). Les distributions simulées sont renormalisés au nombre d'amas dans les données. Les graphiques inférieurs représentent le rapport des données et simulation. }
                 %{ Distribution of the on-track cluster width (left) and seed charge (right)  in data and simulation for the OB2 geometry, for the current (default) and newly measured (updated) cross talk parameters~($XT$).  The simulated distributions are rescaled to the number of clusters in data.  The bottom plots represent the data to simulation ratios. }


%The cross talk parameters influence profoundly the cluster shape, but not the total cluster charge. The change of the cross talk might cause that the clusters which were slightly above the clustering threshold before do not have to pass it now and therefore are not reconstructed anymore and vice versa. In addition, with the change of the cross talk parameters, the cluster position and its resolution change in simulation, resulting in small changes in tracking. Therefore the effect of a change of the cross talk may propagate up to some of the physics analyses, e.g. in searches for appearing/disappearing tracks. The object discriminators, mainly the b-tagging discriminators, which are strongly dependent on the tracking, are also influenced by it. The impact on the other physics objects and analyses, not largely dependent on tracking, is expected to be negligible.

Les paramètres de la diaphonie influencent profondément la forme des amas, mais pas leur charge totale. Le changement de la diaphonie pourrait faire en sorte que les amas qui étaient précédement légèrement au-dessus du seuil de clustering soient à présent en dessous et ne soient donc plus reconstruits, et vice versa. De plus, avec la modification des paramètres de la diaphonie, la position de l'amas et sa résolution changent lors de la simulation, entraînant de petits changements dans la reconstruction des traces. Par conséquent, l'effet d'une modification de la diaphonie peut se propager jusqu'à certaines des analyses de physique, par exemple dans les recherches de signatures de nouvelle physique qui cherchent des traces qui apparaissent tardivement ou disparissent dans le trajectographe. L'identification des objets de physique, qui dépendent fortement du trajectographie, comme par exemple le b-tagging (identification des jets comme issues de quarks b) , peuvment également être influencés par un changement des paramètres de la diaphonie. L'impact sur les autres objets et analyses de physique, qui ne dépendent pas fortement du trajectographie, devrait être négligeable.

%It was shown that the cross talk parameters evolve as a function of the fluence and therefore they should be remeasured and updated regularly. Also more frequent measurements of the cross talk parameters could help us to understand why the cross talk is decreasing with respect to the previous measurement. The literature is stating that with an increasing fluence, the inter-strip capacitance should increase and the inter-strip resistance decrease. Both these changes should lead to the increase of cross talk, what was not observed. However the inter-strip resistance and capacitance are not the only factors influencing the signal formation, but there is more complex network of capacitances and resistances which has its impact on the cluster shape.  Note that the cross talk was measured only in the deconvolution mode, so it should be updated for the peak mode as well. Nevertheless, the peak mode is not used in standard data-taking, but it could help us to investigate the decrease of the cross talk, by disentangling the effects of the deconvolution from other effects.

Il a été montré que les paramètres de la diaphonie évoluent en fonction de la fluence et doivent donc être réévalués et mis à jour régulièrement. De plus, des mesures plus fréquentes des paramètres de la diaphonie pourraient nous aider à comprendre pourquoi la diaphonie diminue par rapport à la mesure précédente. La littérature~\cite{Hartmann:2017gzy} indique qu'avec une fluence croissante, la capacité entre les pistes devrait augmenter et la résistance inter-pistes diminuer. Ces deux changements devraient entraîner une augmentation de la diaphonie, ce qui n’a pas été observé. Cependant, la résistance et la capacité entre les pistes ne sont pas les seuls facteurs qui influencent la formation du signal. Il existe un réseau plus complexe de capacités et de résistances qui ont un impact sur la forme del'amas. Notez que la diaphonie a été mesurée uniquement pour le mode principale d'échantillage du signal (mode déconvolution). Elle faudrait donc également être mise à jour pour le mode ``peak''. Bien que ce dernier mode ne soit pas utilisé dans la prise de données standard, il pourrait nous aider à étudier la diminution de la diaphonie en distinguant les effets de la déconvolution des autres effets.

%The cosmic data recorded in absence of magnetic field, which were used for the cross talk measurement, posed several constraints and difficulties. First, due to the trigger conditions there is insufficient statistics in the disks and endcaps. Even if the trigger would be re-designed, due to the $\eta$ distribution of the cosmics, the data-taking would have to be long in order to collect sufficient statistics, which is difficult to arrange in the tight CMS schedule. The second large issue is related to the tracker timing when it samples the collected signal. The tracker has no special timing configuration for the cosmics and therefore the collision timing is always used.  As the cross talk depends on timing, we need to use for the cross talk measurement only cosmics which arrive to the given module at the same time as a particle produced in the pp collision. Another ambiguity comes from the computation of  the particle arrival time, which extrapolates all particles to the interaction point, even though they do not have to pass through there. Both the pseudorapidity coverage and the timing issues would be solved if it is possible to arrange a new data-taking of non suppressed collision data without magnetic field. But such data-taking is not compatible with the CMS program and priorities.

Les données cosmiques enregistrées en l'absence de champ magnétique, qui ont été utilisées pour la mesure de la diaphonie, ont posé plusieurs contraintes et difficultés. Tout d'abord, en raison des conditions de déclenchement, les statistiques sur les disques et les bouchons sont insuffisantes. Même si le déclenchement était re-conçu, en raison de la direction d'arrivée des cosmiques, la prise de données devrait être longue afin de collecter suffisamment de statistiques dans ces zones-là, ce qui est difficile à organiser dans le calendrier serré du CMS. Le deuxième problème important est lié à la synchronisation du trajectographe lorsqu'il échantillonne le signal collecté. Le trajectographe n'a pas de configuration de synchronisation spéciale pour les cosmiques et, par conséquent, la synchronisation des collisions est toujours utilisée. Comme la diaphonie varie en fonction du temps dans l'échantillonage du signal, nous ne devons utiliser pour la mesure de la diaphonie que les cosmiques dont le temps d'arrivée au module donné coinciderait avec celui d'une particule produite lors d'une collision pp. Une autre ambiguïté provient du calcul du temps d'arrivée des particules, qui extrapole toutes les particules au point d'interaction, même si elles ne sont pas passé par là. Tous ces problèmes pourraient être résolus s'il était possible d'organiser une nouvelle prise de données de collision en mode spécial d'acquisition (sans soustraction ni suppression de charge) et sans champ magnétique. Mais une telle prise de données n'est pas compatible avec le programme et les priorités de CMS.

%After the update of the cross talk parameters and conditions such as gains and noise updated by other members of the tracker local reconstruction group, the description by the simulation of the in-time clusters in data was largely improved. It was identified that there are still several parameters which are outdated and could be  updated in future, however these parameters do not largely change the cluster description. In this thesis on one hand I found out that the simulation is simplified and sometimes not realistic, but on the other hand developing more sophisticated models would not largely improve the description of the clusters in data by simulation. I also identified that already some existing parts of the simulation have only a negligible impact, i.e. the diffusion, and it could be good to evaluate what parts of the simulation are really needed in order to reduce the time needed for the event simulation.

Après la modification des paramètres de la diaphonie et des conditions tels que les gains et le bruit mis à jour par les autres membres du groupe de reconstruction locale du trajectographe, la description par simulation des amas synchronisés avec les croisements de faisceaux (``en temps'') dans les données a été grandement améliorée. Il a été identifié qu'il existe encore plusieurs paramètres qui sont obsolètes et pourraient être mis à jour à l'avenir, mais ces paramètres ne modifient pas largement la description des amas. Dans cette thèse d'une part, j'ai découvert que la simulation est simplifiée et parfois non réaliste, mais d'autre part, le développement de modèles plus sophistiqués n'améliorerait pas grandement la description des amas dans les données par simulation. J'ai également constaté que certaines parties de la simulation ont un impact négligeable, à savoir la diffusion, et il pourrait être utile d'évaluer quelles parties de la simulation sont réellement nécessaires pour réduire le temps nécessaire à la simulation d'événements.

%Several improvements could be still included in the simulation of the out-of-time clusters. First, the pulse shape has changed due to the change of the APV parameter and therefore the pulse shape should be updated in order to reduce the simulated charge by a correct fraction. Secondly, it was shown that the cross talk depends on timing. The cross talk parameters were derived for in-time clusters and are not correct for out-of-time clusters. Ideally, the cross talk parameters should be parameterized as a function of the particle arrival time to the module. The out-of-time clusters need also to be well simulated as these clusters might be used by the tracking algorithm to reconstruct the tracks of the particles.

Plusieurs améliorations pourraient encore être incluses dans la simulation des amas non synchrones avec des collisions. Tout d'abord, la forme du signal a changé en raison de la modification de la configuration de la puce APV mite au problème d'inefficacités de reconstruction des hits discutés précédement. Par conséquent, la forme du signal doit être mise à jour afin de réduire la charge simulée d'une fraction correcte. Deuxièmement, il a été montré que la diaphonie dépend du temps. Les paramètres de la diaphonie ont été dérivés pour les amas ``en temps'' et ne sont pas corrects pour les amas pas ``en temps''. Idéalement, les paramètres de la diaphonie devraient être estimés en fonction du temps d'arrivée des particules au module. Les amas pas ``en temps'' devraient également être mieux simulés, car ces amas pourraient être utilisés par l’algorithme de trajectographie pour reconstruire les traces des particules.


%Few other interesting studies could be performed with the taken cosmic data and here I discuss few examples. In the past it was observed that there is a left-right asymmetry in the charge sharing. This effect could be as well studied with these data. Moreover, it is possible to study the cross talk and cluster properties as a function of the position where the track intercepts the strip plane, close and far from the strip. In the past it was also observed that there is an evolution of the cluster seed charge as a function of the strip number within one APV chip, this study could be repeated with these data. All these studies could help to better understand the formation of the signal, the features of the sensors such as non-uniformities in the electric field, and also the changes in the sensors properties resulting from their irradiation. 

Pour aller plus loin avec les données cosmiques enregistrées, quelques d’études intéressantes qui pourraient être réalisées je propose ici. Dans le passé, on observait une asymétrie gauche-droite dans le partage des charges. Cet effet pourrait aussi être étudié avec les données que j'ai utlilisées. De plus, il est possible d'étudier les propriétés de la diaphonie et d'amas en fonction de la position où la trace croise le plan des pistes, à proximité ou à distance de la piste. Dans le passé, il a également été observé qu’il existe une évolution de la charge portée par la piste de plus grande charge dans l'amas en fonction du numéro dei la piste dans une puce APV, cette étude pouvrait être répétée avec ces données. Toutes ces études pourraient aider à mieux comprendre la formation du signal, les caractéristiques des capteurs telles que les non-uniformités dans le champ électrique, ainsi que les modifications des propriétés des capteurs résultant de leur irradiation.

%In the future tracker for the HL-LHC, the clustering will be largely different, only the binary information from the strip/macro-pixel will be sent, therefore the cluster seed charge and cluster charge will not be available and just the cluster width will be known. As the cluster width depends on cross talk, it will also be needed to measure the cross talk in the new tracker. Moreover it will be important to determine and monitor the cross talk in order to design and maintain the threshold for strip/macro-pixel charge to be correctly set to one or zero.

Dans le futur trajectographe pour le HL-LHC, le formation des amas sera complètement différente, seules les informations binaires de la piste/macro-pixel seront envoyées. Par conséquent, la charge de la piste de charge maximal et la charge totale de l'amas ne seront plus disponibles, seule la largeur de l'amas sera être connue. Comme la largeur de amas dépend de la diaphonie, il sera également nécessaire de mesurer la diaphonie dans le nouveau trajectographe. De plus, il sera important de déterminer l'impact de la diaphonie sur le panage de seuil enre 0 et 1 pour la charge de la piste/macro-pixel, afin de fixer correctement le seuil et le surveiller au cours du temps.

\section{Recherche du partenaire supersymétrique du quark top}

%After an introduction of both SM and SUSY in Chapter~\ref{sec:SUSYch}, the search for the stop is presented in Chapter~\ref{sec:stopch} in the single lepton final state with the data of Run~2. This last part of the thesis is discussing a search for physics beyond the standard model. Despite its capability to well describe the majority of observed physics phenomena, the standard model (SM) of particle physics suffers from several shortcomings such as the hierarchy problem or the absence of a dark matter candidate. Due to these shortcomings, theories beyond the SM were proposed and one of them, the supersymmetry, became the most promising one because of its ability to address large part of the SM issues. In this thesis I presented a search for the supersymmetric partner of the top quark, the stop, with the CMS Run~2 data. This search targets the stop pair production, with three different possibilities for the decays of the stops: the two stops decay each to a top quark and a neutralino, or the two stops decay to a bottom quark and a chargino, or each stop decays differently as a mixture of the two previous cases. In all cases, the targeted signal final states contain one lepton, jets and missing transverse energy. 

Après une introduction à la fois SM et SUSY dans le Chapitre ~\ref{sec:SUSYch}, la recherche de le stop est présenté dans le Chapitre~\ref{sec:stopch} dans l'état final de lepton unique avec les données de Run~2. Cette dernière partie de la thèse traite d'une recherche de physique au-delà du modèle standard. Malgré sa capacité à bien décrire la majorité des phénomènes physiques observés, le modèle standard de la physique des particules souffre de plusieurs inconvénients tels que le problème de la hiérarchie ou l’absence d’un candidat pour la matière noire. En raison de ces lacunes, des théories au-delà de la SM ont été proposées et l'une d'entre elles, la supersymétrie, est devenue la plus prometteuse en raison de sa capacité à traiter une grande partie des problèmes de SM. Dans cette thèse, j'ai présentée une recherche du partenaire supersymétrique du quark top, le stop, avec les données CMS Run~2. Cette recherche vise la production de paires des stops, avec trois possibilités différentes pour les désintégrations des stops: les deux stops se désintègrent chacun un quark top et un neutralino, ou la désintégration de deux stops à un quark et un chargino, ou chaque stop désintégrations différemment comme un mélange des deux cas précédents. Dans tous les cas, les états finaux du signal ciblé contiennent un lepton, des jets et une énergie transversale manquante.

%I was involved in three public releases of this search: one based on the data recorded in 2015 corresponding to an integrated luminosity $\int{\mathcal{L}}$ of 2.3~fb$^{-1}$~\cite{Sirunyan:2016jpr}, the second one based on the data recorded at the beginning of 2016 ( $\int{\mathcal{L}}=12.9$~fb$^{-1}$)~\cite{CMS:2016vew} and the last one corresponding to $\int{\mathcal{L}}= 35.9$~fb$^{-1}$ collected during the full 2016 pp data-taking period~\cite{Sirunyan:2017xse}. I was mainly involved in the last analysis, where I was responsible for the estimation of one of the SM backgrounds, in which a Z boson decays to two neutrinos. In this thesis I also exploited a technique for tagging of merged jets originating from a W boson and I showed that the gain of implementing such technique is growing with the integrated luminosity. 

J'ai été impliqué dans trois versions publiques de cette recherche: une basée sur les données enregistrées en 2015 correspondant à une luminosité intégrée $\int{\mathcal{L}} $ de 2.3~fb$^{-1}$~\cite{Sirunyan:2016jpr}, le second basé sur les données enregistrées début 2016 ($\int{\mathcal {L}} = 12.9 $~fb$^{-1} $)~\cite{CMS:2016vew} et le dernier correspondant à $ \int{\mathcal {L}} = 35.9 $ ~ fb$^{-1} $ collecté pendant toute la période de prise de données de 2016 pp~\cite{Sirunyan:2017xse}. J'ai été principalement impliqué dans la dernière analyse, où j'étais responsable de l'estimation de l'un des milieux SM, dans lequel un boson Z se désintègre en deux neutrinos. Dans cette thèse, j'ai également exploité une technique de marquage de jets fusionnés provenant d'un boson W et j'ai montré que le gain de mise en œuvre d'une telle technique augmente avec la luminosité intégrée.

%The searches for the stop pair production in different final states were already performed with the Run~1 data. The analyses exclude stop masses in terms of simplified model spectra with stops decaying to  a top quark and a neutralino up to around 755~GeV for a neutralino mass below 200 GeV. With the increase of the center of mass energy and then also the integrated luminosity, it was soon possible to further probe the stop masses. As it can be seen in Fig.~\ref{fig:figures/resultPlot2}, no excess was found in the full 2016 data ($\int{\mathcal{L}}=35.9$~fb$^{-1}$) with respect to the SM background predictions and therefore exclusion limits were derived. This analysis excluded stop masses up to 1120 GeV for a massless neutralino in terms of simplified model spectra where both stops decay to a top quark and a neutralino as shown in Fig.~\ref{fig:figures/limitT2tt}. In the case where both stops decay to a bottom quark and a chargino or in the case of mixed decay, the stop masses were excluded up to 1000 GeV  and 980 GeV, respectively, presented in Figs.~\ref{fig:figures/limitT2bW2} and \ref{fig:figures/limitT2tb}.

Les recherches pour la production de paires des stops dans différents états finaux ont déjà été effectuées avec les données Run~1. Les analyses excluent les masses d'arrêt en termes de spectres de modèles simplifiés avec des stops se désintégrant en un quark top et un neutralino jusqu'à environ 755~GeV pour une masse de neutrinos inférieure à 200 GeV. Avec l'augmentation du centre d'énergie de masse puis de la luminosité intégrée, il était bientôt possible de sonder davantage les masses de stop. Comme on peut le voir sur la Figure~\ref{fig:figures/resultPlot2}, aucun excédent n'a été trouvé dans les données complètes de 2016 ($ \int{\mathcal {L}} = 35.9 $~fb $^{-1} $) en ce qui concerne les prédictions de bruit du fond SM et, par conséquent, des limites d'exclusion ont été obtenues. Cette analyse a exclu les masses jusqu’à 1120 GeV pour un neutrino sans masse en termes de spectres de modèles simplifiés où les deux stops se désintègrent en un quark top et un neutralino, comme le montre la Fig.~\ref{fig:figures/limitT2tt}. Dans le cas où les deux stops se désintègrent en quark bottom et en chargino ou en cas de désintégration mixte, les masses de stop ont été exclues jusqu’à 1000 GeV et 980 GeV, respectivement, présentées dans les Fig.~\ref{fig:figures/limitT2bW2}  et~\ref{fig:figures/limitT2tb}.

    \insertFigure{figures/resultPlot2} % Filename = label
                 {0.75}       % Width, in fraction of the whole page width
                 {La présentation des données observées (points noirs), des estimations de signal du fond attendu pour trois modèles de signaux différents dans les 31 régions de signal. Le signal du fond  du lepton perdu est affiché en vert, le signal du fond  $ 1 \ell $ de $ W + jets \to 1 \ell $ (c'est-à-dire pas du top) en jaune, le $ Z \to \nu \bar {\nu} $ de signal du fond  en violet et la contribution $ t \bar {t} \to 1 \ell $ en rouge. Le modèle T2tt ($ \tilde{t}_{1} \to t \tilde {\chi}^{0}_{1} $) avec une masse de stop de 900~GeV et une masse LSP de 300~GeV est représenté par une ligne pointillée bleue. Le T2tb ($\tilde{t}_{1} \to t \tilde{\chi}^{0}_{1} /\tilde{t}_{1} \to b \tilde{\chi}^{\pm}_{1}$) et T2bW ($\tilde{t}_{1} \to b \tilde{\chi}^{\pm}_{1} $) scénarios pour une masse de stop de 600~GeV et une masse de neutrinos de 300~GeV sont respectivement représentés par des lignes pointillées roses et vertes. La ligne pointillée rouge sépare les zones de signal des analyses compressées nominales et optimisées~\cite{Sirunyan:2017xse}. }
                 %{ The presentation of the observed data (black dots), background estimates and expected signal for three different signal models in the 31 signal regions. The lost lepton background is shown in green, the $1 \ell$ background from $W+jets \to 1\ell$ (i.e. not from top) in yellow, the $Z \to \nu \bar{\nu}$ background in violet and the $t\bar{t} \to 1\ell$ contribution in red. The T2tt model ($\tilde{t}_{1} \to t \tilde{\chi}^{0}_{1}$ ) with a stop mass of 900~GeV and an LSP mass of 300~GeV is shown in blue dashed line. The T2tb ($\tilde{t}_{1} \to t \tilde{\chi}^{0}_{1} /\tilde{t}_{1} \to b \tilde{\chi}^{\pm}_{1}$) and T2bW ($\tilde{t}_{1} \to b \tilde{\chi}^{\pm}_{1}$) scenarios for a stop mass of 600~GeV and a neutralino mass of 300~GeV are shown in pink and green dashed lines, respectively. The red dashed line separates the signal regions of the nominal and optimized compressed analyses~\cite{Sirunyan:2017xse}. }

    \insertFigure{figures/limitT2tt}
                 {0.49}       % Width, in fraction of the whole page width
                 {Les limites d'exclusion à 95\% CL sur le modèle T2tt correspondant à une luminosité intégrée de 35.9~fb$^{-1}$ ~\cite{Sirunyan:2017xse}. L'interprétation est fournie dans le contexte du SMS, dans le plan de la masse de stop par rapport à la masse du LSP. Le code de couleur indique la limite supérieure de 95 \% CL sur la section transversale en supposant un rapport de branchement de 100\% dans $ \tilde {t}_{1} \to t \tilde{\chi}^{0}_{1} $. Les contours rouge et noir représentent respectivement les limites d'exclusion attendues et observées. Les lignes bleue et magenta dans le tracé de gauche montrent les limites attendues des analyses $ 1 \ell $ et $ 0 \ell $, respectivement. La zone sous la courbe épaisse noire indique la région exclue des points de signal. }

                 %{ The exclusion limits at 95\% CL on the T2tt model corresponding to an integrated luminosity of 35.9~fb$^{-1}$ ~\cite{Sirunyan:2017xse}. The interpretation is provided in the context of SMS, in the plane of the stop mass vs the LSP mass. The color code shows the 95\% CL upper limit on the cross section assuming a branching ratio of 100\% in $ \tilde{t}_{1} \to t  \tilde{\chi}^{0}_{1} $. The red and black contours represent the expected and observed exclusion limits, respectively. The blue and magenta lines in the left plot show the expected limits of $1 \ell$ and $0 \ell$ analyses, respectively.  The area below the black thick curve indicates the excluded region of the signal points.  }


    \insertFigure{figures/limitT2bW2} % Filename = label
                 {0.5}       % Width, in fraction of the whole page width
                 { Les limites d'exclusion à 95\% CL sur le modèle T2bW correspondant à une luminosité intégrée de 35.9~fb$^{-1}$~\cite{Sirunyan:2017xse}. L'interprétation est fournie dans le contexte du SMS, dans le plan de la masse de stop vs masse LSP. Le code couleur indique la limite supérieure de 95\% CL sur la section. Les contours rouge et noir représentent respectivement les limites d'exclusion attendues et observées. La zone sous la courbe épaisse noire indique la région exclue des points de signal. }
                 %{ The exclusion limits at 95\% CL on the T2bW model corresponding to an integrated luminosity of 35.9~fb$^{-1}$~\cite{Sirunyan:2017xse}. The interpretation is provided in the context of the SMS, in the plane of the stop mass vs LSP mass. The color code shows the 95\% CL upper limit on the cross section. The red and black contours represent the expected and observed exclusion limits, respectively. The area below the black thick curve indicates the excluded region of the signal points.  }


    \insertFigure{figures/limitT2tb}
                 {0.49}       % Width, in fraction of the whole page width
                 {Les limites d'exclusion à 95\% CL sur le modèle T2tb correspondant à une luminosité intégrée de 35.9~fb $^{-1}$~\cite{Sirunyan:2017xse}. L'interprétation est fournie dans le contexte du SMS, dans le plan de la masse de stop par rapport à la masse du LSP. Le code de couleur indique la limite supérieure de 95\% CL sur la section transversale en supposant un rapport de branchement de 50\% dans $\tilde{t}_{1} \to t \tilde {\chi}^{0}_{1} $ et 50\% dans $\tilde{t}_{1} \to b \tilde{\chi}^{\pm}_{1} $. Les contours rouge et noir représentent respectivement les limites d'exclusion attendues et observées. Les lignes bleue et magenta dans le tracé de gauche montrent les limites attendues des analyses $ 1 \ell $ et $ 0 \ell $, respectivement. La zone sous la courbe épaisse noire indique la région exclue des points de signal. }
                 %{ The exclusion limits at 95\% CL on the T2tb model corresponding to an integrated luminosity 35.9~fb$^{-1}$ ~\cite{Sirunyan:2017xse}. The interpretation is provided in the context of SMS, in the plane of the stop mass vs the LSP mass. The color code shows the 95\% CL upper limit on the cross section assuming a branching ratio of 50\% in $ \tilde{t}_{1} \to t  \tilde{\chi}^{0}_{1} $ and 50\% in $ \tilde{t}_{1} \to b  \tilde{\chi}^{\pm}_{1} $. The red and black contours represent the expected and observed exclusion limits, respectively. The blue and magenta lines in the left plot show the expected limits of $1 \ell$ and $0 \ell$ analyses, respectively.  The area below the black thick curve indicates the excluded region of the signal points.  }



%Despite the stop exclusion in the TeV range, there is still room for the natural supersymmetry and therefore the effort to search for stops is not diminishing. According to the CMS predictions, with an integrated luminosity of 3000~fb$^{-1}$ envisioned to be collected at the HL-LHC, it will be possible to probe the stop masses up to 2~TeV. An upgrade of the LHC to higher center of mass energies would permit us to go even further in the stop masses. 

Malgré de l’exclusion du stop dans la gamme TeV, il reste de la place pour la supersymétrie naturelle et, par conséquent, l’effort de recherche des stops ne diminue pas. Selon les prévisions de la CMS, avec une luminosité intégrée de 3000~fb$^{-1} $ devant être collectée au HL-LHC, il sera possible de sonder les masses de stop jusqu'à 2~TeV. Une mise à niveau du LHC vers l’énergie dans le centre de masse plus élevé  nous permettrait d'aller encore plus loin dans les masses de stop.

%Before migrating to future projects, there are several directions how to improve the presented stop analysis for its future update based on data with a larger integrated luminosity. On the side of the SM backgrounds it is desired to use the data-driven methods for the background estimations in order to minimize a dependence on the simulation and consequently reduce the systematic uncertainties. The data-driven estimates are based on control regions, these control regions should be defined very similarly as the signal regions to avoid uncertainties due to extrapolations in variables.  These extrapolations rely on the shapes of variables in simulation, and if not well modeled they largely increase the systematic uncertainties on the background estimates. On the side of the signal we discussed that the simplified event simulation is not reliable in the region of the 2D mass plane where $\Delta m(\mathrm{stop, neutralino}) \sim m(\mathrm{top})$ at low neutralino mass. This could be solved by using the full simulation of the detector in this particular kinematic region. Then also the analysis strategy can be reoptimized. It is possible to reoptimize the signal regions by finding optimal cuts but also trying to find better discriminating variables. Another option is to use more object taggers. Nowadays there is a variety of taggers targeting the W boson and top quark identification, not only targeted as a  single merged jet, but also as completely resolved jets or partially merged jets. We could also consider not to perform a cut and count analysis but move to machine learning techniques providing discriminant between signal and background we can cut on. 

Avant de migrer vers des projets futurs, il existe plusieurs directions pour améliorer l'analyse de stop présenté pour sa future mise à jour basée sur des données avec une luminosité intégrée plus grande. Du côté des contextes SM, il est souhaitable d'utiliser les méthodes basées sur les données pour les estimations de bruit du fond afin de minimiser la dépendance à la simulation et, par conséquent, de réduire les incertitudes systématiques. Les estimations du bruit du fond sur les données sont basées sur des régions de contrôle, ces régions de contrôle doivent être définies de manière très similaire aux régions de signal pour éviter les incertitudes dues aux extrapolations dans les variables. Ces extrapolations reposent sur la forme des variables dans la simulation et, si elles ne sont pas bien modélisées, elles augmentent largement les incertitudes systématiques sur les estimations du bruit du fond. Du côté du signal, nous avons discuté du fait que la simulation d'événement simplifiée n'est pas fiable dans la région du plan de masse 2D où $ \Delta m (\mathrm{stop, neutralino}) \sim m(\mathrm{top}) $ pour faible masse neutrino. Cela pourrait être résolu en utilisant la simulation complète du détecteur dans cette région cinématique particulière. Ensuite, la stratégie d'analyse peut également être réoptimisée. Il est possible de réoptimiser les régions de signal en trouvant des coupes optimales mais en essayant également de trouver de meilleures variables discriminantes. Une autre option consiste à utiliser plus  d'objet tagueurs. De nos jours, il existe une variété de tagueurs ciblant le boson W et l’identification des quark top, non seulement ciblés comme un jet fusionné unique, mais également comme des jets complètement résolus ou des jets partiellement fusionnés. Nous pourrions également envisager de ne pas effectuer d’analyse de coupure et de comptage, mais de passer à des techniques d’apprentissage automatique permettant d’établir une distinction entre le signal et le bruit du fond.

%Within the CMS collaboration a large variety of SUSY searches is performed. The stops are being sought in several different final states and the other predicted SUSY particles are targeted by dedicated searches as well. Many analyses search for SUSY within the context of the minimal supersymmetric standard model but there are also several searches which go beyond, e.g. searches allowing violation of the R-parity. In general the SUSY searches are interpreted in terms of simplified spectra. A lot of efforts are being done to reinterpret these searches in more realistic models. Another part of activities is the combination of searches to reach a better discovery/exclusion potential. The CMS experiment and in general the collider experiments do not provide the sole opportunity to search for supersymmetry. For example, supersymmetry can also be studied via direct dark matter detection experiments. 

Au sein de la collaboration CMS, une grande variété de recherches SUSY est effectuée. Les stops sont recherchés dans plusieurs états finaux différents et les autres particules SUSY prédites sont également ciblées par des recherches dédiées. De nombreuses analyses recherchent SUSY dans le contexte du modèle supersymétrique minimal, mais il existe également plusieurs recherches allant au-delà, par ex. recherches permettant de violer la R-parité. En général, les recherches SUSY sont interprétées en termes de spectres simplifiés. De nombreux efforts sont déployés pour réinterpréter ces recherches dans des modèles plus réalistes. Une autre partie des activités est la combinaison de recherches pour atteindre un meilleur potentiel de découverte/exclusion. L'expérience CMS et, en général, les expériences de collisionneur ne constituent pas la seule opportunité de rechercher la supersymétrie. Par exemple, la supersymétrie peut également être étudiée par des expériences de détection directe de matière noire.

%In summary, there are still many options for natural SUSY in term of parameter space and also many options how to search for it. Collider experiments are one of these options, and with increasing luminosity, energy and optimization of analyses they are providing great opportunities to go beyond the current limits and further probe the uncovered phase-space.

En résumé, il existe encore de nombreuses options pour SUSY naturel en termes d'espace de paramètres, ainsi que de nombreuses options pour le rechercher. Les expériences de collisionneur sont l'une de ces options et, avec l'augmentation de la luminosité, de l'énergie et de l'optimisation des analyses, elles offrent de grandes possibilités pour dépasser les limites actuelles et explorer davantage l'espace de phase non couvert.

%----------------------------------------------------------------------------------------------------------------------------------------------------------------------------------
