\chapternonum{Conclusion}

%TODO past/present

This thesis is divided into two parts, first part is dedicated to the CMS detector and especially the CMS silicon strip tracker. The studies and measurements in Chapter~\ref{sec:HIPch} were motivated by the observed inefficiencies in the tracking during the 2015 and 2016. The first analysis of this chapter performs study of the HIP events as a possible explanation of these inefficiencies. Shortly, it was realized that the HIP effect itself is not responsible for the inefficiencies in tracking, but not optimal settings of APVs was found the main cause.  The APV settings were quickly changed and second data were taken in order to perform analysis of the HIP events which do not suffer from the inefficiencies introduced by the APV settings. This fact and also the conditions of the data-taking has provided a possibility to perform a first HIP probability measurement at CMS. The HIP probability per pileup has been computed computed for each layer/wheel/ring of the silicon strip tracker and it has been found to be of the order of~$10^{-6}-10^{-5}$ depending on the tracker layer/wheel/ring. This analysis also focuses on the fake clusters induced by the HIP events. 

In Chapter~\ref{ch:simu} the simulation of clusters in the silicon strip tracker is discussed. The dependency of the cluster charge, seed charge and width on the parameters used in the simulation has been shown. It has been identified that the outdated cross talk parameters lead to the discrepancies of the clusters seed charge and width description between data and simulation. This observation motivated the measurement of the cross talk parameters from the 2018 CRUZET VR data. Unfortunately, due to trigger condition, only cross talk in barrel could be measured and a new cross talk for the disks and endcaps had to be evaluated with help of both data and simulation. The new cross talk parameters improved largely the description of data by simulation in the cluster seed charge and cluster width and therefore they have been included into the official release of the CMS software.

The Chapter~\ref{sec:stopch} describes performed Run~2 searches for the top squark pair production in the single lepton final state and especially focuses on the analysis of the full 2016 dataset corresponding to the integrated luminosity of 35.9~fb$^{-1}$. This chapter discusses the whole analysis chain and highlights my contributions to this analysis. No excess from the standard model has been observed in any of the studied stop decay modes which are $\tilde{t}_{1} \to t  \tilde{\chi}^{0}_{1} $, $\tilde{t}_{1} \to b  \tilde{\chi}^{\pm}_{1}$, and $ \tilde{t}_{1} \to t  \tilde{\chi}^{0}_{1}/\tilde{t}_{1} \to b  \tilde{\chi}^{\pm}_{1} $ and therefore the exclusion limits has been derived in terms of simplified model spectra in the plane of the stop vs. LSP masses. The strongest limit on the stop masses has been obtained for the decay mode where both stops decay to b-quark and neutralino $\tilde{t}_{1} \to t  \tilde{\chi}^{0}_{1}$. In this case the full 2016 analysis is excluding the stop masses up to 1120~GeV for a massless LSP. This result is reaching the naturalness limit on the stop mass which was identified to be around 1~TeV. At the end of this chapter the naturalness bound is revisited and the constraints on the masses of the SUSY particles from the observed cold dark matter density are also discussed. The final thoughts are about the possibility of the SUSY theories beyond the MSSM.
