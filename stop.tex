\chapter{Stop}

\section{Introduction and motivation}

The supersymmetry was found to be the most popular extention of the Standard Model, due to its capabality to adress many shortcomings of the SM. Among others it provides solution to the naturalness probelm and provieds dark matter candidate. Supersymmetry introduces a supersymmetric partner to each SM particle, which have the same quantum numbers except of spin differing by 1/2. This chapter is focused on the production of the suppersymetric partners of the stop quarks, reffered as ``stops''~($\tilde{t}$). There are two scalar stops $\tilde{t}_{R}$ and  $\tilde{t}_{L}$ as there is left and right component of the SM fermion field. These two stops mix into mass eigenstates $\tilde{t}_{1}$ and $\tilde{t}_{2}$,  $\tilde{t}_{1}$ being the lighter one. Due to the naturalnes constraint the ligter stop should have mass in TeV range. Morover the Higss mass measurements give condition that $\sqrt{m_{\tilde{t}_{1}} m_{\tilde{t}_{2}}}$ should be around 600~GeV. In this chapter a SUSY model, in which the R-parity is conserved and the LSP is the lightest neutralino~($\tilde{\chi}^{0}_{1}$), is considered. Furthermore only direct production of stop pair is assumed.

Depending on the mass difference between stop and neutralino, $\Delta m =  \tilde{\chi}^{0}_{1}$, several decay modes of the stop quark are possible. The stops can decay via two, three or four body decays to final states with b or c quarks.



-intro what we focus on

- mtoivation
	two stpos of different mass, stop1 and stop2
	short conclusion of susy chapter - light stop, 1-lep, high xsection -> basic or sth more like Alex?
	single lepton high br and low backgrounds
	targeting few posssible decay chains
	can be destinguished from backgrounds when cutting on MET, MT...
	32\% of directly produced stops decay to final states with one lepton
-signature  and possible decays
	SMS, T2TT, T2bW. T2tb, decay cascades, feynam diagrams
	r parity conservation
	

-plane od delta M
	search in delta M plane


-general intro
	- I worked on more versions of analyses, but only one introfuced and than the differences, and then some chosen personal contribution

-delta M around top mass challenging kinematics -> looks like SM tt -> this region is called stealthy region
-for t2bW when W becomes on-shell also difficult kinematics



\subsection{Overview/General idea}
-give general idea
	-overview

reduce SM background
binned approach in variables which tend to reduce signal

\subsection{Signal topologies}

The preseneted analysis focus on the kinematic region where $\Delta m > m(W)+m(b)$ and on three different decay modes.

All final , but the kinematics differ depending on the decay mode and $\Delta m$. The  $\Delta m$ plane is shown in Fig.~\ref{fig:figures/dmplane}.

    \insertFigure{figures/dmplane} % Filename = label
                 {0.99}       % Width, in fraction of the whole page width
                 { dm plane ~\cite{Aad:2014kra}. }


The stop quark can either decay to top quark and neutralino, or b quark and chargino
-what they are, how they look like
-t2tt, t2bw,...
    \insertTwoFigures{figures/stopdecays}
                 {figures/T2tt} % Filename = label
                 {figures/T6bbW} % Filename = label
                 {0.45}       % Width, in fraction of the whole page width
                 { decay diagrams ~\cite{website:SUSYdiagrams}. }

    \insertFigure{figures/T4tbW} % Filename = label
                 {0.99}       % Width, in fraction of the whole page width
                 { mixed decay diagram ~\cite{website:SUSYdiagrams}. }
-> the motivate variables and backgrounds





\subsection{Variables}
-put picture of some of my presentetions with rounds (midterm probably)i
leptonic diagram~\cite{CMS:2016vew}
-variables
	MET significance?!
	MT -formula; suppress largely tt1l background; and also single top and W+jets; kinematic endpoint at W mass
	MT2W - formula; explanation; supress lost lepton (already suppressed by requiring no additional lepton, but not enough); tries to reconstruct the event under tt2l and one undetected lepton assumption; for signal large delta M leads to alrge MT2W, while small delta M has lage MT2W; endpoint at top mass
	tmod - formula; chi2 like variable how well the event agrees with tt2l hypothesis, similar behavior to Mt2W; removing some of the terms from oifficial topness helps the discrimination; works better at low jet multiplicities than MT2W
	min dphi (jet, MET) - formula
        Njets
	Mlb

\subsection{Backgrounds}
-backgrounds 
	what type of
-how they look like, how they can be distingusehed from signal
	tt2l - lost lepton does not obey MT<MW
	W+jets - offshel Ws - no endpoint at W mass
	Znunu no bound Mt<MW

\subsection{Triggers, data and simulated samples}

	single lepton and MET triggers
	double lepton - to check kinematics of lost lepton
	single photon - to check MET resolution

-MC samples
	fastsim

\subsection{Objects and event selection}
	vertex selection 
	lepton selection - high pt isolated lepton, no veto lepton to reduce tt2l
	isolated track veto - tracker isolation to avoid taus
	hadronic tau veto (2015) - additional to the isolation, or selects additional taus
	jets - ak4 (pt>30, eta<2.4), medium working point for btag(CsVv2) (2015)
	MET - sum of PF candidates, type-1 corrected - jet energy correction applied to the jets in MET calculation, MET filters


\section{Search strategy - latest analysis}
- preselection (baseline selection), nr of jets, min dphi, MT, MET
then binning in 4 search variables
	N jets - we expect 4, but jet can be lost if neutralino and chargino are mass degenerated (soft jet), or if jets are merged
        	(ICHEP 2,3 and 4+ jets, now 2 and three jets bins merged)
	tMod - during ichep only at some search regions otherwise MT2W, now with more stats, this variable can be used everywhere (tmod<0 compressed, tmod>10 large dm, in between bulk)
	Mlb >175(T2bW or W+jets like), Mlb<175 (T2tt) invariant mass of closest b jet to lepton and lepton
		fir regions with high mlb and tmod large contribution of W+jets -> modification of b-tagging - 1 and more tight b-jets
	MET
	-> 27 search regions (exclusive)

+ compressed mW<dm<mt
	same preselection plus
	ISR -> 5th jet, leading jet not b-tagged
	soft lepton - pT<150GeV
	boosted top quarks dphi(lep,MET)<2
	min dphi(jet,MET)>0.5
        -> 4 MET bins
in totality 31 signal regions

\subsection{Baseline selection}
	exactly one good lepton
	MT, MET
	MT2W (2015) below and above 200GeV -> in compressed spectra the MT2W is small (similar to background), in large mass spliting it can strongly suppress tt2l
	3 or 4 jets - boosted topologies (2015)
        compressed spectra (2015) tmod instead of MT2W + two jets - jets are soft, the only visible ones are from b-tag hadronization

\subsection{Signal regions}
	- first vertex in event pass the good quality criteria
	- pass one lepton selection
	- reject additional (veto) leptons
	- at least two jets
	-at leats one b-tag
	-MT>150
	-dphi>0.8
	+ compressed, boosted high delta M, low delta M, High delta M regions (in 2015)

-\subsection{Control regions}



-binning in
	MET
	MT2W (2015)
	

\subsection{Undefined}
-SF, rewighting

-compressed spectra
	ISR jets
	recoil, enhance MET

\subsection{Background estimation}

-data driven + simulation
-general idea
-MET extrapolation?
-transfer factors
-scale factors - take into account differences in lepton, b-tagging etc efficiency
-composition of backgrouns - percentage

\subsubsection{Z to nunu}

-ttZ, WZ, ZZ
-2015 dataset taken from simulation -> small stats
-large in high MET and MT2W regions

-2017
-ZZ still taken from simulation
-3l scale factor -> normalization

\subsection{Single lepton}

-sensitive to MET and MT
-W+jets, tt1l
- tt1l -> constrained kinematics of W due to top, so MT tail is just due to MET resolution
-W+jets, no kinematics constraint, MT tail due to of shell W production (W width) 
-W+jets, zero b-tag control region
-tt1l -> negligible, estimation from simulation

-1 lep neutrino from W decay

ESTIMATIN:
-data driven, similar as for the lost lepton
-CR contamination N CR,data,Wjets,0btag = N CR,data,0btag - N CR,MC,ninWjets,0btag
-MET and NJET extrapolation (2015)
-W+b modelling systematics

\subsection{Lost lepton}

-dominant background after MET and MT selections
-tt2l wit both Ws decaying leptonically and one lepton lost
-largest contribution is the tt2l, then single top and then ttV and diboson processes.
-dilepton control region
-misreconstruction of lepton - 1 lepton and large MET
-two enutrinos lare MT and MT2W

ESTIMATON:
- N data,SR (ll) / N data,CR (ll) = N MC,SR (ll) / N MC,CR(ll) -> get N data,SR
-MET and Njet extrapolation (2015) -> additional transfer factor
-good modeling of njet and MET needed -> mismodelling in ssimu leads to mismodelling in TF -> special emu CR -> additional SF due to mismodelling
-gamma plus jets -> MET resolution > bin migrations - gamma pt spectrum reveighted to match neutrino pt spectrum, from reweighted events METmodified = reconstructed-MET + pTgamma


-systematics

\subsection{Results and interpretation}

\section{Differences in previous analyses}

\section{Selected topics}

\subsection{W-tagging}

\textbf{Motivation}

	high delta M regim - boost -> jets merge

\textbf{Techniques}
	larger radius jets
	tau ratios - N subjetiness
	different masses-> grooming techniques
	
\textbf{Results}
	end of 2015 - slide 8,9 - results on different ak8 W-tagging categories (18/11); lumi=2 1/fb
	mid of 2016 -update of previous study - lumi=2.26 fb; ak8+ak10 tagging; tables slide 7, 11 -> mo improvement with ak8, but slight improvement with ak10 


\textbf{Perspectives}
	-results from Sicheng with top tagging
	-resolved top tagger - 3ak4 jets
	-second presentation (31/01 same as 12 feb?), slide 6 - interesting plot, slide 10,11 significance table
        -merged top tagger - boosted objects
		slide 4, 5, 6

\subsection{Depndence of discriminating variables on pileup}
-not much to say
	
