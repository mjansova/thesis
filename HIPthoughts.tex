The problem for physics:
%https://twiki.cern.ch/twiki/bin/viewauth/CMS/SiStripHitEffLoss
%https://indico.cern.ch/event/450121/contributions/1946079/attachments/1217237/1777959/HIP_GB_160126.pdf
%https://indico.cern.ch/event/560224/contributions/2265347/attachments/1320462/1980048/WGM_HIP_Boudoul.pdf
%https://indico.cern.ch/event/560226/contributions/2277448/attachments/1324704/1988050/wgm_vfp_change_ebutz.pdf
%https://indico.cern.ch/event/536853/contributions/2269134/attachments/1320885/1980813/Hall_DPG_0816.pdf -> about APV design
%https://indico.cern.ch/event/570778/contributions/2308697/attachments/1340097/2017731/hip-caltech.pdf -> hip story, mitigation, impact
%-------------------------
increase of instantaneous luminosity
 With increasing instantaneous luminosity we observe:

    lower cluster charge
    lower S/N
    lower hit efficiency
    shorter tracks
    lower track efficiency 

-tracks fewer hits:
- hit inefficiency
-reduced cluster charge
-relaxed? cluster charge cut? _> there is a change, different procedure
-less tracks in  2015RunD tahn in MC, not visinle in RunC  !!!
-The effect is lumi dependent, but also BX dependent (electronics recovers during gaps)
-lower cluster charge, less hits and also less reconstructed vertices -> tracks get shorter
-problems with cc and cluster position -> loss of hit
-more reasons for track to stop (nr of lost hits, nr of lost consecutive hits, ...)
-mitigation: ccc retuning
-mitigation: inactive hits, retune matching hits in stereo modules -> but can produce a lot of fake tarcks (applied?) -> not large efffect but in general worsenes tracking performance but helps in B-tagging
-mitigation: HIP simulation -> looked reasonable at that time
%-------------------------
actions taken
%------------------------
-discussion with the people who created APV etc
-simulation: HIP math model, function and fit code -> mitigation
-recontruction: Recipe How to Reconstruct VirginRaw Data and reco datasets (new ZS, e.g. iterative median etc)
-Inefficient modules identification/Classification in VR data -identify modules with missing hit and categorize what is happening
-hit efficiency studies
-impact of HIP on cluster charge and S/N 
-adjusting the baseline level: in order not to be that much sensitive to teh drop, did not help, only spurious clusters from noise appeared
-Evolution of mean cluster charge with bx, with old and new APV settingsi (as a function of time, bx, inst lumi, etc...)
-parameters in APV - considering change

%-------------------------
HIP on the physics:
%-------------------------

Jet/MET
- shorter tracks are ok, no big problems for jet and met at first -> nothing seen as a function of BX
- there is only small effect for plow pT jets
- but then: Large MET tails arising from iterations in muon tracking steps, which lead to spurious muon or charged hadron with high pt and large error
	- reconstruction issue influencing both data and MC
	- main source: not high purity tracks in PF algorithm
	- this causes problems to analyses with a lot of fake MET (fully hadronic searches) or high pt muon searches
	-> bad muon filter + bad charged hadron filter    
- problem with JEC - probably problem with photon energy scale

Muon:
-medium and tight muon ID efficiency -> drop with the inst lumi -> significant impact
-muon tracking efficiency drop -> not seen in MC
-nr of primary vertices drop -> not seen in MC
-increased muon fake rate due to mitigation

Electron:
-electron ID and reconstruction influenced by HIP

Tau:
-some effect, but well mitigated with changes in tarcking

BTV:
%https://indico.cern.ch/event/544743/contributions/2219210/attachments/1299735/1939738/20160628_xpog.pdf
- hit multiplicity and selected tracks multiplicity disagreement between data and simu
- an effect up to 10\% on the SV reconstruction
-disagreement in number of b-tagged jets
-> problems with scale factors, dependent on luminosity
-> mitigation - improved track selection, relaxin  number of hits in strip tracker, while keepinh nr of hits in pixels high
- 0.65 b tagging efficiency without hip, with hip up to only 0.5


Physics analyses:
-bad muons: %https://indico.cern.ch/event/605717/contributions/2443762/attachments/1398823/2133673/gp-badmuons-190117.pdf
-bad muons create fake MET O(0.1\%) effect but
-in Z->mumu + 0 jets completely dominates the MET tail
-bad muon is poorly reconstructed copy of the good muon
	-inside-out tracking reconstructs a muon
	-outside-in tracking tries to also reconstruct it, and sometimes produces also a fake track
- disappearring tracks searches largely influenced -> in general really influences searches which plays with tracks
