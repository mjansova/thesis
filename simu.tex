%Remove the word tuning
%Explain how the simulation is working → how to simulate cluster, describe how it is done
%Present what is the situation
%Description of width not so great
%Not put too much emphasis on the other tests
%Tehn found xtalk is responsible
%Cross talk measurement → 
%Limitations
%Then xtalk from MC + ePerADC

\clearpage

\setcounter{secnumdepth}{4}
\chapterwithnum{Silicon strip tracker simulation}
\setcounter{secnumdepth}{5}


\section{CMS simulation}

The simulated samples are vital part of many analyses. For the physics analyses purposes they are used to copmare the theoretical signal and background with the measured data.  Further, the simulations are also important in development and understaning of specific analysis methods and in derivation and valiadation of calibrations, efficiencies and resolutions.


%steps
The CMS simulation workflow~\cite{Banerjee:2007zz, Hildreth:2017vpw, Hildreth:2015kps, website:simuBasics } is divided into several steps. At the beginning of the simulation chain the physics events are generated and then the generated final state particles are sent through the simulated detector. Following step is the simulation of response of electron electronics to particle traversing the detector. The otput of this producedure are RAW data, which can be later reconstructed and slimmed for the purposes of physics analyses. The overview of the simulation steps, which will be described in larger detail in following subsections, can be seen in Fig.~\ref{fig:figures/SimulationFlow}~\cite{website:simuBasics}. The production of the simulated samples is handled centrally~\cite{Boudoul:2015bkp} by the CMS collaboration.

    \insertFigure{figures/SimulationFlow} % Filename = label
                 {0.99}       % Width, in fraction of the whole page width
                 { A diagram of simulation workflow. The four-vectors of generated particles together with the detector description enter to the Geant4 simulation which output are simulated hits in the detector volume. Optionally the simulated hits from pile-up interactions can be added on top of the simulated hist from physics and the mixture of these hits are digitized in the electronics simulation. In this step the description of the electronics, for eaxample the noise and detector conditions, for example the temperature, is added. The output of digitization are RAW data~\cite{website:simuBasics}. }

\section{Monte Carlo event generators}

The Monte Carlo~(MC) generators are basic tool designed to produce physics events according to a physics model. In the majority of cases three kinds of generators are used in CMS~\cite{website:generation, website:generationIntro}. 

\textbf{General-purpose generators}
These are for example Pythia8~\cite{Sjostrand:2014zea} or Herwig++~\cite{Bahr:2008pv}. The provide the best possible description of the result of the proton collision. To generate outgoing particles originating from the interaction of colliding particles, many theoretical models and aspects has to be plugged in the generation process, such as the description of soft and hard interactions~(in leading order), parton distribution functions~(PDFs), initial and final state radiation~(ISR and FSR), multiple parton interactions, hadronization of partons and decay of particles~\ref{}.

\textbf{Matrix Element calculators}
The generators such as Powheg~\cite{Oleari:2010nx} or MadGraph5\_aMCatNLO~\cite{Alwall:2014hca} were developed to provide next-to-leading order~(NLO) claculations. These calculators give the final state description on the parton level which needs to be plugged into one of the general-purpos generators to proceed with  the full hadronization.

\textbf{Specific generators}
These genrators are used to generate specific kind of events e.g. diffractive or cosmic events.

-> output: particle four-vectors

\section{Detector simulation}

To be able to compare the data and simulations, the generated particles need  to be propagated through the volume of detector. This is achieved via GEANT4~\cite{Agostinelli:2002hh ,Lefebure:1999wja} toolkit into which detailed description of the CMS detector, its active and dead material dimensions, hierarchy and properties, is plugged. The GEANT4 sends the generated particles through the detector and simulates the interactions with material and modells the physics processes which happen during the passage of the particles through detector. The output of this procedure are simulated hits left by particles interacting with the active volumes of the subdetectors. The simulated hits can originate from primary particles generated by the MC generator, or from the secondary particles which are result of the GEANT4 simulation process.

The simulation of the pile-up events is done separately from the simulation of the events of interests. The input to  the GEANT4 simulation of in-time and out-of-time pile-up is pool of Minimum Bias single interaction events.

This full simulation~(FullSim) is very time intensive and thus it is not suitable to simulate samples for which huge number of events is needed. For this purposes the fast simulation~(FastSim)~\cite{Sekmen:2017hzs, CMS:2010spa, Giammanco:2014bza} was developed as an alternative to the FullSim. The FastSim uses simplified detector geometry and interactions with material, what speeds the simulation by factor of around 100. The comparison of physics objects of FullSim and FastSim shows that FastSim is reliable alternative that reproduces the FullSim with around 10\% accuracy~\cite{Abdullin:2011zz, Sekmen:2017hzs} The FastSim is widely used to produce for example Supersymmetry samples, where large scans with different parameter values are needed. 

\section{Simulation of the detector response to the particle signal}

The next step is to simulate the collection of the signal obtained from the GEANT4 and the response of the readout electronics to this signal. This step is called a digitization and its input is merged collection of physics events of interest and pile-up events. There are three domains providing digitization of given siubdetectors, which are SimTracker, SimCalorimetry and SimMuon~\cite{iwebsite:simdigi}. The digitized saples are in the RAW format and can be further reconstrcutedi in a similar way as data.

\section{Simulation of the silicon strip tracker to the particle signal}


A) Geant 4:~\cite{Lefebure:1364020}
CMSTrackerHit
One CMSTrackerHit object is created
for each new particle entering a Tracker-like component (delta-rays are considered as new particles)
for each sensitive detector unit
The information provided by the CMSTrackerHit class is:
1) the entry and exit point of the particle in the local reference frame of the detector unit,
2) the energy of the particle when it enters the detector unit,
3) the identification of the track and of the detector unit,
4) the time that the particle has been alive until it enters the detector,
5) the total amount of energy deposited by the particle along its trajectory in the detector.


    \insertFigure{figures/geantDeposit} % Filename = label
                 {0.5}       % Width, in fraction of the whole page width
                 { ~\cite{website:simuBasics}. }


B) charge collection (CR2009\_338)
- track in sensitive volume is divided into small segments of the same size (compared to the sensor thickness)
-the energy is divided between these segments and for each segment the energy depesition is fluctuated by a GEANT4 routine (ask?)
-the charge(energy) from each track segment is drifted towards strips/backplane
- there is a diffusion of the charge in the perpendicular(ask?) plane
- the drift is not straight but as mg. field is present, the charge carriers are deflected -> injected knowledge about the lorentz angle
- the collected charge is multiplied by gains and converted to the ADCs
- inject interstrip cross-talk to change the resolution of clusters and tracks - charge collected by neighbors (ask before or after ePerADC?)
-then zero-suppression

C)Flow(from Nuttens slides)
- signal from GEANT4
-divide charge deposit (SiLinearcChargeDivider.cc)
-Cross-talk (SiTrivialInduceCahrgeOnStrip.cc)
-Remove dead channels <-database (SiStripDigitizerAlgorithm.cc, write something about condition databasei and global tag)
-APV killer for HIP (SiStripDigitizerAlgorithm.cc, VR only)
-baseline shift (VR only, SIGaussianiTailNoiseAdder.cc)
-baseline tilt (VR only)
-Noise <- database (SiGaussianNoiseAdder.cc)
-Gain at APV level <- database (SITRivialDigitalConverter.cc)
-CMN (VR only, SiGaussianiTailNoiseAdder.cc)
-Pedestal <- database (VR only, SIGaussianiTailNoiseAdder.cc)
-Digitization (including saturation SiStripDigitizerAlgorithm, SiStripDigitizer)
-to be reconstructed

-parameters in simulation picture, + highligt the ones which can chhange with time
-several changes in the detector

a) temperature
- used in difussion constant evaluation -> this constant is used to compute the drift time of the charge carriers in the sensor
-charge drift is one of the forst steps
-temperature of detector changed several times (in 2015 to -20, in to 2018 to ask?)
(-more parameters used in simu are probably temperature dependent!)

b)depletion and applied voltage
SiHitDigitizer.cc
double timeNormalisation = (moduleThickness*moduleThickness)/(2.*depletionVoltage*chargeMobility);
SiLinearChargeCollectionDrifter.cc (drift time in the sensor)
double driftTime = -timeNormalisation*log(1.-2*depletionVoltage*thicknessFraction/(depletionVoltage+appliedVoltage))+chargeDistributionRMS; 
-> the drift time is dependent on dpeletion and applied voltage!!!
SignalPoint drift (EnergyDepositUnit edu, Localvector drift, moduleThickness, timeNormalisation)

c) noise
Noise in ZeroSuppression inputs:
void addNoise(std::vector<double> &in,size\_t& minChannel, size\_t& maxChannel, int numStrips, float noiseRMS)
where noiseRMS used in: noiseRMS*theElectronPerADC/gainValue; gain and noiseRMS from database, describe what are the Gain values (G1, G2, GSim) -> in here it looks like it is gsim
genNoise->generate(numStrips, threshold, noiseRMS, generatedNoise); genNoise(new GaussianTailNoiseGenerator(rndEngine)); <- is it there, what does it do?

d) gains
-used when adding noise
-used when converting to ADC
DigitalVecType convert(const std::vector<double>& analogSignal, edm::ESHandle<SiStripGain> \&gainHandle, unsigned int detid)
-again changes with time

e) coupling (cross talk)
- can differ beacuse of aging of detector?
double chargeDeposited(size\_t strip,size\_t Nstrips,double amplitude,double chargeSpread,double chargePosition)
Determine integral and fraction of signal
-integralUpToStrip = (strip == 0) ? 0. : (normal\_cdf( strip, chargeSpread, chargePosition) );
-integralUpToNext = (strip+1 == Nstrips) ? 1. : ( normal\_cdf( strip+1, chargeSpread, chargePosition) );
-percentOfSignal = integralUpToNext - integralUpToStrip;
-return percentOfSignal * amplitude / geVperElectron
charge spread depends on applied and depletion voltage

void induce(collection\_type collection\_points,StripGeomDetUnit det,localAmplitudes,size\_t recordMinAffectedStrip, size\_t recordMaxAffectedStrip,TrackerTopology *tTopo)
-loop over all signal points 
-for each signal point define chargePosition(from signalpoint); first and until strip (chargePоsition±Nsigma*chargeSpread )
-the loop over strips 
-compute charge depositied on strip =chargeDeposited( strip, Nstrips, signalpoint->amplitude(), chargeSpread, chargePosition)
-Strip range affected by deposit (affectedFromStrip, affectedUntilStrip) -> loop over affected strips to apply induce charge
	-affectedStrip(localAmplitude)  += chargeDepositedOnStrip * coupling.at(abs( affectedStrip - strip ))

%magnetic field setting

%in detector dir
%TS2018_001_2 -> MC generators, detector simulatios
%CERN-THESIS-2017-300 - Event simulation
%TS2017_028_2 - simulation super short
