%Remove the word tuning
%Explain how the simulation is working → how to simulate cluster, describe how it is done
%Present what is the situation
%Description of width not so great
%Not put too much emphasis on the other tests
%Tehn found xtalk is responsible
%Cross talk measurement → 
%Limitations
%Then xtalk from MC + ePerADC

\clearpage

\setcounter{secnumdepth}{4}
\chapterwithnum{Silicon strip tracker simulation}
\setcounter{secnumdepth}{5}


\section{CMS simulation}

%motivation
-steps overview:
    \insertFigure{figures/SimulationFlow} % Filename = label
                 {0.99}       % Width, in fraction of the whole page width
                 { Simulation workflow~\cite{website:simuBasics}. }

\section{Generation}
produce MC samples for wide variaty of physicall processes -> MC event processing chain
generate HEP events -> generate outgoing particles prouced in interactions of incoming ones

Two kinds of generators
a) general-purpose generators, eg Pythia8 and Herwig++
- best possible description of the result of the hadron collision
-these generators contain theoretical models for a large range of physics ascpects (hard and soft interactions, parton distributions, initial and final state parton showers, multiple interactions, fragmentation and decay)

b) Matrix Element (ME) calculators, eg Powheg, MadGraph5_aMCatNLO
-produce final state at parton level -> then the group a) generators can be used to finish the event up to hadronized level

c) specific generators for eg diffractive events or cosmic events

-> output: particle four-vectors

\section{Simulation}
a) detector simulation - geant 4
-full-scale  simulation of the CMS - based on geant4
-detailed description of the hierarchy of volumes, materials, sensitive and dead parts of detector
-input are generated particles -> tehy are sent through detector: their deposits are traced + the underlying physics caused by passage through material is simulated -> interaction in detector saved as simulated hits (for example energy loss plus position within a sensitive volume)
-the sim hits can come from primary (generated) or secondary (result of geant 4) particles
-the simulation moduel is called: OscarProducer
-output: SimHits which contein (in general), which contain information about particle traversing a simualtion volume (dE/dx, entry and exit point, (check)....)

i) fullsim 

-full-scale Geant4 detector simulation
-steps of fullsim:
	-modelling of interaction region
	-modeleling of the passage of particles through the detector volume
	-modelling of the physics processes when particle passes teh detector volume
	-simulation of PU (multiple interactions per bx, events overlay)

ii) fastsim
-fatster (simulation of events speeded up ~100 times)
-> possible to simulate more physics processes
-used to produce large SUSY signal model scans
-simplified geometry and simplified model of interaction in material

b) detector digitization - response of the readout electronics


- three domains: SimTracker,SimCalorimetry,SimMuon -> each of these simulate response of several subdetectors
- these three modules are used after simulation of PU
- pile up simulation -> MixingModule - simulates pile up (more interactions at 1 bx), but also the fact that electronics read the signal longer than 1bx (this simulation is optional) - taken from pool of MinBiassingle events
-physics events~(signal or hard scattering events) plus pile up merged before digitization  
- collection of the signal and forming electronic signal
-> these samples then can eb reconstructed

%fullsim
%fastsim
%steps -> geant4, ....

%sources
%https://twiki.cern.ch/twiki/bin/view/CMSPublic/WorkBookGenIntro
%https://twiki.cern.ch/twiki/bin/view/CMSPublic/WorkBookSimDigi
%https://twiki.cern.ch/twiki/bin/view/CMSPublic/SWGuideSimulation
%https://indico.cern.ch/event/596660/timetable/ fastsim
%https://indico.cern.ch/event/596660/contributions/2412428/attachments/1411446/2159057/Simulation_Basics_2_17.pdf
%https://indico.cern.ch/event/596660/contributions/2412429/attachments/1411828/2159813/SekmenFSDHowITWorks170213.pdf fastsim

\section{Simulation in tarcker}

%geant4
%digitization
%magnetic field setting

%in detector dir
%TS2018_001_2 -> MC generators, detector simulatios
%CERN-THESIS-2017-300 - Event simulation
%TS2017_028_2 - simulation super short

