%Remove the word tuning
%Explain how the simulation is working → how to simulate cluster, describe how it is done
%Present what is the situation
%Description of width not so great
%Not put too much emphasis on the other tests
%Tehn found xtalk is responsible
%Cross talk measurement → 
%Limitations
%Then xtalk from MC + ePerADC

\clearpage

\setcounter{secnumdepth}{4}
\chapterwithnum{Improvement of the silicon strip tracker simulation~\label{ch:simu}}
\setcounter{secnumdepth}{5}

In particle physics, a direct comparison between and theory cannot be made due to the complex nature of the interactions and the apparatus. By consequence, the reconstructed event data are directly compared to the ones \textcolor{red}{ simulated by the MC technique} for measurement or hypothesis test purpose. Further, the simulations are also important for the development and understanding of specific analysis methods and for derivation and validation of calibrations. The detector performance indicators such as resolutions and efficiencies rely as well to a large extent on the simulation.

In Section~\ref{sec:CMSsimu} the general simulation workflow of CMS is introduced. In the following Section~\ref{sec:trackerSimu} a deeper explanation of the simulation of particle charge deposits in the CMS silicon strip tracker is given. In this Section the effects of tracker ageing and simplifications used in simulation are also discussed. The chapter ends with Section~\ref{sec:xtalk} presenting a new measurement of the cross talk parameters in the tracker and their impact in the simulation on cluster properties.

\section{An introduction to the CMS simulation~\label{sec:CMSsimu}}

The CMS simulation workflow~\cite{Banerjee:2007zz, Hildreth:2017vpw, Hildreth:2015kps} is divided into several steps. The beginning of the simulation chain starts with the generation of the physics events. In this step the  particles resulting from the interactions between colliding particles are generated as well as their kinematics. The final state particles produced by the generator are then sent to GEANT4 to pass through the ``simulated'' detector in which these particles interact and leave energy deposits. The last step is the simulation of the response of the detector and its electronics to particle traversing the detector. The output of this procedure is RAW data, which can be later reconstructed and slimmed for the purposes of the physics analyses. The overview of the simulation steps, which are described in larger detail in the following subsections, can be seen in violet in Fig.~\ref{fig:figures/SimulationFlow}~\cite{website:simuBasics}. The official production of the simulated samples is handled centrally~\cite{Boudoul:2015bkp} by the CMS collaboration.

    \insertFigure{figures/SimulationFlow} % Filename = label
                 {0.99}       % Width, in fraction of the whole page width
                 { A diagram of the simulation workflow. The four-vectors of generated particles together with the detector description enter to the GEANT4 simulation whose output are simulated hits. Optionally the simulated hits from pileup interactions can be added on top of the simulated hits from the process of interest and the mixture of these hits is digitized during the electronics simulation. In this step the description of the electronics, for example the noise and the detector conditions (like the temperature), is added. The output of the digitization is the RAW data. Later steps are similar for both data and simulated events~\cite{website:simuBasics}. }


\subsection{Step 1: Monte Carlo event generation}

The Monte Carlo~(MC) generator is a \textcolor{red}{complex software} designed to produce physics events according to a physics model. In the majority of cases three kinds of generators are used by the CMS collaboration~\cite{website:generation, website:generationIntro}. 

\textbf{General-purpose generators: }
These are for example Pythia~8~\cite{Sjostrand:2014zea} or Herwig++~\cite{Bahr:2008pv}. They provide the complete description of the result of the proton collision. To generate outgoing particles originating from the interaction of colliding particles, many theoretical models and aspects have to be plugged in the generation process, such as the description of soft and hard interactions mainly in the leading order, parton distribution functions~(PDFs), initial and final state radiations~(ISR and FSR), multiple parton interactions, hadronization of partons and decay of particles.

\textbf{Matrix Element calculators: }
The generators such as Powheg~\cite{Oleari:2010nx} or MadGraph5\_aMCatNLO~\cite{Alwall:2014hca} were developed to provide the next-to-leading order~(NLO) calculations. These calculators give the final state description at the parton level and their output needs to be plugged into one of the general-purpose generators to generate radiations and proceed with the full hadronization.

\textbf{Specific generators: }
These generators are used to generate specific kind of events e.g. diffractive or cosmic events.

\subsection{Step 2: Simulation of the particle interactions in the detector}

To be able to describe data by simulations, the generated particles need to be propagated through the volume of the CMS detector. This is achieved via the GEANT4~\cite{Agostinelli:2002hh, Lefebure:1999wja} toolkit, into which a detailed description of the CMS detector, its active and inactive material dimensions, layout and properties, are plugged. The GEANT4 toolkit propagates the generated particles through the detector, simulates the interactions with material and models the physics processes which happen during the passage of the particles through the detector. This procedure gives as output the simulated hits left by particles interacting with the active volumes of the subdetectors. These simulated hits can originate from primary particles generated by the MC generator, or from the secondary particles which are the result of the GEANT4 simulation process.

The simulation of the pileup events is done separately from the simulation of the events of interest. The input to the GEANT4 simulation of in-time and out-of-time pileup is a pool of Minimum Bias single interaction events. To simulate a given luminosity profile a specific distribution of the number of interactions per bunch crossing has to be chosen. 


The complete simulation performed with GEANT4, also called Full Simulation~(FullSim), is very CPU intensive. \textcolor{red}{Due to this} a Fast Simulation~(FastSim)~\cite{Sekmen:2017hzs, CMS:2010spa, Giammanco:2014bza} was developed as an alternative to the FullSim. The FastSim uses simplified detector geometry and interactions with material, this speeds the simulation by a factor of around 100. The interaction with material is in the FastSim taken into account on the hit level. The comparison of physics objects of FullSim and FastSim shows that FastSim is a reliable alternative that reproduces the FullSim high level objects within around 10\% accuracy~\cite{Sekmen:2017hzs, Abdullin:2011zz}. The FastSim is widely used to produce for example samples of supersymmetric processes, where large scans with different parameter values are needed. Typically for one fixed parameter set, few tens of millions of events are produced.

\subsection{Step 3: Simulation of the detector and electronics response to the particle signal}

To obtain the output electronic signal from the detector, the signal hits from GEANT4 are collected by the subdetectors. Then the response of the readout electronics to the collected signal is simulated. This step is also called ``digitization'' and its direct input are hits from events of interest and pileup events. There are three domains providing digitization of the given subdetectors, which are SimTracker, SimCalorimetry and SimMuon~\cite{website:simdigi}. The digitized samples are in the RAW format and can be further reconstructed in a similar way as data.


\section{Simulation of the silicon strip tracker response to the particle signal~\label{sec:trackerSimu}}

In order to rely on the simulated samples, they have to describe the data as well as possible. In the case of the tracker, the output of the standard simulation is composed of zero suppressed data, which are further reconstructed into clusters during the reconstruction step. These clusters are used for tracking and physics object reconstruction, therefore a good and reliable description of data by simulated samples at the cluster level is necessary.

The following Figs.~\ref{fig:figures/clusterchargeRescaledalll0to0simulation} to \ref{fig:figures/seedTOB} show data and simulation comparisons of several important cluster quantities. A zero bias data sample recorded in 2017 and corresponding to run 305064 in fill 6298 is compared to a simulated sample of minimum bias events produced in the FullSim configuration. Both data and simulated samples were produced in the deconvolution mode. To avoid fake clusters and clusters originating from out-of-time pileup, only quantities related to the on-track clusters, i.e. clusters associated to reconstructed tracks, are used along this chapter.
 
The charge collected in a sensor from a given track divided by the track length in this sensor, further referred to as ``cluster charge'', is shown in Fig.~\ref{fig:figures/clusterchargeRescaledalll0to0simulation}. The division of the cluster charge by the track length gives the cluster charge per unit of length. Therefore the cluster charge does not depend on the different track lengths caused by the different thickness of the sensors or the inclination of the tracks.  The data to simulation ratio fluctuates around 1 within around $\pm 0.1$ in the bulk of the cluster charge distribution, but the description worsens for very small and very large cluster charges. Overall the cluster charge distribution in simulation is narrower than in data. 


    \insertFigure{figures/clusterchargeRescaledalll0to0simulation} % Filename = label
                 {0.5}       % Width, in fraction of the whole page width
                 { Data and simulation comparison of on-track cluster charge divided by the track length. The simulated distribution is rescaled to the number of clusters in data. The bottom plot represents the data to simulation ratio. }

The cluster width depends on the sensor geometry and therefore has to be shown separately for different geometries. The possible geometries and their location in the tracker can be found in Table~\ref{tab:trackerGeometries}. The data to simulation comparisons for inner and outer barrel geometries are shown in Figs.~\ref{fig:figures/widthTIB}~and~\ref{fig:figures/widthTOB}, respectively. In all cases there is a very poor description of the cluster width in data by the simulation. The mentioned dependency of the cluster width on the geometry can be seen for example when comparing the cluster width between the different barrel geometries. The IB1 sensors have a smaller pitch with respect to the IB2 ones and therefore the IB1 clusters are in average wider, as in the IB1 case a larger number of strips would collect the charge from the same track. The sensors of IB2 and OB1 have a similar pitch, but the OB1 sensors are thicker which leads to a larger cluster width in the OB1 than in IB2, because the particle path in a sensor is longer. Moreover the cross talk is larger for OB1 than IB2, leading to a larger enhancement in OB1 cluster width compared to IB2.


    \insertTwoFigures{figures/widthTIB} % Filename = label
                 {figures/clusterwidthTIBl1to2simulation}
                 {figures/clusterwidthTIBl3to4simulation} % Filename = label
                 {0.45}       % Width, in fraction of the whole page width
                 { Distribution of the on-track cluster width for the IB1 (left) and IB2 (right) geometries in data and simulation. The simulated distribution is rescaled to the number of clusters in data. The bottom plots represent the data to simulation ratio for a given geometry. \textcolor{red}{The pitch of the IB1 sensors is 80~$\mu$m and of IB2 sensors 120~$\mu$m, all IB sensors have thickness of 320~$\mu$m. } }

    \insertTwoFigures{figures/widthTOB} % Filename = label
                 {figures/clusterwidthTOBl1to4simulation}
                 {figures/clusterwidthTOBl5to6simulation} % Filename = label
                 {0.45}       % Width, in fraction of the whole page width
                 { Distribution of the on-track cluster width for the OB2 (left) and OB1 (right) geometries in data and simulation. The simulated distribution is rescaled to the number of clusters in data. The bottom plots represent the data to simulation ratio for a given geometry. \textcolor{red}{The pitch of the OB2 sensors is 122/183~$\mu$m and of OB1 sensors 122~$\mu$m, all OB sensors have thickness of 500~$\mu$m. }}


Another cluster quantity is the charge on the strip having collected the largest charge among all strips in the cluster rescaled to track path in the sensor, which is later referred to as ``cluster seed charge''. The cluster seed charge shown in Figs.~\ref{fig:figures/seedTIB}~and~\ref{fig:figures/seedTOB} is also dependent on the module geometry and therefore is presented for the four different barrel geometries. In these plots, it can be seen that the simulation in average predicts in all cases a lower cluster seed charge, a clear trend in data to simulation ratio is present. It can also be observed that with a larger pitch, the seed charge increases, as the charge from a larger part of the sensitive volume is read by one strip. Moreover, differences in the seed charge for IB2 and OB1 are observed. However these two parts of the detector have a similar pitch size and the difference in the module thickness is taken into account by rescaling the collected charge by the path length. The difference between the two geometries comes in fact from a larger cross talk in OB1, leading to a smaller seed charge.


    \insertTwoFigures{figures/seedTIB} % Filename = label %TDO continue here
                 {figures/clusterseedchargeRescaledTIBl1to2simulation}
                 {figures/clusterseedchargeRescaledTIBl3to4simulation} % Filename = label
                 {0.45}       % Width, in fraction of the whole page width
                 { Distribution of the on-track cluster seed charge for the IB1 (left) and IB2 (right) geometries in data and simulation. The simulated distribution is rescaled to the number of clusters in data. The bottom plots represent the data to simulation ratio for a given geometry. }

    \insertTwoFigures{figures/seedTOB} % Filename = label %TDO continue here
                 {figures/clusterseedchargeRescaledTOBl1to4simulation}
                 {figures/clusterseedchargeRescaledTOBl5to6simulation} % Filename = label
                 {0.45}       % Width, in fraction of the whole page width
                 { Distribution of the on-track cluster seed charge for the OB2 (left) and OB1 (right) geometries in data and simulation. The simulated distribution is rescaled to the number of clusters in data. The bottom plots represent the data to simulation ratio for a given geometry. }


To summarize, the simulation tends in general to overestimate the cluster width and to underestimate the charge on the seed strip. Large variations in the data to simulation ratio are observed. Therefore it is necessary to identify the sources leading to the observed disagreements between data and simulation, and to propose possible corrections.

To achieve such goal, the simulation flow of the silicon strip tracker (step~3) is described in detail in the rest of this section, starting from GEANT~4 output in Subsection~\ref{sec:G4out} and going through all the simulation steps in a detector unit up to the final digitization in Subsections~\ref{sec:divide} to \ref{sec:digitize}. The potential issues in the simulation approach are identified and discussed at each of these steps. For simplicity, in the majority of cases, the cluster seed charge and width distributions are shown only for the OB2 geometry which has the largest number of layers. As the standard data-taking mode is the zero suppression mode with APVs operating in the deconvolution mode, the simulation steps are described for these conditions. At the end a possibility of simulating virgin raw data is briefly discussed is Section~\ref{sec:VRsimu}.

%To understand how the clusters are simulated and what part of the simulation can lead to an incorrect cluster description, the simulation flow of the silicon strip tracker has first to be described. The generated particles of simulation (step~1) are  sent to the GEANT4 toolkit (step~2) through the simulated detector where they interact with matter. The format of the GEANT4 output is discussed in Section~\ref{sec:G4out}. The GEANT4 energy deposits in form of charge carriers are in step~3 propagated through the tracker sensors, collected by the strips, read by the electronics and zero suppressed. The zero suppressed simulated events can then be reconstructed similarly as data.

%The third simulation step, also referred to as  ``digitization'' starts with division of energy deposit simulated by GEANT4 along the path of the track passing through given module. The energy deposit is reduced if that particle arrives before or after the optimal arrival time.  The division of energy deposit and its reduction due to timing is described in Section~\ref{sec:divide}. In following Section~\ref{sec:drift} the divided energy deposits are drifted through the sensor towards strips and the effects acting on drift such as the thermal diffusion or hall effect (Lorentz angle) are discussed. Then the drifted energy deposits are divided between strips and converted to electrons in Section~\ref{sec:induce}. In this section the charge sharing between strips, the cross talk, due to the capacitive coupling between strips is also described. On top of the charge induced on the strips, the noise is added in Section~\ref{sec:digitize}. In the same section the signal is converted to the ADCs and rescaled by gains in order to mimic differences in modules of the tracker. The strips which were marked as bad in the CMS tracker are masked and the resulting signal is zero suppressed. At the end a possibility of simulating virgin raw data is briefly discussed is Section~\ref{sec:VRsimu}.


%Other possibilities are shortly discussed as well.

% To achieve good data and simulation agreement, the propagation has to be corrected for effects which happen physically in the sensor, for example the thermal diffusion. Other set of factors is applied on top of the simulated signal to mimic effects of electronics on the signal acquired during the data-taking. The analogue signal with all corrections applied, is digitized. The flow of the simulation steps is shown in Fig.~\ref{fig:figures/trackerSimulationFlow}. 


%    \insertFigure{figures/trackerSimulationFlow} % Filename = label
%                 {0.99}       % Width, in fraction of the whole page width
%                 { A diagram of the of the simulation steps in the CMS tracker. The energy deposited produced by GEANT4 is divided into track segments and drifted towards the strips. Then the charge is induced on the strips. The dead channels are removed and on top of the signal the noise is added. The resulting signal is converted to digis and zero suppressed. }

%TODO naming convention! digitization or simulation?

\subsection{The GEANT4 output~\label{sec:G4out}}

A simulated hit in the tracker produced by the GEANT4 toolkit is stored in the CMSTrackerHit~\cite{Lefebure:1364020} object. The later is created for each particle entering in the tracker and for each sensitive detector unit, which in case of the silicon strip tracker represents one side of sensors of one module. The CMSTrackerHit object stores the information on the particle entry and exit points in the reference system of the detector unit~$(x_{i}, y_{i}, z_{i})$. The other quantities are the energy~($E_{ent}$) of the particle at the point where track enters the detector unit and the total amount of energy~($E_{loss}$) deposited in the detector unit by the given particle. The time-of-flight~($TOF$) of the particle from the primary interaction point to the detector unit, the identification number of the detector unit~($DetID$) and of the track~($trackID$) are stored as well. A sketch of a detector unit and the information stored by GEANT4 per detector unit and track is shown in Fig.~\ref{fig:figures/geantDeposit}. In the following, the detector unit will be referred to as ``module''.

    \insertFigure{figures/geantDeposit} % Filename = label
                 {0.5}       % Width, in fraction of the whole page width
                 { Sketch of a detector unit and the corresponding information saved for each simulated hit by GEANT4. The hits are stored with the detector unit~($DetID$) and track~($trackID$) identification numbers. They also contain the particle entry energy~($E_{ent}$), the total energy deposited~($E_{loss}$) by the particle and the time-of-flight~(TOF) of the particle from the interaction point to the detector unit. The entry and exit points of the particle in the local frame~$(x_{i},y_{i},z_{i})$ are stored as well~\cite{website:simuBasics}. }

\subsection{Dividing the energy deposit simulated by GEANT4 along the path of the track and the timing of the tracker~\label{sec:divide}}

GEANT4 provides the information on the total energy $E_{loss}$ deposited in the detector unit, in the form of the charge carriers created in the module. But depending on the arrival time of particle to the module, a different fraction of this total energy is read due to the electronic signal shape. To determine this fraction, the time-of-flight~($TOF$) of a given particle is compared to the time-of-flight of a reference photon traveling at the speed of light. The reference photon is assigned to a given bunch crossing and its time-of-flight is computed from the speed of light and the distance between the interaction point and the entry point of the given particle in the given module. The maximum amount of charge is read when the time-of-flight difference between the reference photon and the given particle is zero. Larger is the difference from zero, smaller is the collected fraction of the particle charge. Because of the bunch crossing assignment, the time-of-flight difference between the reference photon and the particle can be either negative or positive. If the particle originates from the studied bunch crossing the difference is always negative or equal to zero, but if it was created before (OOT PU), the difference can be positive as well. Therefore according to the delay between the reference photon and the particle at the entry point to the module, the time response of the module to the signal left by the particle can be determined. The schema of this time response~(pulse shape) to the signal currently used in simulation can be seen in Fig.~\ref{fig:figures/timeResponse}. This shape corresponds to the electronic pulse shape convoluted by a physical signal (flat with a duration  of 20~ns). It can be noticed that the time response of the module to the particle signal can take several tens of ns. It is longer than the time between collisions, which is 25~ns, and therefore in the bunch crossing of interest it is possible to read signal from preceding and succeeding bunch crossings. This effect is one of the sources of the OOT pileup. \textcolor{red}{The other sources are particles arriving with a delay, for example slow particles and loopers.}

    \insertFigure{figures/timeResponse} % Filename = label
                 {0.5}       % Width, in fraction of the whole page width
                 { The \textcolor{red}{approximated} deconvolution pulse shape plugged into the simulation. \textcolor{red}{This pulse shape was derived from the laser measurements~\cite{Delaere:1061284} and its maximum was aligned to the 0~ns  corresponding to the reference~(photon) time, for which the maximum of the charge is collected.}}

GEANT4 hit stores only a total energy deposition information per one module but in the real data-taking, the particle looses its energy as it traverses the module. To mimic this effect, the track is divided into a certain number of equidistant segments. Currently the number of sectors is evaluated as a fixed factor multiplied by the expected number of strips reading the particle signal. The total energy reduced by the time response value, as described in the previous text, is then divided into the track segments and for each segment it is fluctuated according a Landau function. The fluctuated charge per segment is then normalized to keep the sum of the segment energies equal to the initial energy. The local coordinates of the track segment together with the fluctuated normalized energy deposit are saved in an object referred to as ``energy deposit unit''.

In the energy division procedure there are several simplifications which can lead to discrepancies between the properties of simulated clusters and clusters in data. The GEANT4 provides only the total deposited charge, therefore the charge has to be divided into discrete and equidistant tracks segments. Moreover it is assumed that the energy loss is constant along the track, but according to the Bethe-Bloch formula there is an evolution of the energy loss with the particle momentum. Other issues of this step are discussed in detail in the following paragraphs.

\textbf{Cut-off on the delta ray production~$\delta$ }

The charge of each track segment is being fluctuated by the GEANT4 routine. This routine is initialized with a parameter called $\delta$ specifying a cut-off on the delta ray production. The delta rays are already part of the previous GEANT4 output and therefore this cut-off has to be the same as the one in the previous step in order to avoid any mismatch. The dependency of the cluster charge, cluster seed charge and cluster width on the delta ray cut-off, which was varied up to $\pm 20\%$ around its default value of 0.120425, is shown in Figs.~\ref{fig:figures/clusterchargeRescaledalll0to0delta}~and~\ref{fig:figures/seedwidthTOBdelta}. From these figures it can be concluded that the change of distributions of the cluster properties is almost negligible in the given range of the cut-off variation. 

    \insertFigure{figures/clusterchargeRescaledalll0to0delta} % Filename = label
                 {0.5}       % Width, in fraction of the whole page width
                 {  On-track cluster charge in simulation for different values of the delta ray production cut-off~$\delta$. The simulated distributions are normalized by the same factor as in Fig.~\ref{fig:figures/clusterchargeRescaledalll0to0simulation}. In the ratio plot the default simulation is divided by different simulation scenarios. }

    \insertTwoFigures{figures/seedwidthTOBdelta} % Filename = label %TDO continue here
                 {figures/clusterwidthTOBl1to4delta}
                 {figures/clusterseedchargeRescaledTOBl1to4delta} % Filename = label
                 {0.45}       % Width, in fraction of the whole page width
                 {  On-track cluster width (left) and cluster seed charge (right) in simulation for the OB2 geometry and for different values of the delta ray production cut-off~$\delta$.   The simulated distributions are normalized by the same factors as in the left plots of Figs.~\ref{fig:figures/widthTOB} and \ref{fig:figures/seedTOB}, respectively.  In the ratio plots the default simulation is divided by different simulation scenarios. }

\newpage
\textbf{Pulse shapes}

The deconvolution pulse shape used in simulation, presented in Fig.~\ref{fig:figures/timeResponse}, was determined by the parametrization of results obtained in~\cite{Delaere:1061284}. These results were obtained from laser studies of the TOB modules before the start of the LHC operation. Fig.~\ref{fig:figures/timeResponseReal} presents the results, when a single strip was hit by the laser. In this plot, the resulting pulse shape from the hit strip and the two neighbors on each side is shown. The neighboring strips read non-zero charge purely due to the electronic cross talk, which is a result of the coupling of the strips via inter--strip capacitance as shown in Fig.~\ref{fig:figures/capacitanceNetwork}. 

%As mentioned in previous chapter, there are two main sources of cross talk. First is caused by the fact, that when charge carriers are moving inside the sensor, they are inducing charge on more strips, not only the hit one. The amount of the induced charge is time dependent. i


    \insertFigure{figures/timeResponseReal} % Filename = label
                 {0.5}       % Width, in fraction of the whole page width
                 { The deconvolution pulse shape obtained from focusing a laser on one strip of the TOB module. The charge read by the first and second neighboring strips on both sides are shown as well. The largest signal is collected by the hit strip, a lower fraction by its first neighbors and the lowest fraction by second neighboring strips. The black curves correspond to the theoretical fits of the data, while the dots are measured data. The vertical lines indicate the time of the maximum of the pulse shapes~\cite{Delaere:1061284}.  }
%TODO By default, cross talk in this chapter means capacitive coupling between strips.

It can be noticed that there are several differences between the pulse shape obtained from laser measurements and the one in simulation. First, the times corresponding to the measured maxima of the pulse shapes are not same for the hit strip and for its neighbors. This fact is omitted in the simulation. Secondly, measured pulse shape can have negative values, while it is not the case in simulation. Thirdly, a pulse shape does not take into account any ageing effects of the modules of the CMS tracker. Moreover in 2016 the VFP parameter changed in order to avoid saturation of the electronics, which resulted in a change of the pulse shape in data~\cite{website:vfp}. This change was not propagated into the simulation. Therefore the dynamics is not correctly modeled, but in average the on-track cluster charge is well described. \textcolor{red}{In order to provide more realistic simulation of clusters, the pulse shapes should be updated in future.} 

Furthermore there is a dynamics which modifies the pulse shape and which is not modeled in the simulation. For example, the only charge carriers which are present in the simulation are holes drifting to the strips, but the presence of electrons is omitted.

\textbf{The backplane correction}

In 2009, discrepancies in the cluster position between data recorded in peak and deconvolution modes were observed~\cite{website:backplane}. The disagreement was found to be worse for thick sensors and the problem was quickly identified as an issue of the charge collection from the whole sensor. In the deconvolution mode the charge carriers which are created close to the backplane are not read as efficiently as those produced close to the strips and therefore the cluster position in the deconvolution mode is shifted with respect to the peak data, where such problem is not present. To correct the cluster position between peak and the deconvolution data, an ad-hoc correction called ``backplane correction'' is applied on top of the deconvolution data during the reconstruction step. \textcolor{red}{The derived backplane correction is not precise and its inaccuracy is further compensated by the alignment.} As in the deconvolution mode the strip sensor appears effectively thinner, this effect does not only impact the cluster position, but it could also influence the cluster charge, the cluster seed charge and the cluster width. However, this effect is not considered in simulation.

% TODO are not puls shapes difefrent for thick and thin sensors??
%ISSUES
%puls shape problems - puls shape same for all strips (actually reveighted at once), does not have any undershoot, is pretty outdated from here CMS NOTE 2007/027
%charge division itself, eloss does not have to be linear
%delta rays are missing (maybe): delta cutoff in MeV, has to be same as in Geant (0.120425 MeV corresponding to 100um range for electrons)
%puls shape -> must take into account the drift time of holes
% Automatically generated using parametrizePulse::generateCode(low=-30, high=35, step=0.1)
% That pulse shapes correspond to the ones described in CMS NOTE 2007/027
% with tau=50ns and delta=20ns



\subsection{Charge drift of the energy deposits through the sensor~\label{sec:drift}}

To simulate  the drift to the electrodes of the charge carriers created after the passage of a particle through a module, the energy deposits have to be propagated separately from each track segment to the strips. In simulation the energy deposits are not directly converted to the charge carriers, but the energy deposits are drifted as the charges would do. 

As in the barrel region the electric field is perpendicular to the magnetic field, the drift of the charge carriers is deflected from the direction of electric field, i.e. the local z-axis, by the Lorentz angle. Thus the drift direction of the charges towards the surface of the sensor must be corrected accordingly to the Lorentz angle and the local magnetic field. 

Moreover the charge carriers undergo diffusion in the silicon volume during the drift and therefore the energy deposit from each track segment is collected smeared at the surface due to the diffusion effects. The spread of the energy distribution is computed from the knowledge of the drift time and a diffusion constant, which depends on the type of material through which the charge drifts and the temperature of the sensor. The drift time from the track segment to the sensor surface depends on:

\begin{itemize}
\item \textbf{The sensor thickness}: The drift time is longer for thicker sensors
\item \textbf{The depletion and applied voltages}:  The former is the voltage which fully depletes the sensor while the later is the total voltage applied, including the depletion voltage. Both influence how fast the charge carriers drift
\item \textbf{The charge mobility}: Depending on the material properties, its temperature, the electric field and the concentration of the charge carriers, the charge mobility influences how fast the charge carriers drift. For example the mobility of holes is lower than the one of the electrons
\item \textbf{The coordinates of the track segment}: They are used to determine the distance that the charge  has to drift through
\end{itemize}


The output from the previous step, as described in Subsection~\ref{sec:divide}, is in the form of energy unit after having applied now the drift and diffusion procedures and is stored as an object referred to as ``signal point'', which contains the information about its coordinates at the surface, the energy and its spread.

In order to keep the processing time relatively short, the description of the drift is simplified. It does not take into account any radiation damages of the sensors, ageing of modules, or non-uniformities of the electric filed. However, the charge drift depends on several parameters which can be subject of ageing. The description of these parameters and the impact of a variation around their default values in the  simulation is presented in the following paragraphs. 

\textbf{Applied~($AV$) and depletion~($DV$) voltages}

As mentioned, the spread of the diffused energy distribution depends on the drift time which depends on the applied and depletion voltages. The depletion voltage changes during the operation of the tracker because of the ageing of modules, but this information is not updated in the simulation. The applied voltage has not been changed during the operation of CMS, but it will become necessary once the depletion voltage will be larger than the applied one. To evaluate the effect of a voltage change on the cluster charge, seed charge and width, these distributions are shown for several values of depletion and applied voltages in Figs.~\ref{fig:figures/clusterchargeRescaledalll0to0AV}~to~\ref{fig:figures/seedwidthTOBDV}. The depletion and applied voltages, whose default values are 170~V and  300~V, respectively, are varied in the figures within a $\pm 20\%$ around the default. It can be concluded that the following change of the cluster properties is negligible.

%\ref{fig:figures/seedwidthTOBAV}~\ref{fig:figures/clusterchargeRescaledalll0to0DV}

    \insertFigure{figures/clusterchargeRescaledalll0to0AV} % Filename = label
                 {0.5}       % Width, in fraction of the whole page width
                 { On-track cluster charge in simulation for different values of the applied voltage~$AV$.  The simulated distributions are normalized by the same factor as in Fig.~\ref{fig:figures/clusterchargeRescaledalll0to0simulation}. In the ratio plot the default simulation is divided by different simulation scenarios. }

    \insertTwoFigures{figures/seedwidthTOBAV} % Filename = label %TDO continue here
                 {figures/clusterwidthTOBl1to4AV}
                 {figures/clusterseedchargeRescaledTOBl1to4AV} % Filename = label
                 {0.45}       % Width, in fraction of the whole page width
                 {  On-track cluster width (left) and cluster seed charge (right) in simulation  for the OB2 geometry and for different values of the applied voltage~$AV$. The simulated distributions are normalized by the same factors as in the left plots of Figs.~\ref{fig:figures/widthTOB} and \ref{fig:figures/seedTOB}, respectively.   In the ratio plots the default simulation is divided by different simulation scenarios. }


    \insertFigure{figures/clusterchargeRescaledalll0to0DV} % Filename = label
                 {0.5}       % Width, in fraction of the whole page width
                 {  On-track cluster charge in simulation for different values of the depletion voltage~$DV$.   The simulated distributions are normalized by the same factor as in Fig.~\ref{fig:figures/clusterchargeRescaledalll0to0simulation}.  In the ratio plot the default simulation is divided by different simulation scenarios. }

    \insertTwoFigures{figures/seedwidthTOBDV} % Filename = label %TDO continue here
                 {figures/clusterwidthTOBl1to4DV}
                 {figures/clusterseedchargeRescaledTOBl1to4DV} % Filename = label
                 {0.45}       % Width, in fraction of the whole page width
                 {  On-track cluster width (left) and cluster seed charge (right) in simulation  for the OB2 geometry and for different values of the depletion voltage~$DV$.  The simulated distributions are normalized by the same factors as in the left plots of Figs.~\ref{fig:figures/widthTOB} and \ref{fig:figures/seedTOB}, respectively.   In the ratio plots the default simulation is divided by different simulation scenarios. }

\textbf{Temperature~($T$) and charge mobility~($\mu$)}

The diffusion constant $D$ can be expressed as

\eq{driftEquation}
{
   D=  \frac{\mu k_{B} T}{q},
}
where $\mu$ is the charge mobility in units $cm^{2}/(V \cdot s)$, $k_{B}$ the Boltzmann constant, $T$ the absolute temperature of the material and $q$ the electrical charge of the particle. Between Run~1 and Run~2 the temperature of the tracker was changed from 4$^{\circ}$C to -15$^{\circ}$C and between 2017 and 2018 from -15$^{\circ}$C to -20$^{\circ}$C. As this change was not propagated into the simulation, it is therefore important to evaluate its effect. The cluster charge, width and seed charge are shown in Figs.~\ref{fig:figures/clusterchargeRescaledalll0to0T}~and~\ref{fig:figures/seedwidthTOBT} for the default value of temperature $T=273$~K used in simulation as well as for variations up to $\pm 20\%$. No dependency of the cluster charge, width and seed charge on the temperature is observed. 

The charge mobility $\mu$ intrinsically also depends on the temperature, as well as on the concentration of defects which change in time and therefore this parameter should in principle also be adjusted in the simulation. Figs.~\ref{fig:figures/clusterchargeRescaledalll0to0mu}~and~\ref{fig:figures/seedwidthTOBmu} are not revealing any change of the cluster charge, width and seed charge when altering the default charge mobility of $310~\mathrm{cm^2/(V \cdot s)}$ by up to $\pm 20\%$. 

In summary, there is almost no change in the simulated cluster properties due to parameters influencing diffusion of the charge carriers because the magnitude of diffusion is much smaller than the pitch size.


    \insertFigure{figures/clusterchargeRescaledalll0to0T} % Filename = label
                 {0.5}       % Width, in fraction of the whole page width
                 { On-track cluster charge in simulation for different values of the temperature~$T$.   The simulated distributions are normalized by the same factor as in Fig.~\ref{fig:figures/clusterchargeRescaledalll0to0simulation}.  In the ratio plot the default simulation is divided by different simulation scenarios. }

    \insertTwoFigures{figures/seedwidthTOBT} % Filename = label %TDO continue here
                 {figures/clusterwidthTOBl1to4T}
                 {figures/clusterseedchargeRescaledTOBl1to4T} % Filename = label
                 {0.45}       % Width, in fraction of the whole page width
                 {  On-track cluster width (left) and cluster seed charge (right) in simulation  for the OB2 geometry and for different values of the temperature~$T$.  The simulated distributions are normalized by the same factors as in the left plots of Figs.~\ref{fig:figures/widthTOB} and \ref{fig:figures/seedTOB}, respectively.   In the ratio plots the default simulation is divided by different simulation scenarios. }

    \insertFigure{figures/clusterchargeRescaledalll0to0mu} % Filename = label
                 {0.5}       % Width, in fraction of the whole page width
                 { On-track cluster charge in simulation for different values of the charge mobility~$\mu$.   The simulated distributions are normalized by the same factor as in Fig.~\ref{fig:figures/clusterchargeRescaledalll0to0simulation}.  In the ratio plot the default simulation is divided by different simulation scenarios. }

    \insertTwoFigures{figures/seedwidthTOBmu} % Filename = label %TDO continue here
                 {figures/clusterwidthTOBl1to4mu}
                 {figures/clusterseedchargeRescaledTOBl1to4mu} % Filename = label
                 {0.45}       % Width, in fraction of the whole page width
                 { On-track cluster width (left) and cluster seed charge (right) in simulation  for the OB2 geometry and for different values of the charge mobility~$\mu$.  The simulated distributions are normalized by the same factors as in the left plots of Figs.~\ref{fig:figures/widthTOB} and \ref{fig:figures/seedTOB}, respectively.   In the ratio plots the default simulation is divided by different simulation scenarios. }

%then we have the lorentz angle
\textbf{Lorentz angle~($LA$)}

The Lorentz angle used in simulation is dependent on the electric and magnetic field and therefore the Lorentz angle should be updated with the change of voltage. The effect of the Lorentz angle on the simulated cluster charge, width and seed charge is shown in Fig.~\ref{fig:figures/clusterchargeRescaledalll0to0LA}~and~\ref{fig:figures/seedwidthTOBLA}. The Lorentz angle injected into the simulation is specific for each module and it is read from the offline condition database. This database contains information about tracker conditions, such as Lorentz angle, noise, list of bad modules and gains.  The distribution of the Lorentz angles in the OB2 geometry per cluster was fitted by  a Gaussian distribution to obtain its mean value and standard deviation. Then a mean value altered by up to twice the standard deviation was injected into the simulation to obtain the varied simulated samples. It can be concluded that within the range of injected Lorentz angle values there is a negligible effect on the description of the shown cluster properties. The Lorentz angle impacts the position of the cluster and therefore within the tracker local reconstruction group, there are intents to remeasure and update the Lorentz angle values to improve the cluster position in simulation.


    \insertFigure{figures/clusterchargeRescaledalll0to0LA} % Filename = label
                 {0.5}       % Width, in fraction of the whole page width
                 { On-track cluster charge in simulation for different values of the Lorentz angle~$LA$.   The simulated distributions are normalized by the same factor as in Fig.~\ref{fig:figures/clusterchargeRescaledalll0to0simulation}.  In the ratio plot the default simulation is divided by different simulation scenarios. }

    \insertTwoFigures{figures/seedwidthTOBLA} % Filename = label %TDO continue here
                 {figures/clusterwidthTOBl1to4LA}
                 {figures/clusterseedchargeRescaledTOBl1to4LA} % Filename = label
                 {0.45}       % Width, in fraction of the whole page width
                 { On-track cluster width (left) and cluster seed charge (right) in simulation for the OB2 geometry and for different values of the Lorentz angle~$LA$.  The simulated distributions are normalized by the same factors as in the left plots of Figs.~\ref{fig:figures/widthTOB} and \ref{fig:figures/seedTOB}, respectively.   In the ratio plots the default simulation is divided by different simulation scenarios. }


%ISSUES
%charge is induced also by electrons
% charge is beiing induced - no cross talk due to induction.
%point-like track segment - spread i  charge should be already there...
%temperature changes with time
%lorentz angle changes as well because of different voltage
%ultimately the voltage itself play a large role
%charge mobility

\subsection{Induced charge on the strips~\label{sec:induce}}

Once the energy deposits are drifted to the surface, the amount of electrons induced on the given aluminium strips is simulated. The energy cannot be divided between strips purely based on geometrical criteria, due to the capacitive coupling (cross talk) between the silicon strips. The cross talk induces that the charge collected by one strip is partially shared with its first and second neighboring strips on both sides. The sharing with second neighboring strips is larger in the deconvolution mode than the peak mode. In the peak mode the sharing with second neighboring strips is negligible.   

The signal point at the surface is associated to the closest strip. There the energy is converted to a number of electrons corresponding to a given charge. The charge is then divided between the neighboring strips according to the energy spread originating from the diffusion. This is repeated for all signal points. If one strip obtains signal from many signal points, these charge deposits are summed. On top of the assigned charge to the strips, the cross talk effect is applied. The simulation iterates over all strips from left to right, and shares the charge of a given strip in between its neighbors.

At this step, several more parameters have to be plugged into the simulation. All these parameters can evolve with the ageing of the modules and therefore the impact of their change on the cluster properties is discussed in following paragraphs. 

\textbf{The cross talk~($XT$)}

While in reality the current is already induced on the aluminium strips once the electrons and holes start to drift to the backplane and silicon strips, respectively, the simulation is simplifying the description, by decomposing these steps: first only holes are drifted towards the strips, second the charge on the strips is collected and then the cross talk is applied. The cross talk could change in time due to radiation damages of modules. The change in cross talk would have a small impact on the cluster charge, but the cluster width and cluster seed charge would change largely as the sharing between strips is impacted. In Figs.~\ref{fig:figures/clusterchargeRescaledalll0to0simulation}~to~\ref{fig:figures/seedTOB}, it was observed that the simulation describes well the cluster charge in data, but there is a large disagreement between data and simulation for the cluster width and seed charge, therefore a change in cross talk would be a good candidate to explain the observed discrepancies.  

To evaluate, what would be the impact of a non-correct description of the cross talk, Figs.~\ref{fig:figures/clusterchargeRescaledTOBl1to4XT}~and~\ref{fig:figures/seedwidthTOBXT} show the cluster charge, width and seed charge distributions for several cross talk configurations. In these plots it can be indeed seen that the cluster width and seed charge, which are poorly described in simulation, are strongly dependent on the cross talk. 

To evaluate if there is a cross talk parametrization which could be injected into simulation to describe the measured data, a matrix of simulated samples was produced. In these samples both the charge sharing between the first and second neighboring strips was varied. The cluster width distribution of each simulated sample was compared to data, based on a $\chi^{2}$ test. The sample with the best $\chi^{2}$ is shown in green in Figs.~\ref{fig:figures/clusterchargeRescaledTOBl1to4XT}~and~\ref{fig:figures/seedwidthTOBXT}. In the figures it can be noticed, that a decrease of the charge sharing between the central strip and both its first and second neighboring strips, with respect to the values in the default simulation, is required in order to better describe data. After this procedure the data and simulation agreement for the cluster width and seed charge is vastly improved, but it is still not perfect as the parameters and conditions which cause minor source of discrepancies have to be updated as well. The remaining discrepancies could be then caused by a simplification of the cluster simulation. More details about the cross talk and the dedicated measurement performed during this thesis are given in Section~\ref{sec:xtalk}.


    \insertFigure{figures/clusterchargeRescaledTOBl1to4XT} % Filename = label
                 {0.5}       % Width, in fraction of the whole page width
                 { Data and simulation comparison of the on-track cluster charge divided by the track length for the OB2 geometry and for different values of the cross talk~$XT$. The first of the three XT numbers corresponds to the fraction of charge induced on the seed strip, the second and third number represent the fractions of charge induced on each first and second neighboring strips, respectively. The simulated distributions are rescaled to the number of clusters in data. The bottom plot represents the data to simulation ratios.}

    \insertTwoFigures{figures/seedwidthTOBXT} % Filename = label %TDO continue here
                 {figures/clusterwidthTOBl1to4XT}
                 {figures/clusterseedchargeRescaledTOBl1to4XT} % Filename = label
                 {0.45}       % Width, in fraction of the whole page width
                 { Distribution of the on-track cluster width (left) and cluster seed charge (right) in data and simulation for the OB2 geometry and for different values of the cross talk $XT$. The first of the three XT numbers corresponds to the fraction of charge induced on the seed strip, the second and third number represent the fractions of charge induced on each first and second neighboring strips, respectively. The simulated distributions are rescaled to the number of clusters in data. The bottom plot represents the data to simulation ratios. } 


\textbf{The energy to electron conversion factor~($Ee$)}

With the ageing of the modules, the conversion of the energy into a number of electrons can evolve. The data and simulation comparisons of the cluster charge, width and seed charge are shown in Figs.~\ref{fig:figures/clusterchargeRescaledalll0to0Ee}~and~\ref{fig:figures/seedwidthTOBEe}. A set of simulations, produced by changing the conversion factor by up to $\pm 20\%$ with respect to its default value of $3.61 \times 10^{-9}$~GeV, is plotted as well. In these figures it can be noticed that a change in the conversion factor between the deposited energy and the electron multiplicity by up to $\pm 20\%$ leads to significant changes in all cluster quantities. However, it can be concluded that a single change of this conversion factor does not allow to provide a good data and simulation agreement in all these three quantities at the same time. Moreover as there realistic gains are applied, the possible evolution of the conversion factor is already taken into account by the gain calibration procedure. The details are discussed in the next section when the factor converting electrons to ADCs is introduced.

    \insertFigure{figures/clusterchargeRescaledalll0to0Ee} % Filename = label
                 {0.5}       % Width, in fraction of the whole page width
                 { Data and simulation comparison of the on-track cluster charge divided by the track length for the OB2 geometry and different values of the energy to electrons conversion factor~$Ee$. The simulated distributions are rescaled to the number of clusters in data.  The bottom plot represents the data to simulation ratios. }

    \insertTwoFigures{figures/seedwidthTOBEe} % Filename = label %TDO continue here
                 {figures/clusterwidthTOBl1to4Ee}
                 {figures/clusterseedchargeRescaledTOBl1to4Ee} % Filename = label
                 {0.45}       % Width, in fraction of the whole page width
                 { Distribution of the on-track cluster width (left) and cluster seed charge (right) in data and simulation for the OB2 geometry and different values of the energy to electrons conversion factor~$Ee$.  The simulated distributions are rescaled to the number of clusters in data.  The bottom plots represent the data to simulation ratios. }

%TODO more about xtalk
%in reallity not like this at all, the carriers drift and induce charge
%association to the closest strip - well if the center is close to middle, this is dangerous
%cross talk not correct on the edges, first and last strips
%no dynamics, sequential cross talk
%maximal cluster charge and seed charge at different time because of the puls shapes
%the emasurements of cross talk  are done in a way that not only capacitive coupling is inside

\subsection{Conversion of the analog signal to a digital one~\label{sec:digitize}}

Before the digitization of the signal, few more effects must be considered. First, in the detector around 4\% of channels are flagged as bad. The information about bad channels is stored in the condition database, read by the simulation, and if a bad channel has a non-zero value of its simulated charge, it is set to zero. Secondly, to mimic real conditions, the noise, which is stored in the condition database, has to be added to each strip. The noise per strip is determined from the the fluctuation of the strip charge after pedestal and CMN subtraction in data runs in absence of collision. The noise is stored in ADCs, therefore it has to be converted to a number of electrons by a dedicated conversion factor. Moreover because of differences between modules in the detector, the noise has to be rescaled by the gains of the modules to correct for module differences. The gain used in simulation~(Gsim) is set to be the G1 gain described in Section~\ref{sec:localreco} determined from the simulation. Another option for the Gsim is to smear randomly the G1 value determined from data to take into account the fluctuations of G1 over time and the accuracy of its measurement. The G2 gain for simulation is then calculated from a simulated $t\bar{t}$ sample with all the other conditions applied, by targeting the MPV of the MIP response to be 300~ADC/mm.

 The sum of signal and noise on the strips in electron units is then converted to ADCs by a conversion factor and scaled by Gsim gain to mimic the output from the detector. During the reconstruction step, both data and simulation are corrected by 1/(G1$\times$G2) to eliminate differences in modules, and compensate for the drop of signal due to losses in the signal collection and during the signal transmission. Therefore it would be possible not to apply any gain on the simulated data as in the simulation itself no inefficiencies or differences between similar modules are introduced. And indeed, for the first simulations, no gains were applied, but later it was decided to apply gains to be able to simulate saturation effects. In FED, the charge in ADCs is truncated into 8-bit range, therefore the Gsim gain is applied to mimic the realistic output of detector. The simulated data in ADCs are zero suppressed and stored as digis.

The gains, noise as well as the conversion factor between electrons and ADCs change during the operation of the CMS because of the change of operating conditions and radiation damage of the modules. The evolution of these quantities is discussed in the following paragraphs.   
 
\textbf{Gains and noise}

The noise is strongly dependent on the tracker operating temperature and both noise and gains change with the radiation damage. Therefore they need to be measured several times per year and updated regularly for needs of the data reconstruction. These values are not propagated to the simulation, which is using conditions from a different condition database, updated with a much smaller frequency. The difference of noise and gains between data and simulation can lead to different zero suppression results and also different reconstruction results. If the noise is larger in one case than the other, it can appear that a strip with a given charge can fail the S/N threshold criterion in one case and pass in the other. This effect can lead to discrepancies in the cluster width and charge between data and simulation. To evaluate the current status, the cluster seed noise rescaled by gains, further called only as ``seed noise'', is shown in Fig.~\ref{fig:figures/clusterseednoiseTOBl1to4simulation}. From these plots it is obvious that the difference in the seed noise between data and simulation is huge. This finding triggered the update of these conditions by the responsibles of the tracker local reconstruction group. The gains in the simulation will be updated during 2018.

For the update of 2018 conditions in simulation, the G1 gain, noise and bad components conditions are taken directly from data, the Gsim gain is taken to be a smeared version of G1 and the G2 gain is recomputed by a strategy similar as before.

    \insertFigure{figures/clusterseednoiseTOBl1to4simulation} % Filename = label
                 {0.5}       % Width, in fraction of the whole page width
                 { Data and simulation comparison of the on-track cluster seed noise rescaled by gains for the OB2 geometry. The simulated distribution is rescaled to the number of clusters in data. The bottom plot represents the data to simulation ratio. }


\textbf{The electrons to ADCs conversion factor~($eADC$)}

Similarly, as the conversion factor $Ee$, the conversion factor from electrons to ADCs $eADC$ is sensitive to the ageing of modules. The comparisons of data with different simulations are shown in Figs.~\ref{fig:figures/clusterchargeRescaledalll0to0eADC}~and~\ref{fig:figures/seedwidthTOBeADC} for the cluster charge, width and seed charge, the simulations are using an $eADC$ conversion factor within up to $\pm 20\%$ from the default value of 247~$\mathrm{ADC^{-1}}$. By its nature the impact on the cluster quantities of the variations of $eADC$ is very similar to the case with varied $Ee$ factor and again the conversion factor $eADC$ alone cannot be taken as the responsible for the non-description of the data by simulation. 

As already suggested the evolution of the conversion factors is already taken into account by applying the gains. In data the (G1 $\times$ G2) factor corrects for the electronics effects and the differences at the sensor level by tuning the MPV of the MIP response to be 300~ADC/mm. The gain calibration procedure then guarantees that the cluster charge is stable through the operation of CMS even if due to radiation effects the inefficiency, for example in the charge collection, can increase. In simulation, the (G1 $\times$ G2), if well calibrated, also targets the charge to be the one of the MIP. Therefore the product of $Ee$ and $eADC$ conversion factors is already fixed by this calibration procedure. 


    \insertFigure{figures/clusterchargeRescaledalll0to0eADC} % Filename = label
                 {0.5}       % Width, in fraction of the whole page width
                 { Data and simulation comparison of the on-track cluster charge divided by the track length for the OB2 geometry and different values of the conversion factor~$eADC$. The simulated distributions are rescaled to the number of clusters in data.  The bottom plot represents the data to simulation ratios. }

    \insertTwoFigures{figures/seedwidthTOBeADC} % Filename = label %TDO continue here
                 {figures/clusterwidthTOBl1to4eADC}
                 {figures/clusterseedchargeRescaledTOBl1to4eADC} % Filename = label
                 {0.45}       % Width, in fraction of the whole page width
                 { Distribution of the on-track cluster width (left) and cluster seed charge (right) in data and simulation for the OB2 geometry and different values of the conversion factor~$eADC$.  The simulated distributions are rescaled to the number of clusters in data.  The bottom plots represent data the to simulation ratios. }

%TODO plot seed noise
%TODO plot e per ADC
%TODO G1, G2, Gsim
%TODO all the hits are accumulated for each evet!
%TODO write something about database
 
\subsection{Virgin raw simulation~\label{sec:VRsimu}}

Besides the standard zero suppressed simulation, the tracker digitization step brings a possibility to simulate Virgin Raw data. In this case, several steps need to be added to the simulation steps described above. After inducing the charge on the strips, the APV saturation due to the HIP effect can be simulated. This option is by default switched off. Then the APV baseline is shifted down from its designed position of 127~ADC, accordingly to the amount of charge collected by a given APV and the number of channels reading the charge. The strip noises, CMNs and pedestals are finally added and the result is stored.

%TODO the baseline shift is not correct according to me 
%TODO - there is a diffusion of the charge in the perpendicular(ask?) plane
% TODO	-particle goes straight (ask? meaning the track goes straight?)
% TODO charge spread depends on applied and depletion voltage ???

%\newpage

\section{The cross talk measurement to improve the simulation~\label{sec:xtalk}}

As identified in the above sections a change in the cross talk parameters could lead to a better description of data by simulation. There are several possibilities how to update the cross talk parameters. The first possibility was already introduced in Subsection~\ref{sec:induce}, by tuning the cross talk on data to simulation comparisons. A matrix of samples with each of the fractions of charge shared to the first and second neighbors changed in steps, can be produced. Then the cross talk which is best improving the description of data by simulation is chosen. Before such tuning of the cross talk, it is necessary to perform an update of the bad channels, noise and gains in the simulation as their update after the tuning could destroy the obtained data-simulation agreement. Indeed, the noise and gains are measured conditions, which are influencing the cluster properties and therefore it is indispensable to update them prior to the cross talk tuning.

%and it is to produce a matrix of samples, each changing the two free cross talk parameters by step. Then it can be estimated via a $\chi^{2}$ test, which simulated cluster width distribution fits the one in data.

The second option is to measure the cross talk parameters from data. The measurement from data has an advantage that it is not sensitive to the other potentially outdated parameters in simulation like the mentioned tuning. In following Subsection~\ref{sec:xtalkb}, the results on a cross talk measurement are presented and then compared with the results obtained by the tuning of cross talk on data to simulation comparisons. Because of the low data statistics in TID and TEC, the cross talk could be measured only in barrel. To obtain new cross talk parameters for TID and TEC, a third option was developed, which combines measurement from data and the tuning based on data to simulation comparisons. It is presented in Subsection~\ref{sec:tuning}.

\subsection{The cross talk  measurement in the barrel~\label{sec:xtalkb}}

\textcolor{red}{In this section first the data-taking used for the cross talk measurement is discussed. It is followed by the description of methodology for the cross talk measurement and discussion on the dependence of the cross talk on time. Later, both results of the cross talk measurement and their validation are given. In the last subsection, the evolution of the cross talk with the radiation is studied. } 


\subsubsection{\textcolor{red}{Data-taking for the cross talk measurement}}

The cross talk parameters used in simulation were measured using the 0~T cosmic VR data at the beginning of Run~1. After having identified that the outdated cross talk values could lead to large discrepancies between data and simulation, the decision  to remeasure the cross talk was taken. The new data for the cross talk measurement were taken in the virgin raw mode at the end of March 2018. These data are CRUZET, \textcolor{red}{i.e. 0~T cosmics } , implying that there was no magnetic field and no collisions in the detector during the data taking, therefore only cosmic muons were triggered. The trigger used during this data-taking, was first designed for triggering cosmics during collisions~(CDC), which was requested by the tracker alignment group. Its advantage is that it increases the probability of triggering muons which passed through the tracker. The CDC trigger bit~\cite{website:trigger} assumes that the muon passes through the whole CMS in around 30~ns, implying that the top leg of the muon, \textcolor{red}{i.e. the part of the muon track in  the top part of the detector}, should be seen in the top part of the detector one bunch crossing before seeing the bottom leg in the bottom part of the detector. Moreover the two legs should be back to back, leading to the requirement on the angle between the legs to be $\pi \pm 1/6\pi$. To further enhance cosmics passing through the tracker, the following criteria on $\Phi$ and $\eta$ are imposed. The top leg must pass the requirement $|\Phi|<1/2\pi \pm 1/3\pi$ and the bottom leg $|\Phi|<3/2\pi \pm 1/3\pi$, the pseudorapidity of muon is required to satisfy $|\eta|<1.202$. The requirement on $\eta$ results in muons which are almost exclusively measured by DTs only. In totality 145854 events were collected and saved, corresponding to runs 312627, 312972, 312973, 312974 and 312977.

The absence of the magnetic field is important to decompose the effects of the Lorentz angle from the cross talk in the data. Furthermore the data need to be in the VR mode in order not to suppress the strips which have a non-zero charge because of the cross talk but is small enough to be zero suppressed. 


\subsubsection{\textcolor{red}{The cross talk measurement method}}

When there is no cross talk effect, the total charge $Q$ should be collected by only one strip in case where the track passes only through the sensitive volume of the sensor belonging to that strip. But as shown in Fig.~\ref{fig:figures/crossTalk}, due to the cross talk effect, the charge is shared with two neighboring strips on each side and the charge collected by this seed strip is not $Q$ but only a fraction $x_{0}$ of the original charge $Q$. The neighboring strips obtain smaller fractions $x_{1}$ and $x_{2}$ of this charge $Q$. The total charge has to be conserved, leading to the requirement on the fractions to be 
\eq{fracs}
{
1=x_{0}+2x_{1}+2x_{2}. 
}

At first attempt, the strategy of the previous cross talk measurement was reused, a selection on the angle of the track imposed in order to reduce the probability of the track traversing the sensor area to belong to more than one strip. The selection on the track was chosen to be the same as in the previous measurement:

\eq{narrowTrack}
{
|\tan \theta \cos \varphi| < 0.5 \times (p/t),
}
where $\theta$ and $\varphi$ are the local track angles defined in Fig.~\ref{fig:figures/localCoordinates}, $p$ is the pitch between the strips and $t$ is the sensor thickness. This selection is ensuring that the track projected to the xz plane is close to be perpendicular to the x-axis. The track passing such a criterion is shown in Fig.~\ref{fig:figures/moduleCT} and is denoted as ``accept''. On the contrary tracks like the one denoted as ``reject'' are rejected by this criterion. Furthermore, quality criteria on the number of hits in the track ($N_{hit}>6$) and on the $\chi^{2}$  ($\chi^{2}/dof < 10$) of the track are imposed to select only good tracks. 

The cross talk can then be measured from the strip charges within clusters originating from the accepted tracks. For the purpose of the cross talk measurement, each cluster is reconstructed from pedestal and CMN subtracted digis with no truncation of strips with negative charge. We reconstruct a cluster by forcing it to have five strips, two on each side around the central one. Then the fraction of shared charge can be computed as


\eq{fractions}
{
x_{i} = \frac{\eta_{\pm i}}{1+2\eta_{\pm 1}+2{\eta_{\pm 2}}} ,
}
with $\eta_{i}$ defined as
\eq{etas}
{
\eta_{\pm i} = \frac{q_{\pm i}}{q_{0}} = \frac{x_{i}}{1-2x_{1}-2{x_{2}}} ,~\mathrm{with}~i=1,2,
}
where $x_{1}$ is the fraction of charge received by the first neighboring strip,  $x_{2}$ is the fraction of charge received by the second neighboring strip and $x_{0}$ is obtained from the normalization condition $x_{0} = 1-2x_{1}-2x_{2}$. $q_{0}$ is the charge on the seed strip and $q_{i}$ the charge registered by the neighboring strips. The distributions of $q_{\pm 1}/q_{0}$ and $q_{\pm 2}/q_{0}$ are each filled twice per cluster and fitted separately by a Gaussian function, avoiding the tails. The mean values of these functions are inserted in the Eq.~\ref{eq:fractions} in order to compute the fraction of charge shared with the neighboring strips. The cross talk depends on the pitch size, strip length and width, and thus it has to be measured separately for each geometry. 
%In the table, only the results of the cross talk measurement in the barrel is present, because of the pseudorapidity cut of the chosen trigger for the current data-taking, the tracks passing in the disks and endcaps are suppressed. 
%The results on the cross talk measurement performed at Run~1, which are currently used in simulation, together with the results of the presented measurement, which is described later, can be seen in Table~\ref{tab:measuredXtalk}. 

    \insertFigure{figures/crossTalk} % Filename = label
                 {0.5}       % Width, in fraction of the whole page width
                 { A schema of charge sharing between neighboring strips. }

    \insertFigure{figures/moduleCT} % Filename = label
                 {0.5}       % Width, in fraction of the whole page width
                 { A schematic view of the track selection for the cross talk measurement. }

\subsubsection{\textcolor{red}{The dependence of the cross talk on time}}

As discussed, the observed pulse shapes for the seed strip and the neighboring strips are different and they peak at different time (see Fig.~\ref{fig:figures/timeResponseReal}), leading to the conclusion that the cross talk evolves as a function of time. To test such hypothesis the time information of the muon is required. The time of the muon is given at the interaction point and details about its computation can be found in Section~\ref{sec:muonTiming}. In the data collected in 2018 two kinds of muons were observed: Ones having hits only in the top part of the DTs as sketched in the left part of Fig.~\ref{fig:figures/muonTracks}, further referred to as ``top muons'',  and the other having hits only in the bottom part of the DTs as shown in right part of Fig.~\ref{fig:figures/muonTracks}, in later referred to as ``bottom muon''. In around 90\% of the cases both top and bottom muons are present in one event, as they are two legs of the same muon, but both are associated with a different time. 

The goal of this analysis is to find the muons which passed the IP at the same time as a muon in collision would be produced, \textcolor{red}{i.e. as a function of the CMS clock}. Having access to the time information it is possible to study the evolution of the cross talk as a function of time. The muon originating from pp collisions would be produced at $time_{IP}^{InOut}=0$ and would move from the IP outside similarly as the bottom muon. This is not the case for the top muon, which travels from outside towards the IP. There can be two times associated to one track reconstructed in the tracker, one coming from the bottom muon leg and the second from the top muon leg. Only the time of the bottom muon leg is taken into account as this muon leg has the same direction as the muon produced in the pp collision. 

As also displayed in the right part of Fig.~\ref{fig:figures/muonTracks}, there are two kinds of hits left in the tracker by cosmics: One in the top part of the tracker in green, referred to as ``top hits'' and others in blue in the bottom part, referred to as ``bottom hits''. The timing of the tracker, i.e. the sampling of a pulse shape in cosmic events, is the same as in collisions and is tuned for collisions.  The situation is simpler for bottom hits, for which in case of the collision and cosmic muon being at the IP at the same time, the arrival time to the given module has to be the same. In case of top hits, the cosmic muon goes from outside towards the given module and then to the IP, while for the collision muon the direction is opposite, it starts at IP and then passes through a given module and continues towards outside of CMS. Therefore in this case both cosmic and collision muons with the same time at the IP reach the given module in the top part of tracker at a different time, specifically the cosmic muon arrives there before the collision one. For this reason of timing mismatch in the top part of the tracker, only hits in the bottom part of the tracker are used.

    \insertFigure{figures/muonTracks} % Filename = label
                 {0.99}       % Width, in fraction of the whole page width
                 { A schema of possible configurations of muon legs and hits in the tracker. }


%The distribution of time and free inverse beta for the bottom legs of the muons are shown in Fig.~\ref{fig:figures/mutimebeta}. As expected, the free inverse beta has  a largest population around one, as the direction of the bottom legs is from the IP outwards and that the muon speed is close to the speed of light. For the further analysis only muon tracks with positive free inverse beta are used. 
The distribution of time for the bottom legs of the muons is shown in Fig.~\ref{fig:figures/timeBot}. In principle, the time distribution for cosmic muons should be uniform due to the random arrival time of cosmics to the interaction point. But as mentioned in~\cite{Chatrchyan:2009ig}, the trigger efficiency is the highest for muons arriving to the interaction point in a time window of $\pm 5$~ns around the collision time. The time distribution in Fig.~\ref{fig:figures/timeBot} peaks around $-3.5$~ns. The muons with a time equal to zero passed the interaction point at the same time that a collision muon would be produced at IP. Twice the standard deviation of the time distribution is 13.2~ns. To evaluate which fraction of the spread is caused by the time resolution in DTs, the timing distribution from DTs of muons produced in collisions is shown in Fig~\ref{fig:figures/timeColl} with a Gaussian fit. The collision muons are produced at a time equal to zero, therefore there is no ambiguity on the production time as for the cosmic muons. But a bias originates from the fact that the muon time is computed with an assumption that muon travels at the speed of light. Twice the standard deviation of the collision muon time computed from the fit gives 3.3~ns. The cosmic muon timing distribution is more than four times larger because of the natural spread in arrival time of the cosmics. Another bias in the time measurement can appear due to the assumption used in the time computation that the muon passes through the interaction point.
 

%    \insertTwoFigures{figures/mutimebeta} % Filename = label
%                 {figures/timeBot}
%                 {figures/freeInverseBetaBot} % Filename = label
%                 {0.45}       % Width, in fraction of the whole page width
%                 {(left) Time distribution of bottom muon legs in 2018 CRUZET VR data. (right) The free inverse beta distribution of bottom muon legs in 2018 CRUZET VR data.}
    \insertFigure{figures/timeBot} % Filename = label
                 {0.5}       % Width, in fraction of the whole page width
                 { The distribution of the time $time_{IP}^{InOut}$ at the IP of the bottom muon legs in 2018 CRUZET VR data. In red is displayed the result of a Gaussian fit.  }


    \insertFigure{figures/timeColl} % Filename = label
                 {0.5}       % Width, in fraction of the whole page width
                 { Time distribution of muons originating from pp collisions recorded during the 2017 era~F. In red is displayed the result of a Gaussian fit.  }


The timing of the pulse shape sampling in the tracker is tuned in such a way that the cluster seed charge is required to be maximal. Such timing adjustments are done with particles originating from collisions, their production time is thus fixed to zero. Therefore to evaluate if the cosmic muon time is synchronized with the collision one, the cluster seed charge distribution can be studied as a function of the muon time as depicted in the left plot of Fig.~\ref{fig:figures/chargeVsTime}. As mentioned earlier, there is a bias in the time measurement coming from the fact that cosmic muons do not have to pass through the interaction point, but that the muon time computation is extrapolating the muon time to the interaction point. This effect could shift the maximum of the seed charge away from the time equal to zero. We can evaluate what is the dependency of the maximum of the cluster seed charge as a function of the closest distance of track from the IP. This closest distance, further referred only as ``distance'', is determined as the closest approach distance of the track to the IP. Four different cases of tracks passing at different distances to the IP are shown in Fig.~\ref{fig:figures/chargeVsTime} (left).  It can be noticed that the maxima of the cluster seed charge for the four cases are at different positions and none of them is around 0~ns. All cluster seed charge distributions peak at negative times and \textcolor{red}{larger in absolute value} is the distance, smaller is the corresponding peak time. Because of the observed dependence of the maximal cluster seed charge with the distance from the IP, the muon time has to be corrected as a function of this distance. A time correction factor can be estimated from a linear fit of the time positions~($t_{max}$) of the maxima obtained from the Gaussian fits in the left plot of~\ref{fig:figures/chargeVsTime} as a function of the distance~($d$). The fit displayed in the right plot of Fig.~\ref{fig:figures/chargeVsTime} is giving the following relation between time and distance

\eq{driftEquation}
{
t_{max} = (0.05 \pm 0.01) \times d  - (7.9\pm 0.4),
}
leading to the conclusion that the cluster seed charge is maximal for muon which passed directly through the interaction point at a time equal to $-7.9 \pm 0.4 $. The dependency of the time with the track distance can be resolved by subtracting the factor $ 0.05 \times d $ from the muon time.


    \insertTwoFigures{figures/chargeVsTime} % Filename = label
                 {figures/chargePerUnitLayersBinsbottommuBottom}
                 {figures/chargePerUnitLayersBinsfitbottommuBottom}
                 {0.45}       % Width, in fraction of the whole page width
                 {(left) The cluster seed charge profiles as a function of time for four different distances of the muon track from the IP in 2018 CRUZET VR data, the results of Gaussian fits are also displayed. (right) The time of the maxima extracted from the fits performed in the left plot as a function of the mean track distance from the IP, these points are fitted with a straight line. }

As suggested above, the cross talk depends on time. To prove this hypothesis, the $\eta_{\pm 1}$ and $\eta_{\pm 2}$ are studied as a function of time. As the available statistics does not allow to consider the $\eta_{\pm 1}$ and $\eta_{\pm 2}$ distribution per bin of 1~ns, overlapping time windows of 4~ns are used for the muons. The $\eta_{\pm 1}$ and $\eta_{\pm 2}$ distributions are filled for each time window and then fitted by Gaussians. The estimated means are shown in Fig.~\ref{fig:figures/etaAsTime}  for the OB2 geometry and allow to evaluate the time evolution of the charge sharing. Each bin of this figure corresponds to the middle of the 4~ns interval in which the fit was performed, for example the time equal to zero is in reality integrated over clusters from muons arriving to the IP in a time window of $0 \pm 2$~ns. The dependence of $\eta_{\pm 1}$ and $\eta_{\pm 2}$ as a function of time in Fig.~\ref{fig:figures/etaAsTime} confirms that the time window in which to perform the cross talk measurement has to be selected with care. The time window $-8 \pm 2$~ns, for which the cluster seed charge is maximal, is therefore of special interest for the cross talk measurement as it corresponds to the collision conditions. The width of the time window is also close to the time spread in the DTs shown in Fig.~\ref{fig:figures/timeColl}. The $\eta_{\pm 1}$ and $\eta_{\pm 2}$ distributions resulting from muons within the $-8 \pm 2$~ns time window are shown in Fig.~\ref{fig:figures/etaAsTime}, together with the Gaussian fits.  The width of these distributions is due to the strip noise.  The tails in charge sharing are caused by inclination of track or diffusion.


    \insertTwoFigures{figures/etaAsTime} % Filename = label
                 {figures/etaOneaAsTimeTOB2bottommuBottom}
                 {figures/etaTwoAsTimeTOB2bottommuBottom} % Filename = label
                 {0.45}       % Width, in fraction of the whole page width
                 { The time evolution of the charge shared with the first (left) and second (right) neighboring strips divided by the seed charge for the OB2 geometry in 2018 CRUZET VR data.}

    \insertTwoFigures{figures/xtalkTOB2} % Filename = label
                 {figures/narrowTrackSharing1TOB2bottommuBottom}
                 {figures/narrowTrackSharing2TOB2bottommuBottom} % Filename = label
                 {0.45}       % Width, in fraction of the whole page width
                 { The $\eta_{\pm 1}$ (left) and  $\eta_{\pm 2}$ (right) distributions of bottom muon tracks in the time window of $-8 \pm 2$~ns for the OB2 geometry in 2018 VR CRUZET data. The results of Gaussian fits are indicated in red.}


\subsubsection{\textcolor{red}{The results on the cross talk measurement}}

The cross talk parameters are computed from Eq.~\ref{eq:fractions} for all barrel geometries in the time window of $-8 \pm 2$~ns from the Gaussian fits of the corresponding  $\eta_{\pm 1}$ and $\eta_{\pm 2}$ distributions as shown for OB2 in Fig.~\ref{fig:figures/xtalkTOB2}. The measured values together with the values currently used in the simulation corresponding to an old measurement are recorded in Table~\ref{tab:measuredXtalk}. It can be noticed that in all cases the new sharing of the charge, i.e. the cross talk, decreases. There is no information in the documentation if, during the previous cross talk measurement, any timing requirement was taken into account and therefore part of the change can originate from including the time information to evaluate the cross talk. In addition,  there is also a real change in the cross talk expected from the radiation effects. 

\begin{table}[h]
\begin{center}
\begin{tabular}{|l|l|l|l|l|}
\hline
Geometry & Type & $x_{0}$ & $x_{1}$ & $x_{2}$ \\
\hline
\hline
IB1 & current measurement & $ 0.836 \pm 0.009 $ & $0.070 \pm 0.004 $ & $0.012 \pm 0.002 $ \\
IB1 & currently in simulation & $ 0.775 $ & $ 0.096 $ & $0.017 $  \\
\hline
IB2 &  current measurement & $0.862 \pm 0.008 $ & $0.059 \pm 0.003 $ & $0.010 \pm  0.002 $  \\
IB2 & currently in simulation &  $0.830 $ & $0.076 $ & $ 0.009$   \\
\hline
OB2 &  current measurement & $0.792 \pm 0.009 $ & $0.083 \pm 0.003 $ & $0.020 \pm 0.002$  \\
OB2 & currently in simulation &   $0.725 $ & $0.110 $ & $ 0.027 $  \\
\hline
OB1 &  current measurement &  $0.746 \pm 0.009 $ & $0.100 \pm 0.003 $ & $0.027 \pm 0.002 $  \\
OB1 & currently in simulation &  $0.687 $ & $0.122 $ & $ 0.034 $ \\
\hline
\end{tabular}
\caption[Table caption text]{The cross talk measured in 2018 CRUZET VR data and the cross talk values used currently in simulation for barrel geometries. }
\label{tab:measuredXtalk}
\end{center}
\end{table}


Ref.~\cite{Hartmann:2017gzy} discusses that, due to the surface radiation damages of the silicon strip sensors, the inter--strip capacitance increase and inter--strip resistance decrease, and consequently an increase in cross talk is expected. In the presented measurement a decrease of the cross talk is observed contrary to the literature. The cross talk change can also depend on the change of the strip implant to backplane capacitance, of the strip implant to the aluminium strip capacitance, and of the depletion and applied voltages. Therefore the expected direction of the cross talk change is difficult to evaluate as one could thus assume that the other capacitances may have changed as well.

\subsubsection{Validation of the new cross talk parameters~\label{sec:validation}}

To evaluate if the new cross talk measurement contributes to an improvement in the data and simulation comparison,  the cross talk parameters are injected into the simulation and compared with data and the default simulation. For this validation the 2018 minimum bias simulated samples and 2018 zero bias data from run 317649 are used. The cluster charge, width and seed charge distributions for barrel geometries are shown in Fig.~\ref{fig:figures/clusterchargeRescaledTIBXTm}~to~\ref{fig:figures/seedTOBXTm}. In these plots it is obvious, that newly measured cross talk parameters largely improve the overall cluster width and seed charge description. The cluster charge remains almost unchanged, as the cross talk influences only the charge sharing between channels, not the total charge.  

For OB2, the measured cross talk fractions $(x_{0}, x_{1}, x_{2})$ are ($0.792 \pm 0.009 $, $0.083 \pm 0.003 $, $0.020 \pm 0.002$). When tuning the cross talk parameters in Section~\ref{sec:induce} to best describe the cluster width in data, the best obtained fractions are (0.814, 0.070, 0.023), which suggest even a smaller cross talk than the measured one, but no uncertainties were evaluated for these tuned parameters. 

Even though the large improvement, there are still discrepancies in the description of data by simulation. These discrepancies could be caused by the outdated conditions, such as gains and noise. These conditions can have an impact on the cluster width, charge and seed charge description and therefore they are as well being currently updated. The discrepancies can also arise from the fact that the simulation is simplified and many parameters in the simulation are outdated as discussed previously. Thus the cross talk in simulation does not have to purely describe the cross talk itself, but it can compensate for other effects as well. 


    \insertTwoFigures{figures/clusterchargeRescaledTIBXTm} % Filename = label
                 {figures/clusterchargeRescaledTIBl1to2XTm} % Filename = label
                 {figures/clusterchargeRescaledTIBl3to4XTm} % Filename = label
                 {0.45}       % Width, in fraction of the whole page width
                 { Distribution of the on-track cluster charge in data and simulation for the IB1 (left) and IB2 (right) geometries for the current (default) and newly measured (updated) cross talk parameters~($XT$).  The simulated distributions are rescaled to the number of clusters in data.  The bottom plots represent data to simulation ratios. }

    \insertTwoFigures{figures/widthTIBXTm} % Filename = label %TDO continue here
                 {figures/clusterwidthTIBl1to2XTm}
                 {figures/clusterwidthTIBl3to4XTm}
                 {0.45}       % Width, in fraction of the whole page width
                 { Distribution of the on-track cluster width in data and simulation for the IB1 (left) and IB2 (right) geometries, for the current (default) and newly measured (updated) cross talk parameters~($XT$).  The simulated distributions are rescaled to the number of clusters in data.  The bottom plots represent the data to simulation ratios. }

    \insertTwoFigures{figures/seedTIBXTm} % Filename = label
                 {figures/clusterseedchargeRescaledTIBl1to2XTm} % Filename = label
                 {figures/clusterseedchargeRescaledTIBl3to4XTm} % Filename = label
                 {0.45}       % Width, in fraction of the whole page width
                 { Distribution of the on-track cluster seed charge in data and simulation for the IB1 (left) and IB2 (right) geometries, for the current (default) and newly measured (updated) cross talk parameters~($XT$).  The simulated distributions are rescaled to the number of clusters in data.  The bottom plots represent the data to simulation ratios. }


    \insertTwoFigures{figures/clusterchargeRescaledTOBXTm} % Filename = label
                 {figures/clusterchargeRescaledTOBl1to4XTm} % Filename = label
                 {figures/clusterchargeRescaledTOBl5to6XTm} % Filename = label
                 {0.45}       % Width, in fraction of the whole page width
                 { Distribution of the on-track cluster charge in data and simulation for the OB2 (left) and OB1 (right) geometries, for the current (default) and newly measured (updated) cross talk parameters~($XT$).  The simulated distributions are rescaled to the number of clusters in data.  The bottom plots represent the data to simulation ratios. }

    \insertTwoFigures{figures/widthTOBXTm} % Filename = label %TDO continue here
                 {figures/clusterwidthTOBl1to4XTm}
                 {figures/clusterwidthTOBl5to6XTm}
                 {0.45}       % Width, in fraction of the whole page width
                 { Distribution of the on-track cluster width in data and simulation for the OB2 (left) and OB1 (right) geometries, for the current (default) and newly measured (updated) cross talk parameters~($XT$).  The simulated distributions are rescaled to the number of clusters in data.  The bottom plots represent the data to simulation ratios. }

    \insertTwoFigures{figures/seedTOBXTm} % Filename = label
                 {figures/clusterseedchargeRescaledTOBl1to4XTm} % Filename = label
                 {figures/clusterseedchargeRescaledTOBl5to6XTm} % Filename = label
                 {0.45}       % Width, in fraction of the whole page width
                 { Distribution of the on-track cluster seed charge in data and simulation for the OB2 (left) and OB1 (right) geometries, for current (default) and newly measured (updated) cross talk parameters~($XT$).  The simulated distributions are rescaled to the number of clusters in data.  The bottom plots represent the data to simulation ratios. }


\subsubsection{\textcolor{red}{The cross talk evolution with ageing of the tracker}}

Fig.~\ref{fig:figures/xtalkratio1} shows the ratio of  the new $x_{1}$ parameter to the default value used in the simulation for the four barrel geometries as a function of the distance from the beamline. This ratio is fitted by a straight line. From the slope of $0.0012 \pm 0.0005$ it can be seen that the change in the parameter $x_{1}$ depends on the distance of a geometry from the beamline and is the largest for the closest geometry. As the fluence of particles decreases as a function of distance from the beamline, the closest geometry suffers therefore from the largest fluence, its radiation damage is the largest and consequently its cross talk change is the largest.  


    \insertFigure{figures/xtalkratio1} % Filename = label
                 {0.5}       % Width, in fraction of the whole page width
                 {The ratio of the new measurement of the $x_{1}$ parameter (updated) to the default value currently used in simulation (default) as a function of the distance from the beamline of the four different barrel geometries. The result of a linear fit is indicated in red. }

%Fig.~\ref{fig:figures/xtalkratio1} suggests that the cross talk changes as a function of the fluence. 
But the sensors in different geometries are not the same and therefore for a given fluence the radiation damage of different sensors is different. To decouple the two effects and evaluate purely the fluence dependence, the cross talk can be studied for different layers of a single geometry. The time dependence of the $\eta_{\pm 1}$ and $\eta_{\pm 2}$ distributions is shown in Fig.~\ref{fig:figures/etaAsLayers} for the four layers of OB2 separately. The plots were obtained similarly as the plots in Fig.~\ref{fig:figures/etaAsTime}, but this time the time window for each bin is 6~ns in order to gain in statistics. The charge sharing is larger for the first two layers, compared to the layers three and four. In  $\eta_{\pm 1}$ distribution there is also a visible difference in the charge sharing between layer three and four, whose ratio varies from one by up to $\pm$10\%. 

The first two layers are composed of double-sided modules, while the second two have only one side of sensors. In the double-sided modules, the sensors are back to back, therefore a particle going from the IP outside the CMS is entering one sensor from the side of the strips and the second from the backplane. The formation of the signal could be different depending from which side the particle enters the sensor and therefore there can be a different cross talk between the mono and stereo sides of one module. The $\eta_{\pm 1}$ and $\eta_{\pm 2}$ as a function of time, for decomposed mono and stereo sensors of modules in the OB2 layers one and two are shown in Fig.~\ref{fig:figures/etaAsMonoStereo}. In these plots, in the majority of bins of the $\eta_{\pm 1}$ and $\eta_{\pm 2}$  distributions, the sharing for modules in one layer is similar or larger for stereo sensors than mono. This would explain that the first two layers of OB2, which are composed of double-sided modules, exhibit a larger charge sharing to the first neighboring strips than the second two, which are single-sided, as discussed in Fig.~\ref{fig:figures/etaAsLayers}. 

%The evidence for pure dependency of cross talk on the radiation can then be observed by the change of $\eta_{\pm 1}$ between layer three and four in the same plot.


    \insertTwoFigures{figures/etaAsLayers} % Filename = label
                 {figures/etaOneaAsTimeLayerTOB2bottommuBottom}
                 {figures/etaTwoAsTimeLayerTOB2bottommuBottom} % Filename = label
                 {0.45}       % Width, in fraction of the whole page width
                 { The time evolution of the charge carried by the first (left) and second (right) neighboring strips divided by the seed charge for the four layers of the OB2 geometry in the 2018 CRUZET VR data. The ratio plots are shown in the bottom, the blue line corresponds to the ratio of L1 to L2, the violet line corresponds to the ratio of the L2 to L3, and the gray line corresponds to the ratio of the L3 to L4 distributions. }


    \insertTwoFigures{figures/etaAsMonoStereo} % Filename = label
                 {figures/etaOneaAsTimeMonoStereoTOB2bottommuBottom}
                 {figures/etaTwoAsTimeMonoStereoTOB2bottommuBottom} % Filename = label
                 {0.45}       % Width, in fraction of the whole page width
                 { The time evolution of the charge carried by the first (left) and second (right) neighboring strips divided by the seed charge for the mono and the stereo sensors of first two layers of TOB in the 2018 CRUZET VR data. The ratio plots are shown in the bottom, the blue line corresponds to the ratio of L1 mono to stereo, and the gray line corresponds to the ratio of the L2 mono to stereo distributions. }


\subsection{\textcolor{red}{Cluster seed charge evolution with time}}

\textcolor{red}{In the data used for the cross talk measurement}, it is also possible to study the evolution of the charge on the seed strip and the neighboring strips as a function of the muon arrival time as shown in Fig.~\ref{fig:figures/nm2EvolutionCanTOB2merged}. These charge evolutions can be compared to the past measurements of the pulse shapes depicted in Fig.~\ref{fig:figures/timeResponseReal}, obtained from focusing a laser on one strip of the TOB module. In order to compare these figures, the time axis of the Fig.~\ref{fig:figures/timeResponseReal} should be reverted  (from positive to negative values) so that it represents the muon arrival time. Indeed a muon which arrives with a positive delay with respect to the reference (i.e. after the reference) is on the rising part (on the left) of its pulse shape and has not yet reached its maximum. Similarly to the past measurement, the neighboring strips peak at the higher arrival times than the seed one, what corresponds to the earlier positions at the pulse shapes. Moreover, negative contributions are observed for the second neighboring strips,  the time range not being sufficient to observe a similar behavior for the first neighbors. This undershoot is also clearly visible in Fig.~\ref{fig:figures/timeResponseReal}. The ratio of the maximal charge on the first to the second neighboring strip is similar for both measurements, around three. However, there is a large difference between these two measurements in the ratio of the maximal charge of the first neighbors to the seed strip: This ratio is around three for the past measurement, but around five for the current measurement. 

%Although the change in the pulse shapes properties is not an issue for the cross talk measurement, this evolution is not understood and could be investigated in the future. Understanding the cause of such an effect could give us an insight into why the cross talk decreased while the opposite is expected.

    \insertFigure{figures/nm2EvolutionCanTOB2merged} % Filename = label
                 {0.5}       % Width, in fraction of the whole page width
                 { The charge on the strip for the seed strip and two neighbors on each side for the OB2 geometry. The black curve corresponds to the seed strip, blue and red curves to the first and second neighboring strips respectively. The neighbors on the left side from the seed are depicted in the lighter shades. }

\subsection{Evaluation of the cross talk for TID and TEC~\label{sec:tuning}}

%TODO some blba bla here - in principle we could rescale by the parametrization we saw before, but it is not that easy
%plot of seed charge


Because of the insufficient statistics of clusters for TID and TEC in the available CRUZET VR data, the cross talk for these partitions had to be evaluated differently. For the barrel, the change of $x_{1}$ scales linearly with the distance from the beamline and it can be expected that the same applies for TID and TEC. But the fluence and module geometries in TID and TEC are more complicated than in the case of the barrel and the $x_{1}$ change for the TID and TEC does not evolve linearly as a function of distance from beamline. Moreover there is no guarantee that the cross talk currently used in simulation is correct.  

The change in cross talk in the TID and TEC rings can be studied in the cluster seed charge distribution, which is a direct proxy to the $x_{0}$ parameter. The data to simulation ratio of the mean cluster seed charge is shown in Fig.~\ref{fig:figures/seedratioTIDTEC}. The ratio is always larger than one, revealing that the mean seed charge is larger in data than in simulation and consequently the charge sharing is smaller in data than in simulation. No clear trend as a function of the distance from the beamline is observed in this ratio.

    \insertFigure{figures/seedratioTIDTEC} % Filename = label
                 {0.5}       % Width, in fraction of the whole page width
                 {The data to simulation ratio of the mean cluster seed charge for rings of TID and TEC.}

As the cluster seed charge is a proxy to $x_{0}$, the updated $x_{0}^{u}$ can be evaluated as

\eq{x0}
{
x_{0}^{u} =\frac{\langle Q_{seed}^{data} \rangle}{\langle Q_{seed}^{simu} \rangle  } x_{0}^{d},
}
where $\langle Q_{seed}^{data} \rangle$ and $\langle Q_{seed}^{simu} \rangle$ are the mean cluster seed charge in data and simulation, respectively, and $x_{0}^{d}$ is the default value of $x_{0}$ currently used in simulation. 

The change of $x_{2}$ is found to be constant for all the barrel geometries and it is evaluated from a fit to be 0.76. We decided to apply this factor to rescale $x_{2}$ in TID and TEC. The updated value $x_{2}^{u}$ is computed as 
\eq{x2}
{
x_{2}^{u} = 0.76~x_{2}^{d},
}
with  $x_{2}^{d}$ being the default value of $x_{2}$ currently used in simulation. The $x_{2}$ parameter is small in absolute and therefore the other parameters do not largely depend on it. 

The updated value of $x_{1}$ can then be evaluated from the normalization condition $x_{1} = (1-x_{0}-2x_{2})/2$. The updated values of cross talk can be seen in Table.~\ref{tab:measuredXtalkTODTEC}.

\begin{table}[h]
\begin{center}
\begin{tabular}{|l|l|l|l|l|}
\hline
Geometry & Type & $x_{0}$ & $x_{1}$ & $x_{2}$ \\
\hline
\hline
W1a &  new values & 0.8571 & 0.0608 & 0.0106 \\
W1a &  currently in simulation & 0.786 & 0.093 & 0.014 \\
\hline
W2a &  new values & 0.8861 & 0.049 & 0.008 \\
W2a &  currently in simulation & 0.7964 & 0.0914 & 0.0104 \\
\hline
W3a &  new values & 0.8984 & 0.0494 & 0.0014 \\
W3a &  currently in simulation & 0.8164 & 0.09 & 0.0018 \\
\hline
W1b &  new values & 0.8827 & 0.0518 & 0.0068 \\
W1b &  currently in simulation & 0.822 & 0.08 & 0.009 \\
\hline
W2b &  new values & 0.8943 & 0.0483 & 0.0046 \\
W2b &  currently in simulation & 0.888 & 0.05 & 0.006 \\
\hline
W3b &  new values & 0.8611 & 0.0573 & 0.0121 \\
W3b &  currently in simulation & 0.848 & 0.06 & 0.016 \\
\hline
W4 &  new values & 0.8881 & 0.0544 & 0.0015 \\
W4 &  currently in simulation & 0.876 & 0.06 & 0.002 \\
\hline
W5 &  new values & 0.7997 & 0.077 & 0.0231 \\
W5 &  currently in simulation & 0.7566 & 0.0913 & 0.0304 \\
\hline
W6 &  new values & 0.8067 & 0.0769 & 0.0198 \\
W6 &  currently in simulation & 0.762 & 0.093 & 0.026 \\
\hline
W7 &  new values & 0.7883 & 0.0888 & 0.0171 \\
W7 &  currently in simulation & 0.7828 & 0.0862 & 0.0224 \\
\hline
\end{tabular}
\caption[Table caption text]{The updated cross talk for rings of TID and TEC. }
\label{tab:measuredXtalkTODTEC}
\end{center}
\end{table}


The validation plots of the cluster charge, width and seed charge are shown in Figs.~\ref{fig:figures/clusterchargeRescaledTIDTECXTm} to \ref{fig:figures/seedTIDTECXTm} for two chosen examples which are W1a (left) and W5 (right). By definition of the cross talk update procedure there is an improvement in the description of the cluster seed charge. The updated cross talk also leads to the large improvement in the description of the cluster width.

    \insertTwoFigures{figures/clusterchargeRescaledTIDTECXTm} % Filename = label
                 {figures/clusterchargeRescaledTIDl1to1XTm} % Filename = label
                 {figures/clusterchargeRescaledTECl5to5XTm} % Filename = label
                 {0.45}       % Width, in fraction of the whole page width
                 { Distribution of the on-track cluster charge in data and simulation for the W1a (left) and W5 (right) geometries, for current (default) and newly updated cross talk parameters~($XT$).  The simulated distributions are rescaled to the number of clusters in data.  The bottom plots represent the data to simulation ratios. }

    \insertTwoFigures{figures/widthTIDTECXTm} % Filename = label %TDO continue here
                 {figures/clusterwidthTIDl1to1XTm}
                 {figures/clusterwidthTECl5to5XTm}
                 {0.45}       % Width, in fraction of the whole page width
                 { Distribution of the on-track cluster width in data and simulation for the W1a (left) and W5 (right) geometries, for the current (default) and newly updated cross talk parameters~($XT$).  The simulated distributions are rescaled to the number of clusters in data.  The bottom plots represent the data to simulation ratios. }

    \insertTwoFigures{figures/seedTIDTECXTm} % Filename = label
                 {figures/clusterseedchargeRescaledTIDl1to1XTm} % Filename = label
                 {figures/clusterseedchargeRescaledTECl5to5XTm} % Filename = label
                 {0.45}       % Width, in fraction of the whole page width
                 { Distribution of the on-track cluster seed charge in data and simulation for the W1a (left) and W5 (right) geometries, for the current (default) and newly updated cross talk parameters~($XT$).  The simulated distributions are rescaled to the number of clusters in data.  The bottom plots represent the data to simulation ratios. }

\newpage

\subsection{Conclusion}

This section presents the results of a new cross talk measurement. The cross talk parameters used in simulation were measured at the beginning of Run~1, and as demonstrated in this section, the cross talk changes with the ageing of the detector, therefore it had to be remeasured. We show that the cross talk depends on the particle arrival time to the module, which probably was not taken into account during the Run~1 measurements. 

The newly measured cross talk improves the description of the cluster seed charge and width in  simulation although discrepancies between data and simulation can still be observed. These discrepancies can be caused by the outdated database conditions and therefore they are being updated for the 2018 simulated samples. If after the update of database conditions some discrepancies remain between data and simulation, it can be the sign that the simulation is over-simplified and that the update of the cross talk is not sufficient to well describe the cluster properties in data.

%, it a sign that the conversion factor between energy and electrons or electons and ADC has to be updated as well. These conversion factors have only minor influence on the cluster width. The cross-talk was found to be only parameter in the simulation which strongly influences the cluster width, therefore if the update of database conditions leads to large discrepancies in cluster width, it means that the simulation is over-simplified and update of cross-talk is not sufficient to well described the cluster propertiesin data. In such case the cross talk can be tuned to best described data as described in section~\ref{sec:induce}. In this case the cross talk parameter would not purely account for the cross talk effect but would compansate for other effects not described in the simulation.

%TODO blabla how we need to (and are proceeding) with cnhange of database conditions
%TODO some blabla, different data and put into official release of CMSSW

Due to the limitations of the 2018 CRUZET VR data, it is not possible to measure the cross talk in the disks and endcaps. A newly designed procedure to update the cross talk parameters in these regions rescales the cross talk currently used in simulation  by well motivated factors. 

%The update of parameters leads to a better agreement between data and simulation in cluster width and seed charge for rings of TEC and TID.
%To update cross talk also for disk and endcapes, dedicated data-taking would be required. Another option is again to deetrmine cross-talk parameters from the data and simulation comparison as described in section~\ref{sec:induce}. If the second approach is chosen, it is necesary first to update detector conditions, which are bad channels, noise and gains. These conditions are obtained from data and can affect the cluster width and charge. Then the matrix of the simulations with different cross talk parameters would be produced and these simulations would be compared to data. The agreement between data and simulations would be established based on a $\chi^{2}$ test of the cluster width distribution compatibility. The cross talk parameters of the sample with the best $\chi^{2}$ can then be injected into simulation. 


The new obtained cross talk parameters considerably improve the cluster width and seed charge description and therefore they are now integrated into the official release of the CMS software and are going to be used for the central production of the simulated samples. The impact of the cross talk change on the high level objects is not expected to be large as the cluster charge remains the same and only the charge sharing differs, but there could be a non-negligible impact on tracking due to several reasons. Because of the different charge sharing and threshold effects in the clustering, the hit position computation can give different results, this has an impact on the tracking. The change of the hit position and cluster width also influences the hit resolution, from which the cluster position error used in the tracking is calculated. Due to the reduced cluster width, previously merged clusters might be distinguished, which is also influencing tracking. Because of the decrease of the cluster width and the smaller sharing of the charge, the OOT PU clusters which did not pass the clustering threshold can pass it now and vice versa, again playing a role in tracking.   

% @MJ@ TODO write abut the high level quantities

%, leading to a conclusion, that the cross talk needed in simulation is compensating for other non-descriptions. To achieve a good data and simulation agreement, we have decided to plug into simulation the parameters  which describe the data the best, as presented in Section~\ref{sec:induce}. Also the cross talk could not be measured for disk and endcaps due to the geometrical restrictions of muons, but the approach of Section~\ref{sec:induce} can give results on all partitions.

%The chosen approach is to first update detector conditions, which are bad channels, noise and gains. These conditions are obtained from data and can affect the cluster width and charge. Then the matrix of the simulations with different cross talk parameters is produced and these simulations are compared to data. The agreement between data and simulations is established based on $\chi^{2}$ test of cluster width distribution compatibility. The cross talk parameters of sample with the best $\chi^{2}$ can then be injected into simulation. The cross talk parameters influence largely the cluster width and seed charge, but much less the cluster charge. If after tuning there is need for better description of cluster charge or seed charge, one of the conversion factors, either energy per electron or electron per ADC can be tuned similarly. These conversion factors are influencing the cluster width only slightly.


%TODO plug the measured cross talkinto the simulation
%TODO should I say sth more about other hits and tracks we do not use?
%In ideal case it would be best to acquire 0T collision VR data, which is not CMS priority. 

%actually the calue of cross talk we inject into MC can compensate for other effects and therefore does not have to agree with simualtions.
%\subsubsection{Cross talk tuning}

%[TODO This will come after the conditions are updated...]

  %the fit of charge distribution
  %NO.   NAME      VALUE            ERROR          SIZE      DERIVATIVE 
  % 1  Constant     3.39435e+03   7.97716e+00   4.20941e-03   2.42023e-05
  % 2  Mean        -1.82332e+00   1.70530e-01  -1.23780e-04   1.02297e-03
  % 3  Sigma        2.05388e+01   7.24243e-01   6.33711e-06   3.88536e-03

%dist<20
%TIB1
%x0 0.725506 +- 0.00815634
%x1 0.100057 +- 0.00297439
%x2 0.0371896 +- 0.00279007

%TIB2
%x0 0.795078 +- 0.0104821
%x1 0.0845729 +- 0.00427595
%x2 0.0178882 +- 0.00303061

%TOB2
%x0 0.704474 +- 0.00391049
%x1 0.112763 +- 0.00156127
%x2 0.0349995 +- 0.00117704

%TOB1
%x0 0.725506 +- 0.00815634
%x1 0.100057 +- 0.00297439
%x2 0.0371896 +- 0.00279007
