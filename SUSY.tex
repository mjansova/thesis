\clearpage

\setcounter{secnumdepth}{4}
\chapterwithnum{Supersymmetry as a possible extension of the standard model~\label{sec:SUSYch}}
\setcounter{secnumdepth}{4}

In this chapter the basics of the standard model are discussed. In section~\ref{sec:SM} the description of the elementary particles and interactions is presented as well as an idea on how the standard model was derived. This section ends with a discussion of several shortcomings of the standard model, which motivate physicists to formulate beyond the standard model theories. Section~\ref{sec:SUSY} then describes supersymmetry~(SUSY), which is a promising extension of the standard model, due to its ability to solve many of the standard model issues. Within this section are also discussed how some searches for supersymmetry can be performed and  the results of the Run~1 SUSY searches are presented. 

\section{The standard model and its shortcomings~\label{sec:SM}}

The Standard Model~(SM) of particle physics is based on Quantum Field Theory~(QFT) and gauge symmetries~\cite{9783527406012}. It was developed during 1960s, but it was only in 2012 that the last predicted particle of the SM, the Higgs boson, was discovered by the ATLAS and CMS collaborations~\cite{Chatrchyan:2012xdj, Aad:2012tfa}. The SM describes all fundamental interactions (except gravity), namely the electromagnetic, weak, and strong interactions, and characterizes all known elementary particles. 

%There are two kinds of particles in the SM, fermions with half-integer spin and bosons with integer spin. TODO do not forget

The interactions in the SM are mediated via an exchange of (gauge) bosons which have an integer spin. There are 12 bosons with a spin of one. The mediators of the electromagnetic and weak interactions are the massless photon~$\gamma$ and the massive $W^{\pm}$ and $Z$ bosons, respectively. The $W^{\pm}$ and $Z$ bosons acquire their mass through a spontaneous breaking of the electroweak symmetry as explained later in the text. The gauge bosons of the strong interaction are eight massless gluons, each holding an unique color charge which is a combination of one color and one anticolor. The gluons are massless, indicating that the strong symmetry is unbroken. The last boson belonging to the SM is a scalar boson, the Higgs boson, which arises from the electroweak symmetry breaking. 

In addition  to bosons, the particles of matter are fermions, thus carrying half integer spin. Fermions can be divided into leptons and quarks. There are three generations of leptons formed respectively by the electron~$e$, the muon~$\mu$, and the tau~$\tau$, with their corresponding neutrinos~($\nu_{e},~\nu_{\mu},~\nu_{\tau}$). Leptons have an integer charge of $\pm e$ with $e$ being the elementary electric charge and do not carry any charge of color. Therefore interact only via electromagnetic and weak interactions as they can have non-zero quantum numbers with respect to these interactions, which are the weak isospin $(t_{3})$ and hypercharge $Y$. There are also three generations of quarks, the first one is formed by up~($u$) and down~($d$) quarks, the second by charm~($c$) and strange~($s$) quarks and the third one by top~($t$) and bottom~($b$) quarks. The $u,~c,~t$ quarks have a charge of $2/3~e$ while the $d,~s,~b$ quarks have a charge of $-1/3~e$. Each quark exists in three color versions (red, green and blue) and thus quarks can participate in the strong interaction. Quarks can hold weak isospin and hypercharge and thus they can also interact via electromagnetic and weak interactions. Due to the phenomena referred to as ``color confinement''~\cite{Alkofer:2006fu}, quarks are always bound inside a composite object called ``hadron'' and cannot be found isolated in nature. Each SM fermion has an antiparticle denoted with a bar over its symbol. The antiparticle differs from the particle by the sign of its charge and its projection of the spin to the z-axis. The overview of all SM particles can be seen in Fig.~\ref{fig:figures/SMparticles}.



    \insertFigure{figures/SMparticles} % Filename = label
                 {0.5}       % Width, in fraction of the whole page width
                 { Overview of the particles present in the standard model.}

The symmetry group of the SM is

\eq{SMgroup}
{
SU(3)_{C} \otimes SU(2)_{L} \otimes U(1)_{Y},
}
where $C$ stands for the color charge of the strong interaction, $L$ for left-handed particles, which have, unlike the right-handed ones, a non-zero weak isospin, and $Y$ represents the weak hypercharge. The SM group and the interactions originating from it, are discussed part by part in the following sections~\ref{sec:QCD} to \ref{sec:EWbreaking} built on the quantum field theory. As the standard model is a field theory which requires invariance towards certain transformations of its Lagrangian, to better understand the derivation and the features of the SM, the basics of the quantum field theory and gauge transformations are described in the following sections. Finally, the last Section~\ref{sec:shortcomings} presents the shortcomings of the SM.





%-sm intro, more about sm can be seen in this book~\cite{9783527406012} %griffiths
%-renormalizable quantumn field theory, derived from gauge symmetries ~\cite{tHooft:1971qjg, Weinberg:1967tq}
%-formulation started in 1960s, completed in 2012 with discovery of higgs boson~\cite{Chatrchyan:2012xdj, Aad:2012tfa}
%-describes all known particle and fundamental interactions (except of gravity)  
%- the SM is $SU(3)_{C} \otimes SU(2)_{L} \otimes U(1)_{Y} $ group as will be shown later
%- the SM there are matter fileds which are quark and leptons -> half spin particles - fermions. The quarks have colors and electroweak charges (intract strongly and electroweakly) while leptons have only electroweak charges (only EW interaction).
%- within sm there are three generations of leptons and three generations of quark , each quark is present in three colors. Up to know, there is no explanation why there are three lepton generations.
%-leptons have full number charge, while quarks do not (smae chareg in up or wown generation)
%-top row q = +2/3e , where e is the electron charge, 
%- bottom row q = -1/3 e.
%-interaction -> exchange of boson
%-carrier of force
% TODO The masses of the SM particles cannot be predicted, they have to be measured



\subsection{Quantum field theory and gauge symmetries~\label{sec:QFT}}

In classical mechanics, the motion of a given system can be calculated by solving the Euler-Lagrange equation~\cite{9783527411887}. This equation can be generalized in order to build a relativistic theory, in which the space and time coordinates must be treated similarly. In the relativistic case, the Euler-Lagrange equation is given by the formula

\eq{EL}
{
 \partial_{\mu}\left (\pdv{\mathcal{L}}{(\partial_{\mu}\phi_{i})}\right) = \pdv{\mathcal{L}}{\phi_{i}},
}
where the physics describing the system is contained with a Lagrangian $\mathcal{L}$ which depends on the fields $\phi_{i}$ and their space-time derivatives $\partial_{\mu}\phi_{i}$. 

In the case of a scalar (spin-0) field, the Lagrangian $\mathcal{L}_{KG}$ can be written in the form 

\eq{kglagrangian}
{
 \mathcal{L}_{KG} = \frac{1}{2}(\partial_{\mu} \phi)(\partial^{\mu} \phi) - \frac{1}{2} \left(\frac{mc}{\hbar} \right)^{2} \phi^{2},
}
where $\phi$ is a single scalar field, $m$ is its mass, $\hbar$ is the Planck constant and $c$ is the speed of light. From now on, the standard convention of $\hbar = c = 1$ is used. The scalar field Lagrangian defined in Eq.~\ref{eq:kglagrangian} and plugged into the Euler-Lagrange Eq.~\ref{eq:EL} gives the Klein-Gordon equation

\eq{kgEq}
{
 \partial_{\mu}\partial^{\mu} \phi + m^{2} \phi = 0,
}
which describes a scalar particle of mass $m$. 

The field of a half-spin particle is a four-component spinor field $\psi$. The Euler-Lagrange equation applied on the field $\bar{\psi}$ using the Lagrangian $\mathcal{L}_{D}$


\eq{dirlagrangian}
{
 \mathcal{L}_{D} = i \bar{\psi} \gamma^{\mu} \partial_{\mu} \psi -m \bar{\psi} \psi
}
where $\gamma_{\mu}$ are the Dirac matrices, gives the Dirac equation describing a half-spin particle of mass $m$:

\eq{dirEq}
{
  i \gamma^{\mu} \partial_{\mu} \psi - m \psi = 0.
}
Equations for particles with spin one can be derived similarly.

In the above Lagrangians defined by Eq.~\ref{eq:kglagrangian}~and~\ref{eq:dirlagrangian}, only non-interacting fields are present. To include interactions between fields, the impact of local and global transformations of the fields on the corresponding Lagrangian must be studied. The Dirac Lagrangian given by Eq.~\ref{eq:dirlagrangian} is invariant under the global phase transformation

\eq{globalTrans}
{
\psi \to e^{i\theta} \psi, 
}
with the phase $\theta$ being an arbitrary real number. But this Lagrangian is not invariant under the local phase transformation 

\eq{localTrans}
{
\psi \to e^{i\theta (x)} \psi,
}
where the phase $\theta(x)$ is this time dependent on the space-time coordinate $x$. To preserve the invariance of the Lagrangian of Eq.~\ref{eq:dirlagrangian}, the term 

\eq{newTerm}
{
-(q\bar{\psi}\gamma^{\mu}\psi)A_{\mu}, 
}
with the field $A_{\mu}$ which transforms as 

\eq{transfrom}
{
A_{\mu} \to A_{\mu} + \partial_{\mu} \lambda
}
can be added to the Dirac Lagrangian. The field $A_{\mu}$ is a new vector~(spin-1) field. To obtain the full Lagrangian, the free field Lagrangian for the vector field $A_{\mu}$ must also be added to the Dirac Lagrangian in Eq.~\ref{eq:dirlagrangian}. The summed Lagrangian is locally invariant only in the case when the field $A_{\mu}$ is massless and therefore the term for the free field Lagrangian of the field $A_{\mu}$ can be written as

\eq{freeA}
{
    \frac{-1}{4}F^{\mu\nu}F_{\mu\nu}, ~\mathrm{where}~ F^{\mu\nu} \equiv \partial^{\mu}A^{\nu} - \partial^{\nu}A^{\mu}.
} 
The total Lagrangian is then

\eq{qedL}
{
    i \bar{\psi} \gamma^{\mu} \partial_{\mu} \psi -m \bar{\psi} \psi -  \frac{1}{4}F^{\mu\nu}F_{\mu\nu} -(q\bar{\psi}\gamma^{\mu}\psi)A_{\mu},
} 
which generates the quantum electrodynamics~(QED), where the field $A_{\mu}$ corresponds to the electromagnetic potential.

The global transformation of a field $\psi$ can be understood as the multiplication of this field by an unitary matrix $U$ ($\psi \to U \psi$). In the given example of quantum electrodynamics, the size of matrix is $1 \times 1$ and therefore the symmetry of this theory is referred to as ``$U(1)$ gauge invariance'' as the group of all such matrices is $U(1)$. A similar strategy of a local phase invariance of the Lagrangian can be applied on other groups, which was found to be the way how to generate the standard model.

\subsection{The electroweak interaction}

In 1954, Yang and Mills~\cite{Yang:1954ek} applied local invariance on the $SU(2)$ group to describe the weak interaction and later Glashow, Salam and Weinberg~\cite{Glashow:1961tr, Salam:1968rm, Weinberg:1967tq} showed, that if the group $SU(2) \otimes U(1)$ is considered, the weak and electromagnetic interactions can be unified. Moreover they divided the left and right chiral components of the fermion fields. The doublet $\Psi_L$ is composed by two left-handed fermion field spinors, left-handed charged lepton and its corresonding neutrino. The singlet $\Psi_R$ consists of one right-handed fermion field spinor. The locally invariant Lagrangian of the electroweak~(EW) interaction (without symmetry breaking) can be written as following

\eq{EWlagrangian}
{
\mathcal{L}_{EW} = - \frac{1}{4} \sum_{a=1}^{3} F_{\mu\nu}^{a} F^{a\mu\nu} - \frac{1}{4} B_{\mu\nu}B^{\mu\nu} +  i \bar{\Psi_L} \gamma^{\mu} D_{\mu} \Psi_{L} +  i \bar{\Psi_R} \gamma^{\mu} D_{\mu}  \Psi_{R},
}
where $\Psi_{R}~(\Psi_{L})$ is the right-handed (left-handed) component of the fermion fields and $D_{\mu}$ is the covariant derivative. The tensors $F_{\mu\nu}$ are composed of the vector fields $W^{a}_{\mu}$ and their derivatives, and the tensor $B_{\mu\nu}$ is composed of derivatives of the vector field $B_{\mu}$, similarly as shown in Eq.~\ref{eq:freeA}.

In the case of the $SU(2)$ group the covariant derivative $D_{\mu}$ is 

\eq{weakCovariant}
{
   D_{\mu} = \partial_{\mu} - ig\sum_{a=1}^{3}t^{a}W_{\mu}^{a},~\mathrm{with}~a=1,2,3.
}
The matrices $t^{a}$ are the generators of the $SU(2)$ group composed by Pauli matrices and $g$ is a constant. The $t_{3}$ component is called the weak isospin. The covariant derivative $D_{\mu}$ for the $U(1)$ group is

\eq{weakCovariant}
{
   D_{\mu} = \partial_{\mu} - ig'YB_{\mu},
}
where $Y$ is the weak hypercharge and $g'$ is a constant. The charge $Q$ of a particle is then given by the relation between its weak isospin and hypercharge: $Q= t_{3} + \frac{1}{2}Y$.


The group of the electroweak interaction is often denoted as $SU(2)_{L} \otimes U(1)_{Y}$, where $L$ is related to the difference of behavior between left-handed (doublet) and right-handed (singlet) fields w.r.t. the weak interaction. This group produces two massless gauge fields $W^{1}$ and $W^2$ which mix and create the $W^{+}$ and $W^{-}$ bosons. These bosons interact only with the left-handed components of the fermion field (maximum parity violation). The remaining $W^{3}$ and $B$ gauge fields interact with both the left- and right-handed fermions and they mix into the $Z$ boson and the photon $\gamma$. As mentioned previously all these bosons have to be massless in order to preserve gauge invariance, but the $W^{\pm}$ and $Z$ bosons were expected to be massive because the weak interaction is a short range interaction. Later, the discovery of the $W_{\pm}$ and $Z$ bosons confirmed that they are massive and have mass of 80~GeV and 91~GeV, respectively. Because these bosons are massive, the electroweak symmetry must be broken. It is also important to note that the EW Lagrangian gives a maximum parity violation for neutrinos because only the left component of the neutrino field exists.

%The electric charge can be is $e= g\mathrm{sin}\theta_{W} = g'\mathrm{cos}\theta_{W} $, where $\theta_{W}$ is Weinberg mixing angle which was experimentally measured to be of around $30^{\circ}$.
%The weak isospinhas only non-zero value for left-handed components.
%-isospin, hypercharge TODO
%-no right neutrino  TODO

\subsection{Quantum Chromodynamics}

The theory of strong interaction, called Quantum Chromodynamics~(QCD), is based on the $SU(3)$ group. The corresponding Lagrangian of QCD is

\eq{QCDlagrangian}
{
\mathcal{L}_{QCD} = -\frac{1}{4} \sum_{a=1}^{8} F_{\mu \nu}^{a} F^{a \mu \nu} + \sum_{j=1}^{n_f} \bar{q}_{j}(i D_{\mu}\gamma^{\mu} -m_{j})q_{j} ,
}
where the quark fields $q_{j}$ are summed over the number of different quark flavors $n_{f}$, and $m_{j}$ are the associated masses. A quark field $q_{j}$ is composed of three quark spinors, one for each color. The tensors $F_{\mu \nu}$ are a combination of gluon fields $g_{\mu}^{a}$ and their derivatives. The index $a$ runs over eight independent color charges of gluons. The covariant derivative in this case is

\eq{QCDdervative}
{
   D_{\mu} = \partial_{\mu} - i\sqrt{4 \pi \alpha_{s}} \sum_{a=1}^{8} T^{a} g_{\mu}^{a},~a=1,...,8 , 
}
where $\alpha_{s}$ is the strong coupling constant and $T^{a}$ are the generators represented by the Gell-Mann matrices. 

%Particles which can interact via strong interaction must carry a color charge, the quarks carry the red, green or blue color, while in case of gluons there are eight different combinations of one color and one anticolor.
The coupling of colored objects is weak at short distances~(asymptotic freedom)~\cite{Gross:1973id}, but it grows with distance~(confinement)~\cite{Wilson:1974sk}. Therefore colored objects always have to be bound inside colorless hadrons, where they are quasi-free, and can never be observed separately. There are two kinds of hadrons, the baryons holding three quarks of different colors and the mesons composed of two quarks, one of a certain color and the other of the corresponding anticolor.

%-asymptotic freedom - the coupling depends on the distance, , it is very weak at short distances (asymtotic freedom) nut grows in distance (confinment - bound hadron states)
%-confinment -> color must remain neutral , not possinle to separate individual quarks and gluons, always are bound in colorless hadrons - baryons of mesons.
%-hadronization - formation of colorless objects (say more) 
%-color

%INTERACTIONS
%-sm lagrangian (CERN-thesis-2017-005) -> without higgs
%-interactions (electroweak, QCD)
%TODO -19 free parameters (nine fermion masses, one scalar mass, three coupling parameters, four quark mixing parameters, higgs vacuum expectation value, strong cp violating phase)
%TODO -perturbative theory (LO, NLO)



\subsection{The electroweak symmetry breaking~\label{sec:EWbreaking}}

As discussed, the $W^{\pm}$ and $Z$ bosons are massive, but the mass term for these bosons cannot be incorporated into the Lagrangian of the electroweak interaction, because it would break the invariance of the Lagrangian under a local phase transformation. This problem is solved by the ``Brout-Englert-Higgs mechanism''~(BEH), based on a phenomenon referred to as ``spontaneous symmetry breaking'' of the SU(2) symmetry~\cite{Englert:1964et, Guralnik:1964eu}. The BEH mechanism introduces a new scalar doublet field $\Phi$, defined as

\eq{doubletHiggs}
{
    \Phi = \binom{\phi^{+}}{\phi^{0}}.
}
The Lagrangian for this field and its interactions can be expressed as 

\eq{lagHiggs}
{
    \mathcal{L}_{Higgs} =  (D_{\mu}\Phi)^{\dagger} (D^{\mu}\Phi) - V(\Phi^{\dagger}\Phi),
}
where $V(\Phi^{\dagger}\Phi)$ is the Higgs potential and the covariant derivative $D_{\mu}$ is

\eq{higgsCovariant}
{
   D_{\mu} = \partial_{\mu} - ig\sum_{a=1}^{3}t^{a}W_{\mu}^{a} - ig'YB_{\mu},~\mathrm{with}~a=1,2,3.
}

The Higgs potential $V(\Phi^{\dagger}\Phi)$ is defined by the equation

\eq{potHiggs}
{
    V(\Phi^{\dagger}\Phi) =  - \frac{1}{2} \mu^{2}\Phi^{\dagger}\Phi + \frac{1}{4} \lambda(\Phi^{\dagger}\Phi)^{2},
}
where $\mu$ and $\lambda$ are real parameters. Because both $\mu^{2}$ and $\lambda$ are positive numbers, the potential $V(\Phi^{\dagger}\Phi)$ takes the shape of a ``Mexican hat'', as shown in Fig.~\ref{fig:figures/mexicanHat}. The shape of the potential is such that the value of field $\Phi$ at the ground state, i.e. the vacuum expectation value~(VEV) of the field $\Phi$, is non-zero. The ground state is degenerated and can be chosen to be

    \insertFigure{figures/mexicanHat} % Filename = label
                 {0.4}       % Width, in fraction of the whole page width
                 { The shape of the Higgs potential $V$ for the complex field $\Phi$ with positive values of real parameters $\mu^{2}$ and $\lambda$. The parameter $v$ denotes the vacuum expectation value.}

\eq{solutionHiggs}
{
    \langle 0 | \Phi | 0 \rangle = \frac{1}{\sqrt{2}}\binom{0}{v} ,
}
with

\eq{vDef}
{
v = \sqrt{\frac{\mu^{2}}{\lambda}}
}
being the ground state energy of the field $\Phi$. Then the excitation of the field $\Phi$ can be written as follows

\eq{solutionHiggs2}
{
    \Phi = \frac{1}{\sqrt{2}}\binom{0}{v+H},
}
where $H$ is the scalar field called the Higgs boson. The vector bosons $W^{\pm}$ and $Z$ become massive via the interaction with the Higgs field present in the first term of the Higgs Lagrangian of Eq.~\ref{eq:lagHiggs}. The masses of the fermions can be also generated via the interaction of the fermion field  $\Psi$ with the Higgs boson $H$ by adding  to the Lagrangian of the SM Yukawa terms of type $\lambda_{Y} \bar{\Psi}_{L} \Phi \Psi_{R} + h.c$ with $\lambda_{Y}$ representing the Yukawa coupling proportional to the mass of the fermion. The full Lagrangian of the standard model in the above defined notation can then be written as

\eq{SMlag} 
{
 \mathcal{L}_{SM} = \mathcal{L}_{QCD} + \mathcal{L}_{EW} +\mathcal{L}_{Higgs} + \mathcal{L}_{Yukawa}.
} 

It can be noticed  that the Yukawa interaction with the Higgs boson flips the chirality of a fermion from left to right and vice versa. Therefore the masses of neutrinos cannot be generated in this way, as there are only left-handed neutrinos in the SM. The parameters $\mu$ and $\lambda$ of the Higgs potential are not predicted by the standard model, i.e. they are free parameters which are measured experimentally. The mass~($m_{H,0}$) of the Higgs boson at tree level depends on the parameter $\mu$: 

\eq{treeMass}
{
m_{H,0} = \sqrt{2}\mu,
}
and therefore is as well not predicted by the SM. 

%Fermions get mass via interaction with $\Phi$ filed~\cite{Weinberg:1967tq}
%-hiigs flips the chirality, this is why neurino cannot have a mass within SM
%ELECTROWEAK symmetry breaking: TODO strat here
%Yang-Mills~\cite{Yang:1954ek} -> nonabelian gauge theory
%-new scalar field predicted by Higgs Englert and Brout in 1964~\cite{Higgs:1964pj, Englert:1964et}
%-discovery in 2012 by CMS and ATLAS
%-> in this theory gauge bosons are masless - gauge symmetry do not alow mass terms in lagrangian
%-based on spontaneous symmetry breaking principles -> apperance og goldstone~\cite{Goldstone:1961eq} bosons (one for each generator of broken symmetry?!) , goldosnes are masless spin-0
%-> do not speak about goldstone
%-non-zero ground state - vev
%-degenerated state -> infinite nr of minima on on circle of phi(1) and phi(2)- complex field -> this gives us a chance to fix phi as we want. 
%-1960s - the goldstone bosons cancel and give mass to other bosons -> generation of mass for Ws and Z bosons -> Hoggs mechanism ~\cite{Englert:1964et, Higgs:1964ia, Guralnik:1964eu,}% Higgs:1966ev}
%-parameter v is vacum expectation value - v =sqrt(-mh2/lambda)
%-Goldstone theorem -> masless states -> masless states are absorbed by the evctor bosnons
%-lambda and higgs mass must be experimentally measured
%- v and mH value? (Hoss)
%Higgs lagrangian
% TODO \section{Feynman diagrams} ?
%+perturbative theory

\subsection{Limitations of the standard model~\label{sec:shortcomings}}

Even though the standard model proved to be very successful in describing and predicting results of a large part of high energy physics experiments, phenomena which cannot be explained by the SM are observed as well~\cite{9783527406012}. For this reason it is widely believed that the standard model is an effective theory  of a more fundamental one. Before discussing the possible extensions of the SM in Section~\ref{sec:SUSY}, some of the shortcomings and open questions of the standard model are first briefly described.


\textbf{The naturalness problem}

As mentioned in Section~\ref{sec:EWbreaking}, the mass of the Higgs boson at the tree level is defined by Eq.~\ref{eq:treeMass}, but this mass must be corrected for contributions of virtual particles. An example of a virtual fermion loop contributing to the Higgs mass is shown in Fig.~\ref{fig:figures/fermionCorr}. The mass of the Higgs boson can be thereby decomposed in $m_{H,0}$ as:

\eq{HiggsMass}
{
m_{H}^{2} \approx m_{H, 0}^{2} + \Delta m_{H}^{2},
}
where $\Delta m_{H}^{2}$ is the correction from the virtual fermion loops. This correction is quadratically divergent and depends quadratically on the fermion masses due to their Yukawa couplings. Therefore the largest correction to the Higgs mass comes from the virtual top quarks which are the heaviest fermions. More precisely, the correction $\Delta m_{H}^{2}$ follows the relation

\eq{HiggsMassT2}
{
\Delta m_{H}^{2} \propto m_{f}^{2} \Lambda^2,
}
where $m_f$ is the fermion mass and $\Lambda$ is a cutoff on the momentum of the considered virtual particle. This cutoff is expressing up to which scale the standard model is valid and is usually taken to be the Planck mass $m_{P} \sim  10^{19}$~GeV. Knowing the term $\Delta m_{H}^{2}$, the Higgs mass can be written as

\eq{HiggsTuning}
{
m_{H}^{2} \sim m_{H, 0}^{2} + k ~m_{P}^{2},
}
where $k$ includes constants and SM couplings. 

    \insertFigure{figures/fermionCorr} % Filename = label
                 {0.3}       % Width, in fraction of the whole page width
                 { A virtual fermion contributing to the mass of the Higgs boson. }

The mass of the Higgs boson was experimentally measured to be of 125~GeV~\cite{Chatrchyan:2012xdj, Aad:2012tfa}, which is orders of magnitude lower than Planck mass. This mismatch between the order of magnitude of the Higgs and the Planck masses, for which there is no physical reason, is referred to as the ``hierarchy problem''. Moreover to obtain this relatively small mass of the Higgs boson, there must be large cancellation between the two terms $m_{H, 0}^{2}$ and $k ~m_{P}$  in Eq.~\ref{eq:HiggsTuning}. The cancellation of the terms has to happen in around thirty orders of magnitude. Such fine-tuning is not natural and this problem of fine-tuning is referred to as the ``naturalness problem''.

%corrwctions to Higgs boson mass
%to compute cross section, all quantum loop corrections has to be taken into account
%fermions an vector boson masses proctected from diverging by mechanism within the SM
%but no mechanism for Higgs mass: $mh^2~ mh0^2+k mPlanck^2 $ - parameters mh0, k and mPlanck a priori unrelated. But these parameters must be fine tuned in order to obtain mass of Higgs (mh<<mPlanck) -> not natural
%called hierarchy problem - no reason to expect a large hierarchy between electroweak scale and planck scale
%-34 digits
%-picture higgs loop

\textbf{Dark matter and dark energy}

Cosmological observations suggest that the ordinary matter and energy, described by the standard model, account for only $\sim 5\%$ of the total mass~(energy) of the universe~\cite{Bertone:2004pz, Bennett:2012zja}. The remaining $\sim$95\% is divided between the dark matter~($\sim$27\%) and dark energy~($\sim$68\%).

Dark matter has been observed only indirectly as it does not emit any radiation. Therefore if it is asocciated to an elementary particle, it should have no color or electric charge. The first observation supporting the dark matter existence came from the measurement of the rotation curves of galaxies~\cite{Zwicky:1937zza, Rubin:1980zd}. These curves show the dependency of the star orbital velocity on the distance of the star from the center of galaxy. The rotation curves can be theoretically computed and it was found out that the measured and theoretical curves agree at short distances. With increasing the distance the observed curves remain constant, while theoretically they are expected to decrease~\cite{Bertone:2004pz}. This phenomenon can be explained by the presence of a halo of new particles which interact by gravitational force. It was also found out that these particles must be stable and non-relativistic~(cold). The requirement that the dark matter must be cold comes from the observations of the large scale distribution of the dark matter in the universe. However it was shortly realized that there is no candidate for dark matter within the standard model.

Dark energy arises from the need to introduce a cosmological constant in the Einstein equation~\cite{Sami:2009jx}. Without it, it would not be possible to explain the accelerated expansion of the universe. Dark energy can be interpreted as a vacuum energy, but there is a mismatch of $\sim$120 orders of magnitude between the vacuum energy estimation from the cosmological constant and quantum field theory calculations.

%-5 percent of baryonic matter
%-27 percent of dark matter
%-68 dark energy
%-measuremnt of rotation curves of galaxies - first dark matter hypothesis
%-gravitational interaction, but not electromagnetic -> dark matter
%-from observations several constraints on dark matter - not short -lived and not baruonic, gravitationally interacting, low kinetic energy (cold -> it cannot be neutrino)
%-> no good candidate within the SM
%From cosmological observations we expect dark matter mass of order of 100~GeV
%- how many percent?
%-microwawe background
%-no radiation of DM -> no collor, no electric charge

%3)Dark energy
%-cosmological constant (lambda) in einsteins equation necessary to explain the observed expansion of universe
%-> cosmologica constant can be interpreted as a vacuum energy

\textbf{Other shortcomings of the standard model}

Another problem of the SM is the asymmetry between matter and antimatter. In our universe there is an abundance of matter, even though the matter is supposed to have been produced in the same amount as antimatter during the big bang.  Weak interaction violates the combined charge conjugation and parity~(CP) symmetry~\cite{Kobayashi:1973fv}, which could lead to an asymmetry between matter and antimatter. But the known sources of the CP violation cannot describe so large asymmetry and therefore there is no mechanism within the SM, which could explain the observed asymmetry between matter and antimatter.

Then, as mentioned, there are only left-handed neutrinos in the SM and therefore their masses cannot be generated via the interaction with the Higgs boson, which flips left-handed fermions to right-handed ones. But neutrino oscillations have been observed~\cite{Fukuda:1998mi, Ahmad:2001an} and are only possible if neutrinos are massive. Thus one needs to introduce a mechanism to generate mass terms for neutrinos in the SM Lagrangian. However, this fact in  itself is not a conclusive argument in favor of a physics beyond the SM~(BSM).

%Another argument to try to extend the the standard model is to include the gravitational force, which is not presently part of the standard model and therefore the SM is only effective theory. 
As mentioned earlier in this chapter, the gravitation is not included in the SM. This model is based on a QFT model which is based on the special relativity. Currently, there are big efforts to combine the general relativity and quantum field theory to formulate a theory of quantum gravity.  

Finally, in the SM, the coupling constants are dependent on the energy. At higher energies the constants of weak, electromagnetic and strong interactions become of similar strengths, giving a hope for unification of these interactions at a large energy scale. There is no certainty that the three forces are unified, but the idea of grand unification is historically based on the success or previous unifications,  such as of electricity and magnetism, or electromagnetic and weak interactions, and  therefore unification of the three forces is expected. A requirement for the unification of the interactions is the convergence of the coupling constants at a large energy scale. This unification is not achieved within the SM and would request a theory beyond the SM.
 
%4)Matter-antimatter assymetry
%-matter and antimatter should be produced in smae amount at big bang
%-but our world dominated by matter
%5)Neutrino masses
%-neutrinos oscialte from one flavour to other -> this can only happen when neutrinos are masive and have different mass states than flavour states

%6)Strong CP phase
%strong QCD lagrangian introducing the phase theta - close to zero, despite the theoreticla arguments that it should not be like this

%7)Quantum gravity
%-gravity not described by SM
%-desired to unify general relativity with QFT
%8)Unification of forces
%-possibility to unify all interactions
%-> of couplingconstants

%9)open questions
%-in SM large differences between quarks
%-why there should be three fermion families

%remaining stuff:
%-19 free parameters (nine fermion masses, one scalar mass, three coupling parameters, four quark mixing parameters, higgs vacuum expectation value, strong cp violating phase)
%-perturbative theory (LO, NLO)

\section{Supersymmetry~\label{sec:SUSY}}

To address the mentioned shortcomings of the standard model, many extensions of the SM were proposed over the past decades. In general there are several possibilities how to formulate tehories beyond the standard model. The extension can be achieved by, for example, adding ``extra dimensions''~\cite{Patrignani:2016xqp}, adding new symmetries, postulating new particles or proposing new interactions. An example of a theory adding a new symmetry is SUperSYmmetry~(SUSY)~\cite{Martin:1997ns}, which became very popular due to its capability to solve many issues of the SM. SUSY started to be developed in the 1970's around an idea of introducing a spin symmetry relating fermions and bosons.
  
%- said before: SM works fine, but we need to extend it -> we can add either additional symetries, space-time dimensions or field content
%- one of the possibility how to extend is susy 
%- it addresses many issues of SM

Supersymmetry introduces the symmetry operator $Q$, which acts on fermions~$f$ and bosons~$b$ in the following way:

\eq{SUSYop}
{
Q \mid f \rangle \to \mid b \rangle ,
}
\eq{SUSYop2}
{ 
Q \mid b \rangle \to \mid f \rangle .
}
The operator $Q$ changes the spin of the particle by $1/2$, therefore transforms a fermion to a boson and vice versa, but does not change any other quantum number or particle property. Within this symmetry $Q$, each SM fermion has a bosonic SUSY partner with the same quantum numbers except of the spin, and similarly each SM boson has an associated fermionic SUSY partner. The supersymmetric partners of the SM particles are referred to as ``sparticles''. To build a theory which is able to reproduce the standard model, the operator $Q$ must satisfy the following (anti)commutator relations~\cite{Haag:1974qh, Coleman:1967ad}:

\eq{comutators}
{
\{Q,Q^{\dagger}\} = P^{\mu}, \; \{Q,Q\} =\{Q^{\dagger},Q^{\dagger}\}= 0, \; [P^{\mu}, Q] = [P^{\mu}, Q^{\dagger}] = 0,
}
where $P^{\mu}$ is the four-momentum operator. It can be noticed that $-P^{2}$, which is the mass-squared operator, commutes with both $Q$ and $Q^{\dagger}$ and therefore a particle and its SUSY partner have the same mass. But as no SUSY partners have been observed, this symmetry must be broken. Many symmetry breaking scenarios can be built, leading to many possible realizations of the supersymmetric theory. 

The naming convention for SUSY particles is to add the prefix ``s'' to the SUSY partners of fermions, therefore ``sfermions'' are bosons. The SUSY partners to bosons get the suffix ``ino'', for example the SUSY partner of gluon is the gluino, which is a fermion. In addition, the symbols of the superpartners have a tilde.

There are many realizations of supersymmetry, but further in the text only the minimal supersymmetric standard model, which is the most popular one, is considered.


%------------------------------------------------------------------------------------------------------------------------------------------
%-around 70's
%-Golfand and Likhtman -> new symmetry Q -> Q|f> -> |b>; Q|b> -> |f> (transformation) -> later Haag, Lopuszanski and Sohnius said that such symmetry corresponds to supersymmetry
%-to each fermion a boson  with same quantum numbers (except of spin)
%-particles in supermulitples, where there is same number of fermionic and boisonic degrees of freedom
%-mass degeneration of particles in supermultiplet (from commutation relation of Q)
%-particles in supermultiplet have same quantum, numbers under the SU(3)xSU(2)xU(1) transformation
%-supermultiplet SM particle + susy partner, just differening by spin (1/2) -> bit more complicated I guess
%-parters of fermions are sfermions, and partner of bosons are inos, tilde for susy particles
%-partner of praticle is superpartner and they form superfield
%-spin differs by 1/2
%-same interactions of SUSY particles as the SM ones (for example only scalar partners of left handed fermions interact with partners of W boson)
%-two SUSY Higgs boson doublets are needed (in order to keep the theory renormalizable)
%-superpartner should have the same mass -> not observed -> susy must be broken (for now we just add a term into the lagrangian)
%-motivation:
%	-solve hierarchy problem (superpartnes have equal masses and cancel the loop corrections) - in case of "soft breaking" susy prevents the quadratic divergencies and there are only logarithmic + small fine tuning
%	-> naturalness of susy related to the mass difference between particle and its superpartner (Q: then if the susy partner of eg electron is very heavy does not it induce the divergencies? )

%-susy breaking - not much known about its mechanism, there are several hypothesis (models)
%-susy solves naturalnes problem, bosons opposite sign of corerction to delta mass -> fine tuning can be removed, if the coupling constants are the same, just differ by sign (and it is actually true)
%	-> the loop diagram
%-susy soles dm candidate - lsp
%-susy solves the unification - susy modifies the energy evolution of coupling constants 

%-constraints on lsp from dm relic density~\cite{Ade:2015xua}
%-commputation relations of susy operators?
%------------------------------------------------------------------------------------------------------------------------------------------

\subsection{The minimal supersymmetric standard model}

The Minimal Supersymmetric Standard Model~(MSSM)~\cite{Martin:1997ns} is a supersymmetric extension of the SM which adds a minimum of new particles, but does not introduce any additional gauge interactions. For example, in the MSSM there are two Higgs doublets in order to avoid gauge anomalies and to be able to have Yukawa couplings of both up- and down-type quarks. Both Higgs doublets have a non-zero vacuum expectation value. The Lagrangian of the new theory can be written as

\eq{lagSUSY}
{
    \mathcal{L} =  \mathcal{L}_{MSSM} +  \mathcal{L}_{soft} = \mathcal{L}_{free} + \mathcal{L}_{int} + \mathcal{L}_{soft}    ,
}
where $\mathcal{L}_{MSSM}$ is the supersymmetric part of the Lagrangian and the term $\mathcal{L}_{soft}$ introduces the ``soft'' breaking of supersymmetry. The breaking of supersymmetry must be mild in order to end up with only a small tuning of the Higgs boson mass. 


The large advantage of supersymmetry is that, in its unbroken form, it is able to solve the hierarchy problem. The virtual correction from the fermion loop $\Delta m_{H}^{2}$ to the Higgs boson mass as shown in Fig.~\ref{fig:figures/fermionCorr2} is in the context of the MSSM proportional to

\eq{HiggsMassSUSY}
{
\Delta m_{H}^{2} \propto (m_{f}^{2} - m_{\tilde{f}}^2) \mathrm{ln} \left(\frac{\Lambda}{m_{\tilde{f}}}\right),
}
where $m_{f}$ is the mass of the fermion $f$, $m_{\tilde{f}}$ the mass of the sfermion $\tilde{f}$ and $\Lambda$ a cutoff. The $\Delta m_{H}^{2}$ term is equal to zero if the masses of the particle ($f$) and its partner ($\tilde{f}$) are the same and therefore no fine-tuning is needed. The virtual contributions to the Higgs mass in the context of the MSSM are visualized in Fig.~\ref{fig:figures/fermionCorr2}. The fermion and sfermion loop contributions are of the same value but with an opposite sign and therefore cancel. 


    \insertFigure{figures/fermionCorr2} % Filename = label
                 {0.3}       % Width, in fraction of the whole page width
                 { Virtual fermion and boson loops contributing to the mass of the Higgs boson.}

In the broken theory though the masses are not the same anymore and the term $\Delta m_{H}^{2}$ has a logarithmic divergence. Larger is the mass difference between the particle and sparticle, more fine-tuning is needed. Therefore to preserve the naturalness of the theory, it is required that SUSY is broken only slightly and that the supersymmetric partners do not have a mass hugely larger than the SM particles. Especially, as the largest virtual  contribution to the Higgs mass comes from the top quark, the mass difference between the top quark and top squark (also called ``stop'') should be reasonably small, leading to a constraint on the top squark mass to be of order of around 1~TeV~\cite{Martin:1997ns, Barbieri:1987fn, Papucci:2011wy}. Newly explored natural SUSY models even permit the stop mass to go up to 3~TeV~\cite{Baer:2016bwh}.


The spectrum of the SUSY partners can be seen in Tab.~\ref{tab:SUSYspectrum} both as gauge and mass eigenstates. Indeed, the left and right components of the third generation of squarks, associated to the left and right chiral components of the fermion fields, mix together due to the large mass of the third generation of quarks and therefore large Yukawa couplings which are proportional to $m/v$, where $m$ is the fermion mass and $v = \mu/\sqrt{\lambda}$ is related to the Higgs potential. So  for example the $\tilde{t}_{L}$ and  $\tilde{t}_{R}$ stop eigenstates mix into the mass eigenstates  $\tilde{t}_{1}$ and  $\tilde{t}_{2}$,  $\tilde{t}_{1}$ being the lightest one. The same happens for the tau lepton.

%The mass eigenstates can be as well found in Tab.~\ref{tab:SUSYspectrum}. 

As a consequence of the electroweak symmetry breaking, the charged fields $\tilde{W}_{1,2}$ mix into the positive and negative winos~($\tilde{W}^{\pm}$) and the neutral fields $\tilde{B}_{0}$ and $\tilde{W}_{3}$ mix into the zino~($\tilde{Z}$) and photino ($\tilde{\gamma}$). The higgsinos mix with these flavor eigenstates of the SUSY partners of the $SU(2) \otimes U(1)$ gauge bosons to give rise to the gaugino mass eigenstates. Thus the two charged higgsinos~($\tilde{H}_{u}^{+},~\tilde{H}_{d}^{-}$) mix with $\tilde{W}^{\pm}$ to form four charginos~($\tilde{\chi}_{1,2}^{\pm}$). The four neutralinos $\tilde{\chi}_{1,2,3,4}^{0}$ are a mix of the zino, photino and the two neutral higgsinos~($\tilde{H}_{u}^{0},~\tilde{H}_{d}^{0}$). 


\begin{table}[h]
\begin{center}
\begin{tabular}{|c|c|c|c|}
\hline
Names & Spin  & Gauge Eigenstates & Mass Eigenstates  \\
\hline
      \rule{0pt}{3ex}   &   & $\tilde{u}_{L}~\tilde{u}_{R}~\tilde{d}_{L}~\tilde{d}_{R}$  & (same) \\
squarks & 0 & $\tilde{s}_{L}~\tilde{s}_{R}~\tilde{c}_{L}~\tilde{c}_{R}$  & (same) \\
        &   & $\tilde{t}_{L}~\tilde{t}_{R}~\tilde{b}_{L}~\tilde{b}_{R}$  & $\tilde{t}_{1}~\tilde{t}_{2}~\tilde{b}_{1}~\tilde{b}_{2}$ \\
\hline
         &   & $\tilde{e}_{L}~\tilde{e}_{R}~\tilde{\nu}_{e}$  & (same) \\
sleptons & 0 & $\tilde{\mu}_{L}~\tilde{\mu}_{R}~\tilde{\nu}_{\mu}$  & (same) \\
         &   & $\tilde{\tau}_{L}~\tilde{\tau}_{R}~\tilde{\nu}_{\tau}$  & $\tilde{\tau}_{1}~\tilde{\tau}_{2}~\tilde{\nu}_{\tau}$ \\
\hline
neutralinos \rule{0pt}{3ex} & 1/2 & $\tilde{B}_{0}~\tilde{W}_{3}~\tilde{H}_{u}^{0}~\tilde{H}_{d}^{0}$  & $\tilde{\chi}_{1}^{0}~\tilde{\chi}_{2}^{0}~\tilde{\chi}_{3}^{0}~\tilde{\chi}_{4}^{0} $ \\
\hline
charginos  \rule{0pt}{3ex} & 1/2 & $\tilde{W}_{1,2}~\tilde{H}_{u}^{+}~\tilde{H}_{d}^{-}$  & $\tilde{\chi}_{1}^{\pm}~\tilde{\chi}_{2}^{\pm} $ \\
\hline 
gluino & 1/2 & $\tilde{g}$  & (same) \\
\hline
gravitino  \rule{0pt}{3ex} & 3/2 & $\tilde{G}$  & (same) \\
\hline
\end{tabular}
\caption[Table caption text]{The gauge and mass eigenstates of the sparticles~\cite{Martin:1997ns}. The subscripts $L,R$ denote the left and right chiral components of the sparticle fields. The mixing of the first two generations of sfermions is considered to be negligible. }
\label{tab:SUSYspectrum}
\end{center}
\end{table}

The MSSM brings 105 new free parameters on top of the 19 free parameters in the standard model. Most of these parameters come from the supersymmetry breaking part of the Lagrangian. It is not possible to search for SUSY in such a large parameter-space and therefore the collider experiments usually search for SUSY in the context of simplified models, referred to as ``Simplified Model Spectra'', which reduce considerably the number of free parameters. More details on these simplified models are given in Section~\ref{sec:SMS}. Another option to reduce the number of parameters is to apply well motivated assumptions. For example the pMSSM allows to reduce the number of parameters to 19, defined at the electroweak scale~\cite{Berger:2008cq}.

The left ($\tilde{t}_{L}$) and right ($\tilde{t}_{R}$) stop eingenstates mix into the mass eignestates ($\tilde{t}_{1},~\tilde{t}_{2}$). In the phenomenological MSSM, which reduces number of free parameters by assuming values of majority of them, the masses of the $\tilde{t}_{1}$ and $\tilde{t}_{2}$ can be determined from the mass matrix $M_{\tilde{t}}$ in the ($\tilde{t}_{L},~\tilde{t}_{R}$) basis~\cite{Passehr:2017ufr}. The mass matrix $M_{\tilde{t}}$ can be written as

%from 100 to 19 (or sth like this)

\eq{stopM}
{
M_{\tilde{t}} =
\begin{pmatrix}
m_{\tilde{t}_{L}}^{2} + m_{t}^{2} + \Delta_{\tilde{u}_{L}} & m_t(A_{t}^{*} - \mu ~\cot \beta) \\
m_t(A_{t} + \mu ~ \cot \beta) &  m_{\tilde{t}_{R}}^{2} + m_{t}^{2} + \Delta_{\tilde{u}_{R}}  \\
\end{pmatrix}
,
}
where $m_{\tilde{t}_{L}}$ and $m_{\tilde{t}_{R}}$ are the masses of the left and right component of the stop field, respectively, $m_{t}$ is the top quark mass,  $\mu$ is the Higgs mass parameter, $\beta$ is the ratio of the Higgs VEVs, $A_{t}$ is the trilinear coupling and the $\Delta$ terms are coming from the quadrilinear interactions of the squarks and the  Higgs boson which represent small contributions compared to the other terms. 


Through the radiative corrections, the Higgs mass is sensitive to the stop mass. The Fig.~\ref{fig:figures/plotsstophiggs} shows the dependence of the Higgs mass on two parameters, $m_{\tilde{t}_{R}}$ and $X_{t}$ which is defined as
\eq{xt}
{
X_{t} \equiv A_{t}+ \mu ~ \cot \beta .
}
The other pMSSM parameters are fixed to the physically motivated values. Therefore the observed value of the Higgs mass puts a constraint on the stop parameters. It can be noticed, that even when masses of  $\tilde{t}_{L}$ and $\tilde{t}_{R}$ are around 3~TeV a configuration of parameters being compatible with the observed Higgs mass can be found. 

 %specifically the parameters present in the Eq.~\ref{eq:stopM} as shown in Fig.~\ref{fig:figures/plotsstophiggs}, where the dependence of the Higss boson mass on the $X_{t}$ and $m_{\tilde{t}_{R}}$ parameters is depicted. 

%%%%%%%%%%%%%%%%%%%%%%%%%%%%%%%%%%
%%%%%%%%%%%%%%%%%%%55STARTHERE

    \insertFigure{figures/plotsstophiggs} % Filename = label
                 {0.7}       % Width, in fraction of the whole page width
                 {The predicted mass of the lightest Higgs boson $M_{h}$ in dependence on the ratio of the stop parameters $X_{t}$ and $m_{\tilde{t}_{R}}$ for three different values of $m_{\tilde{t}_{R}}$. For the computation the parameters were fixed to $m_{\tilde{t}_{L}} = m_{\tilde{t}_{R}}+100$~GeV, $A_{t}= \abs{1.3 m_{\tilde{t}_{R}} + \mu ~ \cot \beta} $, $\mu$=1~TeV and $\tan \beta = 50$. The gray horizontal line corresponds to the observed Higgs mass.  More details can be seen in~\cite{Passehr:2017ufr}.  } % Caption

%The Ref.~\cite{Baer:2016bwh} shows that the mass of the lighter stop $(\tilde{t}_{1})$ up to the 3~TeV can lead to the natural SUSY  and examines the stop discovery and exclusion potential for current and future colliders as depicted in Fig.~\ref{collidersstop}. The figure reveals, that the stop mass of 3~TeV cannot be probed at current center-of-mass energies of 13~TeV and machine delivering collisions at center-of-mass energy of 33~TeV would be needed.  [TODO: maybe this discussion should not really be here, what do you think?]

%    \insertFigure{figures/collidersstop} % Filename = label
%                 {0.45}       % Width, in fraction of the whole page width
%                 { Potential for discovery (5$\sigma$) and exclusion (95\%) of the current LHC with center-of-mass energy of 13~TeV, HL-LHC with center-of-mass energy of 13~TeV, HE-LHC with center-of-mass energy of 33~TeV,  and pp collider with center-of-mass energy of 100~TeV given the stop mass $\tilde{t}_{1}$~\cite{Baer:2016bwh}.  }


%The chargino and neutralino mass eigenstates can be as well obtained from the mass matrices~\cite{Martin:1997ns}. 
The Ref.~\cite{Gunion:1987yh} discusses that in the case when Higgs parameter $\mu$ is much smaller than the bino and wino masses, the lightest chargino and neutralino are almost mass degenerate. The results of the thesis~\cite{Duarte:2017fkm} show that if $\mu = 200$~GeV, $\tan \beta =10$ and the bino mass is of 3~TeV, the mass difference between the lightest chargino and neutralino is 5~GeV maximum, depending on the mass of the wino.

In the total MSSM Lagrangian, in principle there could be terms which would violate the baryon or lepton number conservation. Under such conditions, the proton could decay, what was not experimentally observed~\cite{Nishino:2009aa}. To avoid such violation, a symmetry referred to as ``R-parity''~($P_{R}$) is considered. The $P_{R}$ being defined as

\eq{Rparity}
{
P_R=(-1)^{3(B-L)+2s },
}
where $B$ is the baryon number, $L$ is the lepton number and $s$ the spin of the particle. All SM particles have $P_{R}=1$ and their SUSY partners $P_{R}=-1$. The consequence of R-parity conservation is that sparticles can be produced only in even number and each of them can decay only into an odd number of sparticles. Also the lightest supersymmetric particle~(LSP) must be stable. In that case the LSP is a dark matter candidate and it can be either the lightest neutralino, sneutrino or gravitino, depending on the specific realization of the MSSM. 

%From the cosmological observations of the relic density~\cite{Ade:2015xua}, constraint on the mass of the dark matter mass to be of order of 100~GeV was imposed.


Another advantage of the MSSM is that the energy dependence of the coupling constants of the SM interactions is modified as a result of introducing sparticles. Consequently these coupling constants can be unified at a large energy scale referred to as ``GUT scale''. Within the supersymmetry it is also possible to create models of supergravity by imposing the locality of the supersymmetry.

%\textbf{conservation of R-party by construction }
%	-> pair production of sparticles
%        -> decay only to odd nr of sparticles
%        -> LSP is stable -> dark matter candidate
%-in susy lagrangian there can be interaction between susy particles and sm particles -> there can be lepton or baryon number valiation
%-but this was restricted by SM , because proton could decay, wat was not observed~\cite{Nishino:2009aa}
%-to avoid this: R-parity requirement added:   , where B is baryon number, L is lepton numbver and s is spin of particle
%-it must be conserved
%-all susy particles negative r-parity, sm ones positive
%	-> lsp is stable (interacts only weakly - good dm candidate)
%	-> decay to odd number of susy particles
%	-> at colliders, susy particles produced in pair
%-there are rpv models, but the lsp is not stable
%LSP, unification of forces and in some conditions it can describe quantum gravity

%TODO 



%------------------------------------------------------------------------------------------------------------------------------------------
%-most used SUSY relization is MSSM -> minimal -> adding the minimum number of fields~(particles) to the SM to become supersymmetric
%- no additional gauge interactions
%-table of supermultiplets?! (CERN-THESIS-2015-390) -> these are the particles before "mixing"
%-adding sfermions and gauginos - left and right handed fermions -> e.g two selectrons
%-for Higgs more complicated- one higgsino is not enough, but second SU(2) doublet is needed to avoid a gauge anomaly. (give mass to up and down type of quarks?) -> even there must be two SM doublets
%-both higgs doublets have to have non-zero vev
%-large mixing between sfermions states (because of large Yukawa coupling which is dependent on mass?), mixing of the second and first generation smaller, du to smaller Yukawa
%-it is likely that right-handed states are lighter than left-handed
%-winos and binos mix after ew symmetry breaking -> winos, zino, photino , but they mix to give mass eingenstates
%	-4 neutralinos which are mix of neutral bino, wino and higgsinos
%	-two charginos (each can be negative or positive) - which are mix of charged winos and higgsinos
%-only gluinos do not mix to give some mass eingenstates
%-mass eigenstates do not have to be flavor eigenstates
%-combinations of electroweak gauginos and higgsions make charginos and neutralinos
%-mixing between left and right superpartners 
%-spectrum of sparticles
%-MSSM - more than 100 new parameters than in SM
%	-majority from symmetry breaking
	%-it can be constrained by using pMSSM (with 19 free parameters), where no assumptions on the breaking mechanism 
%-105 mssm + 19 sm parameters


%\textbf{pmssm}
%-> too many parameters - problem for phenomenological and experimental models
%-> pMSSM - phenomenological MSSM -> reduction of number of parameters by assuming
%	- there is no new source of CP vilation
%	-lightest neutralino is the LSP
%	-other assumptions on the sfermion masses, trilinear couplings and flavor violation
%	->reduction of parameters to 19
%		-higgsino mass parameter and pseudo-scalar higgs mass 
%		-ration of Higgs vauum expectation values
%		-soft gaugino masses (bino, wino, gluino
%		-sfermion masses
%		-trilinear couplings

\subsection{Simplified Model Spectra~\label{sec:SMS}}

The parameter-space of the MSSM can be reduced by fixing the sparticles decay modes and branching ratios and the mass hierarchy between sparticles. Such restrictions give only a limited number of possible MSSM models called ``Simplified Model Spectra''~(SMS)~\cite{Alves:2011wf, Alwall:2008ag, Chatrchyan:2013sza}.

This thesis focuses on the search for the top squark~(stop) and therefore only specificities of the stop production and its decay within the SMS are described in larger detail. The lighter stop~($\tilde{t}_{1}$) is expected to be in natural SUSY lighter than gluino and therefore the production of the stop can be mediated via gluino or a stop pair can be produced directly in the pp interaction. An example of a direct stop pair production can be seen in the left part of Fig.~\ref{fig:figures/stopProd}, the gluino mediated production is depicted in the right part of the same figure.


    \insertTwoFigures{figures/stopProd} % Filename = label
                 {figures/T2tt}
                 {figures/T5tttt} % Filename = label
                 {0.45}       % Width, in fraction of the whole page width
                 {(left) An example of diagram of direct stop pair production. In this diagram both stops decay to a top quark and an LSP. (right)  An example of diagram of gluino mediated stop production. In this case gluinos are produced in pairs and both decay to a top quark and a top squark. Both stops, as previously, decay to a top quark and an LSP. } % Caption

The stops can decay directly to a top quark and an LSP which is the lightest neutralino, or via an intermediate chargino as shown on example in Fig.~\ref{fig:figures/T6bbWW}. The mass of the chargino is fixed by the relation

\eq{charginomass}
{
m_{\tilde{\chi}_{1}^{\pm}} = x ~m_{\tilde{t}_{1}} + (1-x) ~m_{LSP},
}
where $m_{\tilde{\chi}_{1}^{\pm}}$ is the chargino mass, $m_{\tilde{t}_{1}}$ the stop mass, $m_{LSP}$ the neutralino mass and $x$ is a fixed fraction between 0 and 1, which is for CMS searches usually chosen to be 0.25, 0.5 or 0.75. The CMS also studies scenario in which one stop decays directly to the neutralino and second via the chargino. In this scenario the chargino and neutralino mass are almost degenerate, the chargino is 5~GeV heavier than the neutralino. Such model is motivated by the mentioned low mass difference between chargino and neutralino in case when bino and wino masses are larger than the parameter $\mu$.

    \insertFigure{figures/T6bbWW} % Filename = label
                 {0.45}       % Width, in fraction of the whole page width
                 {An example of diagram with an intermediate chargino in the stop decay. In this diagram both stops decay to a bottom quark and a chargino. Both charginos then decay to a W boson and an LSP.   } % Caption

The directly produced stop pair decays into two b-quarks, two W bosons and two neutralinos. The neutralinos interact only weakly and escape the detector unmeasured. Although undetected particles can be spotted by the presence of the missing transverse energy, which can be especially large in case of final states with more neutralino(s) and neutrino(s). Therefore SUSY signal can have an unique experimental signature compared to the standard model processes.

In general, the results of searches are interpreted in the context of the SMS. If no excess from the SM is observed, the exclusion limits on the sparticle masses are derived in the plane of the two of them, in the further described case in the plane of the stop mass versus the LSP mass.

%lhc reference~\cite{Alves:2011wf, Alwall:2008ag}
%cms reference~\cite{Chatrchyan:2013sza}
%-considering small number os sparticles -> others are too heavy -> considering mass hierarchy and branching ratios
%-limits in planes of sparticles
%-part of the parameter space of the MSSM
%-gluinos heavier than the lighter stop
%-described set of particles, their possible production and decay chain
%-in simplified models considering only production process of the primary particles?
%-each aprticle diracte or cascade decay
%-each decay ends with lsp (neutralino, gravitino)
%-relationship between particle masses, production cross section, decay modes, branching ratios (usually 100\%)
%-stop decay channels
%-mass of intermediate particle
%$m_int = x m_mother + (1-x)m_LSP $
%T2 (T6) prefix - qaurk-squark production
%T2, T2bb, T2tt, T6ttww, T2bW
%	-each squark - two body decay to lighter flavours and chargino/neutralino


%\textbf{stop (third generation)}
%-stops gives the largest conbtribution to the higgs mass
%-there should not be much tuning in the higgs mass -> constraint on stop mass to be at energy level of order of 1~TeV (max) -> then fine tuning of the order of 10\%
%-phase space of stop decay?!
%-stop 1 or sbottom 1 probably the lightest squarks  -> because of mixing between R and L 
%-motivation and signature
%-lsp -large missing ET -> leaves detector undetected
%-derct stop production or gluino mediated


\subsection{Results on the SUSY searches in Run 1}

%-combined 1 and 2 lep: http://cms-results.web.cern.ch/cms-results/public-results/publications/SUS-14-015/index.html
%-one lep alone: http://cms-results.web.cern.ch/cms-results/public-results/publications/SUS-13-011/index.html
%-fully hadronic: http://cms-results.web.cern.ch/cms-results/public-results/publications/SUS-13-023/index.html , http://cms-results.web.cern.ch/cms-results/public-results/publications/SUS-14-001/index.html
%-summarized results: https://twiki.cern.ch/twiki/bin/view/CMSPublic/SUSYSMSSummaryPlots8TeV

In Run~1, many analyses targeting the SUSY particles and especially the stop quark production were performed. The naturalness constraint and the mixing of gauge eigenstates of the stops suggest that the lightest stop $\tilde{t}_{1}$ could be relatively light, about 1~TeV, and therefore accessible at the LHC energies. The combined results of stop decaying either to $t \tilde{\chi}_{1}^0$ or $c \tilde{\chi}_{1}^0$ are shown in Fig.~\ref{fig:figures/T2tt2015}. This figure shows the exclusion limits on the stop and LSP masses in three different kinematic regions. The region where the stop mass is smaller than the mass of the LSP is kinematically forbidden, but once the masses satisfy $m_{\tilde{t}} > m_{LSP}$ but $m_{\tilde{t}} - m_{LSP} < m_{W}$ the stop can decay either to $c \tilde{\chi}_{1}^0$ or to four bodies via off-shell top quarks and W bosons. The decay of the stop to the $c \tilde{\chi}_{1}^0$ is possible via loops or flavor changing neutral currents.  Once  $m_{\tilde{t}} - m_{LSP} > m_{W}$ but also  $m_{\tilde{t}} - m_{LSP} < m_{t}$, the three body decay via off-shell top quark is possible. When  $m_{\tilde{t}} - m_{LSP} > m_{t}$  the stop can decay to $t \tilde{\chi}_{1}^0$. The analyses whose results are presented in the Fig.~\ref{fig:figures/T2tt2015} do not only differ by the coverage of the kinematic regions, but also by the analysis strategy, for example they require different number of leptons in the final state. In general, less leptons are in the final state, larger is the branching ratio, and therefore the expected yield. But on the other hand analyses with more leptons suffer from less SM background contributions.  The strongest exclusion limit on the stop mass was put by analysis~\cite{Khachatryan:2016oia} (light blue in the Fig.~\ref{fig:figures/T2tt2015}) searching for fully hadronic final states. This analysis is interpreted in the context of of SMS and excludes top squark masses up to 755~GeV when the neutralino mass is below 200~GeV. On the diagonals, where $\Delta m \sim m_{W}$ and $\Delta m \sim m_{t}$ the signal final state topologies are very similar to the ones of SM background, the analyses have a low sensitivity in these regions and therefore there are no limits on them with the Run~I data.

%-analyses preformed in 0,1 and 2-lep
%	- hadronic suffer with large backgroun but high production cross section
%	-while petonic low background but low cross section
%	-three kind of regions + kinematically forbidden when amss of the stop is lower than mass of neutralino
%        -for mass diffference between stop and neutralino smaller than E, decay to c neutralino, othervise to top nutralino
%        -different multiplicity of leptons and difefrent search techniques
%	-going to build on  SUS-13-011
%-emphasise that stop should be light?

    \insertFigure{figures/T2tt2015} % Filename = label
                 {0.7}       % Width, in fraction of the whole page width
                { The exclusion limits on the stop and LSP masses for the $t \tilde{\chi}_{1}^0$ and $c \tilde{\chi}_{1}^0$ decay modes. The results from several Run~1 analyses are combined in this plot~\cite{website:SUSYresRunI}.   } % Caption

The summary plots of excluded masses of other SUSY particles by different SUSY searches at Run 1 performed by the ATLAS and CMS collaborations can be seen in Fig.~\ref{fig:figures/summaryRun1}.


    \insertTwoFigures{figures/summaryRun1} % Filename = label
                 {figures/ATLASSUSYSummaryRun1}
                 {figures/CMSSUSYSummaryRun1} % Filename = label
                 {0.45}       % Width, in fraction of the whole page width
                 {The exclusion limits on masses of SUSY particles by ATLAS~\cite{website:SUSYresRunIATLAS} (left) and CMS~\cite{website:SUSYresRunI} (right) collaborations. In the right plot the exclusion in dark shades is given for the massless LSP and exclusion in light shades for m(mother) - m(LSP) = 200 GeV. The branching ratios in this plot are assumed and therefore these result serve as an upper limit. In case of the left plot, the assumptions are directly depicted in the plot.}

As show in Fig.~\ref{fig:figures/xsectionsstrong} the production cross section of the stop pair falls steeply for heavier stop masses, therefore the search is statistically limited and it becomes difficult to largely extend the limits on the stop mass given the center-of-mass energy or instantaneous luminosity remains unchanged. But the increases of the center-of-mass energy between Run~1 and Run~2 from 8~TeV to 13~TeV, permits further probing of the stop masses as the parton luminosity increase around eight times for the stop pair production given the stop mass is of 700~GeV as shown in Fig.~\ref{fig:figures/PartonLuminosity13to8}.  

    \insertFigure{figures/xsectionsstrong} % Filename = label
                 {0.7}       % Width, in fraction of the whole page width
                 { The cross sections of selected SUSY processes~\cite{website:LHCxsec}. }

    \insertFigure{figures/PartonLuminosity13to8} % Filename = label
                 {0.99}       % Width, in fraction of the whole page width
                 {(left) The ratios of 13~TeV to 8~TeV LHC parton luminosities as a function of produced particle mass $M_{X}$. (right) The value of this ratio for several SM and beyond SM processes.  The ratio for production of 700~GeV stop quark pair is 8.4~\cite{Hoecker:2236645}.}
\newpage
