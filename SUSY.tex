\chapter{Supersymmetry as a possible extension of the Standard Model}

the particle physics is described by the SM which is formulated within quantum field theory model
-brief intorducation to standard model its particles and interactions
-then electrowek breakinga nd higgs mechanism
- then shortcommings of SM, need for extension
-the most popular extenion of SM is supersymmetry -> symmetry between fermions and bosons

\section{Standard Model and its shortcomings}

Standard Model~(SM) of particle physics is model based on Quantum Field Theory~(QFT) and derived from gauge symmetries~\cite{9783527406012}. The symmetry group of the SM is

\eq{SMgroup}
{
SU(3)_{C} \otimes SU(2)_{L} \otimes U(1)_{Y},
}

where $C$ stands for the color charge of strong interaction, $L$ left-handed particles, which have, unlike the right-handed ones, non-zero weak isospin and $Y$ for hypercharge. The SM group will be discussed part by part in following sections.

The Standard Model started to be formulated in 1960s and was finished in 2012, when last particle of the SM, the Higgs boson, was discovered by ATLAS and CMS experiments~\cite{Chatrchyan:2012xdj, Aad:2012tfa}. It describes all fundamental interactions (except gravity), which are electroweak~(EW) and strong interactions and all known particles. There are two kinds of particles in the SM, fermions with half-integer spin and bosons with integer spin.

Fermions can be divided into leptons and quarks. There are three generation of leptions which are electron~$e$, muon~$\mu$, tau~$\tau$ generation with their corresponding neutrinos~($\nu_{e},~\nu_{\mu},~\nu_{\tau}$). Leptons have integer charge in multiples of elementary charge~$e$ and do not have color and therefore interact only via electroweak interaction as they can have non-zero ispospin and hypercharge. There are also theree generation of quarks, first is formed by up~($u$) and down~($d$) quarks, second by charm~($c$) and strange~($s$) quarks and third by top~($t$) and bottom~($b$) quarks. The $u,~c,~t$ have charge of $2/3~e$ and $d,~s,~b$ have charge of $-1/3~e$. Each quark exist in three color versions, red, green and blue and thus quarks can participate in strong intearction. Quarks can hold isospin and hypercharge and thus the can also interact via electroweak interaction. Due the the phenomena reffered as ``color confinement'', quarks are always bound in hadron and cannot be separated.

The interaction in SM are mediated via an exchange of (gauge) bosons. There are 12 bosons with spin of one. There mediators of the electroweak interaction is masless photon~$\gamma$, massive $W^{\pm}$ and $Z$ bosons. The $W^{\pm}$ and $Z$ bosons gain their mass through breaking of electroweak symmetry as explained in later text. The gauge bosons of strong interaction are eight masless gluons, each holding unique color charge which is combination of colors and anticolors. The gluons are masless, indicating that strong symmetry is unbroken. The last particle belonging to the SM is Higgs boson which arises from EW symmetry breaking. The overview of all SM particles can be seen in Fig.~\ref{fig:figures/SMparticles}.


    \insertFigure{figures/SMparticles} % Filename = label
                 {0.5}       % Width, in fraction of the whole page width
                 { Overview of the particles present in the Standard Model.}

%-sm intro, more about sm can be seen in this book~\cite{9783527406012} %griffiths
%-renormalizable quantumn field theory, derived from gauge symmetries ~\cite{tHooft:1971qjg, Weinberg:1967tq}
%-formulation started in 1960s, completed in 2012 with discovery of higgs boson~\cite{Chatrchyan:2012xdj, Aad:2012tfa}
%-describes all known particle and fundamental interactions (except of gravity)  
%- the SM is $SU(3)_{C} \otimes SU(2)_{L} \otimes U(1)_{Y} $ group as will be shown later
%- the SM there are matter fileds which are quark and leptons -> half spin particles - fermions. The quarks have colors and electroweak charges (intract strongly and electroweakly) while leptons have only electroweak charges (only EW interaction).
%- within sm there are three generations of leptons and three generations of quark , each quark is present in three colors. Up to know, there is no explanation why there are three lepton generations.
%-leptons have full number charge, while quarks do not (smae chareg in up or wown generation)
%-top row q = +2/3e , where e is the electron charge, 
%- bottom row q = -1/3 e.
%-interaction -> exchange of boson
%-carrier of force
% TODO The masses of the SM particles cannot be predicted, they have to be measured


The SM is a field theory, which is requiring invariance towards certain transformations of involved Lagrangians. To better understand the derivation and the features of the SM, the basics of quantum field theory and gauge transformations are given in following sections. Then, building on the QFT, more details about fundamental interactions are given.

\subsection{Quantum field theory and gauge symmetries}

In classical mechanics, the motion of some system can be calculated by sloving the Euler-Lagrange equations~\cite{9783527411887}. These equations can be generalized in order to build a relativistic theory, in which the space and time coordinates must be treated similarly. In the relativistic case, the classical Euler-Lagrange equation is generalized into formulation

\eq{EL}
{
 \partial_{\mu}(\pdv{\mathcal{L}}{(\partial_{\mu}\phi_{i})}) = \pdv{\mathcal{L}}{\phi_{i}},~i=1,2,3,4
}


The basic building block of a QFT is a Lagrangian which depends on fileds $\phi_{i}$ and  their space-time drivatives. In the case of scalar field (spin-0) the Lagrangian can be written in form 

\eq{kglagrangian}
{
 \mathcal{L}_{Klein-Gordon} = \frac{1}{2}(\partial_{\mu} \phi)(\partial^{\mu} \phi) - \frac{1}{2} (\frac{mc}{\hbar})^{2},
}

where $m$ is a mass, $\hbar$ is Planck constant and $c$ is the speed of light of the. From now on, the standard convention of $\hbar = c = 1$ is used. The scalar field Lagrangian~\ref{eq:kglagrangian} plugged into the Euler-Lagrange equation~\ref{eq:EL} gives Klein-Gordon equation

\eq{kgEq}
{
 \partial_{\mu}\partial^{\mu}_\phi + m^{2} \phi = 0
}

describing a scalar particle of mass $m$. The field of half-spin particle is a four-component spinor field $\psi$. The solution of Euler-Largrange equation w.r.t. to $\bar{\psi}$ using Lagrangian


\eq{dirlagrangian}
{
 \mathcal{L}_{Dirac} = i \bar{\psi} \gamma^{\mu} \partial_{\mu} \psi -m^{2} \bar{\psi} \psi
}

is Dirac equation describing half-spin particle of mass $m$:

\eq{dirEq}
{
  i \gamma^{\mu} \partial_{\mu} \psi - m \psi = 0.
}

Equations describing  particles with different spins can be derived similarly.

In the equations above, only non-interacting fields are present. To include interactions between fields, the impact of local ang global transformations of the fileds on the corersponding Lagrangian must be studied.  The Dirac Lagrangian~\ref{eq:dirlagrangian} is invariant under global phase transformation
\eq{globalTrans}
{
\psi \to e^{i\theta} \psi, 
}

with phase $\theta$ being arbitrary real number. But this Lagrangian is not invariant under local phase transformation 

\eq{localTrans}
{
\psi \to e^{i\theta (x)} \psi,
}

where phase $\theta(x)$ is this time dependent on the space time coordinate. To preserve invariance of Lagrangian~\ref{eq:dirlagrangian}, the term $-(q\bar{\psi}\gamma^{\mu}\psi)A_{\mu}$, with $A_{\mu}$ which transforms as $A_{\mu} \to A_{\mu} + \partial_{\mu} \lambda$ can be added to the Dirac Lagrangian. The $A_{\mu}$ is a new vector~(spin-1) field. To obtain full Lagrangian, also the free field Lagrangian for vector field must be added to the Dirac Lagrangian~\ref{eq:dirlagrangian}. The summed Lagrangian is locally invariant only in case when field $A_{\mu}$ is masless. Such lagrangian generates quantum electrodynamics~(QED), where the field $A_{\mu}$ corresponds to the electromagnetic potential.

Thei global transformation of potential $\psi$ can be uderstood as multiplication of a field by an unitary matrix $U$ ($\psi \to U \psi$). In given example of quantum electrodynamics, the size of matrix is $1 \times 1$ and therefore it is $U(1)$ theory as the group of such matrcies is U(1). Similar strategy of global and local phase invariance of Lagrangian can be appliead on other groups, which was found to be the way how to generate the Standard Model.

\subsection{Electroweak interaction}

In 1954 Yang and Mills~\cite{Yang:1954ek} applied local and global invariance on $SU(2)$ group to describe weak interaction and later Glashow, Salam and Weinberg~\cite{Glashow:1961tr, Salam:1968rm, Weinberg:1967tq} shown, that if group $SU(2) \otimes U(1)$ is considered, the weak and electormagnetic interaction can be unified. Moreover they divided the left and right chiral components of the fermion fields into $\Psi_L$ composed of two spinors~(doublet) and one-spinor~(singlet) $\Psi_R$. The locally invariant Lagrangian of electroweak intractions (without symmetry breaking) was found to be

\eq{EWlagrangian}
{
\mathcal{L}_{EW} = - \frac{1}{4} \sum_{a=1}^{3} F_{\mu\nu}^{a} F^{a\mu\nu} - \frac{1}{4} B_{\mu\nu}B^{\mu\nu} +  i \bar{\Psi_L} \gamma^{\mu} D_{\mu} \partial_{\mu} \Psi_{L} +  i \bar{\Psi_R} \gamma^{\mu} D_{\mu} \partial_{\mu} \Psi_{R},
}

where in case of $SU(2)$ group the covariant derviative $D_{\mu}$ is

\eq{weakCovariant}
{
   D_{\mu} = \partial_{\mu} - ig\sum_{a=1}^{3}t^{a}W_{\mu}^{a},~a=1,2,3
}

where matrices $t^{a}$ are generators of group composed by Pauli matrices and $g$ is a constant. The $t_{3}$ component is called weak isospin. The covariant derivative $D_{\mu}$ for $U(1)$ group is

\eq{weakCovariant}
{
   D_{\mu} = \partial_{\mu} - ig'YB_{\mu},
}

where $Y$ is the weak hypercharge and $g'$ is a constant . The charge $Q$ of a particle is then given by relation between its isopin and hypercharge $Q= t_{3} + \frac{1}{2}Y$.

In presented equations $B_{\mu}$ and $W_{\mu}^{a}$ are the gauge fields, $\Psi_{R,L}$ is the right and left component of fermion field, $D_{\mu}$ is the covariant derivative and $\gamma_{\mu}$ are the Dirac matrices. The tensors $F_{\mu\nu}$ is composed of fields $W^{a}_{\mu}$ and their derivatives and tensor $B_{\mu\nu}$ is composed  of derivatives of $B_{\mu}$ field.

The group of electroweak interactions is often denoted as $SU(2)_{L} \otimes U(1)_{Y}$, where $L$ is related to difference of behavior of left- and right-handed fileds w.r.t weak interactions and $Y$ denotes the weak hypercharge. This group produces two masless gauge fields $W^{1}$ and $W^2$ which mix and create $W^{+}$ and $W^{-}$ bosons. These bosons interact only with left-handed componets of the fermion field (maximum parity violation). The remaining $W^{3}$ and $B$ gauge fields interact with both left- and right-handed fermions and they mix into $Z$ boson and electromagnetic $\gamma$. As mentioned previously all these bosons have to be masless in order to perserve local and global gauge invariance, but experimentally, the $W^{\pm}$ and $Z$ bosons were found to be a massive and therefore the electroweak symmetry must be broken. It is also important to note, that EW Lagrangian gives maximum parity violation for neutrino and therefore there is only left component of the neutrino field exist, not right one.

%The electric charge can be is $e= g\mathrm{sin}\theta_{W} = g'\mathrm{cos}\theta_{W} $, where $\theta_{W}$ is Weinberg mixing angle which was experimentally measured to be of around $30^{\circ}$.
%The weak isospinhas only non-zero value for left-handed components.
%-isospin, hypercharge TODO
%-no right neutrino  TODO

\subsection{Qunatum Chromodynamics}

The theory of strong intreaction, called Quantum Chromodynamics~(QCD) is based on $SU(3)$ group. The corresponding lagranigian of QCD is

\eq{QCDlagrangian}
{
\mathcal{L}_{QCD} = -\frac{1}{4} \sum_{a=1}^{8} F_{\mu \nu}^{a} F^{a \mu \nu} + \sum_{j=1}^{n_f} \bar{q}_{j}(i D_{\mu}\gamma^{\mu} -m_{j})q_{j} ,
}

where $q_{j}$ are the quark fields which are summed over the number of different favours $n_{f}$ and $m$ is a mass. The quark filed $q_{j}$ is composed of three quark spinors, one for each color. Tensors $F_{\mu \nu}$ are combination of gluon fields $g_{\mu}$ and its derviatives. The covariant derivative in this case is

\eq{QCDdervative}
{
   D_{\mu} = \partial_{\mu} - i\sqrt{4 \pi \alpha_{s}} \sum_{a=1}^{8} t^{a} g_{\mu}^{a},~a=1,...,8 , 
}

where $\alpha_{s}$ is strong coupling constant and $g_{\mu}$ are gluon fields. The objects which can interact via strong interactions must have a color, which is in case of quarks red, green or blue and in case of gluon a eight different combinations of colors and anticolors.

The coupling of colored objects is weak at short distances~(asymptotic freedom)~\cite{Gross:1973id}, but it grows with distance~(confinment)~\cite{Wilson:1974sk}. Therefore colored objects always have to be bound inside colorless hadrons, where they are quasi-free and never can be observed separately. There are two kinds of hadrons, baryons holding three quarks of different colors and mesons composed of two quarks, one of certain color and second of corresponding anticolor.

%-asymptotic freedom - the coupling depends on the distance, , it is very weak at short distances (asymtotic freedom) nut grows in distance (confinment - bound hadron states)
%-confinment -> color must remain neutral , not possinle to separate individual quarks and gluons, always are bound in colorless hadrons - baryons of mesons.
%-hadronization - formation of colorless objects (say more) 
%-color

%INTERACTIONS
%-sm lagrangian (CERN-thesis-2017-005) -> without higgs
%-interactions (electroweak, QCD)
%TODO -19 free parameters (nine fermion masses, one scalar mass, three coupling parameters, four quark mixing parameters, higgs vacuum expectation value, strong cp violating phase)
%TODO -perturbative theory (LO, NLO)



\subsection{Electroweak symmetry breaking~\label{sec:EWbreaking}}

As discussed, the $W^{\pm}$ and $Z$ bosons are massive, but the mass term for these bosons cannot be incomporated into the Largangian of electroweak interaction, becuase it would break the invariance of Lagrangian under local phase transformation. This problem was solved by ``Higgs mechanism'', based on phenomenon reffered as ``spontaneous symmetry breaking'' of SU(2) symmetry~\cite{Englert:1964et, Higgs:1964ia, Guralnik:1964eu}. The Higgs mechanism introduces a new complex scalar field $\phi$. The Lagrangian for this field and its interactions can be expressed as 

\eq{lagHiggs}
{
    \mathcal{L}_{Higgs} =  (D_{\mu}\Phi)^{\dagger} (D^{\mu}\Phi) - V(\Phi^{\dagger}\Phi),
}

where $D_{\mu}$ is the the covariant derivativei, $V(\Phi^{\dagger}\Phi)$ is the Higgs poptential and $\Phi$ is doublet of scalar fields defined as


\eq{doubletHiggs}
{
    \Phi = \binom{\phi^{+}}{\phi^{0}}.
}

The Higgs potential $V(\Phi^{\dagger}\Phi)$ is

\eq{potHiggs}
{
    V(\Phi^{\dagger}\Phi) =  - \frac{1}{2} \mu^{2}\Phi^{\dagger}\Phi + \frac{1}{4} \lambda(Phi^{\dagger}\Phi)^{2},
}

%$\Phi$ is a complex field. It is self-interacting SU(2)L doublet, weak hypercharge 1/2

    \insertFigure{figures/mexicanHat} % Filename = label
                 {0.5}       % Width, in fraction of the whole page width
                 { A shape of the Higgs potential $V$ for the complex field $\phi$ with positive values of real parameters $\mu^{2}$ and $\lambda$.}

where $\mu$ and $\lambda$ are real parameters. In case when both $\mu^{2}$ and $\lambda$ are positive numbers, the potential $V(\Phi^{\dagger}\Phi)$ takes shape of a ``Mexican hat'', as shown in Fig.~\ref{fig:figure/mexicanHat}. The shape of the potential is such, that the value of field $\Phi$ at the ground state, i.e. vacuum expectation value~(VEV) of the field $\Phi$ is non-zero. The ground state is degenarate and can be chosen to be

\eq{solutionHiggs}
{
    \langle 0 \Phi 0 \rangle = \frac{1}{\sqrt{2}}\binom{0}{v} ,
}

with

\eq{vDef}
{
v = \sqrt{\frac{\mu^{2}}{\lambda}}
}

being the energy of the ground state of field $\Phi$. Then the excitation of the field $\Phi$ can be written as follows

\eq{solutionHiggs2}
{
    \Phi = \frac{1}{\sqrt{2}}\binom{0}{v+H},
}

where $H$ is the Higgs boson. The vector bosons $W^{\pm}$ and $Z$ become massive via interaction with Higgs field present in the first term of Higgs Lagrangian~\ref{eq:lagHiggs}. The masses of the fermions can be also generated via interaction of fermion field  $\Psi$ with Higgs field $H$ by adding new term of type $\lambda_{Y} \bar{\Psi}_{L} \Phi \Psi_{R} + h.c.$ to the lagrangian of the Standard Model, with $\lambda_{Y}$ representig a Yukawa coupling which is dependent on mass of the fermion. It can be noticed that the interaction with the Higgs field flips the chirality of fermion from left to right and vice versa. Therefore masses of neutrinos cannot be generated in this way, as there are only left-handed neutrinos in the SM. The parameters $\mu$ and $\lambda$ of the Higgs potential are not predicted by the Standard Model and are measured experimentally. The mass of the Higgs boson at tree level, $m_{H} = \sqrt{2}\mu$, is dependent on parameter $\mu$ and therefore is as well not predicted by the SM. 

%Fermions get mass via interaction with $\Phi$ filed~\cite{Weinberg:1967tq}
%-hiigs flips the chirality, this is why neurino cannot have a mass within SM
%ELECTROWEAK symmetry breaking: TODO strat here
%Yang-Mills~\cite{Yang:1954ek} -> nonabelian gauge theory
%-new scalar field predicted by Higgs Englert and Brout in 1964~\cite{Higgs:1964pj, Englert:1964et}
%-discovery in 2012 by CMS and ATLAS
%-> in this theory gauge bosons are masless - gauge symmetry do not alow mass terms in lagrangian
%-based on spontaneous symmetry breaking principles -> apperance og goldstone~\cite{Goldstone:1961eq} bosons (one for each generator of broken symmetry?!) , goldosnes are masless spin-0
%-> do not speak about goldstone
%-non-zero ground state - vev
%-degenerated state -> infinite nr of minima on on circle of phi(1) and phi(2)- complex field -> this gives us a chance to fix phi as we want. 
%-1960s - the goldstone bosons cancel and give mass to other bosons -> generation of mass for Ws and Z bosons -> Hoggs mechanism ~\cite{Englert:1964et, Higgs:1964ia, Guralnik:1964eu,}% Higgs:1966ev}
%-parameter v is vacum expectation value - v =sqrt(-mh2/lambda)
%-Goldstone theorem -> masless states -> masless states are absorbed by the evctor bosnons
%-lambda and higgs mass must be experimentally measured
%- v and mH value? (Hoss)
%Higgs lagrangian
% TODO \section{Feynman diagrams} ?
%+perturbative theory

\subsection{Issues of the Standard Model}

Even though the Standard Model prooved to be very successful in describing the physics results of high energy physics experiments, there are observed phenomena which SM cannot explain. Therefore it is widely belived that the Standard Model is part of a larger theory. Before discussing the extention of the SM, some of the shortcommings and open quations of the Standard model are briefly described first.


\textbf{The naturalness problem}

As mentioned in section~\ref{sec:EWbreaking} the mass of the Higgs boson at the tree level is $m_{H} = \sqrt{2}\mu$, but this mass must be corrected for the contribution of the virtual paricles. The correction is especially large in case of virtual fermions due to their coupling constant with the Higgs boson. The coupling constant depends lineraly on the mass of the fermion, thus it is largest for the top quark which is the heaviest fermion. Therefore the largest correction to the Higgs mass comes from virtual top quarks. 

    \insertFigure{figures/fermionCorr} % Filename = label
                 {0.3}       % Width, in fraction of the whole page width
                 { A virtual fermion contributing to mass of the Higgs boson.}

An exaple of virtual fermion contributing to the Higgs mass is shown in Fig.~\ref{fig:figures/fermionCorr}. The mass of the Higgs boson can be decomposed in $m_{H,0}$ which is the mass at the tree level and $\Delta m_{H}$ which is the correction from virtual fermions

\eq{HiggsMass}
{
m_{H}^{2} = m_{H, 0}^{2} + \Delta m_{H}^{2}.
}

It can be shown that the correction $\Delta m_{H}$ follows relation

\eq{HiggsMassT2}
{
\Delta m_{H}^{2} \propto m_{f}^{2} \Lambda^2,
}

where $m_f$ is mass of a fermion and $\Lambda$ is a cutoff on the momentum of the virtual particle. This cutoff is expersing up to which scale the Standard Model is valid and is usually taken to be the Planck mass with $m_{Planck} \sim  10^{19}$~GeV. Knowing the term $\Delta m_{H}$, the Higgs mass can be expressed as

\eq{HiggsTuning}
{
m_{H}^{2} \sim m_{H, 0}^{2} + k \cdot m_{Planck},
}

where $k$ includes the constants and couplings of the SM. The mass of the Higgs boson was experimentally mesured to be of 125~GeV~\cite{Chatrchyan:2012xdj, Aad:2012tfa}, which is orders of magnitude lower than the Planck mass. This mismatch between the order of magnitude of the Higgs and and the Planck mass, for which there is no physics reson, is reffered as ``hierarchy problem''. Morover to obtain this relatively small mass of the Higgs boson, there must be large cancellation between the two terms of Eq.~\ref{eq:HiggsTuning}. The cancelation of the terms has to be up to O(30) of decimal places. Such fine-tuning is not natural and this problem of fine tuning is reffered as ``naturalness problem''.

%corrwctions to Higgs boson mass
%to compute cross section, all quantum loop corrections has to be taken into account
%fermions an vector boson masses proctected from diverging by mechanism within the SM
%but no mechanism for Higgs mass: $mh^2~ mh0^2+k mPlanck^2 $ - parameters mh0, k and mPlanck a priori unrelated. But these parameters must be fine tuned in order to obtain mass of Higgs (mh<<mPlanck) -> not natural
%called hierarchy problem - no reason to expect a large hierarchy between electroweak scale and planck scale
%-34 digits
%-picture higgs loop

\textbf{The dark matter and energy}

The indirect cosmological observation suggest that the ordinary matter and energy, the matter and energy described by the Standard Model, account only for $\sim 5\%$ of the total mass~(energy) of the universe~\cite{Bertone:2004pz, Gaitskell:2004gd, Bennett:2012zja}. The remaining $\sim$95\% is divided between the dark matter~$\sim$27\% and dark energy$\sim$27\%.

The dark matter has been observed only indirectly, it has no color or electric charge and therefore do not radiate. The first observation supprting the dark matter existance came from the mesurement of rotation curves of galaxies. The curves show the dependency of the star orbital velocity on the distrance of the star from the center of galaxy. These curves can be theoretically computed and it was measured, that the mesured and theoretical curves agree at short distances. With increasing distance the observed curves remain constant, but theoretically they should decrease~\cite{Bertone:2004pz}. This phenomena can be explained by presence of halo of new particles, which interact by gravitational force. It was also found out that these particles must be long-lived and they should have mass of order of 100~GeV.

The dark energy arises from the nned of cosmological constant in the Einstein equation. Without it it would not be possible to explaine the accelerated expansion of the universe. The dark energy can be interpreted as a vacuum energy, but there is a mismatch of order of magnitude $\sim$120 in the vacuum energy estimation from the cosmological constant and quantum field theory calculations.

%-5 percent of baryonic matter
%-27 percent of dark matter
%-68 dark energy
%-measuremnt of rotation curves of galaxies - first dark matter hypothesis
%-gravitational interaction, but not electromagnetic -> dark matter
%-from observations several constraints on dark matter - not short -lived and not baruonic, gravitationally interacting, low kinetic energy (cold -> it cannot be neutrino)
%-> no good candidate within the SM
%From cosmological observations we expect dark matter mass of order of 100~GeV
%- how many percent?
%-microwawe background
%-no radiation of DM -> no collor, no electric charge

%3)Dark energy
%-cosmological constant (lambda) in einsteins equation necessary to explain the observed expansion of universe
%-> cosmologica constant can be interpreted as a vacuum energy

\textbf{Other issues of the Standard Model}

One of the other problems is, the assymetry between the matter and antimatter. In our universe there is abundance of matter, even though, the matter should be produced in same amount as antimatter. There is no mechanism within the SM, which could explain such a large baryon aymmetry.

As mentioned, there is only left-handed neutrino in the Standard Model and therefore its mass cannot be generated via interaction with Higgs boson, which filps left-handed fermions to right-handed ones. But observed neutrino oscillations~\cite{Fukuda:1998mi, Ahmad:2001an} are only possible if the neutrinos are massive. This fact itself is not conclusive argument in favour of physics beyond the SM, but it needs to be understood how the mass of neutrinos can be generated.

Other argument for physics beyon the Standard Model is also, that the gravitational force is not a part of the Standard Model. Currently, there are big efforts to combine general relativity and quantum field theory to formulate theory of quantum gravity.  

In the Standard Model, the coupling constants are dependent on energy. At higher energies the constants of weak, electromagnetic and strong interaction become of similar strenghts, giving a hope for unification of these interaction at large energy scale. But this unification cannot be achieved within the SM and to unify forces, a beyond the standard model thoery is needed.
 
%4)Matter-antimatter assymetry
%-matter and antimatter should be produced in smae amount at big bang
%-but our world dominated by matter
%5)Neutrino masses
%-neutrinos oscialte from one flavour to other -> this can only happen when neutrinos are masive and have different mass states than flavour states

%6)Strong CP phase
%strong QCD lagrangian introducing the phase theta - close to zero, despite the theoreticla arguments that it should not be like this

%7)Quantum gravity
%-gravity not described by SM
%-desired to unify general relativity with QFT
%8)Unification of forces
%-possibility to unify all interactions
%-> of couplingconstants

%9)open questions
%-in SM large differences between quarks
%-why there should be three fermion families

%remaining stuff:
%-19 free parameters (nine fermion masses, one scalar mass, three coupling parameters, four quark mixing parameters, higgs vacuum expectation value, strong cp violating phase)
%-perturbative theory (LO, NLO)

\section{Supersymmetry}

To address the mentioned shortcomings of the Standard Modle, many extention of the SM were proposed over the past years. There are in general two kind of theories extending the SM~(beyon the Standard Model theries). First group of theroies in general add new space-time dimensions, which are generally reffered as ``extra dimensions'' theories~\cite{Patrignani:2016xqp}. The second group of theories add new symetries. The Supersymmetry~(SUSY)~\cite{Martin:1997ns}, which became very popular due to its capability to solve many issues of the SM. The SUSY started to be developed since 1970's around an idea of introducing new symmetry between fermions and bosons.
  
%- said before: SM works fine, but we need to extend it -> we can add either additional symetries, space-time dimensions or field content
%- one of the possibility how to extend is susy 
%- it addresses many issues of SM

The Supresymmetry introduces symmetry operator $Q$, whcih acts on fermions~$f$ and bosons~$b$ in following way:

\eq{SUSYop}
{
Q \mid f \rangle \to \mid b \rangle \to  , \; Q \mid b \rangle \to \mid f \rangle
}

The operator $\hat{Q}$ changes the spin of the particle by $1/2$, therefore transforms fermion to boson and vice versa, but do not change any other quantum number or particle properties. This symmetry $Q$ yields to the symmetry within which each SM fermion have bosonic SUSY partner with the same quantum number except of spin, and similarly each SM boson has asocciated fermionic SUSY partner. The supersymmetric partners of the SM particles are reffered as sparticles. To build a theory which is able to reproduce the SM, the operator $Q$ must satisfy following (anti)commutatior relations~\cite{Haag:1974qh, Coleman:1967ad}:

\eq{comutators}
{
\{Q,Q^{\dagger}\} = P^{\mu}, \; \{Q,Q\} =\{Q^{\dagger},Q^{\dagger}\}= 0, \; [P^{\mu}, Q] = [P^{\mu}, Q^{\dagger}] = 0,
}

where $P^{\mu}$ is the four-momentum operator. It can be noticed that $-P^{2}$, which is the mass-squared operator, commutes with both $Q$ and $Q^{2}$ and therefore a particle and its partner have the same mass. But as no SUSY partners have been observed, this symmetry must be broken. 

The naming convention for SUSY paricles adds prefix ``s'' to the SUSY partners of fermions, therefore sfermions are bosons. The SUSY partners to bosons get suffix ``ino'', for example the aprtner of gluon is gluino, which is fermion. The symbols of superpartners are dentoed with tilde.

There are many realizations of supersymmetry, but in further text only Minimal Supersymmetric Standard Model, which is the moset used one, is considered.


%------------------------------------------------------------------------------------------------------------------------------------------
%-around 70's
%-Golfand and Likhtman -> new symmetry Q -> Q|f> -> |b>; Q|b> -> |f> (transformation) -> later Haag, Lopuszanski and Sohnius said that such symmetry corresponds to supersymmetry
%-to each fermion a boson  with same quantum numbers (except of spin)
%-particles in supermulitples, where there is same number of fermionic and boisonic degrees of freedom
%-mass degeneration of particles in supermultiplet (from commutation relation of Q)
%-particles in supermultiplet have same quantum, numbers under the SU(3)xSU(2)xU(1) transformation
%-supermultiplet SM particle + susy partner, just differening by spin (1/2) -> bit more complicated I guess
%-parters of fermions are sfermions, and partner of bosons are inos, tilde for susy particles
%-partner of praticle is superpartner and they form superfield
%-spin differs by 1/2
%-same interactions of SUSY particles as the SM ones (for example only scalar partners of left handed fermions interact with partners of W boson)
%-two SUSY Higgs boson doublets are needed (in order to keep the theory renormalizable)
%-superpartner should have the same mass -> not observed -> susy must be broken (for now we just add a term into the lagrangian)
%-motivation:
%	-solve hierarchy problem (superpartnes have equal masses and cancel the loop corrections) - in case of "soft breaking" susy prevents the quadratic divergencies and there are only logarithmic + small fine tuning
%	-> naturalness of susy related to the mass difference between particle and its superpartner (Q: then if the susy partner of eg electron is very heavy does not it induce the divergencies? )

-susy breaking - not much known about its mechanism, there are several hypothesis (models)
%-susy solves naturalnes problem, bosons opposite sign of corerction to delta mass -> fine tuning can be removed, if the coupling constants are the same, just differ by sign (and it is actually true)
%	-> the loop diagram
-susy soles dm candidate - lsp
-susy solves the unification - susy modifies the energy evolution of coupling constants 

-constraints on lsp from dm relic density~\cite{Ade:2015xua}
%-commputation relations of susy operators?
%------------------------------------------------------------------------------------------------------------------------------------------

\subsection{Minimal Supersymmetric Standard Model}

The Minimal Supersymmetric Standard Model~(MSSM) is supersymmetric extension of SM which adds the minimum of new particles. It also does not introduce any additional gauge interactions. The Lagrangian of the new theory can be written as

\eq{lagSUSY}
{
    \mathcal{L} =  \mathcal{L}_{MSSM} +  \mathcal{L}_{soft} = \mathcal{L}_{free} + \mathcal{L}_{int} + \mathcal{L}_{soft}    ,
}

where $\mathcal{L}_{MSSM}$ is the supersymmetric part of lagrangian and term $\mathcal{L}_{soft}$ introduces ``soft'' breaking of the supersymmetry. The breaking of the supersymmetry must be mild in order to end up only with small tuning in the mass of the Higgs boson. The large advantage of supersymmetry is, that in its unbroaken form, it is able to solve the hierarchy problem. The radiative correction from the fermion loop as shown in Fig.~\ref{fig:figures/HiggsMassSUSY} to the Higgs boson mass $\Delta m_{H}^{2}$ in context of SUSY is proportional to

\eq{HiggsMassSUSY}
{
\Delta m_{H}^{2} \propto (m_{f}^{2} - m_{b}^2) \mathrm{ln}(\frac{\Lambda}{m_{b}}),
}

where $m_{f}$ is mass of fermion, $m_{b}$ mass of boson and $\Lambda$ a cutoff. This term is zero in case that mass of particle and its partner is degenerate and therefore no fine-tuning is needed. The contributions to the Higgs mass in context of MSSM are visualized in Fig.~\ref{fig:figures/fermionCorr2}, within which the fermion and sfermion loop contrinutions have are of same value but opposite sign and therefore cancel. In the broken theory the mases are not the same and larger the difference between the particle and its partner, more fine-tuning is needed. Therefore to preserve the naturalness of the theory, it is required that it is broken only slightly and the supersymmetric partners do not have mass hugely larger than the SM particles. Especially, as the largest radiative correction to the Higgs amss comes from the top quark, the mass difference between the top quark and top squark should be reasonable small, leading to a constraint of the top squark mass to be of order of 1~TeV.  

    \insertFigure{figures/fermionCorr2} % Filename = label
                 {0.3}       % Width, in fraction of the whole page width
                 { A virtual fermion and boson contributing to mass of the Higgs boson.}

The spectrum of SUSY partners can be seen in Tab.~\ref{tab:SUSYspectrum} in column ``Gauge Eigenstates''. In the MSSM there are two Higgs doublets in order  to avioid gauge anomaly. Both Higgs doublets ahve non-zero vacuum expectation value. As a consequence of electroweak symmetry breaking, the charged fields $W^{1,2}$ mix into positive and negative winos~($\tilde(W)$) and the neutral fields $B_{0}$ and $W_{3}$ mix into zino~($\tilde{Z}$) and photino($\tilde{\gamma}$). Then the higssinos mix with the flavor eignestates of the SUSY partners of $SU(2) \otimes U(1)$ bosons to give rise to the mass eigenstates. The two charged higgsinos~($\tilde{H}_{u}^{+}~\tilde{H}_{d}^{-}$) mix with $\tilde{W}^{\pm}$ to form two charginos~($\tilde{\chi}_{1,2}^{\pm}$) in negative and poitive version. The four neutralinos $\tilde{\chi}_{1,2,3,4}^{0}$ are mix of zino, photino and two neutral higgsinos~($\tilde{H}_{u}^{0}~\tilde{H}_{d}^{0}$). There is also mixing between the left and right componet of the third generation of squarks and stau, due to the large mass of thrid generation of quarks and tau and therefore large Yukawa coupling. The mass eigenstates can be as well found in Tab.~\ref{tab:SUSYspectrum}. The subscripts $L,R$ denote left and right component of the  fermion filed.

\begin{table}[h]
\begin{center}
\begin{tabular}{|c|c|c|c|}
\hline
Names & Spin  & Gauge Eigenstates & Mass Eigenstates  \\
\hline
        &   & $\tilde{u}_{L}~\tilde{u}_{R}~\tilde{d}_{L}~\tilde{d}_{R}$  & (same) \\
squarks & 0 & $\tilde{s}_{L}~\tilde{s}_{R}~\tilde{c}_{L}~\tilde{c}_{R}$  & (same) \\
        &   & $\tilde{t}_{L}~\tilde{t}_{R}~\tilde{b}_{L}~\tilde{b}_{R}$  & $\tilde{t}_{1}~\tilde{t}_{2}~\tilde{b}_{1}~\tilde{b}_{2}$ \\
\hline
         &   & $\tilde{e}_{L}~\tilde{e}_{R}~\tilde{\nu}_{e}$  & (same) \\
sleptons & 0 & $\tilde{\mu}_{L}~\tilde{\mu}_{R}~\tilde{\nu}_{\mu}$  & (same) \\
         &   & $\tilde{\tau}_{L}~\tilde{\tau}_{R}~\tilde{\nu}_{\tau}$  & $\tilde{\tau}_{1}~\tilde{\tau}_{2}~\tilde{\nu}_{\tau}$ \\
\hline
neutralinos & 1/2 & $\tilde{B}_{0}~\tilde{W}_{3}~\tilde{H}_{u}^{0}~\tilde{H}_{d}^{0}$  & $\tilde{\chi}_{1}^{0}~\tilde{\chi}_{2}^{0}~\tilde{\chi}_{3}^{0}~\tilde{\chi}_{4}^{0} $ \\
\hline
charginos & 1/2 & $\tilde{W}_{1,2}~\tilde{H}_{u}^{+}~\tilde{H}_{d}^{-}$  & $\tilde{\chi}_{1}^{\pm}~\tilde{\chi}_{2}^{\pm} $ \\
\hline
gluino & 1/2 & $\tilde{g}$  & (same) \\
\hline
gravitino & 3/2 & $\tilde{G}$  & (same) \\
\hline
\end{tabular}
\caption[Table caption text]{\cite{Martin:1997ns}. }
\label{tab:SUSYspectrum}
\end{center}
\end{table}

In the MSSM Lagrangian, in principle there could be terms which would violate baryon or lepton number conservation. Under such conditions, the proton could decay, what was not experimentally observed~\cite{Nishino:2009aa}. To avoid such violation, the quantity so called ``R-parity''~($P_{R}$) defined as

\eq{Rparity}
{
P_R=(-1)^{3(B-L)+2s },
}

where $B$ is a baryon number, $L$ is the lepton numbers and $s$ spin of the particle, is required to be conserved. All SM particles have $P_{R}=1$ and their SUSY partners $P_{R}=1$. The consequence of R-parity conservation is, that the sparticles can be produced only in even number and each of them can decauy only into odd number of sparticles. Also the lightest supersymmetric particle~(LSP) must be stable. The LSP is a dark matter candidate and it can be either the lightest neutralino or gravitino, depending on the specific realization of MSSM. From the cosmological observations of the relic density~\cite{Ade:2015xua}, constraint on the mass of the dark matter mass to be of order of 100~GeV was imposed.


Another advantage of the MSSM is that the energy dependence of the couplig constants of the SM ineractions is modified as a result of introducing sparticles. Consequently these coupling constants can be unified at large energy scale (GUT scale). Within the supersymmetry it is also posibele to create models of supergravity.

%\textbf{conservation of R-party by construction }
%	-> pair production of sparticles
%        -> decay only to odd nr of sparticles
%        -> LSP is stable -> dark matter candidate
%-in susy lagrangian there can be interaction between susy particles and sm particles -> there can be lepton or baryon number valiation
%-but this was restricted by SM , because proton could decay, wat was not observed~\cite{Nishino:2009aa}
%-to avoid this: R-parity requirement added:   , where B is baryon number, L is lepton numbver and s is spin of particle
%-it must be conserved
%-all susy particles negative r-parity, sm ones positive
%	-> lsp is stable (interacts only weakly - good dm candidate)
%	-> decay to odd number of susy particles
%	-> at colliders, susy particles produced in pair
-there are rpv models, but the lsp is not stable
%LSP, unification of forces and in some conditions it can describe quantum gravity

%TODO 

The MSSM brings 105 new free parameters on top of the 19 free parameters in the Standard Model. It is not possinle to search for SUSY ina  such large parameter phase-space and therefore the collider experiments usually search for SUSY in context of model reffered as ``Simplified Model Spectra'', which reduces the number of free parameters.


%------------------------------------------------------------------------------------------------------------------------------------------
%-most used SUSY relization is MSSM -> minimal -> adding the minimum number of fields~(particles) to the SM to become supersymmetric
%- no additional gauge interactions
%-table of supermultiplets?! (CERN-THESIS-2015-390) -> these are the particles before "mixing"
%-adding sfermions and gauginos - left and right handed fermions -> e.g two selectrons
%-for Higgs more complicated- one higgsino is not enough, but second SU(2) doublet is needed to avoid a gauge anomaly. (give mass to up and down type of quarks?) -> even there must be two SM doublets
%-both higgs doublets have to have non-zero vev
%-large mixing between sfermions states (because of large Yukawa coupling which is dependent on mass?), mixing of the second and first generation smaller, du to smaller Yukawa
-it is likely that right-handed states are lighter than left-handed
%-winos and binos mix after ew symmetry breaking -> winos, zino, photino , but they mix to give mass eingenstates
%	-4 neutralinos which are mix of neutral bino, wino and higgsinos
%	-two charginos (each can be negative or positive) - which are mix of charged winos and higgsinos
-only gluinos do not mix to give some mass eingenstates
%-mass eigenstates do not have to be flavor eigenstates
%-combinations of electroweak gauginos and higgsions make charginos and neutralinos
%-mixing between left and right superpartners 
%-spectrum of sparticles
%-MSSM - more than 100 new parameters than in SM
	-majority from symmetry breaking
	%-it can be constrained by using pMSSM (with 19 free parameters), where no assumptions on the breaking mechanism 
%-105 mssm + 19 sm parameters


%\textbf{pmssm}
%-> too many parameters - problem for phenomenological and experimental models
%-> pMSSM - phenomenological MSSM -> reduction of number of parameters by assuming
%	- there is no new source of CP vilation
%	-lightest neutralino is the LSP
%	-other assumptions on the sfermion masses, trilinear couplings and flavor violation
%	->reduction of parameters to 19
%		-higgsino mass parameter and pseudo-scalar higgs mass 
%		-ration of Higgs vauum expectation values
%		-soft gaugino masses (bino, wino, gluino
%		-sfermion masses
%		-trilinear couplings

\subsection{Simplified Model Spectra}

The parameter-space of the MSSM can be reduced by fixing the sparticle production cross section, their decay modes and branhcing ratios and the mass hierarchy between sparticles. Such restrictions give only limited number of possiblei SUSY models called ``Simplified Model Spectra''~(SMS)~\cite{Alves:2011wf, Alwall:2008ag, Chatrchyan:2013sza}.

This thesis focuses on the search for the top squark~(stop) and therefore only specifities of the stop production and decay within SMS are described in larger detail. The lighter stop is expected to be lighter than gluino and therefore the production of the stop can be mediated via gluino stop pair can be produce directly in the hard scattering. An example of a direct stop pair production can be seen in the left part of Fig.~\ref{fig:figures/stopProd}, the gluino mediated production is depicted in the right part of the same figure.


    \insertTwoFigures{figures/stopProd} % Filename = label
                 {figures/T2tt}
                 {figures/T5tttt} % Filename = label
                 {0.45}       % Width, in fraction of the whole page width
                 {(left) An example of Feynman diagram of the direct stop pair production. (right)  An example of Feynman diagram of the gluino mediated stop production.  [TODO describe more the decay]  } % Caption

The stops can decay directly to the top quark and LSP which is the ligtest neutralino, or via intermediary chargino as shown on example in Fig.~\ref{fig:figures/T6bbWW}. The mass of the chargino is fixed by the relation

\eq{charginomass}
{
m_{\tilde{\chi}_{1}^{\pm}} = x m_{\tilde{stop}_{1}} + (1-x)m_{LSP},
}

where $m_{\tilde{\chi}_{1}^{\pm}}$ is the chargino mass, $m_{\tilde{stop}_{1}}$ is the stop mass, $m_{LSP}$ is the enutralino mass and $x$ is a fixed fraction between 0 and 1, which is usually chosen to be 0.25, 0.5 or 0.75. 

    \insertFigure{figures/T6bbWW} % Filename = label
                 {0.45}       % Width, in fraction of the whole page width
                 {(left) An example of Feynman diagram with intermediary chargino [TODO describe more the decay].   } % Caption

The directly produced stop pair decays eventually into two b-quarks, two W bosons and two neutralinos. Therefore in the final state two b-jets, two neutralinos and leptons and/or jets from the W-bosons are expected. The neutralinos interact only weakly and escape detector unmeasured. Although undetected particles can be spotted by presence of large amount of missing transverse energy, giving such SUSY signal unique exxperimental signature compared to the Standard Model processes.

In general, result of the searches in cotext of the SMS are exclusion limits on the sparticle masses in plane of the two sparticle masses, in the presented case in plane of  stop versus LSP mass.

%lhc reference~\cite{Alves:2011wf, Alwall:2008ag}
%cms reference~\cite{Chatrchyan:2013sza}
%-considering small number os sparticles -> others are too heavy -> considering mass hierarchy and branching ratios
%-limits in planes of sparticles
%-part of the parameter space of the MSSM
%-gluinos heavier than the lighter stop
%-described set of particles, their possible production and decay chain
%-in simplified models considering only production process of the primary particles?
%-each aprticle diracte or cascade decay
%-each decay ends with lsp (neutralino, gravitino)
%-relationship between particle masses, production cross section, decay modes, branching ratios (usually 100\%)
%-stop decay channels
%-mass of intermediate particle
%$m_int = x m_mother + (1-x)m_LSP $
%T2 (T6) prefix - qaurk-squark production
%T2, T2bb, T2tt, T6ttww, T2bW
%	-each squark - two body decay to lighter flavours and chargino/neutralino


%\textbf{stop (third generation)}
%-stops gives the largest conbtribution to the higgs mass
%-there should not be much tuning in the higgs mass -> constraint on stop mass to be at energy level of order of 1~TeV (max) -> then fine tuning of the order of 10\%
%-phase space of stop decay?!
-stop 1 or sbottom 1 probably the lightest squarks  -> because of mixing between R and L 
%-motivation and signature
%-lsp -large missing ET -> leaves detector undetected
%-derct stop production or gluino mediated


\subsection{Run I results on the stop}

-combined 1 and 2 lep: http://cms-results.web.cern.ch/cms-results/public-results/publications/SUS-14-015/index.html
-one lep alone: http://cms-results.web.cern.ch/cms-results/public-results/publications/SUS-13-011/index.html
-fully hadronic: http://cms-results.web.cern.ch/cms-results/public-results/publications/SUS-13-023/index.html , http://cms-results.web.cern.ch/cms-results/public-results/publications/SUS-14-001/index.html

-summarized results: https://twiki.cern.ch/twiki/bin/view/CMSPublic/SUSYSMSSummaryPlots8TeV
