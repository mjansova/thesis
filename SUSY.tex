\chapter{Supersymmetry as a possible extension of the standard model}

In this chapter the basics of the standard model are discussed. In section~\ref{sec:SM} the description of the elementary particles and interactions is presented as well as an idea how the standard model was derived. This section ends with a discussion of several shortcomings of the standard model, which motivate physicists to formulate beyond the standard model theories. Section~\ref{sec:SUSY} then describes supersymmetry, which is a promising extension of the standard model, due to its ability to solve many of the standard model issues. Within this section is also discussed how some searches for supersymmetry can be performed and the results of the SUSY searches performed at the LHC on the Run~1 data. 

\section{Standard model and its shortcomings~\label{sec:SM}}

The Standard Model~(SM) of particle physics is based on Quantum Field Theory~(QFT) and gauge symmetries~\cite{9783527406012}. It was formulated in 1960s, but it is only in 2012 that the last particle of the SM, the Higgs boson, was discovered by the ATLAS and CMS collaborations~\cite{Chatrchyan:2012xdj, Aad:2012tfa}. The SM describes all fundamental interactions (except gravity), namely electromagnetic, weak, and strong interactions, and characterizes all known elementary particles. 

%There are two kinds of particles in the SM, fermions with half-integer spin and bosons with integer spin. TODO do not forget

The interactions in the SM are mediated via an exchange of (gauge) bosons which have integer spin. There are 12 bosons with a spin of one. The mediators of the electromagnetic and weak interaction are the massless photon~$\gamma$, ant the massive $W^{\pm}$ and $Z$ bosons, respectively. The $W^{\pm}$ and $Z$ bosons acquire their mass through a spontaneous breaking of the electroweak symmetry as explained in later text. The gauge bosons of the strong interaction are eight massless gluons, each holding an unique color charge which is a combination of colors and anticolors. The gluons are massless, indicating that the strong symmetry is unbroken. The last boson belonging to the SM is the Higgs boson which arises from the electroweak symmetry breaking. 

In addition  to bosons the particles of matter are fermions of-half integer spin. Fermions can be divided into leptons and quarks. There are three generations of leptons formed respectively by the electron~$e$, the muon~$\mu$, and the tau~$\tau$, with their corresponding neutrinos~($\nu_{e},~\nu_{\mu},~\nu_{\tau}$). Leptons have integer charge of $\pm e$ with $e$ being the elementary electric charge, do not carry a charge of color and therefore interact only via electromagnetic and weak interactions as they can have non-zero quantum numbers with respect to these interactions, which are weak isospin and hypercharge. There are also three generations of quarks, the first one is formed by up~($u$) and down~($d$) quarks, the second by charm~($c$) and strange~($s$) quarks and the third one by top~($t$) and bottom~($b$) quarks. The $u,~c,~t$ quarks have charge of $2/3~e$ while the $d,~s,~b$ quarks have charge of $-1/3~e$. Each quark exists in three color versions (red, green and blue) and thus quarks can participate in strong interactions. Quarks can hold weak isospin and hypercharge and thus they can also interact via electroweak interactions. Due to the phenomena referred as ``color confinement''~\cite{Alkofer:2006fu}, quarks are always bound inside a comosite objects called ``hadrons'' and cannot be found isolated in nature. Each SM fermion has and antiparticle denoted with bar over the symbol, which differs from particle by the sign of the charge and spin. The overview of all SM particles can be seen in Fig.~\ref{fig:figures/SMparticles}.



    \insertFigure{figures/SMparticles} % Filename = label
                 {0.5}       % Width, in fraction of the whole page width
                 { Overview of the particles present in the standard model.}

The symmetry group of the SM is

\eq{SMgroup}
{
SU(3)_{C} \otimes SU(2)_{L} \otimes U(1)_{Y},
}
where $C$ stands for the color charge of strong interaction, $L$ left-handed particles, which have, unlike the right-handed ones, non-zero weak isospin and $Y$ represents weak hypercharge. The SM group and the interactions originating from it, are discussed part by part in following sections building on the quantum field theory. As the standard model is a field theory, which is requiring invariance towards certain transformations of its Lagrangian, to better understand the derivation and the features of the SM, the basics of the quantum field theory and gauge transformations are described in following sections. Finally, the last Subsection~\ref{sec:shortcomings} present the shortcomings of the SM.





%-sm intro, more about sm can be seen in this book~\cite{9783527406012} %griffiths
%-renormalizable quantumn field theory, derived from gauge symmetries ~\cite{tHooft:1971qjg, Weinberg:1967tq}
%-formulation started in 1960s, completed in 2012 with discovery of higgs boson~\cite{Chatrchyan:2012xdj, Aad:2012tfa}
%-describes all known particle and fundamental interactions (except of gravity)  
%- the SM is $SU(3)_{C} \otimes SU(2)_{L} \otimes U(1)_{Y} $ group as will be shown later
%- the SM there are matter fileds which are quark and leptons -> half spin particles - fermions. The quarks have colors and electroweak charges (intract strongly and electroweakly) while leptons have only electroweak charges (only EW interaction).
%- within sm there are three generations of leptons and three generations of quark , each quark is present in three colors. Up to know, there is no explanation why there are three lepton generations.
%-leptons have full number charge, while quarks do not (smae chareg in up or wown generation)
%-top row q = +2/3e , where e is the electron charge, 
%- bottom row q = -1/3 e.
%-interaction -> exchange of boson
%-carrier of force
% TODO The masses of the SM particles cannot be predicted, they have to be measured



\subsection{Quantum field theory and gauge symmetries}

In classical mechanics, the motion of a given system can be calculated by solving the Euler-Lagrange equation~\cite{9783527411887}. These equations can be generalized in order to build a relativistic theory, in which the space and time coordinates must be treated similarly. In the relativistic case, the Euler-Lagrange equation is given by the formula

\eq{EL}
{
 \partial_{\mu}\left (\pdv{\mathcal{L}}{(\partial_{\mu}\phi_{i})}\right) = \pdv{\mathcal{L}}{\phi_{i}},~\mathrm{with}~i=1,2,3,...,
}
where the basic building block of a QFT is the Lagrangian $\mathcal{L}$ which depends on the fields $\phi_{i}$ and their space-time derivatives $\partial_{\mu}\phi_{i}$. 

In the case of a scalar (spin-0) field, the Lagrangian $\mathcal{L}_{KG}$ can be written in the form 

\eq{kglagrangian}
{
 \mathcal{L}_{KG} = \frac{1}{2}(\partial_{\mu} \phi)(\partial^{\mu} \phi) - \frac{1}{2} \left(\frac{mc}{\hbar} \right)^{2} \phi^{2},
}
where $\phi$ is a single scalar field, $m$ is its mass, $\hbar$ is Planck constant and $c$ is the speed of light. From now on, the standard convention of $\hbar = c = 1$ is used. The scalar field Lagrangian defined in Eq.~\ref{eq:kglagrangian} and plugged into the Euler-Lagrange equation~(\ref{eq:EL}) gives the Klein-Gordon equation

\eq{kgEq}
{
 \partial_{\mu}\partial^{\mu} \phi + m^{2} \phi = 0,
}
which is describing a scalar particle of mass $m$. 

The field of an half-spin particle is a four-component spinor field $\psi$. The Euler-Lagrange equation applied on the field $\bar{\psi}$ using the Lagrangian $\mathcal{L}_{D}$


\eq{dirlagrangian}
{
 \mathcal{L}_{D} = i \bar{\psi} \gamma^{\mu} \partial_{\mu} \psi -m \bar{\psi} \psi
}
gives the Dirac equation describing a half-spin particle of mass $m$:

\eq{dirEq}
{
  i \gamma^{\mu} \partial_{\mu} \psi - m \psi = 0.
}

Equations for particles with spin one can be derived similarly.

In the above Lagrangians defined by Eq.~\ref{eq:kglagrangian}~and~\ref{eq:dirlagrangian}, only non-interacting fields are present. To include interactions between fields, the impact of local and global transformations of the fields on the corresponding Lagrangian must be studied. The Dirac Lagrangian given by Eq.~\ref{eq:dirlagrangian} is invariant under the global phase transformation

\eq{globalTrans}
{
\psi \to e^{i\theta} \psi, 
}

with the phase $\theta$ being an arbitrary real number. But this Lagrangian is not invariant under the local phase transformation 

\eq{localTrans}
{
\psi \to e^{i\theta (x)} \psi,
}

where the phase $\theta(x)$ is this time dependent on the space-time coordinate. To preserve the invariance of the Lagrangian of Eq.~\ref{eq:dirlagrangian}, the term 

\eq{newTerm}
{
-(q\bar{\psi}\gamma^{\mu}\psi)A_{\mu}, 
}

with the field $A_{\mu}$ which transforms as 

\eq{transfrom}
{
A_{\mu} \to A_{\mu} + \partial_{\mu} \lambda
}

can be added to the Dirac Lagrangian. The field $A_{\mu}$ is a new vector~(spin-1) field. To obtain the full Lagrangian, also the free field Lagrangian for vector field $A_{\mu}$ must be added to the Dirac Lagrangian~\ref{eq:dirlagrangian}. The summed Lagrangian is locally invariant only in the case when the field $A_{\mu}$ is massless and therefore the term for the free field Lagrangian of the field $A_{\mu}$ can be written as

\eq{freeA}
{
    \frac{-1}{16\pi}F^{\mu\nu}F_{\mu\nu}, ~\mathrm{where}~ F^{\mu\nu} \equiv \partial^{\mu}A^{\nu} - \partial^{\nu}A^{\mu}.
} 

The total Lagrangian is then

\eq{qedL}
{
    i \bar{\psi} \gamma^{\mu} \partial_{\mu} \psi -m \bar{\psi} \psi -  \frac{1}{16\pi}F^{\mu\nu}F_{\mu\nu} -(q\bar{\psi}\gamma^{\mu}\psi)A_{\mu},
} 


which generates the quantum electrodynamics~(QED), where the field $A_{\mu}$ corresponds to the electromagnetic potential.

The global transformation of a field $\psi$ can be understood as the multiplication of this field by an unitary matrix $U$ ($\psi \to U \psi$). In the given example of quantum electrodynamics, the size of matrix is $1 \times 1$ and therefore the symmetry of this theory is reffered as ``$U(1)$ gauge invariance'' as the group of all such matrices is $U(1)$. A similar strategy of global and local phase invariance of the Lagrangian can be applied on other groups, which was found to be the way how to generate the standard model.

\subsection{The electroweak interaction}

In 1954, Yang and Mills~\cite{Yang:1954ek} applied local and global invariance on the $SU(2)$ group to describe weak interaction and later Glashow, Salam and Weinberg~\cite{Glashow:1961tr, Salam:1968rm, Weinberg:1967tq} shown, that if the group $SU(2) \otimes U(1)$ is considered, the weak and electromagnetic interactions can be unified. Moreover they divided the left and right chiral components of the fermion fields into $\Psi_L$ composed of two spinors~(doublet) and one spinor~(singlet) $\Psi_R$. The locally invariant Lagrangian of the electroweak~(EW) interaction (without symmetry breaking) was found to be

\eq{EWlagrangian}
{
\mathcal{L}_{EW} = - \frac{1}{4} \sum_{a=1}^{3} F_{\mu\nu}^{a} F^{a\mu\nu} - \frac{1}{4} B_{\mu\nu}B^{\mu\nu} +  i \bar{\Psi_L} \gamma^{\mu} D_{\mu} \Psi_{L} +  i \bar{\Psi_R} \gamma^{\mu} D_{\mu}  \Psi_{R},
}
where $\Psi_{R}~(\Psi_{L})$ is the right-handed (left-handed) component of the fermion field, $D_{\mu}$ is the covariant derivative and $\gamma_{\mu}$ are the Dirac matrices. The tensors $F_{\mu\nu}$ are composed of the fields $W^{a}_{\mu}$ and their derivatives, and the tensor $B_{\mu\nu}$ is composed of derivatives of the field $B_{\mu}$.

In the case of the $SU(2)$ group the covariant dervative $D_{\mu}$ is 

\eq{weakCovariant}
{
   D_{\mu} = \partial_{\mu} - ig\sum_{a=1}^{3}t^{a}W_{\mu}^{a},~\mathrm{with}~a=1,2,3.
}

The matrices $t^{a}$ are the generators of the $SU(2)$ group composed by Pauli matrices and $g$ is a constant. The $t_{3}$ component is called the weak isospin. The covariant derivative $D_{\mu}$ for the $U(1)$ group is

\eq{weakCovariant}
{
   D_{\mu} = \partial_{\mu} - ig'YB_{\mu},
}
where $Y$ is the weak hypercharge and $g'$ is a constant. The charge $Q$ of a particle is then given by the relation between its weak isospin and hypercharge: $Q= t_{3} + \frac{1}{2}Y$.


The group of electroweak interactions is often denoted as $SU(2)_{L} \otimes U(1)_{Y}$, where $L$ is related to the difference of behavior between left-handed (doublet) and right-handed (singlet) fields w.r.t weak interactions and $Y$ denotes the weak hypercharge. This group produces two massless gauge fields $W^{1}$ and $W^2$ which mix and create $W^{+}$ and $W^{-}$ bosons. These bosons interact only with left-handed components of the fermion field (maximum parity violation). The remaining $W^{3}$ and $B$ gauge fields interact with both left- and right-handed fermions and they mix into the $Z$ boson and the photon $\gamma$. As mentioned previously all these bosons have to be massless in order to preserve local and global gauge invariance, but the $W^{\pm}$ and $Z$ bosons were expected to be massive because the weak interaction is short range interactioni, what was later experimentally confirmed. Because these bosons are massive, the electroweak symmetry must be broken. It is also important to note, that the EW Lagrangian gives a maximum parity violation for neutrinos because only the left component of the neutrino field exists, not the right one.

%The electric charge can be is $e= g\mathrm{sin}\theta_{W} = g'\mathrm{cos}\theta_{W} $, where $\theta_{W}$ is Weinberg mixing angle which was experimentally measured to be of around $30^{\circ}$.
%The weak isospinhas only non-zero value for left-handed components.
%-isospin, hypercharge TODO
%-no right neutrino  TODO

\subsection{Quantum Chromodynamics}

The theory of strong interaction, called Quantum Chromodynamics~(QCD), is based on the $SU(3)$ group. The corresponding Lagrangian of QCD is

\eq{QCDlagrangian}
{
\mathcal{L}_{QCD} = -\frac{1}{4} \sum_{a=1}^{8} F_{\mu \nu}^{a} F^{a \mu \nu} + \sum_{j=1}^{n_f} \bar{q}_{j}(i D_{\mu}\gamma^{\mu} -m_{j})q_{j} ,
}
where the quark fields $q_{j}$ are summed over the number of different quark flavors $n_{f}$ and $m_{j}$ are the masses. A quark field $q_{j}$ is composed of three quark spinors, one for each color. The tensors $F_{\mu \nu}$ are a combination of gluon fields $g_{\mu}$ and its derivatives. The covariant derivative in this case is

\eq{QCDdervative}
{
   D_{\mu} = \partial_{\mu} - i\sqrt{4 \pi \alpha_{s}} \sum_{a=1}^{8} t^{a} g_{\mu}^{a},~a=1,...,8 , 
}
where $\alpha_{s}$ is the strong coupling constant and $t^{a}$ are the generators composed of the Gell-Mann matrices. Particles which can interact via strong interaction must carry a color charge, the quarks carry the red, green or blue color, while in case of gluons there are eight different combinations of one color and one anticolor.

The coupling of colored objects is weak at short distances~(asymptotic freedom)~\cite{Gross:1973id}, but it grows with distance~(confinement)~\cite{Wilson:1974sk}. Therefore colored objects always have to be bound inside colorless hadrons, where they are quasi-free, and can never be observed separately. There are two kinds of hadrons, baryons holding three quarks of different colors and mesons composed of two quarks, one of a certain color and the second one with the corresponding anticolor.

%-asymptotic freedom - the coupling depends on the distance, , it is very weak at short distances (asymtotic freedom) nut grows in distance (confinment - bound hadron states)
%-confinment -> color must remain neutral , not possinle to separate individual quarks and gluons, always are bound in colorless hadrons - baryons of mesons.
%-hadronization - formation of colorless objects (say more) 
%-color

%INTERACTIONS
%-sm lagrangian (CERN-thesis-2017-005) -> without higgs
%-interactions (electroweak, QCD)
%TODO -19 free parameters (nine fermion masses, one scalar mass, three coupling parameters, four quark mixing parameters, higgs vacuum expectation value, strong cp violating phase)
%TODO -perturbative theory (LO, NLO)



\subsection{The electroweak symmetry breaking~\label{sec:EWbreaking}}

As discussed, the $W^{\pm}$ and $Z$ bosons are massive, but the mass term for these bosons cannot be incorporated into the Lagrangian of the electroweak interaction, because it would break the invariance of the Lagrangian under a local phase transformation. This problem is solved by the ``Brout-Englert-Higgs mechanism'', based on a phenomenon referred as ``spontaneous symmetry breaking'' of the SU(2) symmetry~\cite{Englert:1964et, Guralnik:1964eu}. The BEH mechanism introduces a new scalar doublet field $\Phi$. The Lagrangian for this field and its interactions can be expressed as 

\eq{lagHiggs}
{
    \mathcal{L}_{Higgs} =  (D_{\mu}\Phi)^{\dagger} (D^{\mu}\Phi) - V(\Phi^{\dagger}\Phi),
}
where $V(\Phi^{\dagger}\Phi)$ is the Higgs potential and $\Phi$ is a scalar doublet field defined as


\eq{doubletHiggs}
{
    \Phi = \binom{\phi^{+}}{\phi^{0}}.
}

The Higgs potential $V(\Phi^{\dagger}\Phi)$ is defined by the equation

\eq{potHiggs}
{
    V(\Phi^{\dagger}\Phi) =  - \frac{1}{2} \mu^{2}\Phi^{\dagger}\Phi + \frac{1}{4} \lambda(\Phi^{\dagger}\Phi)^{2},
}
where $\mu$ and $\lambda$ are real parameters. Because both $\mu^{2}$ and $\lambda$ are positive numbers, the potential $V(\Phi^{\dagger}\Phi)$ takesi the shape of a ``Mexican hat'', as shown in Fig.~\ref{fig:figures/mexicanHat}. The shape of the potential is such that the value of field $\Phi$ at the ground state, i.e. the vacuum expectation value~(VEV) of the field $\Phi$, is non-zero. The ground state is degenerated and can be chosen to be

    \insertFigure{figures/mexicanHat} % Filename = label
                 {0.4}       % Width, in fraction of the whole page width
                 { A shape of the Higgs potential $V$ for the complex field $\Phi$ with positive values of real parameters $\mu^{2}$ and $\lambda$. The parameter $v$ denotes the vacuum expectation value.}

\eq{solutionHiggs}
{
    \langle 0 | \Phi | 0 \rangle = \frac{1}{\sqrt{2}}\binom{0}{v} ,
}
with

\eq{vDef}
{
v = \sqrt{\frac{\mu^{2}}{\lambda}}
}
being the ground state energy ofi the field $\Phi$. Then the excitation of the field $\Phi$ can be written as follows

\eq{solutionHiggs2}
{
    \Phi = \frac{1}{\sqrt{2}}\binom{0}{v+H},
}
where $H$ is the scalar field called the Higgs boson. The vector bosons $W^{\pm}$ and $Z$ become massive via interaction with the Higgs field present in the first terms of Higgs Lagrangian of Eq.~\ref{eq:lagHiggs}. The masses of the fermions can be also generated via interaction of the fermion field  $\Psi$ with the Higgs boson $H$ by adding new Yukawa term of type $\lambda_{Y} \bar{\Psi}_{L} \Phi \Psi_{R} + h.c$ with $\lambda_{Y}$ representing the Yukawa coupling dependent on the mass of the fermion to the Lagrangian of the standard model. The full Lagrangian of the standard model in the above defined notation can then be written as

\eq{SMlag} 
{
 \mathcal{L}_{SM} = \mathcal{L}_{QCD} + \mathcal{L}_{EW} +\mathcal{L}_{Higgs} + \mathcal{L}_{Yukawa}.
} 

It can be noticed  that the Yukawa interaction with the Higgs boson flips the chirality of a fermion from left to right and vice versa. Therefore the masses of neutrinos cannot be generated in this way, as there are only left-handed neutrinos in the SM. The parameters $\mu$ and $\lambda$ of the Higgs potential are not predicted by the standard model, i.e. there are free parameters which are measured experimentally. The mass of the Higgs boson at tree level dependens on the parameter $\mu$, $m_{H} = \sqrt{2}\mu$, and therefore is as well not predicted by the SM. 

%Fermions get mass via interaction with $\Phi$ filed~\cite{Weinberg:1967tq}
%-hiigs flips the chirality, this is why neurino cannot have a mass within SM
%ELECTROWEAK symmetry breaking: TODO strat here
%Yang-Mills~\cite{Yang:1954ek} -> nonabelian gauge theory
%-new scalar field predicted by Higgs Englert and Brout in 1964~\cite{Higgs:1964pj, Englert:1964et}
%-discovery in 2012 by CMS and ATLAS
%-> in this theory gauge bosons are masless - gauge symmetry do not alow mass terms in lagrangian
%-based on spontaneous symmetry breaking principles -> apperance og goldstone~\cite{Goldstone:1961eq} bosons (one for each generator of broken symmetry?!) , goldosnes are masless spin-0
%-> do not speak about goldstone
%-non-zero ground state - vev
%-degenerated state -> infinite nr of minima on on circle of phi(1) and phi(2)- complex field -> this gives us a chance to fix phi as we want. 
%-1960s - the goldstone bosons cancel and give mass to other bosons -> generation of mass for Ws and Z bosons -> Hoggs mechanism ~\cite{Englert:1964et, Higgs:1964ia, Guralnik:1964eu,}% Higgs:1966ev}
%-parameter v is vacum expectation value - v =sqrt(-mh2/lambda)
%-Goldstone theorem -> masless states -> masless states are absorbed by the evctor bosnons
%-lambda and higgs mass must be experimentally measured
%- v and mH value? (Hoss)
%Higgs lagrangian
% TODO \section{Feynman diagrams} ?
%+perturbative theory

\subsection{Limitations of the standard model~\label{sec:shortcomings}}

Even though the standard model proved to be very successful in describing and predicting results of a large part of high energy physics experiments, phenomena which cannot be explained by the SM are observed as well. For this reason it is widely believed that the standard model is part of a larger theory. Before discussing the extension of the SM in Section~\ref{sec:SUSY}, some of the shortcomings and open questions of the standard model are first briefly described.


\textbf{The naturalness problem}

As mentioned in section~\ref{sec:EWbreaking} the mass of the Higgs boson at the tree level is $m_{H} = \sqrt{2}\mu$, but this mass must be corrected for contributions of the virtual particles. The correction to the square of the Higgs mass from the virtual fermions is quadratically divergent and depends quadratically on the fermion masses due to their Yukawa couplings. Therefore the largest correction to the Higgs mass comes from the virtual top quarks which are the heaviest fermions.

    \insertFigure{figures/fermionCorr} % Filename = label
                 {0.3}       % Width, in fraction of the whole page width
                 { A virtual fermion contributing to the mass of the Higgs boson. }

An example of a virtual fermion contributing to the Higgs mass is shown in Fig.~\ref{fig:figures/fermionCorr}. The mass of the Higgs boson can be decomposed in $m_{H,0}$ which is the mass at the tree level and $\Delta m_{H}$ which is the correction from virtual fermions

\eq{HiggsMass}
{
m_{H}^{2} \approx m_{H, 0}^{2} + \Delta m_{H}^{2}.
}

It can be shown that the correction $\Delta m_{H}$ follows relation

\eq{HiggsMassT2}
{
\Delta m_{H}^{2} \propto m_{f}^{2} \Lambda^2,
}
where $m_f$ is the fermion mass and $\Lambda$ is a cutoff on the momentum of the considered virtual particle. This cutoff is expressing up to which scale the standard model is valid and is usually taken to be the Planck mass $m_{P} \sim  10^{19}$~GeV. Knowing the term $\Delta m_{H}$, the Higgs mass can be written as

\eq{HiggsTuning}
{
m_{H}^{2} \sim m_{H, 0}^{2} + k ~m_{P},
}
where $k$ includes constants and the SM couplings. The mass of the Higgs boson was experimentally measured to be of 125~GeV~\cite{Chatrchyan:2012xdj, Aad:2012tfa}, which is orders of magnitude lower than the Planck mass. This mismatch between the order of magnitude of the Higgs and the Planck masses, for which there is no physical reason, is referred as the ``hierarchy problem''. Moreover to obtain this relatively small mass of the Higgs boson, there must be large cancellation between the two terms $m_{H, 0}^{2}$ and $k ~m_{P}$  in Eq.~\ref{eq:HiggsTuning}. The cancellation of the terms has to happen in around thirty orders of magnitude. Such fine-tuning is not natural and this problem of fine-tuning is referred as the ``naturalness problem''.

%corrwctions to Higgs boson mass
%to compute cross section, all quantum loop corrections has to be taken into account
%fermions an vector boson masses proctected from diverging by mechanism within the SM
%but no mechanism for Higgs mass: $mh^2~ mh0^2+k mPlanck^2 $ - parameters mh0, k and mPlanck a priori unrelated. But these parameters must be fine tuned in order to obtain mass of Higgs (mh<<mPlanck) -> not natural
%called hierarchy problem - no reason to expect a large hierarchy between electroweak scale and planck scale
%-34 digits
%-picture higgs loop

\textbf{Dark matter and dark energy}

Cosmological observations suggest that the ordinary matter and energy, described by the standard model, account for only $\sim 5\%$ of the total mass~(energy) of the universe~\cite{Bertone:2004pz, Gaitskell:2004gd, Bennett:2012zja}. The remaining $\sim$95\% is divided between the dark matter~$\sim$27\% and dark energy~$\sim$68\%.

Dark matter has been observed only indirectly as it does not emmit any radiation. Therefore if it an elementary particle, it has no color or electric charge. The first observation supporting the dark matter existence came from the measurement of rotation curves of galaxies~\cite{Zwicky:1937zza, Rubin:1980zd}. These curves show the dependency of the star orbital velocity on the distance of the star from the center of galaxy. Rotation curves can be theoretically computed and it was found out, that the measured and theoretical curves agree at short distances. With increasing distance the observed curves remain constant, while theoretically they should decrease~\cite{Bertone:2004pz}. This phenomenon can be explained by the presence of a halo of new particles, which interact by gravitational force. It was also found out that these particles must be stable and non-relativistic~(cold). Shortly it was realized that there is no candidate for dark matter within the standard model.

Dark energy arises from the need of the cosmological constant in the Einstein equation. Without it, it would not be possible to explain the accelerated expansion of the universe. Dark energy can be interpreted as a vacuum energy, but there is a mismatch of $\sim$120 orders of magnitude between the vacuum energy estimation from the cosmological constant and quantum field theory calculations.

%-5 percent of baryonic matter
%-27 percent of dark matter
%-68 dark energy
%-measuremnt of rotation curves of galaxies - first dark matter hypothesis
%-gravitational interaction, but not electromagnetic -> dark matter
%-from observations several constraints on dark matter - not short -lived and not baruonic, gravitationally interacting, low kinetic energy (cold -> it cannot be neutrino)
%-> no good candidate within the SM
%From cosmological observations we expect dark matter mass of order of 100~GeV
%- how many percent?
%-microwawe background
%-no radiation of DM -> no collor, no electric charge

%3)Dark energy
%-cosmological constant (lambda) in einsteins equation necessary to explain the observed expansion of universe
%-> cosmologica constant can be interpreted as a vacuum energy

\textbf{Other shortcomings of the standard model}

Another problem of the SM is the asymmetry between matter and antimatter. In our universe there is abundance of matter, even though the matter is supposed to have been produced in the same amount as antimatter during the big bang.  Weak interaction violates the combined charge conjugation and parity~(CP) symmetry~\cite{Kobayashi:1973fv}, which could lead to the asymmetry between matter and antimatter. But the CP violation itself cannot describe that large asymmetry and therefore there is no mechanism within the SM, which could explain this asymmetry between matter and antimatter.

Then, as mentioned, there are only left-handed neutrinos in the standard model and therefore their masses cannot be generated via interaction with the Higgs boson, which flips left-handed fermions to right-handed ones. But neutrino oscillations~\cite{Fukuda:1998mi, Ahmad:2001an} have been observed and are only possible if neutrinos are massive. This fact itself is not conclusive argument in favor of physics beyond the SM, but it needs to be understood how the mass of neutrinos can be generated.

Another argument to try to extend the the standard model is to include the gravitational force, which is not presently part of the standard model and therefore the SM is only effective theory. Currently, there are big efforts to combine general relativity and quantum field theory to formulate a theory of quantum gravity.  

Finally, in the standard model, the coupling constants are dependent on energy. At higher energies the constants of weak, electromagnetic and strong interactions become of similar strengths, giving a hope for unification of these interactions at large energy scale. There is no certainty that the forces are unified, but based on previous unifications of e.g electricity and magnetism, or electromagnetic and waek interaction, the unification of other forces is expected. But this unification is not achieved within the SM and to unify forces, a theory beyond the SM is needed.
 
%4)Matter-antimatter assymetry
%-matter and antimatter should be produced in smae amount at big bang
%-but our world dominated by matter
%5)Neutrino masses
%-neutrinos oscialte from one flavour to other -> this can only happen when neutrinos are masive and have different mass states than flavour states

%6)Strong CP phase
%strong QCD lagrangian introducing the phase theta - close to zero, despite the theoreticla arguments that it should not be like this

%7)Quantum gravity
%-gravity not described by SM
%-desired to unify general relativity with QFT
%8)Unification of forces
%-possibility to unify all interactions
%-> of couplingconstants

%9)open questions
%-in SM large differences between quarks
%-why there should be three fermion families

%remaining stuff:
%-19 free parameters (nine fermion masses, one scalar mass, three coupling parameters, four quark mixing parameters, higgs vacuum expectation value, strong cp violating phase)
%-perturbative theory (LO, NLO)

\section{Supersymmetry~\label{sec:SUSY}}

To address mentioned shortcomings of the standard model, many extensions of the SM were proposed over the past decades. In general there are everal possibilities how to extend the SM and formulate beyond the standard model theories. The extension can be achieved by, for example, adding ``extra dimensions''~\cite{Patrignani:2016xqp}, adding new symmetries, postulating new particles and proposing new interactions. An example of theory adding new symmetries is SUperSYmmetry~(SUSY)~\cite{Martin:1997ns}, which became very popular due to its capability to solve many issues of the SM. The SUSY started to be developed since 1970's around an idea of introducing a new symmetry between fermions and bosons.
  
%- said before: SM works fine, but we need to extend it -> we can add either additional symetries, space-time dimensions or field content
%- one of the possibility how to extend is susy 
%- it addresses many issues of SM

The supersymmetry introduces symmetry operator $Q$, which acts on fermions~$f$ and bosons~$b$ in following way:

\eq{SUSYop}
{
Q \mid f \rangle \to \mid b \rangle \to  , \; Q \mid b \rangle \to \mid f \rangle
}

The operator $Q$ changes the spin of the particle by $1/2$, therefore transforms fermion to boson and vice versa, but do not change any other quantum number or particle properties. This symmetry $Q$ yields to the symmetry within which each SM fermion have bosonic SUSY partner with the same quantum numbers except of spin, and similarly each SM boson has associated fermionic SUSY partner. The supersymmetric partners of the SM particles are referred as ``sparticles''. To build a theory which is able to reproduce the standard model, the operator $Q$ must satisfy following (anti)commutator relations~\cite{Haag:1974qh, Coleman:1967ad}:

\eq{comutators}
{
\{Q,Q^{\dagger}\} = P^{\mu}, \; \{Q,Q\} =\{Q^{\dagger},Q^{\dagger}\}= 0, \; [P^{\mu}, Q] = [P^{\mu}, Q^{\dagger}] = 0,
}

where $P^{\mu}$ is the four-momentum operator. It can be noticed that $-P^{2}$, which is the mass-squared operator, commutes with both $Q$ and $Q^{\dagger}$ and therefore a particle and its SUSY partner have the same mass. But as no SUSY partners have been observed, this symmetry must be broken. Many symmetry breaking scenarios can be built, leading to many possible realizations of supersymmetric theory. 

The naming convention for SUSY particles adds prefix ``s'' to the SUSY partners of fermions, therefore sfermions are bosons. The SUSY partners to bosons get suffix ``ino'', for example the SUSY partner of gluon is gluino, which is fermion. The symbols of superpartners have a tilde.

There are many realizations of the supersymmetry, but in further text only Minimal Supersymmetric standard model, which is the most used one, is considered.


%------------------------------------------------------------------------------------------------------------------------------------------
%-around 70's
%-Golfand and Likhtman -> new symmetry Q -> Q|f> -> |b>; Q|b> -> |f> (transformation) -> later Haag, Lopuszanski and Sohnius said that such symmetry corresponds to supersymmetry
%-to each fermion a boson  with same quantum numbers (except of spin)
%-particles in supermulitples, where there is same number of fermionic and boisonic degrees of freedom
%-mass degeneration of particles in supermultiplet (from commutation relation of Q)
%-particles in supermultiplet have same quantum, numbers under the SU(3)xSU(2)xU(1) transformation
%-supermultiplet SM particle + susy partner, just differening by spin (1/2) -> bit more complicated I guess
%-parters of fermions are sfermions, and partner of bosons are inos, tilde for susy particles
%-partner of praticle is superpartner and they form superfield
%-spin differs by 1/2
%-same interactions of SUSY particles as the SM ones (for example only scalar partners of left handed fermions interact with partners of W boson)
%-two SUSY Higgs boson doublets are needed (in order to keep the theory renormalizable)
%-superpartner should have the same mass -> not observed -> susy must be broken (for now we just add a term into the lagrangian)
%-motivation:
%	-solve hierarchy problem (superpartnes have equal masses and cancel the loop corrections) - in case of "soft breaking" susy prevents the quadratic divergencies and there are only logarithmic + small fine tuning
%	-> naturalness of susy related to the mass difference between particle and its superpartner (Q: then if the susy partner of eg electron is very heavy does not it induce the divergencies? )

%-susy breaking - not much known about its mechanism, there are several hypothesis (models)
%-susy solves naturalnes problem, bosons opposite sign of corerction to delta mass -> fine tuning can be removed, if the coupling constants are the same, just differ by sign (and it is actually true)
%	-> the loop diagram
%-susy soles dm candidate - lsp
%-susy solves the unification - susy modifies the energy evolution of coupling constants 

%-constraints on lsp from dm relic density~\cite{Ade:2015xua}
%-commputation relations of susy operators?
%------------------------------------------------------------------------------------------------------------------------------------------

\subsection{Minimal Supersymmetric Standard Model}

The Minimal Supersymmetric Standard Model~(MSSM)~\cite{Martin:1997ns} is a supersymmetric extension of the SM which adds a minimum of new particles. It also does not introduce any additional gauge interactions. The Lagrangian of the new theory can be written as

\eq{lagSUSY}
{
    \mathcal{L} =  \mathcal{L}_{MSSM} +  \mathcal{L}_{soft} = \mathcal{L}_{free} + \mathcal{L}_{int} + \mathcal{L}_{soft}    ,
}

where $\mathcal{L}_{MSSM}$ is the supersymmetric part of the Lagrangian and term $\mathcal{L}_{soft}$ introduces ``soft'' breaking of the supersymmetry. The breaking of the supersymmetry must be mild in order to end up with only small tuning of the Higgs boson mass. The large advantage of supersymmetry is, that in its unbroken form, it is able to solve the hierarchy problem. The virtual correction from the fermion loop $\Delta m_{H}^{2}$ to the Higgs boson mass as shown in Fig.~\ref{fig:figures/fermionCorr2} is  in context of MSSM proportional to

\eq{HiggsMassSUSY}
{
\Delta m_{H}^{2} \propto (m_{f}^{2} - m_{b}^2) \mathrm{ln} \left(\frac{\Lambda}{m_{b}}\right),
}

where $m_{f}$ is mass of fermion, $m_{b}$ mass of boson and $\Lambda$ a cutoff. This term is zero in case that mass of particle and its partner is the same and therefore no fine-tuning is needed. The virtual contributions to the Higgs mass in context of MSSM are visualized in Fig.~\ref{fig:figures/fermionCorr2}, within which the fermion and sfermion loop contributions are of same value but opposite sign and therefore cancel. In the broken theory the masses are not the same, the term $\Delta m_{H}^{2}$ has logarithmic divergence and larger the mass difference between the particle and sparticle, more fine-tuning is needed. Therefore to preserve the naturalness of the theory, it is required that it is broken only slightly and the supersymmetric partners do not have mass hugely larger than the SM particles. Especially, as the largest  contribution to the Higgs mass comes from the top quark, the mass difference between the top quark and top squark should be reasonably small, leading to a constraint on the top squark mass to be of order of around 1~TeV~\cite{Barbieri:1987fn}. The newly explored natural SUSY models permit the stop masses up to 3~TeV~\cite{Baer:2016bwh}.

    \insertFigure{figures/fermionCorr2} % Filename = label
                 {0.3}       % Width, in fraction of the whole page width
                 { A virtual fermion and boson contributing to mass of the Higgs boson.}

The spectrum of SUSY partners can be seen in Tab.~\ref{tab:SUSYspectrum} in column ``Gauge Eigenstates''. In the MSSM there are two Higgs doublets in order to avoid gauge anomalies and to be able to have Yukawa coupling of both up- and down- type quarks. Both Higgs doublets have non-zero vacuum expectation value. As a consequence of electroweak symmetry breaking, the charged fields $W_{1,2}$ mix into positive and negative winos~($\tilde{W}^{\pm}$) and the neutral fields $B_{0}$ and $W_{3}$ mix into zino~($\tilde{Z}$) and photino ($\tilde{\gamma}$). The higgsinos mix with the flavor eigenstates of the SUSY partners of $SU(2) \otimes U(1)$ gauge bosons to give rise to the gaugino mass eigenstates. The two charged higgsinos~($\tilde{H}_{u}^{+}~\tilde{H}_{d}^{-}$) mix with $\tilde{W}^{\pm}$ to form two charginos~($\tilde{\chi}_{1,2}^{\pm}$) in negative and positive version. The four neutralinos $\tilde{\chi}_{1,2,3,4}^{0}$ are mix of zino, photino and two neutral higgsinos~($\tilde{H}_{u}^{0}~\tilde{H}_{d}^{0}$). There is also mixing between the left and right component of the third generation of squarks and stau, due to the large mass of third generation of quarks and tau and therefore large Yukawa coupling which is proportionate to $m/v$, where $m$ is the fermion mass and $v = \mu/\sqrt(\lambda)$ are the parameters of the Higgs potential. The mass eigenstates can be as well found in Tab.~\ref{tab:SUSYspectrum}. The subscripts $L,R$ denote left and right chiral components of the fermion field.

\begin{table}[h]
\begin{center}
\begin{tabular}{|c|c|c|c|}
\hline
Names & Spin  & Gauge Eigenstates & Mass Eigenstates  \\
\hline
        &   & $\tilde{u}_{L}~\tilde{u}_{R}~\tilde{d}_{L}~\tilde{d}_{R}$  & (same) \\
squarks & 0 & $\tilde{s}_{L}~\tilde{s}_{R}~\tilde{c}_{L}~\tilde{c}_{R}$  & (same) \\
        &   & $\tilde{t}_{L}~\tilde{t}_{R}~\tilde{b}_{L}~\tilde{b}_{R}$  & $\tilde{t}_{1}~\tilde{t}_{2}~\tilde{b}_{1}~\tilde{b}_{2}$ \\
\hline
         &   & $\tilde{e}_{L}~\tilde{e}_{R}~\tilde{\nu}_{e}$  & (same) \\
sleptons & 0 & $\tilde{\mu}_{L}~\tilde{\mu}_{R}~\tilde{\nu}_{\mu}$  & (same) \\
         &   & $\tilde{\tau}_{L}~\tilde{\tau}_{R}~\tilde{\nu}_{\tau}$  & $\tilde{\tau}_{1}~\tilde{\tau}_{2}~\tilde{\nu}_{\tau}$ \\
\hline
neutralinos & 1/2 & $\tilde{B}_{0}~\tilde{W}_{3}~\tilde{H}_{u}^{0}~\tilde{H}_{d}^{0}$  & $\tilde{\chi}_{1}^{0}~\tilde{\chi}_{2}^{0}~\tilde{\chi}_{3}^{0}~\tilde{\chi}_{4}^{0} $ \\
\hline
charginos & 1/2 & $\tilde{W}_{1,2}~\tilde{H}_{u}^{+}~\tilde{H}_{d}^{-}$  & $\tilde{\chi}_{1}^{\pm}~\tilde{\chi}_{2}^{\pm} $ \\
\hline
gluino & 1/2 & $\tilde{g}$  & (same) \\
\hline
gravitino & 3/2 & $\tilde{G}$  & (same) \\
\hline
\end{tabular}
\caption[Table caption text]{The gauge and mass eigenstates of the sparticles~\cite{Martin:1997ns}. The mixing of the first two generations of sfermions is considered to be negligible. }
\label{tab:SUSYspectrum}
\end{center}
\end{table}

In the total MSSM Lagrangian, in principle there could be terms which would violate baryon or lepton number conservation. Under such conditions, the proton could decay, what was not experimentally observed~\cite{Nishino:2009aa}. To avoid such violation, a symmetry referred as ``R-parity''~($P_{R}$) is considered. The $P_{R}$ is defined as

\eq{Rparity}
{
P_R=(-1)^{3(B-L)+2s },
}

where $B$ is a baryon number, $L$ is a lepton number and $s$ spin of the particle. All SM particles have $P_{R}=1$ and their SUSY partners $P_{R}=-1$. The consequence of R-parity conservation is, that sparticles can be produced only in even number and each of them can decay only into odd number of sparticles. Also the lightest supersymmetric particle~(LSP) must be stable. The LSP is a dark matter candidate and it can be either the lightest neutralino, sneutrino or gravitino, depending on the specific realization of the MSSM. 

%From the cosmological observations of the relic density~\cite{Ade:2015xua}, constraint on the mass of the dark matter mass to be of order of 100~GeV was imposed.


Another advantage of the MSSM is that the energy dependence of the coupling constants of the SM interactions is modified as a result of introducing sparticles. Consequently these coupling constants can be unified at large energy scale referred as ``GUT scale''. Within the supersymmetry it is also possible to create models of supergravity.

%\textbf{conservation of R-party by construction }
%	-> pair production of sparticles
%        -> decay only to odd nr of sparticles
%        -> LSP is stable -> dark matter candidate
%-in susy lagrangian there can be interaction between susy particles and sm particles -> there can be lepton or baryon number valiation
%-but this was restricted by SM , because proton could decay, wat was not observed~\cite{Nishino:2009aa}
%-to avoid this: R-parity requirement added:   , where B is baryon number, L is lepton numbver and s is spin of particle
%-it must be conserved
%-all susy particles negative r-parity, sm ones positive
%	-> lsp is stable (interacts only weakly - good dm candidate)
%	-> decay to odd number of susy particles
%	-> at colliders, susy particles produced in pair
%-there are rpv models, but the lsp is not stable
%LSP, unification of forces and in some conditions it can describe quantum gravity

%TODO 

The MSSM brings 105 new free parameters on top of the 19 free parameters in the standard model. Most of these parameters come from the supersymmetry breaking part of the Lagrangian. It is not possible to search for SUSY in such a large parameter-space and therefore the collider experiments usually search for SUSY in context of simplified models, in case of CMS collaboration referred as ``Simplified Model Spectra'', which reduce the number of free parameters.


%------------------------------------------------------------------------------------------------------------------------------------------
%-most used SUSY relization is MSSM -> minimal -> adding the minimum number of fields~(particles) to the SM to become supersymmetric
%- no additional gauge interactions
%-table of supermultiplets?! (CERN-THESIS-2015-390) -> these are the particles before "mixing"
%-adding sfermions and gauginos - left and right handed fermions -> e.g two selectrons
%-for Higgs more complicated- one higgsino is not enough, but second SU(2) doublet is needed to avoid a gauge anomaly. (give mass to up and down type of quarks?) -> even there must be two SM doublets
%-both higgs doublets have to have non-zero vev
%-large mixing between sfermions states (because of large Yukawa coupling which is dependent on mass?), mixing of the second and first generation smaller, du to smaller Yukawa
%-it is likely that right-handed states are lighter than left-handed
%-winos and binos mix after ew symmetry breaking -> winos, zino, photino , but they mix to give mass eingenstates
%	-4 neutralinos which are mix of neutral bino, wino and higgsinos
%	-two charginos (each can be negative or positive) - which are mix of charged winos and higgsinos
%-only gluinos do not mix to give some mass eingenstates
%-mass eigenstates do not have to be flavor eigenstates
%-combinations of electroweak gauginos and higgsions make charginos and neutralinos
%-mixing between left and right superpartners 
%-spectrum of sparticles
%-MSSM - more than 100 new parameters than in SM
%	-majority from symmetry breaking
	%-it can be constrained by using pMSSM (with 19 free parameters), where no assumptions on the breaking mechanism 
%-105 mssm + 19 sm parameters


%\textbf{pmssm}
%-> too many parameters - problem for phenomenological and experimental models
%-> pMSSM - phenomenological MSSM -> reduction of number of parameters by assuming
%	- there is no new source of CP vilation
%	-lightest neutralino is the LSP
%	-other assumptions on the sfermion masses, trilinear couplings and flavor violation
%	->reduction of parameters to 19
%		-higgsino mass parameter and pseudo-scalar higgs mass 
%		-ration of Higgs vauum expectation values
%		-soft gaugino masses (bino, wino, gluino
%		-sfermion masses
%		-trilinear couplings

\subsection{Simplified Model Spectra}

The parameter-space of the MSSM can be reduced by fixing the sparticles production cross section, their decay modes and branching ratios and the mass hierarchy between sparticles. Such restrictions give only limited number of possible MSSM models called ``Simplified Model Spectra''~(SMS)~\cite{Alves:2011wf, Alwall:2008ag, Chatrchyan:2013sza}.

This thesis focuses on the search for the top squark~(stop) and therefore only specifics of the stop production and its decay within the SMS are described in larger detail. The lighter stop~($\tilde{t}_{1}$) is expected to be lighter than gluino and therefore the production of the stop can be mediated via gluino or a stop pair can be produced directly in the pp interaction. An example of a direct stop pair production can be seen in the left part of Fig.~\ref{fig:figures/stopProd}, the gluino mediated production is depicted in the right part of the same figure.


    \insertTwoFigures{figures/stopProd} % Filename = label
                 {figures/T2tt}
                 {figures/T5tttt} % Filename = label
                 {0.45}       % Width, in fraction of the whole page width
                 {(left) An example of a Feynman diagram of the direct stop pair production. In this diagram both stops decay to the top quark and LSP. (right)  An example of a Feynman diagram of the gluino mediated stop production. In this case gluinos are produced in pairs and both decay to a top quark and top squark. Both stops, as previously, decay to the top quark and LSP. } % Caption

The stops can decay directly to the top quark and LSP which is the lightest neutralino, or via intermediary chargino as shown on example in Fig.~\ref{fig:figures/T6bbWW}. The mass of the chargino is fixed by the relation

\eq{charginomass}
{
m_{\tilde{\chi}_{1}^{\pm}} = x m_{\tilde{t}_{1}} + (1-x) m_{LSP},
}

where $m_{\tilde{\chi}_{1}^{\pm}}$ is the chargino mass, $m_{\tilde{t}_{1}}$ is the stop mass, $m_{LSP}$ is the neutralino mass and $x$ is a fixed fraction between 0 and 1, which is usually chosen to be 0.25, 0.5 or 0.75. 

    \insertFigure{figures/T6bbWW} % Filename = label
                 {0.45}       % Width, in fraction of the whole page width
                 {An example of a Feynman diagram with intermediary chargino. In this diagram both stops decay to a bottom quark and chargino. Both charginos then decay to W-boson and LSP.   } % Caption

The directly produced stop pair decays eventually into two b-quarks, two W-bosons and two neutralinos. Therefore in the final state two b-jets, two neutralinos and leptons and/or jets from the W-bosons are expected. The neutralinos interact only weakly and escape the detector unmeasured. Although undetected particles can be spotted by presence of large amount of missing transverse energy, giving such SUSY signal unique experimental signature compared to the standard model processes.

In general, results of the searches in context of the SMS are exclusion limits on the sparticle masses in plane of the two of them, in the described case in the plane of stop versus LSP mass.

%lhc reference~\cite{Alves:2011wf, Alwall:2008ag}
%cms reference~\cite{Chatrchyan:2013sza}
%-considering small number os sparticles -> others are too heavy -> considering mass hierarchy and branching ratios
%-limits in planes of sparticles
%-part of the parameter space of the MSSM
%-gluinos heavier than the lighter stop
%-described set of particles, their possible production and decay chain
%-in simplified models considering only production process of the primary particles?
%-each aprticle diracte or cascade decay
%-each decay ends with lsp (neutralino, gravitino)
%-relationship between particle masses, production cross section, decay modes, branching ratios (usually 100\%)
%-stop decay channels
%-mass of intermediate particle
%$m_int = x m_mother + (1-x)m_LSP $
%T2 (T6) prefix - qaurk-squark production
%T2, T2bb, T2tt, T6ttww, T2bW
%	-each squark - two body decay to lighter flavours and chargino/neutralino


%\textbf{stop (third generation)}
%-stops gives the largest conbtribution to the higgs mass
%-there should not be much tuning in the higgs mass -> constraint on stop mass to be at energy level of order of 1~TeV (max) -> then fine tuning of the order of 10\%
%-phase space of stop decay?!
%-stop 1 or sbottom 1 probably the lightest squarks  -> because of mixing between R and L 
%-motivation and signature
%-lsp -large missing ET -> leaves detector undetected
%-derct stop production or gluino mediated


\subsection{Results on stop searches in Run I}

%-combined 1 and 2 lep: http://cms-results.web.cern.ch/cms-results/public-results/publications/SUS-14-015/index.html
%-one lep alone: http://cms-results.web.cern.ch/cms-results/public-results/publications/SUS-13-011/index.html
%-fully hadronic: http://cms-results.web.cern.ch/cms-results/public-results/publications/SUS-13-023/index.html , http://cms-results.web.cern.ch/cms-results/public-results/publications/SUS-14-001/index.html
%-summarized results: https://twiki.cern.ch/twiki/bin/view/CMSPublic/SUSYSMSSummaryPlots8TeV

In Run~I many analyses targeting the stop quark production were performed. The naturalness constraint and the mixing of gauge eigenstates of the stops suggest, that the lightest stop $\tilde{t}_{1}$ could be relatively light, about 1~TeV and therefore accessible at the LHC energies. The combined results of stop decaying either to $t \tilde{\chi}_{1}^0$ or $c \tilde{\chi}_{1}^0$ are shown in Fig.~\ref{fig:figures/T2tt2015} . This figure shows exclusion limits on the stop and LSP mass in three different kinematic regions. The region where the stop mass is smaller than mass of LSP is kinematically forbidden, but once the masses satisfy $m_{\tilde{t}} > m_{LSP}$ but $m_{\tilde{t}} - m_{LSP} = m_{W}$ the stop can decay either to $c \tilde{\chi}_{1}^0$ or to four bodies via off-shell top quark and W-boson. Once  $m_{\tilde{t}} - m_{LSP} > m_{W}$ but also  $m_{\tilde{t}} - m_{LSP} < m_{t}$, the three body decay via off-shell top quark is possible. When  $m_{\tilde{t}} - m_{LSP} > m_{t}$  the stop can decay to $t \tilde{\chi}_{1}^0$. The analyses in the Fig.~\ref{fig:figures/T2tt2015} do not only differ by the coverage of the kinematic regions, but also by the analysis strategy, for example they require different number of leptons in the final state. In general, less leptons in the final state, larger the production cross-section. But on the other hand analyses with more leptons suffer from smaller SM background.  The strongest exclusion limit on the stop mass was put by analysis CMS-SUS-13-023 searching for fully hadronic final states. This analysis in context of SMS excludes top squark masses up to 755~GeV when the neutralino mass is below 200~GeV.

%-analyses preformed in 0,1 and 2-lep
%	- hadronic suffer with large backgroun but high production cross section
%	-while petonic low background but low cross section
%	-three kind of regions + kinematically forbidden when amss of the stop is lower than mass of neutralino
%        -for mass diffference between stop and neutralino smaller than E, decay to c neutralino, othervise to top nutralino
%        -different multiplicity of leptons and difefrent search techniques
%	-going to build on  SUS-13-011
%-emphasise that stop should be light?

    \insertFigure{figures/T2tt2015} % Filename = label
                 {0.7}       % Width, in fraction of the whole page width
                { The exclusion limits on the stop and LSP mass for the $t \tilde{\chi}_{1}^0$ and $c \tilde{\chi}_{1}^0$ decay modes. The results from several Run~I analyses are combined in this plot~\cite{website:SUSYresRunI}.   } % Caption

\newpage
