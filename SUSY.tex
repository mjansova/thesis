\chapter{SUSY}

the particle physics is described by the SM which is formulated within quantum field theory model
-brief intorducation to standard model its particles and interactions
-then electrowek breakinga nd higgs mechanism
- then shortcommings of SM, need for extension
-the most popular extenion of SM is supersymmetry -> symmetry between fermions and bosons



\section{QFT}

-fields and particles
-lagrangian
-feynman diagram
-probably move that to SM : interactions

\section{SM and its shortcomings}


-sm intro, more about sm can be seen in this book~\cite{9783527406012} %griffiths
-renormalizable quantumn field theory, derived from gauge symmetries ~\cite{tHooft:1971qjg, Weinberg:1967tq}
-formulation started in 1960s, completed in 2012 with discovery of higgs boson~\cite{Chatrchyan:2012xdj, Aad:2012tfa}
-describes all known particle and fundamental interactions (except of gravity)  

SM is field theory which is renormalizable and built on (local) symmetries principles.
The SM is QFT  within the particles and interactions are quantized fields
-there is QCD describing strong interactions and QED describing quantum electrodynamics, unification of weak and elmag

%In QFT a particle is entity localized at position (x,y,z) depends on time. A field is not localized wich spreads over a space-time region. The goal gof the QFT is to compute 

In classical mechanics, the motion is described by Euler-Lagrange equation~\cite{9783527411887}. This equation has only time derivatives and in order to build a relativistic theory, the space derivatives have to be added and therefore Eulare-Lagrange iquation is generalized into formulation

\eq{EL}
{
 \partial_{\mu}(\pdv{\mathcal{L}}{(\partial_{\mu}\phi_{i})}) = \pdv{\mathcal{L}}{\phi_{i}}
}


The basic building block of  QFT is a lagrangian which depends on fileds $\phi_{i}$, their space-time drivatives. In case of scalar field (spin-0) he lagrangian can be written in form 

\eq{kglagrangian}
{
 \mathcal{L} = \frac{1}{2}(\partial_{\mu} \phi)}(\partial^{\mu} \phi) - \frac{1}{2} (\frac{mc}{\hbar})^{2} 
}

which after plugging into the Euler-Lagrange equation~\ref{eq:EL} gives Klein-Gordon equation

\eq{kgEq}
{
 \partial_{\mu)}\partial^{\mu}_\phi) + (\frac{mc}{\hbar})^{2} \phi = 0
}

 describing a scalar particle. Similarly the field of half spin particle is a spinor field $\psi$. The solution of Euler-Largrange equation wrt to $\bar{\psi}$ with Lagrangian


\eq{dirlagrangian}
{
 \mathcal{L} = i \hbar c \bar{\psi} \gamma^{\mu} \partial_{\mu} \psi -mc^{2} \bar{\psi} \psi
}

is Dirac equation describing half-spin particles.

\eq{dirEq}
{
  \gamma^{\mu} \partial_{\mu} \psi - \frac{mc}{\bar{h}} = 0.
}

Similarly, equations for particles with different spins can be derived.

In the equations above, only non-interacting fields are described. The Dirac Lagrangian~\ref{dirlagrangian} is invariant under global phase space transformation~($\psi \to e^{i\theta} \psi$), but is not invariant under local phase transformation $\psi \to e^{i\theta (x)} \psi$. To preserve invariance of Lagrangian~\ref{dirlagrangian}, the term $-(q\bar{\psi}\gamma^{\mu}\psi)A_{\mu}$, with $A_{\mu}$ which transforms as $A_{\mu} \to A_{\mu} + \partial_{\mu} \lambda$ can be added. The $\A_{\mu}$ is a new vector~(spin-1) field. To obtain full Lagrangian, also  the free field Lagrangian for vector field must be also added to Lagrangian~\ref{dirlagrangian}. The summed Lagrangian is locally invariant only in case when field $A_{\mu}$ is masless and it generates electrdynamics, where the field $\A_{\mu}$ corresponds to the electromagnetic potential.

Thei global transformation of potential $\psi$ can be uderstood as multiplication by unitary martux $U$ $\psi \to U \psi$, which was in given example $1 \times \1$. The explained example of quantum electrodynamics is $U(1)$ theory as the group of such matrcies is U(1). Similar strategy of global and local phase invariance of Lagrangian can be appliead on other groups, which was found to be the way how to generate interaction of Standard Model.

PARTICLES:
-wihinn SM two kinds of partciles - half spin fermions and integer spin bosons
-interaction -> exchange of boson
-sm group SU(3)C x SU(2)L x U(1)Y -> strong and electroweak interactions -> 8 gluons + 2W bosons+Z+gamma (fields of G, W and B -> W and B mix to create gamma and Z)
-fermions can be divided into  quarks which participate in strong end electroweak interactions and leptons which undergo only weak interactions

    \insertFigure{figures/SMparticles} % Filename = label
                 {0.99}       % Width, in fraction of the whole page width
                 { List of SM particles. }


INTERACTIONS
-sm lagrangian (CERN-thesis-2017-005) -> without higgs
-interactions (electroweak, QCD)
-19 free parameters (nine fermion masses, one scalar mass, three coupling parameters, four quark mixing parameters, higgs vacuum expectation value, strong cp violating phase)
-perturbative theory (LO, NLO)


ELECTROWEAK symmetry breaking:
Yang-Mills~\cite{Yang:1954ek} -> nonabelian gauge theory
-> in this theory gauge bosons are masless - gauge symmetry do not alow mass terms in lagrangian
-symmetry breaking -> apperance og goldstone~\cite{Goldstone:1961eq} bosons (one for each generator of broken symmetry?!) , goldosnes are masless spin-0
-1960s - the goldstone bosons cancel and give mass to other bosons -> generation of mass for Ws and Z bosons ~\cite{Englert:1964et, Higgs:1964ia, Guralnik:1964eu, Higgs:1966ev}
Higgs lagrangian

\eq{lagHiggs}
{
    \mathcal{L}_{Higgs} =  (D_{\mu}\Phi)^{\dagger} (D^{\mu}\Phi) - V(\Phi),
}

where ``Maxican hat potential'' is (if $\mu^2> 0$ )

\eq{potHiggs}
{
    V(\Phi) =  -\mu^{2}\Phi^{\dagger}\Phi + \lambda(Phi^{\dagger}\Phi)^{2},
}

$\Phi$ is a complex field. It is self-interacting SU(2)L doublet, weak hypercharge 1/2


\eq{doubletHiggs}
{
    \Phi = \binom{\phi^{+}}{\phi^{0}} ,
}

the minimum of potential is at value not SU(2)L x U(1)Y invariant -> $\Phi$ has non-zero vacuum expectation value


\eq{solutionHiggs}
{
    \langle \Phi \rangle_{0} = \frac{1}{\sqrt{2}}U(x)\binom{0}{v} ,
}

with

\eq{vDef}
{
v = \sqrt{\frac{\mu^{2}}{\lambda}}
}

$U(x)$ is unitray transformation that rotates between the dengenerated fields.

The Higgs potential gives mass to bosons via first term of Eq.~\ref{eq:lagHiggs}.

\eq{solutionHiggs2}
{
    \Phi(x) = \frac{1}{\sqrt{2}}U(x)\binom{0}{v+H(x)} ,
}

-> Higgs boson appears when moving from minimum of potential ~\ref{Higgs:1964ia, Higgs:1964pj}

Fermions get mass via interaction with $\Phi$ filed~\cite{Weinberg:1967tq}


issues

1)hierarchy problem
corrwctions to Higgs boson mass
to compute cross section, all quantum loop corrections has to be taken into account
fermions an vector boson masses proctected from diverging by mechanism within the SM
but no mechanism for Higgs mass: $mh^2~ mh0^2+k mPlanck^2 $ - parameters mh0, k and mPlanck a priori unrelated. But these parameters must be fine tuned in order to obtain mass of Higgs (mh<<mPlanck) -> not natural
called hierarchy problem - no reason to expect a large hierarchy between electroweak scale and planck scale

2)dark matter
-measuremnt of rotation curves of galaxies - first dark matter hypothesis
-gravitational interaction, but not electromagnetic -> dark matter
-from observations several constraints on dark matter - not short -lived and not baruonic, gravitationally interacting, low kinetic energy (cold -> it cannot be neutrino)
-> no good candidate within the SM
From cosmological observations we expect dark matter mass of order of 100~GeV

3)Dark energy
-cosmological constant (lambda) in einsteins equation necessary to explain the observed expansion of universe
-> cosmologica constant can be interpreted as a vacuum energy

4)Matter-antimatter assymetry
-matter and antimatter should be produced in smae amount at big bang
-but our world dominated by matter

5)Neutrino masses
-neutrinos oscialte from one flavour to other -> this can only happen when neutrinos are masive and have different mass states than flavour states

6)Strong CP phase
strong QCD lagrangian introducing the phase theta - close to zero, despite the theoreticla arguments that it should not be like this

7)Quantum gravity
-gravity not described by SM
-desired to unify general relativity with QFT

8)Unification of forces
-possibility to unify all interactions

9)open questions
-in SM large differences between quarks
-why there should be three fermion families


\section{BSM physics}

BSM theories
-SM works fine, but we need to extend it -> we can add either additional symetries, space-time dimensions or field content

SUSY
-around 70's
-Golfand and Likhtman -> new symmetry Q -> Q|f> -> |b>; Q|b> -> |f> -> later Haag, Lopuszanski and Sohnius said that such symmetry corresponds to supersymmetry
-to each fermion a boson  with same quantum numbers (except of spin)
-partner of praticle is superpartner and they form superfield
-spin differs by 1/2
-superpartner should have the same mass -> not observed -> susy must be broken (for now we just add a term into the lagrangian)
-motivation:
	-solve hierarchy problem (superpartnes have equal masses and cancel the loop corrections) - in case of "soft breaking" susy prevents the quadratic divergencies and there are only logarithmic + small fine tuning
	-> naturalness of susy related to the mass difference between particle and its superpartner (Q: then if the susy partner of eg electron is very heavy does not it induce the divergencies? )
-MSSM
-most used SUSY relization is MSSM -> minimal -> adding the minimum number of fields to the SM to become supersymmetric
-adding sfermions and gauginos - left and right handed fermions -> e.g two selectrons
-for Higgs more complicated- one higgsino is not enough, but second SU(2) doublet is needed to avoid a gauge anomaly.
-mass eigenstates do not have to be flavor eigenstates
-combinations of electroweak gauginos and higgsions make charginos and neutralinos
-mixing between left and right superpartners 
-spectrum of sparticles
-conservation of R-party by construction $R=(-1)^{2S+3B+L} $
	-> pair production of sparticles
        -> decay only to odd nr of sparticles
        -> LSP is stable -> dark matter candidate
-MSSM - more than 100 new parameters than in SM
-> too many parameters - problem for phenomenological and experimental models
-> pMSSM - phenomenological MSSM -> reduction of number of parameters by assuming
	- there is no new source of CP vilation
	-lightest neutralino is the LSP
	-other assumptions on the sfermion masses, trilinear couplings and flavor violation
	->reduction of parameters to 19
		-higgsino mass parameter and pseudo-scalar higgs mass 
		-ration of Higgs vauum expectation values
		-soft gaugino masses (bino, wino, gluino
		-sfermion masses
		-trilinear couplings
