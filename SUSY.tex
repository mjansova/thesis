\chapter{SUSY}

\section{QFT}

-fields and particles
-lagrangian
-feynman diagram


\section{SM and its shortcomings}

-sm intro
-particles
-interactions (electroweak, QCD)
-19 free parameters (nine fermion masses, one scalar mass, three coupling parameters, four quark mixing parameters, higgs vacuum expectation value, strong cp violating phase)
-perturbative theory (LO, NLO)

issues

1)hierarchy problem
corrwctions to Higgs boson mass
to compute cross section, all quantum loop corrections has to be taken into account
fermions an vector boson masses proctected from diverging by mechanism within the SM
but no mechanism for Higgs mass: $mh^2~ mh0^2+k mPlanck^2 $ - parameters mh0, k and mPlanck a priori unrelated. But these parameters must be fine tuned in order to obtain mass of Higgs (mh<<mPlanck) -> not natural
called hierarchy problem - no reason to expect a large hierarchy between electroweak scale and planck scale

2)dark matter
-measuremnt of rotation curves of galaxies - first dark matter hypothesis
-gravitational interaction, but not electromagnetic -> dark matter
-from observations several constraints on dark matter - not short -lived and not baruonic, gravitationally interacting, low kinetic energy (cold -> it cannot be neutrino)
-> no good candidate within the SM
From cosmological observations we expect dark matter mass of order of 100~GeV

3)Dark energy
-cosmological constant (lambda) in einsteins equation necessary to explain the observed expansion of universe
-> cosmologica constant can be interpreted as a vacuum energy

4)Matter-antimatter assymetry
-matter and antimatter should be produced in smae amount at big bang
-but our world dominated by matter

5)Neutrino masses
-neutrinos oscialte from one flavour to other -> this can only happen when neutrinos are masive and have different mass states than flavour states

6)Strong CP phase
strong QCD lagrangian introducing the phase theta - close to zero, despite the theoreticla arguments that it should not be like this

7)Quantum gravity
-gravity not described by SM
-desired to unify general relativity with QFT

8)Unification of forces
-possibility to unify all interactions

9)open questions
-in SM large differences between quarks
-why there should be three fermion families


\section{BSM physics}

BSM theories
-SM works fine, but we need to extend it -> we can add either additional symetries, space-time dimensions or field content

SUSY
-around 70's
-Golfand and Likhtman -> new symmetry Q -> Q|f> -> |b>; Q|b> -> |f> -> later Haag, Lopuszanski and Sohnius said that such symmetry corresponds to supersymmetry
-to each fermion a boson  with same quantum numbers (except of spin)
-partner of praticle is superpartner and they form superfield
-spin differs by 1/2
-superpartner should have the same mass -> not observed -> susy must be broken (for now we just add a term into the lagrangian)
-motivation:
	-solve hierarchy problem (superpartnes have equal masses and cancel the loop corrections) - in case of "soft breaking" susy prevents the quadratic divergencies and there are only logarithmic + small fine tuning
	-> naturalness of susy related to the mass difference between particle and its superpartner (Q: then if the susy partner of eg electron is very heavy does not it induce the divergencies? )
-MSSM
-most used SUSY relization is MSSM -> minimal -> adding the minimum number of fields to the SM to become supersymmetric
-adding sfermions and gauginos - left and right handed fermions -> e.g two selectrons
-for Higgs more complicated- one higgsino is not enough, but second SU(2) doublet is needed to avoid a gauge anomaly.
-mass eigenstates do not have to be flavor eigenstates
-combinations of electroweak gauginos and higgsions make charginos and neutralinos
-mixing between left and right superpartners 
-spectrum of sparticles
-conservation of R-party by construction $R=(-1)^{2S+3B+L} $
	-> pair production of sparticles
        -> decay only to odd nr of sparticles
        -> LSP is stable -> dark matter candidate
-MSSM - more than 100 new parameters than in SM
-> too many parameters - problem for phenomenological and experimental models
-> pMSSM - phenomenological MSSM -> reduction of number of parameters by assuming
	- there is no new source of CP vilation
	-lightest neutralino is the LSP
	-other assumptions on the sfermion masses, trilinear couplings and flavor violation
	->reduction of parameters to 19
