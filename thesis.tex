
\emptypage

% ============================================================================
%  Ackowledgements
% ============================================================================
\chapternonum{Acknowledgments}
%\chapter{Acknowledgements}

%thank you 'n stuff

\emptypage

% ============================================================================
% Table of contents
% ============================================================================

\singlespace
\dominitoc
\renewcommand{\leftmark}{Contents}

\tableofcontents

\emptypage

% ============================================================================
%   Introducton
% ============================================================================

\onehalfspacing
\setcounter{page}{1}
\pagenumbering{arabic}
%\chapternonum{Introduction}
%\chapter{Introduction}

%\vspace*{0.5cm}

\chapternonum{Introduction}

%The knowledge about particle physics brought us many inventions widely used and appreciated in the society. The particle physics applications are now used in many fields, from medicine to the energy industry. To further understand and broaden our knowledge about the elementary particles and interactions, the Large Hadron Collider was built at CERN. The Large Hadron Collider is providing particle collisions in four detectors, one of them being the Compact Muon Solenoid~(CMS). In the detector the colliding particles interact, leading to the creation of new particles.  The created particles are traversing the detector and leaving there the energy deposits along their path. To be able to reliably reconstruct the picture about what happened in the detector, precise measurement of these deposits is vital.

This thesis has started in the same year as Run~2. In Run~2, the center-of-mass energy, instantaneous luminosity, and the bunch crossing frequency has increased compared to the Run~1 and consequently the fluence received by the detector has increased. With increasing fluence, the detector suffers from larger irradiation which could be a cause of issues in its performance. Moreover, the detector is also ageing with fluence, leading to a change of some of its characteristic. Therefore it is very important to watch closely the detector and avoid the increased fluence to affect the physics analyses.

The description of the Compact Muon Solenoid~(CMS) detector is given in the first Chapter~\ref{sec:detch} together with a brief introduction of the the Large Hadron Collider. This chapter is giving detailed information about the silicon strip tracker, which deeper understanding is required for the following chapters. This chapter also presents the reconstruction of the physics objects corresponding to particles passing through the detector.

In Chapter~\ref{sec:HIPch} I present the CMS strip tracker in even larger detail and in particular the inefficiencies in tracking observed during years 2015 and 2016, resulting from the increased fluence. Because of these inefficiencies, many hits in the detector were not reconstructed, and therefore it was essential to understand the source of them. This effect was believed to be caused by the inelastic nuclear interactions in the sensitive volume of the tracker, leading to creation of the highly ionizing particles (HIP). The front-end electronics of the strip tracker is not designed to cope with the large HIP energy deposits, therefore the HIP interaction saturates the electronics and induces the dead-time. This chapter presents two studies, I performed during my PhD, on the  nuclear interactions, which result in a measurement of the probability that such interaction occurs in the tracker. This measurement is the first measurement of the HIP probability with the CMS data. 

The third chapter, Chapter~\ref{ch:simu}, focuses on the simulation of the CMS silicon tracker. In order to be able to compare the results of the experiment with the theoretical expectations, the interactions of generated particles with the detector must be simulated. This chapter describes how the simulation in the strip tracker is implemented and how it can be improved. I show, that several tracker properties are taken as parameters in the simulation. But with aging of the detector these properties evolve and therefore they need to be remeasured and updated in the simulation. In this chapter I identify, which parameters need to be reevaluated. At the end I present results on the measurement in cosmic data of the cross talk parameters,  which were found to be strongly influenced by the detector ageing. I also show how introducing the newly measured cross talk parameters into simulation improve the description of data by simulation at the hit level.

Particle physics is described by the standard model, which last piece, the Higgs boson, was discovered in 2012 by the CMS and ATLAS collaborations. Although the standard model is now complete and in general describes excellently the physics phenomena, it suffers from several shortcomings. This issue makes us believe that the standard model is an effective theory, which is a part of a bigger theory. Over years, many theories  were proposed and one, referred to as supersymmetry, became of a special interest due to its capability to address many of the standard model shortcomings. Both standard model and supersymmetry are introduced in Chapter~\ref{sec:SUSYch}.

Supersymmetry introduces a new partner to each standard model particle and therefore extensive searches for these particles are performed by the CMS collaboration and other collaborations as well. One of these particles is the supersymmetric partner of the top quark, the stop, which is expected to have a mass around 1~TeV in natural supersymmetry and therefore be accessible at the LHC energies. No evidence for the stops was  found in Run~1, but the increase in luminosity as well as the center-of-mass energy in Run~2 allows us to probe the stop masses beyond the Run~1 exclusion. The last chapter before conclusion, Chapter~\ref{sec:stopch},  presents the search for this supersymmetric partner of the top quark in the single lepton final state with the data of Run~2 recorded in 2016 corresponding to an integrated luminosity of 35.9~fb$^{-1}$. In this analysis I was mainly responsible for estimation of one of the SM backgrounds entering into the targeted signal regions. 


\clearpage

\setcounter{secnumdepth}{4}
\chapterwithnum{The CMS experiment at the LHC}
\setcounter{secnumdepth}{5}

There are several options how to study matter, for example it can be studied via cosmic rays and neutrinos, gravitational lensing or at collider experiments. In Section~\ref{sec:LHC} the largest existing collider, the LHC, and its experiments are introduced. In Section~\ref{sec:CMS} the CMS detector, in which context this thesis is performed, is described. Then in Section~\ref{sec:objects}, object reconstruction techniques of the CMS collaboration applied on the CMS data are introduced. A special interest is put on the topics whose deeper understanding is required for the following chapters of this thesis.

\section{The Large Hadron Collider~\label{sec:LHC}}

The Large Hadron Collider~(LHC)~\cite{CERN-Brochure-2017-002-Eng, Evans:2008zzb} is a particle accelerator with a circumference of 27~km, which is a part of the CERN accelerator complex~\cite{Bruning:2004ej} located near Geneva, Switzerland. The LHC project was approved in 1994 and designed to provide mainly collisions of protons. Part of the LHC operation time is also dedicated to collisions of heavy nuclei, but in this thesis only the proton collisions are described.

A schematic view of the CERN accelerator complex with its main experiments is presented in Fig.~\ref{fig:figures/CCC-v2016}~\cite{Mobs:2225847}. At the beginning of the acceleration process the protons are obtained by stripping electrons from the hydrogen atoms. Protons are then accelerated through a chain of accelerators: Linac2, Booster, Proton Synchrotron~(PS) and Super Proton Synchrotron~(SPS). The proton beams accelerated to 450~GeV are injected from the SPS into the LHC~(both clockwise and anti-clockwise). In the LHC, the beams are further shaped and accelerated with eight radio-frequency~(RF) cavities per beam, currently to the beam energy of 6.5~TeV. The LHC is equipped with 9593 magnets, mainly dipoles and quadrupoles, to bend and focus the beams. There is an ultrahigh vacuum of order of $10^{-10}-10^{-11}$~mbar in the LHC tube to avoid collisions of particles with gas, the vacuum of around $10^{-6}$~mbar is being also used for isolation  of cryomagnets and helium distribution line.

    \insertFigure{figures/CCC-v2016} % Filename = label
                 {0.99}       % Width, in fraction of the whole page width
                 { The CERN accelerator complex~\cite{Mobs:2225847}. }

The proton beams are not homogeneous but composed of trains of bunches, each bunch containing around 100 billions of protons. Bunches within a given train are spaced by 25~ns. Time gaps between two different trains are longer, their duration depends on the beam structure and it starts from few multiples of 25~ns. The LHC uses several beam structures, which are specified by the filling scheme. An example of a beam structure is shown in Fig.~\ref{fig:figures/beamStructure}. The proton beams are accelerated in separate pipes and then collided against each other at four interaction points~(IP). For the moment, the LHC operation is divided into two eras, Run~1 and Run~2. The Run~1 started in 2008 and ended in 2013. During this era the center-of-mass energy of the collisions was first 7~TeV, then increased to 8~TeV in 2012, and the beam structure allowed collisions every 50~ns. The Run~2 started in 2015 and is still ongoing. In the Run~2 the collision center-of-mass energy increased to 13~TeV and the time between collisions was reduced to 25~ns.

    \insertFigure{figures/beamStructure} % Filename = label
                 {0.7}       % Width, in fraction of the whole page width
                 { (top) An example of beam structure. The beam pipes are shown in red and violet and the beams are drawn in green. (bottom) An example showing the structure of two trains. }

The bunches of protons collide at four IPs, at which four main experiments were installed. The two general-purpose detectors, which were designed to cover a wide range of physics, are ATLAS (standing for ``A Toroidal  LHC ApparatuS'')~\cite{Aad:2008zzm} and CMS (standing for ``Compact Muon Solenoid'')~\cite{Chatrchyan:2008aa}. The other two main experiments are ALICE (standing for ``A Large Ion Collider Experiment'')~\cite{Aamodt:2008zz}, mainly focusing on the analysis of heavy-ion collisions and studies of ``quark-gluon plasma'', which existed at the early Universe and the LHCb (standing for ``Large Hadron Collider beauty'')~\cite{Alves:2008zz} specialized in the physics of the b-quark and heavy flavor quarks in general. There are also three smaller experiments along the LHC, one of them is TOTEM~\cite{Anelli:2008zza} which is close to CMS and its principal goal is to measure the total cross section of protons at LHC. LHCf~\cite{Adriani:2008zz}, which is located near to the ATLAS experiment, studies particles moving very close to the proton beams. Finally MoEDAL~\cite{Acharya:2014nyr}, which is close to LHCb, focuses on the search of hypothetical particles, e.g. magnetic monopoles.

One of the main motivations for the LHC and its experiments was the search for the Higgs boson, which was discovered in 2012. A second important topic of research is the search for new particles and more generally for evidence of physics beyond the Standard Model, motivated mainly by cosmological observations, naturalness of the Higgs mass and the desire for unification of known interactions.

\newpage

%TODO here

\section{The Compact Muon Solenoid detector~\label{sec:CMS}}

%The CMS (``Compact Muon Solenoid'') detector~\cite{Chatrchyan:2008aa, CMSproposal} is a multipurpose detector of overall length of 22~m, a diameter of 15~m and a weight of 14~t, located at the IP~5. The designed properties of the CMS  detector were to have high efficiency of muon identification and reconstruction. It was also required have precise momentum resolution in large range of momenta, dimuon mass resolution of order of 1\% at 100 GeV, good charged-particle momentum measurement and resolution~(around 1\%-2\% for 10~GeV muons in majority of CMS coverage) as well as high efficiency in reconstruction of charged-particle tracks. Furthermore it was designed to have precise measurement electromagnetic energy, leading to resolution of diphoton and dielectron mass of around 1\% at 100~GeV. Good missing transverse energy~(MET) and dijet mass resolution was required as well. One of the important requirements was also high efficiency in offline tagging of tau particles and jets originating from b-quarks. These conditions have to be fulfilled in the LHC environment, with a bunch crossing~(BX) frequency of 25~ns, where every bunch crossing leads to tens of inelastic interactions on top of the interaction of the interest, referred as ``pileup''~(PU), resulting in thousands of charged particles in the CMS detector every 25~ns. Because of this large radiation, the detectors and front-end electronics have to be radiation-hard.

The CMS (``Compact Muon Solenoid'') detector~\cite{Chatrchyan:2008aa} is a multipurpose detector of an overall length of 22~m, a diameter of 15~m and a weight of 14~t, located at the IP~5. The designed properties~\cite{CMSproposal} of the CMS  detector were to have high efficiency of muon identification and reconstruction, leading to a requirement on  a dimuon mass resolution of the order of 1\% at 100 GeV. A high quality tracking providing precise charged-particle momentum measurement and resolution around 1\%-2\% for 10~GeV muons in the majority of the CMS coverage. Furthermore it was designed to have best possible measurement the electromagnetic energy, leading to a resolution of the diphoton and dielectron masses of around 1\% at 100~GeV. These conditions have to be fulfilled in the LHC environment, with a bunch crossing~(BX) every 25~ns, where every bunch crossing leads to tens of inelastic interactions on top of the interaction of the interest, referred as ``pileup''~(PU), resulting in thousands of charged particles in the CMS detector every 25~ns. Because of this large radiation, the detectors and front-end electronics have to be radiation-hard.

To achieve the given requirements, CMS was built in layers around a large solenoidal magnet, with endcaps at each side. The CMS magnet is a superconducting solenoid providing a magnetic field of 3.8~T. The magnet is surrounded from outside by the steel yoke which returns the magnetic flux of the solenoid~\cite{tdrMagnet}. The radius of the magnet has to be kept relatively small and therefore the available space between the magnet and interaction point is limited.

Inside the magnet from the interaction point outwards, there is a pixel and silicon strip tracker, followed by the electromagnetic and hadron calorimeter. Outside of the magnet there is the outer hadron calorimeter and the steel return yoke with embedded muon chambers. The CMS detector layout is shown in Fig.~\ref{fig:figures/cmsdetector}~\cite{website:CMSdet}. In the following sections, first the coordinate system of the CMS is described. Then the CMS subdetectors are introduced, starting from the innermost one. In the end, basic information on trigger, luminosity and pileup is given.
%-coverage up to |eta|<5


    \insertFigure{figures/cmsdetector} % Filename = label
                 {0.99}       % Width, in fraction of the whole page width
                 { A schematic layout of the CMS detector~\cite{website:CMSdet}. }

\subsection{Coordinate system and conventions}


The coordinate system used by CMS~\cite{Chatrchyan:2008zzk} is sketched in Fig.~\ref{fig:figures/coordinates}~\cite{Pantaleo:2293435}. In the CMS conventions, the cartesian coordinate system has is centered at the IP, with the x-axis pointing into the center of the LHC, the y-axis going upwards and the z-axis going anti-clockwise along the beam direction. The azimuthal angle $\Phi$ is defined in the x-y plane and is measured from the x-axis. The polar angle $\theta$ is measured from the z-axis and is defined in the y-z plane. Finally the $r$ is a radial coordinate in the x-y plane. In this convention the pseudorapidity $\eta$ is defined as

\eq{pseudorapidity}
{
    \eta =  -\ln \left[ \tan \left( \frac{\theta}{2} \right) \right].
}

The angular distance between two points can be measured with the help of $\Delta R$ defined as

\eq{deltaR}
{
    \Delta R = \sqrt{ \Delta \Phi^2 + \Delta \eta^2}.
}


The transverse momentum $p_{T}$ can be computed from the x and y momentum components as

\eq{pseudorapidity}
{
    p_{T} =  \sqrt{p_{x}^2 + p_y^2 }.
}

As at the interaction point, the beams have no momentum in the x-y plane, the sum of momenta of particles originating from collision in this plane should be zero due to momentum conservation law. Therefore the imbalance of momenta in this plane, the non-zero total $p_{T}$, can be sign of particles which momentum was unmeasured, like neutrinos or hypothetical new particles. 

    \insertFigure{figures/coordinates} % Filename = label
                 {0.6}       % Width, in fraction of the whole page width
                 { The CMS coordinate system with the three axes intercepting at the interaction point, the x-axis pointing inside the LHC ring, the y-axis going upwards and the z-axis pointing anti-clockwise along the beam.~\cite{Pantaleo:2293435}. }

\subsection{The silicon tracker~\label{sec:tracker}}

The silicon tracker~\cite{CMS:1997tlf, CMS:2000eqx} is the innermost subdetector of the CMS detector. The CMS tracker is divided into two parts. The inner part consists of a pixel detector, while the outer part of a strip detector.  Its purpose is to reconstruct tracks from charged particle depositions in silicon sensors, the energy loss $E_{loss}$ lost along their path. In order to recognize from which interaction a given particle originates, the tracker is required to have a capability to reconstruct primary and secondary vertices. A track is the reconstructed path of a particle passing trough the tracker. The primary vertices are the positions where the interaction of interest or the PU interaction occurred, which are determined by the vertex fitting procedure from the tracks~\cite{Ball:2007zza}. The secondary vertices correspond to the place where a particle with a long lifetime decays to other particles. 

On the tracker level, the tracking is performed in 4 steps~\cite{website:slidesTracking, website:twikiTracking}. It starts from the hits, which are reconstructed from particle depositions in the sensors. First, the \textit{track seeding} is performed from two or three 3-D reconstructed hits with a constraint on primary vertex position. Then the algorithm proceeds with \textit{track building} which aims to connect all hits originating from one particle. During the track building the track is propagated to the neighboring layers of the tracker, testing the compatibility of the reconstructed hit with the track by a $\chi^{2}$ test. Once the full track candidate is complete, the \textit{track fitting} is performed to obtain the best parameters of the track and to recompute precise hit position using the track properties. The last step is a \textit{track quality selection} rejecting tracks not fulfilling quality requirements, which are based on the $\chi^{2}$ of the final fit, the number of layers with a hit associated to the track and the probability of the track  to originate from the primary vertex.

The default track reconstruction~\cite{Chatrchyan:2014fea} is using the software referred as the Combinatorial Track Finder~(CTF), based on the combinatorial Kalman filter~\cite{Fruhwirth:1987fm}. The tracking uses an iterative approach: in the first iteration the easiest tracks to find are reconstructed (i.e. the ones with the highest $p_{T}$), then after these tracks are complete their hits are masked in order to avoid duplicities and reduce combinatorics for further iterations of tracks finding. In total there are 12 iterations and each iteration is focused on a specific type of tracks.


Due to the presence of the magnetic field, charged particles are bent according to their momentum and charge, thus the CMS tracker is able to measure the charge sign and the momentum associated to a track. To perform so, the tracker needs to have good spatial resolution and to be extremely radiation-hard due to the large flux of incoming particles. Also the material was chosen carefully, in order to reduce multiple scattering, nuclear interactions or bremsstrahlung in the tracker material.

The flux of particles in the CMS detector decreases with the distance from the interaction point, the flux of charged particles in the barrel at radius of 4~cm is around $10^{8}~\mathrm{cm^{-2}s^{-1}}$ while at radius of 115~cm it decreases to around $3 \times10^{5}~\mathrm{cm^{-2}s^{-1}}$. Consequently, the inner part of the tracker is made of silicon pixels, which are able to measure particle paths and its properties in high particle density environment. In the outer part of the tracker, the particle density is low enough to use silicon strip sensors, which are cheaper mainly because they require less readout channels. Overall, the tracker pseudorapidity coverage is  $|\eta| < 2.5$.


\textbf{Silicon pixel tracker}

%The silicon strip tracker is 5.8~m long with a diameter of 2.5~m. 
Before 2017, the pixel barrel, referred as BPIX,  was located at radius of 4.4--10.2~cm from the IP. The pixel endcaps~(FPIX) extended on each side at 6--9~cm in the z direction. In total, there were three barrel layers and two endcaps layers on each side. In 2017, due to the high radiation exposure, the pixel tracker was replaced. The new pixel tracker has four barrel layers at radius 3-16~cm and three endcap disks at distances of 29.1--51.6~cm. The pixel detector plays crucial role in the track seeding, primary and secondary vertex reconstruction~\cite{CMS:2012sda}. 


%-new pixels - short distance from IP, high radiation -> damage -> need for replacement
%- FPIX - 3 disks on each side 29.1cm -51.6cm  (Forward pixel) ~\cite{CMS:2012sda}
%- BPIX - barrel pixel - 4 cylindrical layers radius of 3-16 cm

\textbf{Silicon strip tracker}

%15148 silicon strip modules

As the following chapters present studies of the silicon strip tracker and its simulation, it is necessary to take a deeper look into its design and properties. The silicon strip tracker is divided into four partitions. Each partition has layers of modules, which have either one side~(mono) or two sides~(stereo) of silicon sensors. In totality the strip tracker is composed of 15148 modules. The Tracker Inner Barrel~(TIB) is the innermost barrel part with two layers of stereo modules succeeded by two layers of mono modules. The Tracker Outer Barrel~(TOB) surrounds the TIB. In the innermost part of TOB, there are two layers of stereo modules, the remaining four layers are mono. On each side of the barrels, Tracker Inner Disks~(TID) and Tracker EndCaps~(TEC) are located. There are three wheels of TID with three module rings and 9 wheels of TEC with four to seven rings of modules on each side. In each wheel there is a mixture of mono and stereo modules, but each ring has either mono or stereo modules. In the barrel region, the silicon tracker is placed in radius between 20~cm up to 1.1~m from the IP. The disks start at 80~cm in the z-direction from the IP and the endcaps reach up to z=2.8~m from the IP. The overall layout of the silicon strip tracker can be seen in Fig.~\ref{fig:figures/cmsTracker}~\cite{Chatrchyan:2014fea}.

    \insertFigure{figures/cmsTracker}
                 {0.9}       
                 {A schema of the upper half of the CMS silicon strip tracker and layout of its partitions. The star represents the IP. The modules in blue are stereo, while the mono ones are shown in black~\cite{Chatrchyan:2014fea}. }

Each mono module holds 320~$\mathrm{\mu m}$ or 500~$\mathrm{\mu m}$ thick silicon sensors with either 512 or 768 silicon strips.  The strip length is between 8 and 25~cm, the distance between strips, called pitch, varies between around 80~$\mathrm{\mu m}$ and 200 ~$\mathrm{\mu m}$. The strip width to pitch ratio is 0.25. In the barrel, the strips are parallel to the z-axis, or tilted by 100~mrad with respect to the z-axis in case of the stereo sensors. In TID and TEC, the strips are aligned to be parallel to $r$. The local module coordinates have zero in the middle of the module, the z-axis goes in the direction from the back-plane to the strips, the y-axis goes along the strips, and the x-axis is perpendicular to the strips and traverses them. The local $\theta$ angle is measured from the z-axis. The local $\Phi$ is defined in the x-y plane and measured from the x-axis. The schematic view of the sensor local coordinates is shown in Fig.~\ref{fig:figures/localCoordinates}. The details about the module geometries for each layer of TIB and TOB and each ring of TID and TEC can be found in Table~\ref{tab:trackerGeometries}~\cite{website:hephyPage}.


    \insertFigure{figures/localCoordinates} % Filename = label
                 {0.6}       % Width, in fraction of the whole page width
                 { The sensor local coordinate system with the three axes intercepting in the middle of the sensor. The x-axis is perpendicular to strips, the y-axis goes along strips and the z-axis pointing from backplane to strips. }

\begin{table}[h]
\begin{center}
\begin{tabular}{|l|l|l|l|l|l|}
\hline
Layer & Type  & \#Strips & Thickness~[$\mathrm{\mu m}$] & Pitch~[$\mathrm{\mu m}$] & Geometry label  \\
\hline
\hline
TIB L1 & stereo & 768 & 320 & 80 & IB1 \\
TIB L2 & stereo & 768 & 320 & 80 & IB1 \\
TIB L3 & mono & 512 & 320 & 120 & IB2  \\
TIB L4 & mono & 512 & 320 & 120 & IB2 \\
\hline
TOB L1 & stereo & 768/512 & 500 & 122/183 & OB2 \\
TOB L2 & stereo & 768/512 & 500 & 122/183 & OB2  \\
TOB L3 & mono & 512 & 500 & 183  & OB2 \\
TOB L4 & mono & 512 & 500 & 183  & OB2 \\
TOB L5 & mono & 768 & 500 & 122  & OB1 \\
TOB L6 & mono & 768 & 500 & 122  & OB1 \\
\hline
TID R1 & stereo & 768 & 320 & 81...112  & W1a \\
TID R2 & stereo & 768 & 320 & 113...143 & W2a  \\
TID R3 & mono & 512 & 320 & 124...158  & W3a \\
\hline
TEC R1 & stereo & 768 & 320 & 81...112 & W1b  \\
TEC R2 & stereo & 768 & 320 & 113...143 & W2b   \\
TEC R3 & mono & 512 & 320 & 124...158  & W3b \\
TEC R4 & mono & 512 & 320 & 113...139  & W4 \\
TEC R5 & stereo & 768 & 500 & 126...156  & W5 \\
TEC R6 & mono & 512 & 500 & 163...205  & W6 \\
TEC R7 & mono & 512 & 500 & 140...172  & W7 \\
\hline
\end{tabular}
\caption[Table caption text]{Module type, strip multiplicity, sensor thickness and pitch and module geometry label for layers or rings of the four silicon strip tracker partitions~\cite{website:hephyPage}. }
\label{tab:trackerGeometries}
\end{center}
\end{table}


The track resolution determined in 2011 before replacing the pixel detector, is in \pt is of order of 1.5\% for non-isolated particles in range 1 < \pt < 10~GeV and $|\eta| < 1.4$ and of order of 2.8\% for particles with \pt = 100~GeV and $|\eta| < 1.4$~\cite{TRK-11-001}. The spatial resolution of the reconstructed hits for different barrel layers of the strip tracker is shown in Fig.~\ref{fig:figures/hitResolution}~\cite{website:hitEff}. The typical strip hit resolution is between 15--45~$\mathrm{\mu m}$ in the barrel region of the silicon strip tracker.


    \insertFigure{figures/hitResolution}
                 {0.6}       
                 {Measured resolution of reconstructed hit position for different layers of strip tracker and size of pitch between the strips. The red color corresponds to hits with charge deposited in one strip, green in two strips and blue in three strips~\cite{website:hitEff}. }

\subsection{The electromagnetic calorimeter}

The electromagnetic calorimeter (ECAL)~\cite{tdrECAL} is a subdetector located at outer side of the silicon tracker. In the barrel region the ECAL extends up to a radius of 1.77~m from the IP. It is an homogeneous, fast, radiation resistant calorimeter with a good energy resolution. It is composed of 75848 lead-tungstate ($\mathrm{PbWo_{4}}$) crystals and its purpose is to measure the energy of electrons and photons. The ECAL consists one barrel~(EB) covering $|\eta|<1.479$ and two endcaps~(EC) extending the coverage up to $|\eta| =3$. A preshower is placed in front of the endcaps in order to separate highly energetic single photons from the photons originating from the decay of neutral pions.

The energy resolution of ECAL was determined~\cite{Chatrchyan:2008aa} to be

\eq{ECALresol}
{
 \frac{\sigma_{E}}{E} = \frac{0.028}{\sqrt{E}} \bigoplus \frac{0.12}{E} \bigoplus 0.003 ,
}
where $E$ is the energy and $\sigma_{E}$ is the energy resolution. The first term is the stochastic part, it corresponds to e.g. fluctuations in number of particles. The second term accounts for noise and the third term covers mainly the non-uniformities, energy leakage and inter-calibration issues.


\subsection{The hadron calorimeter}

The purpose of the hadron calorimeter~(HCAL)~\cite{tdrHCAL} is to measure the energy of strongly interacting particles.  The HCAL is a sampling calorimeter composed of four parts, out of them two are located between the ECAL and the magnet, which are the HCAL Barrel~(HB) and Endcaps~(HE). Both HE and HB have a brass absorber and their active material is made of plastic scintillators. The pseudorapidity coverage of HB is $|\eta|<1.3$, and of HE $1.3<|\eta|<3$. The coverage is further extended up to $|\eta|=5.2$ by the third part called the Forward calorimeter~(HF). Installed 11.2~meters from the IP on both sides, HF is made of steel as an absorber and quartz fibers creating the active volume. The technology of HF is very radiation-hard as around one third of the particles produced in the collisions reaches HF. Because the available radial space between ECAL and the magnet ( i.e. is between r=1.77~m and r=2.95~m ) is not large enough to build calorimeter with enough stopping power, the last part of the calorimeter, the Outer calorimeter~(HO), was added after the magnet. HO is covering the region $|\eta|<1.3$ and stops particles escaping HB, for this reason it is sometimes also referred as ``tail catcher''. The magnet material is in that case used as an absorber for HO. A schematic layout of the HCAL is shown in Fig.~\ref{fig:figures/HCAL}~\cite{Chatrchyan:2008aa}.

    \insertFigure{figures/HCAL}
                 {0.7}       
                 {A schema of one quarter of the CMS HCAL with layout of HB, HE, HF and HO~\cite{Chatrchyan:2008aa}. }

The hadron energy resolution from the combination of ECAL and HCAL (barrel and endcaps)~\cite{Chatrchyan:2009ag} was measured to be


\eq{HCALresol}
{
 \frac{\sigma_{E}}{E} = \frac{0.847}{\sqrt{E}} \bigoplus 0.074 ,
}
where the meaning of the terms is similar as for the ECAL.
%- for HF 1.98/sqrt(E)+0.09 -> higher because of high energy of jets in this region but as divided by energy, it is ok

\subsection{The muon chambers}

Because many interesting physics processes have a signature with muons in the final state, good and precise measurement of muons is one of the main goals of the CMS. This comprises muon identification, momentum measurement and triggering. The good triggering and momentum measurement is achieved with the help of the high magnetic field provided by the solenoid. The measurement of muons is done in three gaseous subdetectors, the Drift Tube~(DT), Cathode Strip Chamber~(CSC) and Resistive Plate Chamber~(RPC) systems~\cite{tdrMuon}. The three muon systems combined have in total 1400 chambers, which in radial direction are placed between around 4~m up to 7.5~m from the IP. The distance in the z-direction between the inner and outer part of muon systems is about 5.5~m-11~m. The layout of the muon systems is shown in Fig.~\ref{fig:figures/muons}~\cite{Chatrchyan:2013sba}.

    \insertFigure{figures/muons}
                 {0.7}       
                 {A schematic view of one quarter of the CMS muon systems and their inner structure~\cite{Chatrchyan:2013sba}. In case of DTs MB stands for ``Muon Barrel'' and in case of CSCs ME stands for ``Muon Endcap''. The notation for RPCs is RB for barrel and RE for endcaps. }

The DTs are located in the barrel region and are partly integrated into the return yoke.  The pseudorapidity coverage is of $|\eta|<1.2$. They are composed of plane cathodes with anode wires in between the planes. The smallest unit of DTs is the drift cell of dimensions 42$\times$13~mm, in which one 50~$\mathrm{\mu m}$ thick anode wire is located. The drift time of charge carriers can be up to around 400~ns.

The CSCs are located in the endcap areas outside of the return yoke. The CSCs coverage is $0.9<|\eta|<2.4$, which is partly overlapping with the DTs. The CSCs also contain cathodes with anode wires in between, but one cathode of the pair is segmented into strips. The DTs and CSCs provide a good triggering of muons which is independent of the rest of the CMS detector.

To ensure the correct bunch crossing identification~(i.e. from which bunch crossing a given muon originates), the complementary RPCs are present in both barrel and endcap regions. RPCs are composed of parallel plates of anodes and cathodes and readout strips. The RPCs are faster than DTs and CSCs, but provide a worse position resolution. The RPCs trigger is independent of the CSCs and DTs.  

The spatial resolution per chamber differs for the three systems, for DTs it is 80-120~$\mathrm{\mu m}$, for CSCs 40-150~$\mathrm{\mu m}$, and for RPCs 0.8-1.2~$\mathrm{cm}$~\cite{Chatrchyan:2013sba}.

In the following sub-section the time measurement in DTs is introduced. The timing can be also measured with CSCs and ECAL in a very similar way, but for the purposes of the thesis, only the knowledge of the time measurement in DTs is needed.


\textbf{Muon timing measurement in DTs~\label{sec:muonTiming}}

A muon going through DTs deposits its energy by ionization of the gas. The created charge carriers drift towards the wires and are read at time $t_{read}$ as shown in the blue arrows in Fig.~\ref{fig:figures/dtTiming}~\cite{Traczyk:1365029}. If it is assumed that the muon was produced in-time with the collision~($t_{IP} = 0$) and that its speed is the speed of light, then the time ($t_{cell}$) when the muon arrives from the IP to the given drift cell can be calculated. With the knowledge of the drift velocity~$v_{drift}$, the difference between $t_{read}$, when muon charge was read at the wire, and $t_{cell}$, when muon came to the cell, can be converted into a distance between them. The distance $ v_{drift} (t_{read} - t_{cell})$,  is used to compute the hit positions where muon crossed the cells, starting from the wire as shown in blue crosses. In case that the production time was assumed correctly~(i.e. muon was created in-time with the collision), all hits should be on one line~(blue crosses). If the particle was not produced ``in-time'' with the collision time, the hits are shifted~(red crosses) and the line which connects them is not straight, but curly. The distance between the reconstructed hit and the hit on the straight line~(red line) is converted to the time~$\delta t$, which measures real arrival time of muon with respect to the expected arrival time, and thus can be used to compute the muon timing~\cite{Traczyk:1365029}.


    \insertFigure{figures/dtTiming} % Filename = label
                 {0.5}       % Width, in fraction of the whole page width
                 { A schema of few drift cells of DTs. The wires are shown as black dots, the drift direction as blue arrows, the expected hits as blue crosses and the reconstructed hits as the red crosses. The distance between reconstructed and expected hits is converted to time denoted as $\delta t$~\cite{Traczyk:1365029}. The assumed time when a in-time particle ($t_{IP} = 0$) would enter the drift cell is denoted as $t_{cell}$ and the time when the charges were read by a wire as $t_{read}$. In the left corner two situations are shown, either the muon was produced in-time and zero $\delta t$ is observed or the muon was not produced in-time ($t_{IP} \neq 0$) leading to the non-zero $\delta t$. }

The timing measurements are combined to produce, among others, the following variables: 

\begin{description}
\item [$\mathbf{time_{IP}^{InOut}}$]
This variable corresponds to the time at which a muon was present at the interaction point, assuming that the muon moves at the speed of light from the IP outside of CMS. It is computed as a weighted average $\bar{t}$ of the measured $\delta t$ values, where in case of the DTs each weight is equal to $N_{s}-2$, with $N_{s}$ being the number of hits in the segment of DTs to which the given hit belongs. The formula is then
\eq{timingResol}
{
 \bar{t} = \frac{1}{N} \times \frac{1}{\sum{N_{s,~i} - 2}} \times \sum{ (N_{s, i} -2) t_i },
}
where $N$ is the number of timing measurements $i$ and $t_{i}$ is a single timing measurement $\delta t$.

The error on the time measurement is computed as
\eq{timingResol}
{
 \sigma^{2} = \frac{1}{N-1} \times \frac{1}{\sum{w_{i}}} \times \sum{(t_i-\bar{t})^2 w_{i} },
}
where the weight $w_{i}$ is defined as $w_i = 1/\sigma_{i}^2$ with $\sigma_{i}$ being a single hit resolution and $\bar{t}$ is the above defined weighted average of the $\delta t$ measurements.

\item[$\mathbf{time_{IP}^{OutIn}}$ ]
The $time_{IP}^{OutIn}$ is the muon time at the interaction point assuming that the muon moves from the outside of the CMS detector towards the IP. During the calculation, each $\delta t$ measurement is increased by twice the time-of-flight~(TOF) of the in-time muon from the IP to the DT cell measuring the $\delta t$ to take into account that OutIn muon which was at IP at the same time as InOut one arrived to the DTs twice TOFs before the InOut muon. Then the computation of timing continues as for the $time_{IP}^{InOut}$.

\item[\textbf{Direction}]
The direction variable provides a simple and robust estimate if the muon traveled from the IP out or in the opposite way. The evaluation takes into account the errors on the $time_{IP}^{OutIn}$ and $time_{IP}^{InOut}$ variables and assumes that the correct time hypothesis has the smallest error.

\item[\textbf{Free} $\mathbf{1/\beta}$]
The free inverse beta (free $1/\beta$) variable is the free $c/v$, where $c$ is the speed of light and $v$ the speed of the muon.  It is obtained from the fit of muon time-of-flight measurements. The word ``free'' indicates that neither the production time, the direction nor the velocity is assumed and that all three are free parameters. The muons originating from collisions reaching the muon chambers, travel from inside towards outside of detector at speed close to the speed-of-light and therefore their free inverse beta is close to one. The free inverse beta for cosmic muons is around minus one due to the opposite direction.

The timing resolution for DTs is further discussed in Chapter~\ref{ch:simu}.

%The timing resolution for DTs in Run~I was determined to be 7-9~ns~\cite{Traczyk:1365029}, compared to 2.3~ns obtained from simulation. The measured resolution is expected to decrease as the detector synchronization improves.

\end{description}
%-on top of it
 %CMS being synchronized "top-down" for cosmic runs, so that a cosmic muon going vertically down is always in-time (or "at the same time" to be more precise). So basically the DT system is timed for muons going straight down. (just like for collisions it's timed for muons going outwards from IP).In practical terms. The "muons" collection takes as timeAtIpInOut what is more or less the mean of the segment times for a muon. Because it's assuming that the muon is propagating in the same direction that the system is synchronized in. From my observations above it looks like this is the case, the system is synchronized for downward-going cosmics. - not the case in the paper

\subsection{Trigger and data acquisition}

As bunches are colliding every 25~ns, thus with a rate of 40~MHz, fast and reliable triggering system~\cite{Khachatryan:2016bia} is required. The data size of one event is approximately 1~MB, therefore if there would not be any dedicated trigger, 40~TB of data per second would have to be stored, what is far beyond the current technical capacities. In CMS there are two levels of triggers which provide physics motivated selection of the interesting events. The first one, called Level-1~(L1), is hardware based and its decision-making is based on information from the muon chambers and the calorimeters. The L1 is capable to take a decision within 3.4~$\mathrm{\mu s}$ and its output rate is up to around 100~kHz. The rest of the event is read upon the decision of the L1 and sent to the second level called High Level Trigger~(HLT). The HLT is a software based trigger providing further selections. At the HLT level information from all subdetectors is read and therefore the full event can be reconstructed. The output rate of the HLT is around 1~kHz. The time period of uninterrupted data-taking is called a run and ideally there is one run per a LHC fill. In case of the detector technical difficulties the run can be interrupted and a new one started. Once the beams are dumped the run is stopped and a new run of cosmics instead of collision data-taking can start.

%The trigger can be set to select different kind of events. In case of Zero Bias trigger there is no requirement. In the Minimum Bias events, the trigger is firing on any bunch crossing with minimal requirement on the activity in the detector. More dedicated trigge triggerr is for example Single Muon trigger, which requires presence of one muon. 

During collision runs, most of the triggers are dedicated to record events useful for the physics analyses (Single Muon, \MET, Double Muon, ...). Part of the triggers can be used for calibration purposes, such as the Zero Bias and Minimum Bias triggers which fire on any bunch crossing leading to a minimal activity in the detector.

\subsection{Luminosity and pileup}

In particle physics experiments it is very important to have information about an expected event rate during a time period. For this evaluation, the following formula can be used

\eq{nev}
{
 \frac{\mathrm{d}N}{\mathrm{d}t} = \sigma \times \mathcal{L},
}
where $\sigma$ is the cross section of process of interest and $\mathcal{L}$ is the instantaneous luminosity given by

\eq{luminosity}
{
 \mathcal{L} = \frac{N^2 f}{4 \pi \epsilon_{n} \beta^{*}}F,
}
where the variables in numerator are $N$ which is the number of protons in one bunch (around $10^{11}$) and $f$ is the bunch crossing frequency (currently 40$\times 10^{6}$~Hz). In the denominator, $\epsilon_{n}$ is the normalized transverse beam emittance and $\beta^{*}$ is the beam amplitude function at the IP. This function is expressing how much the beam is ``squeezed'' before collision, lower the $\beta^{*}$ more the beam is squeezed. At LHC, the  $\epsilon_{n}$ is of the order of 4~mm~$\mathrm{\mu m}$ and $\beta^{*}$ of the order of 0.5~m. Lastly, because of the beam crossing angle, the reduction factor $F$ is introduced. Its value at LHC is typically around 0.95.

By integration of the instantaneous luminosity over time, the integrated luminosity can be obtained:

\eq{intluminosity}
{
 L = \int{ \mathcal{L} \mathrm{d}t}.
}

The LHC was designed to deliver an instantaneous luminosity of $1 \times 10^{34} \mathrm{cm^{-2}s^{-1}}$. At the beginning of Run~1, the instantaneous luminosity was lower than that and it has been increased during the years of LHC operation. Later, because of the smooth running, it was decided to go even beyond the designed luminosity, up to $1.58 \times 10^{34} \mathrm{cm^{-2}s^{-1}}$~\cite{Pralavorio:2272474}. The integrated luminosity delivered to CMS over years 2010--2017 can be seen in Fig~\ref{fig:figures/cmslumi}~\cite{website:CMSlumi}. A large fraction of integrated luminosity is good to be used in the physics analyses, the rest is being discarded due to a detector problem, for example. 


    \insertFigure{figures/cmslumi} % Filename = label
                 {0.6}       % Width, in fraction of the whole page width
                 { The delivered luminosity to the CMS detector for years 2010-2017~\cite{website:CMSlumi}. }


High luminosity is essential to study rare processes, but on the other hand it results in effect called pileup~(PU)~\cite{Bayatian:2006nff}. Pileup particles are particles which are not originating from the interaction of interest at a given bunch crossing. The pileup can be divided into two categories, in-time and out-of-time pileup. The in-time pileup is caused by multiple pp interactions in the event, the maximal peak in-time PU for all data-taking eras of 2017 is shown in Table~\ref{tab:PU}.  The out-of-time pileup~(OOT PU) originates from particles produced before or after the bunch crossing of interest. The OOT PU is caused for example by slow particles looping in the detector for more bunch crossings or due to the duration of integration of the signal charge by the front-end electronics, resulting in the pulse shape which is typically longer than 25~ns, depending on the subdetector. Because of the wide signal pulse shape, the signal from one particle can be read during more bunch crossings.

\begin{table}[h]
\begin{center}
\begin{tabular}{|l|c|}
\hline
Era & Peak PU [interactions/BX]  \\
\hline
A & 9.523  \\
B & 46.035  \\
C & 47.162  \\
D & 59.019  \\
E & 70.478  \\
F & 77.978  \\
G & 4.534  \\
H & 61.603  \\
\hline
\end{tabular}
\caption[Table caption text]{The maximal peak in-time PU for all eras of 2017 data-taking. }
\label{tab:PU}
\end{center}
\end{table}
\newpage

\section{Event and object reconstruction at CMS~\label{sec:objects}}

The events triggered by the HLT are saved in RAW data format which contains the raw information from the detector processed by the HLT. The data in the RAW format are then processed in order to reconstruct physics objects. The output of such reconstruction is saved in the RECO format. Most physics analyses do not need all the information present in the RECO files, and thus these files are slimmed to up to a miniAOD~\cite{Petrucciani:2029414} format in order to save space and computing time (in 2017, there is an effort to slim even further the size of the data, going from miniAOD to nanoAOD format). In this section several physics objects, which are important for the analyses discussed in this thesis, are introduced.


\subsection{The Particle-Flow algorithm}

As it can be seen in Fig.~\ref{fig:figures/CMStransverse}~\cite{Sirunyan:2017ulk} each kind of particle leaves a characteristic signature in the CMS detector. For example, an electron leaves a track in the tracker and then stops in the ECAL where it deposits the rest of its energy. The neutral hadron does not interact in the tracker but can leave energy both in ECAL and then HCAL where it stops.

    \insertFigure{figures/CMStransverse} % Filename = label
                 {0.9}       % Width, in fraction of the whole page width
                 { Transverse view through a sector of the CMS detector. The depositions of different particles in different subdetectors are indicated in the picture~\cite{Sirunyan:2017ulk}. }

The CMS approach to reconstruct an event takes this into account and combines information from all subdetectors at once to reconstruct the objects. The algorithm used for this kind of reconstruction is called the Particle Flow~(PF) algorithm~\cite{Sirunyan:2017ulk}. In general, the deposits left by the particle in the different subdetectors are connected by geometrical constraints.


\subsection{Leptons}

\subsubsection{Muons}

Muons are traversing the whole CMS detector and therefore their tracks can be reconstructed in both the silicon tracker and the muon chambers. The ``standalone muon'' is a muon track reconstructed from the hits in CSC, DT and RPC, while the ``inner track'' muon is reconstructed from the hits in the silicon strip tracker only. There are two options how to reconstruct muon using both information from the muon chambers and the tracker. The first one leads to the collection of muons called ``tracker muons''.  The tracker muons are first reconstructed in the tracker and then they are extrapolated to the muon chambers by matching the inner track with hits in the muon chambers. The second approach matches the standalone muon with an inner track based on geometrical criteria. The resulting collection is called ``global muons''~\cite{Chatrchyan:2012xi}. 

The PF algorithm uses both global and tracker muons to identify a PF muon. The leptons originating from the hard scattering are decay products of e.g. W,Z and H boson (prompt muons) and therefore they are not expected not to have any other activity in the proximity. A requirement on muon isolation is added in order to select a muon and reject jets misidentified as muons or muons from decay of heavy flavor quarks. The selected muons are then tested by several quality requirements to balance the efficiency and purity of the muon selection. Based on passing given criteria, several muon identifications (IDs) are assessed, e.g. loose, medium, tight. Muons are being reconstructed up to $|\eta|=2.4$ with a reconstruction and identification efficiency larger than 96\%. The muon trigger efficiency is higher than 90\%. Using information from both tracker and muon chambers the \pt resolution for muons with 20 <\pt < 100~GeV is 1.3--2.0\% in the barrel region and more than 6\% in the endcaps. For muons with \pt up to 1~TeV the transverse momentum resolution does not go above 10\%~\cite{Chatrchyan:2012xi}.

\subsubsection{Electrons}

The electrons are reconstructed with the track information from the tracker and the energy deposits in the ECAL~\cite{Khachatryan:2015hwa}. Because of bremsstrahlung, an electron looses in average around 30\% of its energy before reaching the ECAL. The energy deposit in the ECAL is reconstructed as a supercluster, taking into account the bremsstrahlung photons. The electron track in the tracker is reconstructed by the GSF algorithm~\cite{Adam:2003kg} starting either from the seed created by a few tracker hits or from the ECAL supercluster and then the track is extrapolated from the seed to the full tracker. In case of a tracker-based seeding, the next step is to match the ECAL supercluster with the electron track. Similarly as for the muon, an isolation criterion is requested for the prompt electron as well as quality requirements on the reconstructed electrons are imposed resulting in several categories of electron IDs e.g. loose, medium, tight.

Electrons are reconstructed up to $|\eta|=2.5$,  with an efficiency higher than 88\% in the \pt range from 10~GeV to 100~GeV and $|\eta|<2$. The energy measurement in ECAL and momentum measurement in tracker are used to estimate the momenta of electrons. The resolution on the momentum measurement is evaluated from $Z \rightarrow ee$ decays and its is between 1.7\%-4.5\% for electrons with  \pt~$\approx$~45~GeV~\cite{Khachatryan:2015hwa}.



\subsubsection{Taus}

In around 2/3 of cases the tau leptons are decaying hadronically to a mixture of charged and neutral hadrons and tau neutrino. The decay is very fast and thus it is difficult to reconstruct the secondary vertex. The algorithm which is used to recognize the hadronically decaying tau is called Hadron-Plus-Strips~(HPS)~\cite{CMS:2016gvn}. This algorithm searches for the neutral pions present in the majority of hadronic tau decays. The neutral pion, originating from tau, decays to a photon pair, the photons later converting to an electron/positron pair. These electrons/positrons bend in the magnetic field and thus broaden the energy deposits in the ECAL in azimuthal direction beyond the size of the reconstructed hadronic tau jet. To take this affect into account, the electromagnetic particles  are reconstructed into fixed size $\Delta \eta \times \Delta \Phi$ window called ``strip''. The strip is first associated with the most energetic electron or photon within the PF jet. Then the algorithm looks within the given window whether other electromagnetic particles are present close to the selected one. If yes, the particle is added to the strip. The algorithm proceeds up to the point when no other electromagnetic particle is present in the $\Delta \eta \times \Delta \Phi$ strip. Then the most energetic electromagnetic particle not belonging to any strip is associated to a new strip and the algorithm proceeds as before. The final algorithm searches for the hadronic taus in topologies with a single hadron, one hadron and one strip, one hadron and two strips, and three hadrons. The probability to misidentify a jet, an electron and a muon as an hadronic tau in data depends on the objects identification requirement and was measured to be around 0.01\%-4\%, 0.1\%-3\%, 0.03\%-0.3\%, respectively. Large differences were found in misidentification rate between data and simulation, leading to data to simulation ratio up to 1.66 in the case of electron and 1.86 in the case of muon. The hadronic tau identification efficiency is of around 50-60\% and is similar for data and simulation~\cite{Khachatryan:2015dfa}.

%The tau identification efficiency is measured in perspective of deriving data to simaltion scale factors. These factors vary from 0.83 to 0.96, depending on the tau identification technique.


\subsection{Photons}

Photons do not leave tracks in the tracker and thus their energy is obtained only from ECAL measurements. The prompt photon identification is based on two categories of observables, its shower shape and isolation from the remaining activity in the ECAL. Again, three photon identifications (IDs) are defined: loose, medium and tight. Achieved photon energy resolution in the barrel region is between 1\%-2.5\% for photons with energy of tens of GeV. In the endcaps the photon energy resolution is of 2.5\%-4\%~\cite{CMS:EGM-14-001}.

\subsection{Jets}

The jets are objects originating from the hadronization of quarks and are clustered with the $\mathrm{anti-}k_{T}$~(ak) clustering algorithm~\cite{Cacciari:2008gp, Cacciari:2011ma}. Within this algorithm the distance between objects is defined as

\eq{antikt}
{   
    d_{ij} = \mathrm{min}({p_{T}}^{-2}(i), {p_{T}}^{-2}(j)) \frac{(\eta_{i} -\eta_{j})^2+ (\Phi_{i} -\Phi_{j})^2}{R^2} =  \mathrm{min}({p_{T}}^{-2}(i), {p_{T}}^{-2}(j)) \frac{\Delta R_{ij}^{2}}{R^2},
}
where $p_{T}(i)$ is the transverse momentum, $\eta_{i}$ is the pseudorapidity and $\phi_{i}$ is the azimuthal angle of the object $i$. To reconstruct PF jets, the algorithm runs over PF reconstructed objects such as electrons, muons, photons, charged and neutral hadrons. The parameter $R$ is a jet radius parameter and for the standard jet reconstruction of Run~2 its value is set to 0.4. In the Run~1 this parameter was chosen to be 0.5. In case of boosted objects decaying to partons, two or more jets originating from boosted parton can be merged. These topologies can be reconstructed as so called ``fat jets'', and for purpose of their reconstruction, the parameter $R$ is increased to 0.8 or 1.0. The $\mathrm{anti-}k_{T}$ algorithm, by definition, clusters first the hard objects with the shortest distance between each other and continues with objects further apart. The clustering stops when no hard enough particle is nearby. The product of the clustering is a jet. The charged component of the jet can be reconstructed only up to $|\eta|=2.4$ due to the tracker coverage, while the neutral component extends up to  $|\eta|=5$.

The measured jet energy does not in general correspond to the energy of parton responsible for the jet, because of several inefficiencies and biases in the energy measurement. Therefore the jet energy must be calibrated and re-scaled by a correction factor in order to have a correct jet energy scale. The jet energy corrections are obtained from simulations. Additional jet energy scale correction is derived from measured dijet, photon+jet, Z+jet, and multijet events to take into account differences between data and simulation~\cite{Khachatryan:2016kdb}. The typical PF jet energy resolution is around 15\% for 10~GeV jets, 8\% for 100~GeV jets, and 4\% for 1~TeV jets. %from page det 

\subsection{b-jets}

In order to know from which quark the jet is originating, flavor tagging techniques were developed. The target of the heavy flavor tagging techniques is to recognize jets originating from a b quark~(b-jets) or a c quark~(c-jets). In the following, only b-jets are discussed. The b-quarks are forming B-mesons, which decay within about 1.5 ns creating a secondary vertex  displaced by few mm up to cm from the primary vertex. The information on a displaced secondary vertex or displaced tracks in the jet, information about particles originating from the secondary vertex and optionally information on soft leptons within the jet are used by b-tagging algorithms to tag jets formed from the b-quarks~\cite{Sirunyan:2017ezt}. Due to the tracker geometry, b-jets can be reconstructed only up to $|\eta|=2.4$.

There are several algorithms, whose purpose is to tag a b-jet. The first one is the Combined Secondary Vertex~(CSVv2) algorithm, which combines information about displaced tracks with information on the secondary vertex. The DeepCSV algorithm improves the performance with respect to the CSVv2 algorithm by using a deep neural network. The Combined Multivariate Analysis~(cMVAv2) technique takes into account a fact, that B-hadrons can decay leptonically and thus soft electrons or muons can be present in the b-jet. In this algorithm, the information about soft leptons is used in combination with other taggers. The efficiency to tag a b-jet versus the probability to misidentify a c or light flavor jet as a b-jet for different taggers is shown in Fig.~\ref{fig:figures/btag}~\cite{Sirunyan:2017ezt}. The performances of taggers were evaluated from simulation.


    \insertFigure{figures/btag} % Filename = label
                 {0.7}       % Width, in fraction of the whole page width
                 { The probability to misidentify a c or light flavor jet as a b-jet versus the efficiency to identify a b-jet for different b-tagging algorithms. These curves were evaluated from simulated $t\bar{t}$+jets events, using jets with \pt > 20 GeV~\cite{Sirunyan:2017ezt}. }

%The performance of b-jet taggers is different in data and event simulations therefore a multiplicative data-to-simulation factor must be used on top of the simulation in order to compensate for the differences.

\subsection{Missing transverse energy}


If the momenta of all particles could be measured, the sum of all momenta in the plane transverse to the beam would be zero due to the momentum conservation law. But as neutrinos escape the detector undetected, an imbalance of momentum in this plane is observed. 

The Missing Transverse Energy~($E_{T}^{miss}$) is the magnitude of the negative vectorial sum of the transverse momenta of all PF particles~\cite{CMS:2016ljj}. Because of the energy thresholds in the calorimeters, inefficiencies in the tracker or non-linearity of calorimeters' response for hadrons, the \MET measurement can be biased and thus an energy correction factor must be used. This factor accounts for effects influencing the \MET and is applied on the transverse momenta of the jets. The \MET is then recomputed with the new transverse momenta of jets. The uncertainty on \MET is evaluated by varying the $p_{T}$ of each kind of object within its resolution. The  \MET  distributions for  $Z \rightarrow ee$ and $Z \rightarrow \mu \mu$ events are shown in Fig.~\ref{fig:figures/met}~\cite{CMS:2016ljj}. These plots show a good agreement between data and simulation. The $Z \rightarrow ee$ and $Z \rightarrow \mu \mu$ events have no genuine source of the \MET , no neutrino in the final states. Therefore the \MET is expected to be zero, but due to the \MET resolution, non-zero values are observed.  


    \insertFigure{figures/met} % Filename = label
                 {0.9}       % Width, in fraction of the whole page width
                 { \MET distributions for $Z \rightarrow \mu \mu$~(left) and $Z \rightarrow ee$~(right) events with the data to simulation ratio in bottom. There is also a contribution from two other groups of processes decaying to the dimuon or dielectron final states. The Top contribution originates from the $t\bar{t}$ and single top processes, in which the W-bosons decay leptonically to charged lepton and neutrino and therefore these events have a genuine source of the \MET. The EWK group of processes consists of diboson, $Z\gamma$ and $W\gamma$ production processes, which are also partly present in the \MET tail due to leptonically decaying W-bosons in some of them~\cite{CMS:2016ljj}. }

The \MET variable plays a crucial role in searches for the physics beyond the Standard Model, as many theories predict stable weekly interacting particles which would enhance the \MET.




\chapterwithnum{Study of highly ionizing particles in the strip tracker}

\section{Tracking inefficiencies at beginning of RunII (or clustering, to be decided)}

\subsection{Observed inefficiencies in tracks reconstruction}
\subsection{Highly ionizing particles as possible explanation}
        (how the nuclear event looks like (inelastic, most energy from recoil))


\section{Strip tracker readout system}

\subsection{Overview}
       (give overview about the readout - how the signal goes)
APV25~\cite{French:2001xb}
FED~\cite{Baird:2002wg}
\subsection{Silicon strip modules}
%TODO check the pitch size

The CMS silicon sensors are formed by n-type bulk, which has on one side uniform n+ implant while on the other p+ strips are located. The implants are connected to reverse bias voltage to completly deplete the bulk of the sensor. The thickness of both p+ and n+ implants is small and negligible compared to the bulk, thus almost whole volume of sensor is depleted. Every p+ strip are connected by a wire bond to a read-out electronics.

%TODO is true 4-6 sensors?, or one sensor with more or less strips?
The tracker modules are consisted by 4-6 sensors, each sensor is having 128 p+ strips. The larger part of modules have one layer of sensors~(mono modules), the other holds two layers of sensors, which are attached back to back and with a strip inclination of $5.7^{\circ}$ against each other~(stereo modules). Thus the stereo modules are able to give 2-D information about the position where particle hitted the module~(hit position). The modules also differ by the pitch size between each strip which can vary from 80 $\mu$m up to 200 $\mu$m depending on the tracker layer and partition.

The particle crossing silicon sensor is leaving energy predominantly via electromagnetic interaction - by ionization of the silicon volume, the electron-hole pairs are produced along the path of a particle. The energy loss in the material can be described by the Bethe-Bloch formula~\cite{Groom:2000sm} as a function of $\beta\gamma = p/Mc$, where $p$ and $M$ is mometum and mass of the interacting particle. The Bethe-Bloch function has a minimum for $\beta\gamma \approx 3$. Majority of relativistc particles are having this minimal value of $\beta\gamma$ and thus they are called be Minimum Ionizing Particles~(MIP).

Under normal circumstances the created charge carriers, electrons and holes, would drift on opposite sides directly towards electrods (n+ and p+ implants). But as in the barrels case a perpendicular magnetic field is present, the charge carrier $q$ is deflected from the direction of electric field due to the Lorentz force

\eq{LorentzEquation}
{
    q(E+v \times B).
}


-cross talk

\subsection{The APV25 readout chip}

The charge collected by each channel~(strip) is read by APV25 chip located at the module. The APV is front-end chip providing amplification and shaping of signal from each channel separately. To achive this all APV chips are equipped by preamplifier, CR-RC shaper, analog pipeline and deconvolution filter. As one APV chip is reading signal from 128 strips, 4-6 chips are present at each module.

The amplified signal is sent to CR-RC shaper to convert strip signal into $\sim$50~ns long voltage pulses. The shaper is providing an output with frequecy of bunch crossings. In case of APV working in the ``peak mode'' the sampled signal at maximum of the pulse shape, corresponding to given bunch crossing is used directly, but usually the APV is operating in ``deconvolution mode'' to reduce the out-of-time pile-up. In the deconvolution mode the weighted sum of the shaper output from three consecutive bunch crossing is used instead.To have a possiblity to optimize the puls shape, the feedback resistors of both preamplifier and shaper as well as  bias current and voltage are fully programable. For the calibration and test of the chip the internal calibration channel is present. This channel enables to inject charge to channels prior to the preamplifier stage.

The puls heights and bunch crossing information for all 128 channels are extracted at the end of the analog pipeline upon the request of trigger. The average signal level from 128 channels can be adjusted within the dynamic range of the APV, in order to reduce the signals exceeding the APV range. The processed signal from two APV chips is multiplexed by APVMUX~\cite{Ball:2007zza} into a single line and converted by laser from an electrical to optical signal, which is sent via analog optical fibre to control room. At the control room the optical signal is recevied by the pin diod which is a part of FED.



\subsection{The Front End Driver}

The FED is recieving data from 96 optical fibers, each sending information from 2 APVs. The data in form of optical signal are converted to electrical signal, they are reordered and synchronized. For each APV input the signals per channel are extracted and digitized into 10-bit range AD counts~(ADC). In the standard operation mode, the ``Zero Supression~(ZS)'', the pedestal subtraction followed by common mode noise~(CMN) subtraction is performed. Pedestal is mean strip activity for given strip, which is evaluated from special ``pedestal runs'' taken few times per year. After pedestal subtraction the CMN is remining noise, which is common to all channels and calculated on event by event basis as a median over the 128 strips. After both subtractions the channels with values lower than zero are truncated to zero. For all the channels the signal-to-noise ratio is checked separately and if too low, the ADC of those channels is set to zero~(zero suppressed). Only information about strips with non-zero ADC values are sent to the CMS data acqusition system~(DAQ). By this procedure the available data are reduced by factor $\sim$60 in order not to overload DAQ system.


For testing purposes the FED is able to operate in others than ZS mode. In case of the ``Virgin Raw~(VR)'' data taking mode no subtractions and suppressions are applied and thus it is suitable mode e.g. for studies of the APV behavior.

\subsection{Offline data treatement}

clustering - hits - signal shared by adjacent channels -> adjacent channels added to cluster when they eceed certain thershold. three threshold algorithm - threshold on strip seed charge, charge of eighbouring strips and threshold on cluster charge.
-tracking -connecting hiots to tracks
recHits - position of the charge despoition -> cumputed from cluster barycenter
-clustering - 8 bit truncation
than tracking algo which is doing: %read a bit more
-seed finding
-pattern recognition
-fitting and smoothing
provides track estimates from combining hits. from the seeds (pair or triplet of hits) the tracks are built by scuceeding Kalaman technique. compatible hits in next layer are evaluated by chi2 computation. If good hit found it iss added to track. Once all hits available -track fitting which starts to evaluate new hits position -> from the seed and it is done interatively for each layer with updated track position. Then smoothing which is backward fitting using the trajectory found. From this we get final track and hits.

%Remove the word tuning
%Explain how the simulation is working → how to simulate cluster, describe how it is done
%Present what is the situation
%Description of width not so great
%Not put too much emphasis on the other tests
%Tehn found xtalk is responsible
%Cross talk measurement → 
%Limitations
%Then xtalk from MC + ePerADC

\clearpage

\setcounter{secnumdepth}{4}
\chapterwithnum{Silicon strip tracker simulation}
\setcounter{secnumdepth}{5}


\section{CMS simulation}

The simulated samples are vital part of many analyses. For the physics analyses purposes they are used to copmare the theoretical signal and background with the measured data.  Further, the simulations are also important in development and understaning of specific analysis methods and in derivation and valiadation of calibrations, efficiencies and resolutions.


%steps
The CMS simulation workflow~\cite{Banerjee:2007zz, Hildreth:2017vpw, Hildreth:2015kps, website:simuBasics } is divided into several steps. At the beginning of the simulation chain the physics events are generated and then the generated final state particles are sent through the simulated detector. Following step is the simulation of response of electron electronics to particle traversing the detector. The otput of this producedure are RAW data, which can be later reconstructed and slimmed for the purposes of physics analyses. The overview of the simulation steps, which will be described in larger detail in following subsections, can be seen in Fig.~\ref{fig:figures/SimulationFlow}~\cite{website:simuBasics}. The production of the simulated samples is handled centrally~\cite{Boudoul:2015bkp} by the CMS collaboration.

    \insertFigure{figures/SimulationFlow} % Filename = label
                 {0.99}       % Width, in fraction of the whole page width
                 { A diagram of simulation workflow. The four-vectors of generated particles together with the detector description enter to the Geant4 simulation which output are simulated hits in the detector volume. Optionally the simulated hits from pile-up interactions can be added on top of the simulated hist from physics and the mixture of these hits are digitized in the electronics simulation. In this step the description of the electronics, for eaxample the noise and detector conditions, for example the temperature, is added. The output of digitization are RAW data~\cite{website:simuBasics}. }

\section{Monte Carlo event generators}

The Monte Carlo~(MC) generators are basic tool designed to produce physics events according to a physics model. In the majority of cases three kinds of generators are used in CMS~\cite{website:generation, website:generationIntro}. 

\textbf{General-purpose generators}
These are for example Pythia8~\cite{Sjostrand:2014zea} or Herwig++~\cite{Bahr:2008pv}. The provide the best possible description of the result of the proton collision. To generate outgoing particles originating from the interaction of colliding particles, many theoretical models and aspects has to be plugged in the generation process, such as the description of soft and hard interactions~(in leading order), parton distribution functions~(PDFs), initial and final state radiation~(ISR and FSR), multiple parton interactions, hadronization of partons and decay of particles~\ref{}.

\textbf{Matrix Element calculators}
The generators such as Powheg~\cite{Oleari:2010nx} or MadGraph5\_aMCatNLO~\cite{Alwall:2014hca} were developed to provide next-to-leading order~(NLO) claculations. These calculators give the final state description on the parton level which needs to be plugged into one of the general-purpos generators to proceed with  the full hadronization.

\textbf{Specific generators}
These genrators are used to generate specific kind of events e.g. diffractive or cosmic events.

-> output: particle four-vectors

\section{Detector simulation}

To be able to compare the data and simulations, the generated particles need  to be propagated through the volume of detector. This is achieved via GEANT4~\cite{Agostinelli:2002hh ,Lefebure:1999wja} toolkit into which detailed description of the CMS detector, its active and dead material dimensions, hierarchy and properties, is plugged. The GEANT4 sends the generated particles through the detector and simulates the interactions with material and modells the physics processes which happen during the passage of the particles through detector. The output of this procedure are simulated hits left by particles interacting with the active volumes of the subdetectors. The simulated hits can originate from primary particles generated by the MC generator, or from the secondary particles which are result of the GEANT4 simulation process.

The simulation of the pile-up events is done separately from the simulation of the events of interests. The input to  the GEANT4 simulation of in-time and out-of-time pile-up is pool of Minimum Bias single interaction events.

This full simulation~(FullSim) is very time intensive and thus it is not suitable to simulate samples for which huge number of events is needed. For this purposes the fast simulation~(FastSim)~\cite{Sekmen:2017hzs, CMS:2010spa, Giammanco:2014bza} was developed as an alternative to the FullSim. The FastSim uses simplified detector geometry and interactions with material, what speeds the simulation by factor of around 100. The comparison of physics objects of FullSim and FastSim shows that FastSim is reliable alternative that reproduces the FullSim with around 10\% accuracy~\cite{Abdullin:2011zz, Sekmen:2017hzs} The FastSim is widely used to produce for example Supersymmetry samples, where large scans with different parameter values are needed. 

\section{Simulation of the detector response to the particle signal}

The next step is to simulate the collection of the signal obtained from the GEANT4 and the response of the readout electronics to this signal. This step is called a digitization and its input is merged collection of physics events of interest and pile-up events. There are three domains providing digitization of given siubdetectors, which are SimTracker, SimCalorimetry and SimMuon~\cite{iwebsite:simdigi}. The digitized saples are in the RAW format and can be further reconstrcutedi in a similar way as data.

\section{Simulation of the silicon strip tracker to the particle signal}


Geant 4:~\cite{Lefebure:1364020}
3.1 CMSTrackerHit
One CMSTrackerHit object is created
for each new particle entering a Tracker-like component (delta-rays are considered as new particles)
for each sensitive detector unit
The information provided by the CMSTrackerHit class is:
1) the entry and exit point of the particle in the local reference frame of the detector unit,
2) the energy of the particle when it enters the detector unit,
3) the identification of the track and of the detector unit,
4) the time that the particle has been alive until it enters the detector,
5) the total amount of energy deposited by the particle along its trajectory in the detector.


%magnetic field setting

%in detector dir
%TS2018_001_2 -> MC generators, detector simulatios
%CERN-THESIS-2017-300 - Event simulation
%TS2017_028_2 - simulation super short

\chapter{SUSY}

\section{QFT}

-fields and particles
-lagrangian
-feynman diagram


\section{SM and its shortcomings}

-sm intro
-particles
-interactions (electroweak, QCD)
-19 free parameters (nine fermion masses, one scalar mass, three coupling parameters, four quark mixing parameters, higgs vacuum expectation value, strong cp violating phase)
-perturbative theory (LO, NLO)

issues

1)hierarchy problem
corrwctions to Higgs boson mass
to compute cross section, all quantum loop corrections has to be taken into account
fermions an vector boson masses proctected from diverging by mechanism within the SM
but no mechanism for Higgs mass: $mh^2~ mh0^2+k mPlanck^2 $ - parameters mh0, k and mPlanck a priori unrelated. But these parameters must be fine tuned in order to obtain mass of Higgs (mh<<mPlanck) -> not natural
called hierarchy problem - no reason to expect a large hierarchy between electroweak scale and planck scale

2)dark matter
-measuremnt of rotation curves of galaxies - first dark matter hypothesis
-gravitational interaction, but not electromagnetic -> dark matter
-from observations several constraints on dark matter - not short -lived and not baruonic, gravitationally interacting, low kinetic energy (cold -> it cannot be neutrino)
-> no good candidate within the SM
From cosmological observations we expect dark matter mass of order of 100~GeV

3)Dark energy
-cosmological constant (lambda) in einsteins equation necessary to explain the observed expansion of universe
-> cosmologica constant can be interpreted as a vacuum energy

4)Matter-antimatter assymetry
-matter and antimatter should be produced in smae amount at big bang
-but our world dominated by matter

5)Neutrino masses
-neutrinos oscialte from one flavour to other -> this can only happen when neutrinos are masive and have different mass states than flavour states

6)Strong CP phase
strong QCD lagrangian introducing the phase theta - close to zero, despite the theoreticla arguments that it should not be like this

7)Quantum gravity
-gravity not described by SM
-desired to unify general relativity with QFT

8)Unification of forces
-possibility to unify all interactions

9)open questions
-in SM large differences between quarks
-why there should be three fermion families


\section{BSM physics}

BSM theories
-SM works fine, but we need to extend it -> we can add either additional symetries, space-time dimensions or field content

SUSY
-around 70's
-Golfand and Likhtman -> new symmetry Q -> Q|f> -> |b>; Q|b> -> |f> -> later Haag, Lopuszanski and Sohnius said that such symmetry corresponds to supersymmetry
-to each fermion a boson  with same quantum numbers (except of spin)
-partner of praticle is superpartner and they form superfield
-spin differs by 1/2
-superpartner should have the same mass -> not observed -> susy must be broken (for now we just add a term into the lagrangian)
-motivation:
	-solve hierarchy problem (superpartnes have equal masses and cancel the loop corrections) - in case of "soft breaking" susy prevents the quadratic divergencies and there are only logarithmic + small fine tuning
	-> naturalness of susy related to the mass difference between particle and its superpartner (Q: then if the susy partner of eg electron is very heavy does not it induce the divergencies? )
-MSSM
-most used SUSY relization is MSSM -> minimal -> adding the minimum number of fields to the SM to become supersymmetric
-adding sfermions and gauginos - left and right handed fermions -> e.g two selectrons
-for Higgs more complicated- one higgsino is not enough, but second SU(2) doublet is needed to avoid a gauge anomaly.
-mass eigenstates do not have to be flavor eigenstates
-combinations of electroweak gauginos and higgsions make charginos and neutralinos
-mixing between left and right superpartners 
-spectrum of sparticles
-conservation of R-party by construction $R=(-1)^{2S+3B+L} $
	-> pair production of sparticles
        -> decay only to odd nr of sparticles
        -> LSP is stable -> dark matter candidate
-MSSM - more than 100 new parameters than in SM
-> too many parameters - problem for phenomenological and experimental models
-> pMSSM - phenomenological MSSM -> reduction of number of parameters by assuming
	- there is no new source of CP vilation
	-lightest neutralino is the LSP
	-other assumptions on the sfermion masses, trilinear couplings and flavor violation
	->reduction of parameters to 19

\chapter{Stop}

\section{Introduction and motivation}

The supersymmetry was found to be the most popular extention of the Standard Model, due to its capabality to adress many shortcomings of the SM. Among others it provides solution to the naturalness probelm and provieds dark matter candidate. Supersymmetry introduces a supersymmetric partner to each SM particle, which have the same quantum numbers except of spin differing by 1/2. This chapter is focused on the production of the suppersymetric partners of the stop quarks, reffered as ``stops''~($\tilde{t}$). There are two scalar stops $\tilde{t}_{R}$ and  $\tilde{t}_{L}$ as there is left and right component of the SM fermion field. These two stops mix into mass eigenstates $\tilde{t}_{1}$ and $\tilde{t}_{2}$,  $\tilde{t}_{1}$ being the lighter one. Due to the naturalnes constraint the ligter stop should have mass in TeV range. Morover the Higss mass measurements give condition that $\sqrt{m_{\tilde{t}_{1}} m_{\tilde{t}_{2}}}$ should be around 600~GeV. In this chapter a SUSY model, in which the R-parity is conserved and the LSP is the lightest neutralino~($\tilde{\chi}^{0}_{1}$), is considered. Furthermore only direct stop pair production is assumed.

Depending on the mass difference between stop and neutralino, $\Delta m =  \tilde{\chi}^{0}_{1}$, several decay modes of the stop quark are possible. The stops can decay via two, three or four body decays to final states with b or c quarks. This chapter discribes search for top squark pair production in pp collisions at sqrt(s)=13~TeV using single lepton events~\cite{Sirunyan:2017xse}. As I was also involved in previous versions of this analysis~\cite{Sirunyan:2016jpr, CMS:2016vew}, the differences between the three analyses are briefly discussed.



-intro what we focus on

- mtoivation
	two stpos of different mass, stop1 and stop2
	short conclusion of susy chapter - light stop, 1-lep, high xsection -> basic or sth more like Alex?
	single lepton high br and low backgrounds
	targeting few posssible decay chains
	can be destinguished from backgrounds when cutting on MET, MT...
	32\% of directly produced stops decay to final states with one lepton
-signature  and possible decays
	SMS, T2TT, T2bW. T2tb, decay cascades, feynam diagrams
	r parity conservation
	

-plane od delta M
	search in delta M plane


-general intro
	- I worked on more versions of analyses, but only one introfuced and than the differences, and then some chosen personal contribution

-delta M around top mass challenging kinematics -> looks like SM tt -> this region is called stealthy region
-for t2bW when W becomes on-shell also difficult kinematics




\subsection{Signal topologies}

The preseneted analysis focus on the kinematic region where $\Delta m > m(W)+m(b)$ and on three different decay modes of the stop pair.  All modes lead to states with two b-jets, two W-bosons and two neutralinos, but the kinematics differ depending on the decay mode and $\Delta m$. The $\Delta m$ plane is shown in Fig.~\ref{fig:figures/dmplane}. In the part of the plane where $\Delta m < m(t)$, often reffered as ``compressed spectra'' region, the decay producs are soft and often not reconstructed. On the other hand when the $\Delta m$ is large, boosted topologies can be expected. 

    \insertFigure{figures/dmplane} % Filename = label
                 {0.99}       % Width, in fraction of the whole page width
                 { dm plane ~\cite{Aad:2014kra}. }

In this analysis thei targeted final states are with one leptonically and one haronically decaying W-boson, resulting in one charged lepton in final state. The advange of this one lepton channel is its relatively high branching ratio, around 32\% of directly produced stop pairs decay to final states with one lepton, and low occurance of the Standard model backgrounds. 

The first of the three considered decay chains of the stop pair shown in the left part of Fig.~\ref{figures/stopdecays} is reffered as ``T2tt'', where both stops decay to top quark and neutralino, followed by decay of each top quark to W-boson and b-quark:

\eq{t2tt}
{
    \tilde{t}_{1} \bar{\tilde{t}}_{1} \to t \bar{t} \tilde{\chi}^{0}_{1} \tilde{\chi}^{0}_{1} \to b \bar{b} W^{+} W^{-} \tilde{\chi}^{0}_{1} \tilde{\chi}^{0}_{1}.
}

The second possibility depicted in the right part of Fig.~\ref{figures/stopdecays} is reffered as ``T2bW'' and in this case both stop quarks decay via intermediary chargino:

\eq{t2bW}
{
    \tilde{t}_{1} \bar{\tilde{t}}_{1} \to b \bar{b} \tilde{\chi}^{+}_{1} \tilde{\chi}^{-}_{1} \to b \bar{b} W^{+} W^{-} \tilde{\chi}^{0}_{1} \tilde{\chi}^{0}_{1}.
}

    \insertTwoFigures{figures/stopdecays}
                 {figures/T2tt} % Filename = label
                 {figures/T6bbWW} % Filename = label
                 {0.45}       % Width, in fraction of the whole page width
                 { decay diagrams ~\cite{website:SUSYdiagrams}. }

In the T2bW the chargino mass is fixed to halfway between the mass of stop and neutralino. The third decay shown in Fig.~\ref{figures/T4tbW} combines the previous two. In this case, reffered as ``T2tb'' one of the stops decay to top quark and neutralino and the second to bottom quark and chargino.

\eq{t2tb}
{
    \tilde{t}_{1} \bar{\tilde{t}}_{1} \to t b \tilde{\chi}^{0}_{1} \tilde{\chi}^{+}_{1} \to b \bar{b} W^{+} W^{-} \tilde{\chi}^{0}_{1} \tilde{\chi}^{0}_{1}.
}

    \insertFigure{figures/T4tbW} % Filename = label
                 {0.5}       % Width, in fraction of the whole page width
                 { mixed decay diagram ~\cite{website:SUSYdiagrams}. }

In T2tb the chargino and neutralino are almost mass degenerate, the chargino mass is fixed by relation $m_{\tilde{\chi}_{1}^{\pm}} = m_{\tilde{\chi}_{1}^{0}} + 5~\mathrm{GeV}$  motivated by cosmological observations. Because of this small difference between chargino and neutralino masses, the W-boson is produced ofshell and its decay producst have low transverse momenta and are probably not reconstructed. If the jets from hadronically decaying W-boson are not reconstructed, only the two b-jets are present in this signal topology, what is being one of the motivation to search for final states with less jets than the four expected.


-what they are, how they look like
-t2tt, t2bw,...

-> the motivate variables and backgrounds



\subsection{Analysis strategy}

Because of the two neutralinos, the signal processes (T2tt, T2bW and T2tb)  can have final state signature largery differing from the background processes coming from Standard Model. This knowledge of the signal and background signature is used to define the baseline search region, in which the SM background is supressed. This baseline search region is then divided to smaller signal regions with help of discriminating variables. These variables are useful to define signal regions where some of the signals are enriched or some fo the backgrounds are more reduced. The remaining backround is estimated from data-driven techniques or simulations for each signal region. In teh baseline search region there are three groups of backgrounds present. 

The leading background is ``Lost lepton'' background mainly coming mainly from $\mathrm{t\bar{t}} \to 2 \ell$ processes, where both W-bosons decay into leptons and one of the charged leptons is lost because of acceptance or other critearia. This background is reduced by applying veto on second lepton, which was misreconstructed, but even after this it remains the largest backround. The subleding background is ``One lepton'' background, which mainly composed of W+jets and $\mathrm{t\bar{t}} \to 1 \ell$ processes in which the W-boson decays leptonically. The last relevant background is denoted ``$Z \to \nu \bar{\nu}$'' and comes from processes such as $t\bar{t}Z$ and $WZ$ in which the Z-boson decays to two neutrinos and one of the W-bosons leptonically.  

The signal and estimated background yields are compared with data in the signal regions and in case that no excess from the SM is observed, the exclusion limits can be put on given signal models.

In the following two subsections \ref{sec:variables}, the variables designed for definition of baseline search region and its division to signal regions will be introduced and justified on the exaples of signal topologies.  It is also discussed why and how much the backgrounds should be suppressed with regard to the choice of the variables. In the end the final definition of the baseline and signal regions for the presented analysis is revealed.

%binned approach in variables which tend to reduce signal
%single top?

\subsection{Variables~\label{sec:variables}}

\textbf{Number of leptons~($N_{\ell}$)}

The presented search focuses on final states with one charged lepton and therefore requirement on number of leptons to be equal to one must be imposed. The lepton is either electron or muon and the events with tau leptons are rejected.

\textbf{Number of jets~($N_{J}$)}

As shown on example in blue on the left of Fig.~\ref{fig:figures/T2ttlep1}, four jets are expected in the final state. But as already discussed, because of the low mass diffference between chargino and neutralino in case of T2tb, the two jets from hadronically decaying W-boson have low \pt and thus they do not have to be reconstructed and only two jets are expected. In case of high $\Delta m$ the decay products of stops can be boosted and thus two or more jets can be merged into one, resulting to reduced number of reconstructed jets in final state. The baseline requirement is to have at least two jets, but this variable is also used for the definition of signal regions targeting different topologies.

    \insertTwoFigures{figures/T2ttlep1}
                 {figures/T2ttlepJET} % Filename = label
                 {figures/T2ttlepMET} % Filename = label
                 {0.45}       % Width, in fraction of the whole page width
                 { decay diagrams ~\cite{website:SUSYdiagrams}. }

\textbf{Number of b-jets~($N_{b}$)}

In the left diagram of Fig.~\ref{fig:figures/T2ttlep1} it can be also noticed that two of the jets in green are originating from the b-quark. For this reason number of b-tagged jets selected by medium working point of CSVv2 are one of the disriminating variables. In the signal regions where W+jets background is dominant, the tighti b-tagging working point is used to further suppress it, as teh jets are expected to be mainly composed of light flavours.

\textbf{Missing transverse energy~(MET or $E_{T}^{miss}$)}

Right diagram of Fig.~\ref{figures/T2ttlep1} shows that in the final staes of the signal processes there are two neutralinos and neutrino originating from the leptonicallyd ecaying W-boson. These particles escape detector and cause missing energy in transverse plane, reffered as ``Missing transverse energy''. This variable is very powerful in rejection of SM backgrounds bacouse they tend to have small values of MET as the sources of the MET are more limited than in case of signal. Therefore the basline selection requires MET to be larger than 250~GeV. The distribution of the MET in the baseline search region for all relevant backrounds and selected signal points is shown in the left plot of Fig.~\ref{fig:figures/METMT}. In this Figure the one lepton background is decomposed to ``1l from top'' composed of $t \bar{t} \to 1\ell$ and ``1l not from top'' populated by W+jets processes. The MET was also found to be good variable for definiton of signal regions.

Because of the two neutrinos and misreconstrcuted lepton in lost lepton background, the MET of this background can be very large

\textbf{Transverse mass of lepton-MET system~($M_{T}$)}

The MET in combination with the $M_{T}$ defined as

\eq{MT}
{
 M_{T} = \sqrt{2 p_{T}^{\ell} E_{T}^{miss} (1 - \mathrm{cos}(\phi)) } ,
}

where $p_{T}^{ell}$ is the transverse momentum of lepton and $\phi$ is the angle between momentum of lepton and MET, ensure dramatic reduction of the SM backround. The $M_{T}$ is  designed to suppress backgrounds where both lepton and MET (neutrino) come from one W-boson. In such case the $M_{T}$ has an endpoint at W-boson mass. As shown in the left diagram of Fig.~\ref{fig:figures/T2ttlep2} for the signal the leptonically decaying W-boson is not the only source of MET and therefore it has no endpoint.

    \insertTwoFigures{figures/T2ttlep2}
                 {figures/T2ttlepMT} % Filename = label
                 {figures/T2ttlepDPHI} % Filename = label
                 {0.45}       % Width, in fraction of the whole page width
                 { decay diagrams ~\cite{website:SUSYdiagrams}. }

The baseline selection on  $M_{T}$ variable was chosen to be  larger than 150~GeV, which supresses hugely mainly one lepton bacgrounds. In case of $t \bar{t} \to 1\ell $ the $M_{T}$ has an endpoint at W-boson mass ($\sim$ 80~GeV). %@MJ@ TODO to be continued
For the lost lepton background there is no endpoint as the MET does not originate from one W-boson.

%TODO plot MT

    \insertTwoFigures{figures/METMT}
                 {figures/LogMET2j} % Filename = label
                 {figures/LogMT2j} % Filename = label
                 {0.45}       % Width, in fraction of the whole page width
                 { decay diagrams ~\cite{website:SUSYdiagrams}. }

\textbf{Minimal azimuthal angle between direction and  one of the two leading jets and MET~(min$\Delta \phi (j_{1,2}, E_{T}^{miss})$ )}

In the backgrounds where the neutrino is only source of the MET it is probable that the neutrino is close to the b-quark  originating from the same top decay. This mainly appears for $t\bar{t} \to 1\ell$ process as shown in the left plot of Fig.~\ref{fig:figures/DPHIMLB}. In case of the signal, as depicted in the right diagram of Fig.~\ref{fig:figures/T2ttlep2}, there are more sources of MET and therefore there is no constraint on this variable. The chosen baseline requirement on min$\Delta \phi (j_{1,2}, E_{T}^{miss})$ to be larger than 0.8 is considerably reducing the SM backgrounds but leads to the relatively small reduction of the signal.

    \insertTwoFigures{figures/DPHIMLB}
                 {figures/LogMDPhi2j} % Filename = label
                 {figures/LogMlb2j} % Filename = label
                 {0.45}       % Width, in fraction of the whole page width
                 { decay diagrams ~\cite{website:SUSYdiagrams}. }
\textbf{Invariant mass of the reconstructed lepton and the closest b-quark~($M_{\ell b}$)}

In case that lepton and b-jet come from one top, there is a bound on the $M_{\ell b}$ variable wich is

\eq{Mlb}
{
 M_{t} \sqrt{1 - \frac{M_{W}^{2}}{M_{t}^{2}} } \approx 153~\mathrm{GeV} ,
}

where $M_{W}$ is the mass of the top quark and $M_{W}$ the W-boson mass. This limit is true for backgrounds coming from $t\bar{t}$ decay, but also for T2tt signal. On the other hand there is no limit for W+jets backgrounds and T2bW signal. Therefore this variable is not suitable for baseline region definition, as puting a constraint on it would considerably reduce the signal but it can be used to define signal regions with enriched one kind of signal and supressed one type of the background. The distribution of  $M_{\ell b}$ for all relevant signals and different backgrounds in the baseline search region is shown in Fig.~\ref{fig:figures/Log_Mlb_2j} TODO

\textbf{Modified topness~($t_{mod}$)}

\textbf{MT2W~($N_{J}$)}

\textbf{Number of W-tags~($N_{J}$)}

-put picture of some of my presentetions with rounds (midterm probably)i
leptonic diagram~\cite{CMS:2016vew}
-variables
	MET significance?!
	MT -formula; suppress largely tt1l background; and also single top and W+jets; kinematic endpoint at W mass
	MT2W - formula; explanation; supress lost lepton (already suppressed by requiring no additional lepton, but not enough); tries to reconstruct the event under tt2l and one undetected lepton assumption; for signal large delta M leads to alrge MT2W, while small delta M has lage MT2W; endpoint at top mass
	tmod - formula; chi2 like variable how well the event agrees with tt2l hypothesis, similar behavior to Mt2W; removing some of the terms from oifficial topness helps the discrimination; works better at low jet multiplicities than MT2W
	min dphi (jet, MET) - formula
        Njets
	Mlb

\subsection{Backgrounds}
-backgrounds 
	what type of
-how they look like, how they can be distingusehed from signal
	tt2l - lost lepton does not obey MT<MW
	W+jets - offshel Ws - no endpoint at W mass
	Znunu no bound Mt<MW

\subsection{Triggers, data and simulated samples}

	single lepton and MET triggers
	double lepton - to check kinematics of lost lepton
	single photon - to check MET resolution

-MC samples
	fastsim

\subsection{Objects and event selection}
	vertex selection 
	lepton selection - high pt isolated lepton, no veto lepton to reduce tt2l
	isolated track veto - tracker isolation to avoid taus
	hadronic tau veto (2015) - additional to the isolation, or selects additional taus
	jets - ak4 (pt>30, eta<2.4), medium working point for btag(CsVv2) (2015)
	MET - sum of PF candidates, type-1 corrected - jet energy correction applied to the jets in MET calculation, MET filters


\section{Search strategy - latest analysis}
- preselection (baseline selection), nr of jets, min dphi, MT, MET
then binning in 4 search variables
	N jets - we expect 4, but jet can be lost if neutralino and chargino are mass degenerated (soft jet), or if jets are merged
        	(ICHEP 2,3 and 4+ jets, now 2 and three jets bins merged)
	tMod - during ichep only at some search regions otherwise MT2W, now with more stats, this variable can be used everywhere (tmod<0 compressed, tmod>10 large dm, in between bulk)
	Mlb >175(T2bW or W+jets like), Mlb<175 (T2tt) invariant mass of closest b jet to lepton and lepton
		fir regions with high mlb and tmod large contribution of W+jets -> modification of b-tagging - 1 and more tight b-jets
	MET
	-> 27 search regions (exclusive)

+ compressed mW<dm<mt
	same preselection plus
	ISR -> 5th jet, leading jet not b-tagged
	soft lepton - pT<150GeV
	boosted top quarks dphi(lep,MET)<2
	min dphi(jet,MET)>0.5
        -> 4 MET bins
in totality 31 signal regions

\subsection{Baseline selection}
	exactly one good lepton
	MT, MET
	MT2W (2015) below and above 200GeV -> in compressed spectra the MT2W is small (similar to background), in large mass spliting it can strongly suppress tt2l
	3 or 4 jets - boosted topologies (2015)
        compressed spectra (2015) tmod instead of MT2W + two jets - jets are soft, the only visible ones are from b-tag hadronization

\subsection{Signal regions}
	- first vertex in event pass the good quality criteria
	- pass one lepton selection
	- reject additional (veto) leptons
	- at least two jets
	-at leats one b-tag
	-MT>150
	-dphi>0.8
	+ compressed, boosted high delta M, low delta M, High delta M regions (in 2015)

-\subsection{Control regions}



-binning in
	MET
	MT2W (2015)
	

\subsection{Undefined}
-SF, rewighting

\subsection{compressed spectra}
	ISR jets
	recoil, enhance MET

\subsection{Background estimation}

-data driven + simulation
-general idea
-MET extrapolation?
-transfer factors
-scale factors - take into account differences in lepton, b-tagging etc efficiency
-composition of backgrouns - percentage

\subsubsection{Z to nunu}

-ttZ, WZ, ZZ
-2015 dataset taken from simulation -> small stats
-large in high MET and MT2W regions

-2017
-ZZ still taken from simulation
-3l scale factor -> normalization

\subsection{Single lepton}

-sensitive to MET and MT
-W+jets, tt1l
- tt1l -> constrained kinematics of W due to top, so MT tail is just due to MET resolution
-W+jets, no kinematics constraint, MT tail due to of shell W production (W width) 
-W+jets, zero b-tag control region
-tt1l -> negligible, estimation from simulation

-1 lep neutrino from W decay

ESTIMATIN:
-data driven, similar as for the lost lepton
-CR contamination N CR,data,Wjets,0btag = N CR,data,0btag - N CR,MC,ninWjets,0btag
-MET and NJET extrapolation (2015)
-W+b modelling systematics

\subsection{Lost lepton}

-dominant background after MET and MT selections
-tt2l wit both Ws decaying leptonically and one lepton lost
-largest contribution is the tt2l, then single top and then ttV and diboson processes.
-dilepton control region
-misreconstruction of lepton - 1 lepton and large MET
-two enutrinos lare MT and MT2W

ESTIMATON:
- N data,SR (ll) / N data,CR (ll) = N MC,SR (ll) / N MC,CR(ll) -> get N data,SR
-MET and Njet extrapolation (2015) -> additional transfer factor
-good modeling of njet and MET needed -> mismodelling in ssimu leads to mismodelling in TF -> special emu CR -> additional SF due to mismodelling
-gamma plus jets -> MET resolution > bin migrations - gamma pt spectrum reveighted to match neutrino pt spectrum, from reweighted events METmodified = reconstructed-MET + pTgamma


-systematics

\subsection{Results and interpretation}

\section{Differences in previous analyses}

\section{Selected topics}

\subsection{W-tagging}

\textbf{Motivation}

	high delta M regim - boost -> jets merge

\textbf{Techniques}
	larger radius jets
	tau ratios - N subjetiness
	different masses-> grooming techniques
	
\textbf{Results}
	end of 2015 - slide 8,9 - results on different ak8 W-tagging categories (18/11); lumi=2 1/fb
	mid of 2016 -update of previous study - lumi=2.26 fb; ak8+ak10 tagging; tables slide 7, 11 -> mo improvement with ak8, but slight improvement with ak10 


\textbf{Perspectives}
	-results from Sicheng with top tagging
	-resolved top tagger - 3ak4 jets
	-second presentation (31/01 same as 12 feb?), slide 6 - interesting plot, slide 10,11 significance table
        -merged top tagger - boosted objects
		slide 4, 5, 6

\subsection{Depndence of discriminating variables on pileup}
-not much to say
	

\chapternonum{Conclusion}

%----------------------------------------------------------------------------------------------------------------------------------------------------------------------------------
%tenses!!!!!!!!!!!
%-------------
%HIP


This thesis is divided into two parts, first part is dedicated to the Compact Muon Solenoid~(CMS) detector and especially the CMS silicon strip tracker. The second part presents a group of searches for supersymmetry~(SUSY) with the Run~2 CMS data. In the first part we study two different topics, which are related to highly ionizing particles (HIP) and simulation of the CMS tracker, respectively. 

Before the beginning of the Large Hadron Collider~(LHC) operation, the highly ionizing particles were identified to cause dead-time in cluster reconstruction leading to some hit inefficiency in the silicon strip tracker. In the beam tests of the tracker modules, both modules with iverter resistor value of 100~$\Omega$ and reduced value of 50~$\Omega$ were studied. It was shown that the dead-time decreases with resistance and therefore for modules with reduced inverter resistor value the mean dead-time is lower than for the nominal one. This observation lead to the decision to equip tracker modules with 50~$\Omega$ resistors instead of 100~$\Omega$. On the other hand, it was observed the decrease of the resistance enhanced the baseline distortions and therefore increased the number of reconstructed fake clusters. But this effect was found to cause only negligible increase in the strip ocupancy. 

When an increase of hit inefficiency in the strip tracker was observed in 2015-2016, a HIP was immediately identified as a possible explanation of it. Following this suggestion, I studied the HIP effect with collision events using a special data format in which no subtraction and suppression were applied at the back-end electronics level and therefore full information about all channels was available. In these data, I confirmed that the HIP induces a large charge on few channels and drives the remaining channels belonging to the same front-end electornics, the APV chip, towards lower charges up to the point when the signal is too small to be converted into light. Reaching this zero light level also induces a decreased spread of the charges on all channels within one APV when excluding the channels reading a large charge. With these data I was able to measure the rate of the HIP events, i.e. the number of events with a HIP interaction to all events, to be around $4 \times 10^{-3}$ for the first layer of Tracker Outer Barrel (TOB). I also evaluated the dead-time for this layer to be up to around 250~ns. In the studies at the PSI beam test, the mean and maximal dead-time for the TOB modules was evaluated to be 100~ns and 275~ns, respectively, therefore eround the vale estimated with the 2016 data.  

%TODO do I need the sentence starting "Reaching this" ?

However the HIP effect was not responsible for the observed inefficiencies and after the largest source of inefficiency was found and fixed, a new set of data provided an opportunity to study the HIP effect in a greater detail decomposed of the previous issue. Because of the data-taking conditions, these data permitted to perform a  measurement of the HIP probability in a cleaner environment. The HIP probability was computed per per pp interaction, i.e. rescaled by the fill peak pileup, in order to have a possibility to apply the probability on any LHC fill. This probability for each layer/wheel/ring of the silicon strip tracker was measured to be of the order of~$10^{-4}-10^{-3}$\% depending on the tracker layer/wheel/ring. The probabilty of the HIP events was already evaluated in the past, from simulations and beam test data, but these prebabilities were derived per particle and therefore they cannot easilly be compared to the current measurement. Moreover during the beam test, different particle composition and spectra were present compared to the CMS environment.

The measured HIP probability  $10^{-4}-10^{-3}$\% means that in case of the 250~ns dead-time and 25~ns bunch spacing, the pileup of the order of $10^4-10^5$ would cause that the APV chip is never fully efficient. The current pileup during the summer 2018 is of the order of 50 interactions per bunch crossing and therefore far from the fully inefficient chip scenario. Knowing the dead-time, the HIP probability and fixing the pilep to 31, which is the pileup of a representative run used for the 2018 hit efficiency measurement, it is possible to estimate the order of magnitude of the hit inefficiency resulting from HIP. The inefficiency was calculeted to be 0.1-1\% in agreement with the 2018 hit efficiency measurement.


Unfortunately, because of the data-taking conditions it was impossible to measure the dead-time. The first data provide better opportunity to measure dead-time, but in them it is diffcult to precisely fix the time of the HIP occurence. To measure dead-time, the best option would be to have data without any trigger rules, in order to have possibility to track given HIP from event to event. On top of it, it would be needed to reconsider the analysis strategy. For example as we would not be interested in the HIP probability anymore, we could spot the HIP event by presence of a large peak, which appears only for few ns and therefore should better fix the time of the HIP event. Moreover it would be needed to make sure, that the fake clusters from baseline distortions are not considered as real clusters and therefore it could seem that the chip is efficient. This should be partly solved by the tracking. 

These data also permitted to study the cluster charge, multiplicity and width  in presence of a HIP event or right after. I found out that the fake cluster multiplicity is increased after the HIP event compared to a standard event not influenced by a HIP. However the HIP probability is low and therefore in average the fake cluster multiplicity resulting from the HIP is of the same order of magnitude as the fake cluster multiplicity when no HIP had occurred. I observed that fake clusters induced by the HIP originate from the baseline distortions, i.e. the large spread in the channels charges. For the first layer of TOB, I also estimated a lower limit on the probability that a track is reconstructed with at least one fake cluster originating from a HIP event. This probability was computed to be 0.002\% and therefore is negligible.
 
%Both of the data-takings had its limitations. In the first one we had difficulties to measure the HIP probability, in the second one the deadtime. In case we 

%which kind of data would be perfect?
%	-theoretically the best option would be data without trigger rules. Then we measured probabilities but we miss deadtime. so we would need subdeetctors in to have tracking. And to link the track to the module with HIP. To measure dead-time we would need trains and to track a given module and asseswhn clusters start to be seen in that module with the help of tracking again. The analyiss would need to be more complex. (Think about that a bit more). Then there is still ambiguity in HIP selection - we would need to require HIP which is alreadys een in the first event in the train and then we would need to trigger all events in a time window (dead time estimated to 250ns, so at least this window)

The HIP probability cannot be decreased but we might investigate if the dead-time, and consequently the hit inefficiency, could be reduced. An option how to decrease a dead-time was identified and studied in the past~\cite{website:hitLoss}. This option is to maximize the pedestal subtracted data in order to decrease the chance that the channels charges are shifted beyond the measurable range. The pedestal is the mean strip activity when no particle is present and in the standard data-taking mode, not full pedestals are subtracted by the back-end electronics, but the pedestals minus offset, resulting in the positive mean channel charges after pedestal subtraction. This offset can be further maximized, but it was found out that this option only increases the fake cluster multiplicity without improving the hit efficiency. Moreover this option reduces the dynamic range, resulting in larger charges to be truncated faster. As this test was performed before the fix of the largest source of inefficiency, it could be an option to repeat it  now.

%dcommon powering scheme of inverters

%	-first studied the hi effect as a possible explanation of the inefficiencies
%	-first study with the CMS data, studied the evolution of baseline and raw digis standard deviation and with their help designed the selection of the HIP, it was found out that the HIP leads to the low baseline and low rms raw what was also observed in the 2nd study provided opportunity to study data not affecte dby the ineffeicincies
	%-the HIP effect affects the cluster charge and multiplicity and in the first study a decrease in the cluster multiplicity form average, i.e. dead time, is observed for aroun 250 ns.
	%-Due to the data-taking properties the deadtime cannot be studied in data after vfp fix, but with these data we have shown that 
	%there is an increased multiplicity of fake clusters after a HIP event. But in average this multiplicity is of the same order of magnitude as the cluster multiplicity of fake clusters when no HIP is present. We have also estimated a lower bound on the probability that a fake cluster from a HIP event is reconstructed into track to be 0.002\%. We observed that these fake clusters originate from the baseline distortiond induced by the HIP event. 
	%-We have computed that the hip probability per pileup for each layer/wheel/ring of the silicon strip tracker and it has been found to be of the order of~$10^{-4}-10^{-3}$\% depending on the tracker layer/wheel/ring. This means that in case of dead time for 250ns and bunch spacing of 25ns the PU of order of $10^4-10^5$  would cause that the APV chip is never fully efficient (ask prob of 10 events before have a HIP) and we are far from that now.

For the future upgrade of the silicon strip tracker for the High Luminosity LHC project (HL-LHC), we must pay attention to the HIP effect already at the design level. Ideally, it would be desired that the electronics is designed in a way that if a large charge is read by one channel, it does not affect the other channels belonging to the same chip as it is the case now. The HIP probability computed in this thesis is related to the module geometry and its distance from the interaction point and therefore cannot be easily extrapolated to the future detector. Assuming that the tracker would remain the same and for a pileup of around 200 interactions per bunch crossing, the HIP rate per chip can be expected to be of the order of $10^{-2}-10^{-1}$ \%. In the reallity, the detector will be upgraded and therefore the HIP probability and induced dead-time will change depending on the design of the tracker modules and the readout electronics.

   
%-HLLHC expected pu 140-200
 
%is there something to be improved
%	-different design of the APV chip. The common powering of inverters causes the drop o baseline but o the other hand it stabilizes the baseline. The resistance of the inverter resistor was changed in order to improve the sensitivity to the HIP event, but it enhances the instabilities of baseline, the basline distrotion, leading to a possible enhancement of the fake cluster reconstruction.

%does it need to be monitored
%	-the HIP probability is now low and tehrefore does not cause a large problems. But once the PU is increasing it can become dangerous, but as I mentioned the PU would have to be of the order of... which is far from teh expected PU at LHC.
 
%how to improve ?
%	-increase of lumi, increase of PU > for HL-LHC there is a development of new chips and therefore in their design we need to pay attention to the HIP effect (does somebody already thinking about that)

%could we prevent by changing baseline ref value ?
%	-the HIP was tried to be mitigated by maximizing CMN. But it was found out taht this maximization elads ony to the appearance of the so-called "anomalous cluster" (more details in HitEffLoss page). This check was though done before the APV change and should be redone. But the problem is that due to the HIP tehre are also fragments causing clusters plus fakes from distrotions which would not be normally reconstructed as they are below zero. So would it be the right thing to do?  %slide 27: https://indico.cern.ch/event/470862/contributions/1146979/attachments/1281240/1903454/common_mode_maximization.pdf

%will we suffer of this at HL-LHC ?
%	-we will ahve a HIP effect but it consequences, the induced dead time and probability will depend on the design of modules.

%something about the APV conditions?
%	-Preamp Feedback Voltage Bias (VFP) Change
%	-not much to say...


%the common biasing scheme of inverters -> stabilize baseline but xtalk effect -> could reduce the HIP problem - check 
%----------------------------------------------------------------------------------------------------------------------------------------------------------------------------------


%SIMU
The simulation of the CMS silicon strip tracker uses real tracker conditions in order to provide realistic results. These conditions change with the tracker operating conditions, e.g. the temperature, but also evolve as a result of tracker ageing, i.e. the radiation damage, for example the cross talk, i.e. the charge shared to the first and second neighboring strips. As the majority of these conditions was never updated since the beginning of the Run~1, we decided therefore to revise the simulation and study the impact of the outdated conditions on the simulation of the clusters in the tracker. I found out that the change of conditions (except noise and gains) has a negligible impact on the cluster properties in many cases, but the change of the cross talk parameters largely impacts the cluster width and seed charge.

After having identified that the cross talk parameters cause large discrepancies in distribution of the cluster shape between data and simulation, a cosmic data-taking period was arranged in absence of  magnetic field, dedicated to re-measure the cross talk parameters. As for the HIP studies, no subtraction and suppression of channels charges were done on the beck-end electronics level. In these data I spotted that the cross talk parameters evolve as a function of time when the particle arrives to a given module. Therefore the cross talk needs to be measured at the correct timing, i.e. as when a particle originating from the pp interaction would reach a given module. Due to the trigger conditions, there was only sufficient statistics to measure the cross talk parameters in the barrel. I measured that the cross talk decreased for all barrel geometries compared to the previous measurement. The sharing to the first neighboring strips in barrel decreased by around 18-27\%, the change being larger for geometries closer to the interaction point i.e. with larger fluence and consequently larger radiation damage. The sharing to the second neighboring strips diminished by around 24\%. To determine the cross talk parameters in disks and endcaps, I corrected the old cross talk parameters based on the change of the cross talk parameters in barrel and data to simualtion comparisons of the mean cluster seed charge in collision events. 


%But it is not know if previous measurement of cross talk paramters took into account the timing dependency and tehrefore it is not guaranteed that it was realistic.
	%- measurement done (differences between barrel and endcaps!) \& cross talk was found to decrease compared to the values originally int he MC
	%- example of TOB2 measurement before and after
	%- found evolution of xtalk as a function of time
	%-cross talk: Said somewhere that it was done last time in 2010 ?

The cross talk parameters influence profoundly the cluster shape, but not the total cluster charge. The change of cross talk might cause that the clusters which were slightly above the clustering threshold before do not have to pass it anymore now and therefore are not reconstructed anymore and vice versa. In addition, with the change of the cross talk parameters, the cluster position and its resolution change in simulation, resulting in small changes in tracking. Therefore the effect of a change of the cross talk may propagate up to the physics analysis, i.e. in searches for appearing/disappearing tracks. The object discriminators, mainly the b-tagging discriminators, which are strongly dependent on the tracking, are also influenced by it. But the impact is not large and for other physics objects and analyses, not largely dependent on tracking, it is expected to be negligible.

It was shown that the cross talk parameters evolve as a function of the fluence and therefore they should be remeasured and updated regularly. Also more frequent measurements of the cross talk parameters could help us to understand why the cross talk is decreasing. The literature is stating that with increasing fluence, the interstrip capacitance should increase and the interstrip resistance decrease. Both these changes should lead to the increase of cross talk, what was not observed. Note that the cross talk was measured only in the deconvolution mode, so it should be  updated for the peak mode as well. Nevertheless the peak mode is not used in standard data-taking, it might be important for some perticular studies, but mainly it could help us investigate the decrease of the cross talk, by disentangling the effects of the deconvolution from other effects.

The cosmic data recorded in absence of magnetic field, which were used for the cross talk measurement, posed several constraints and difficulties. First, due to the trigger conditions there is insufficient statistics in the disks and endcaps. Even if the trigger would be re-designed, due to the $\eta$ distribution of the cosmics, the data-taking would have to be long in order to collect sufficient statistics, which is difficult to arrange in the tight CMS schedule. The second large issue is related to the tracker timing when it samples the collected signal. The tracker has no special timing configuration for the cosmics and therefore the collision timing is always used.  As the cross talk depends on timing, we need to use for the cross talk measurement only cosmics which arrived to the given module at the same time as a particle produced in the pp collision. Another ambiguity comes from the computation of particle arrival time, which extrapolates all particles to the interaction point, even though they did not have to pass through there. Both the pseudorapidity coverage and the timing issues would be solved if it is possible to arrange a new data-taking of non suppressed collision data without magnetic field. But such data-taking is not compatible with the CMS program and priorities.
%The collision timing maximizes the cluster seed charge for particles created during the pp collision coming from the interaction point to the given module, and for cosmics no such maximization is done.


%does it need to be remeasured
%	- As shown the cross-talk evolves with fluence and tehrefore it must be remeasured from time to time.  Interesting feature is that the evolution is opposite than expected and therefore it would be interesting to monitor the cross talk in the vision of understanding the feature.

%do we need a different data
%	-the cruzet cosmics is good, but ideally we would need also some data from endcaps and this is difficult with cosmics as we would need quite a lot of time. The ideal case would be zero tesla collision data in the virgin raw, but this is scifi. These data would help to collect large stats everywhere, plus we would not have problems with timing.


%how can we improve the emasurement and the physics
%	-we could potentionally measure if there is left-right assymetry of cross talk (in past and this work observed average cluster (seed) charge dependency on the cluster position burt between 3); if the cross talk is dependent o the cluster position and track position within a sensor. But the description now is in general good and as mentioned there is only a small dependency of the tracking on the cross talk. In general only the physics analyses which are using tracking i.e. searches fro the appearing/disappearing tracks; or b-tagging discriminators are influenced.
%	-also we should update the cross talk in the peak data and check if the evolution is the same as in deco, the crosss talk decreases, this could give us a hint what is going on

After the update of the cross talk parameters and conditions such as gains and noise, which were updated by other members of the tracker local reconstruction group, the description by the simulation of the in-time clusters in data was largely improved. It was identified that there are still several parameters which are outdated and could be  updated in future, however these parameters do not largely change the cluster description. Because several parts in the simulation have only a negligible impact, i.e. the diffusion, it would be good to evaluate what parts of the simulation are really needed in order to reduce the time needed for the event simulation.

Several improvements could be still included in the simulation of the out-of-time clusters. Firstly, the pulse shape has changed due to the change of the APV parameter and therefore the pulse shape should be updated in order to reduce the simulated charge by a correct fraction. Secondly, it was shown that the cross talk depends on timing. The cross talk parameters were derived for in-time clusters and are not correct for out-of-time clusters. Ideally, the cross talk parameters should be parametrized as a function of the particle arrival time to the module. The out-of-time clusters need also to be well simulated as these clusters might be used by the tracking algorithm to reconstruct a track.

Few other interesting studies could be performed with the taken cosmic data and here I discuss few examples. In the past it was observed that there is a left-right asymmetry between the charge sharing. This effect could be as well studied with these data. Moreover, it is possible to study the cross talk and cluster properties as a function of the position where the track intercepts the strip plane, close and far from the strip. In the past it was also observed that there is an evolution of the cluster seed charge as a function of the strip number within one APV chip, this study could be repeated with these data. All these studies could help to better understand the formation of the signal, the features of the sensors such as non-uniformities in the electric field, and also the changes in the sensors properties reasulting from their irradiation. 


%Overall to prevent large discrepacnies between data and simulation, the paraaters and conditions in the simulation should be more frequently updated

%can we introduce some new things in simu
%	-for now we have achieved improvement and there are no things to be introduced into simu. After updating the conditions everything seems to be better described except the fraction of saturated clusters. This need to be resolved and changes in simulation might be done. On the other hand there are parts of simulation, where the effect is really small, e.g. diffucsion and if we would target the simplification of the code i order to reduce the simulation time, certainly few things could be simplified
 
%are there some problematic things in simualtion
% -in my opinion there is one large problem in simulation taht the parameters are not updated frequently. For some parameters it is not a problem, but for some it is crucial. For example the timing curve changed with the change of APV parameter and was not updated. The pulse shape largely influences teh clsuter properties and can pose really problems
%	- mainly the OOT will not be corectly described, firstly of incorrect pulse shape, but also because of the dependence of the cross talk on the particle arrival time. This could be considered for implementation in the smulation.
%	- then the puls shape is difefrent for the main strip and neighbor, what will cause also incorrect description of oot as we fixed it now for intime

%do we need to change something more frequently or completely because the sensor changes with radiation
%	- as shown except of cross talk also the noise and database conditions change fast with the aging of sensors.


%pulse shape problematics -> the OOT is not well described
%xtalk independent of time in simu
%	-already talked about


In the future tracker for the HL-LHC, the clustering will be largely different, only the binary information from the strip/macro-pixel will be sent, therefore the cluster seed charge and cluster charge will not be available and just the cluster width will be known. As the cluster width depends on cross talk, it will also be needed to mesure the cross talk in the new tracker. Moreover it will be important to determine and monitor the cross talk in order to design and maintain the threshold for strip/macro-pixel charge to be correctly set to one or zero.

%TODO: how is it with AC and DC coupling?

%measure it at HLLHC?
%	-for sure it needs to be determined what is teh cross talk and what it evolution will be (and if we are able to do so, measured from time to time) We need to make sure that the clusters do nto fail the binary criterium because they are too wide, because of too large cross talk. 
%	-soem sensors are going to be DC coupled, things will be different with xtalk
%	-ensure large interstrip resistance (think about xtalk during design) 

%what are the limitations ?
%why outdated ? talk about irradation
%	-fluence causes the aging of modules (defects, surface damage)

%Do we know why it was modified?
%	- not completely uderstood why xtalk is evolving in this way, but the evolution was expected due to surface damages on the sio-si interface

%How to understand the change of cross talk, which goes in opproite direction, can we disentangle it by more frequent measurements?
%	- it might be the cose but it does not have to

%Radiation Damage (https://indico.cern.ch/event/577879/contributions/2740332/attachments/1572803/2483960/CMS-OT_aldi.pdf) - slide 23 reference

%Bulk damage
%	Primary lattice defects (I and V) form higher order defects (V2, VO,...) or even defect clusters, with energy levels in the band gap of Si
%	Depending on energy level and cross section they contribute to leakage current, effective doping concentration, trapping

%Surface damage
%	-Ionizing radiation generates e/h pairs also in SiO2 
%	e much higher mobility than h -> positive charge up of oxide
%	Additional, interface traps with dynamic characteristics
%	Theses lead to increased surface currents, altered electric field in surface region, accumulation of electrons at surface


 %	- what remains unchanged - many parameters were found otdated but not updated week dependence -> should be some update later, but just for consistency

%----------------------------------------------------------------------------------------------------------------------------------------------------------------------------------


%SUSY
%motivation, conclusion
	%-the standard model of the particle physics was over years found to be excelently describing the nature, nut there are several shortcomings
	%-the most popular extension of sm du to its ability to addrss many shortcommings
%what was done and my contribution
	%-search for the supersymmetric partner of teh stop
	%-three posible decay channels, top neutralino, bottom chargino and mix of them
	%-search in one lepton final state + jets + met
	%- limits already at runI but runII due to cnage in energy and larger integrated luminosity enabled to probe stop masses far beyond the runI possibilities
	%-three analyses - my largest contribution to the full 2016
	%-the largest contribution to the full 2016 is estimation of the z to nunu background

%perspectives at HLLHC
%	-the relatively light stops already excluded
%	-the naturalness can be achieved with heavier ones therefore we must keep searching
%	-at the HL-LHC integrated lumi of 3000 fb-1 by CMS to be 2TeV
%	-further probing could be achieved by increase of the center of mass energy
%	-beyond this the more spohisticated analysis techniqus can eb designed to further probe the stop masses - like W or top tagging, new disriminating variables or machine learning teqniques

%perspectives:	What about the search at low mass ?
%perspectives:	How to improve ?
%erspectives:	How to interpret in a realistic model ? Do we have other strong constraints from other measurements/ serches ? ...

The second part of the thesis is discussing that the standard model (SM) of particle physics suffers from several shortcomings such asthe hierarchy problem or absence of the dark matter candidate, but in general over years that it proved to be capable to well describe the majority of observed physics phenomena. Due to these shortcomings theories beyond the SM were proposed and one of them, the supersymmetry, became the most promising one because of its ability to address large part of the SM issues. In this thesis I present searches for the supersymmteric partner of the top quark, the stop, with the CMS Run~2 data. These searches target the stop pair production, with three different possibilitis for the decays of the stops: the two stops decay to a top quark and a neutralino, or the two stops decay to a bottom quark and a chargino, or each stop decays differently as a mixture of the two previous cases. In all cases, the targeted signal final states contain one lepton, jets and missing transverse energy. 

I was involved in three publications of one search: one based on the data recorded in 2015 corresponding to an integrated luminosity $\int \mathcal{L}$ of 2.3~fb$^{-1}$, the second one based on the data recorded at the beginning of 2016 ( $\int \mathcal{L}=12.9$~fb$^{-1}$) and the last one corresponding to $\int \mathcal{L}= 35.9$~fb$^{-1}$ collected during the full 2016 pp data-taking period. I was mainly involved was in the last analysis, where I was responsible for the estimation of one of the SM backgrounds, in which a Z boson decays to two neutrinos. In this thesis I also exploited a technique for tagging of merged jets originating from a W boson and I showed that the gain of implementing such technique is growing with the integrated luminosity. 

The searches for the stop pair production in different final states were already performed with the Run~1 data. The analyses exclude stop masses in terms of simplified model spectra with stops decaying to  atop quark and a neutralino up to around 755 GeV for a neutralino mass below 200 GeV. With the increase of the center of mass energy and then also the integrated luminosity, it was soon possible to further probe the stop masses. No excess was found in the full 2016 data ($\int \mathcal{L}=35.9$~fb$^{-1}$) with respect to the background and therefore exclusion limits were derived. This analysis excluded the stop masses up to 1120 GeV for a massless neutralino in terms of simplified model spectra where both stops decay to a top quark and a neutralino. In the case where both stops decay to a bottom quark and a chargino or in the case of mixed decay, the stop masses were excluded up to 1000 GeV  and 980 GeV, respectively.

Despite the stop exclusion in the TeV range, there is still room for the natural supersymmetry and therefore the effort to search for the stops is not diminishing. According to the CMS predictions, with an integrated luminosity of 3000~fb$^{-1}$ envisioned to be collected at the HL-LHC, it will be possible to probe the stop masses up to 2~TeV. An upgrade of the LHC to higher center of mass energies would permit us to go even further in the stop masses. In addition to an increas of the center of mass energy and the integrated luminosity, we can further improve the discovery potential by optimizing the analyses, for example with the use of the discussed tagging of W and other jets, machine learning categorization of events, and new discriminating variables.

In the CMS a large variety of SUSY searches is performed. The stops are being sought in several differenet final states and the other predicted SUSY particles are targeted by the searches as well. Several searches also allow violation of the R-partity. In general the SUSY searches are interpreted in terms of simplified spectra. A lot of effort is being done to reinterpret these searches in more sophisticated models. The CMS, and the collider experiments in general, do not provide the sole opportunity to search for the supersymmetry. For example, the supersymmetry can also be studied via direct dark matter measurement experiments.

%----------------------------------------------------------------------------------------------------------------------------------------------------------------------------------

\chapternonum{Abstract}


%This thesis started in October 2015, the same year as the beginning of the Run~2 data-taking period (2015-2018) during which  the center-of-mass energy, the instantaneous luminosity and the bunch crossing frequency was increased compared to the Run~1 (2008-2013). Consequently, the possibilities for physics analyses were largely extended with a price of a larger fluence on the detector side, especially on the tracker which is the inner subdetector. With this increasing fluence, the detector suffers from a larger irradiation which could be a cause of performance issues. Moreover, the detector is also ageing with irradiation, leading to a change of some of its characteristics. Therefore it is very important to monitor closely the detector, to keep its stable performance and consequently the potential physics reach.

%Particle physics is described by the standard model whose  last piece, the Higgs boson, was discovered in 2012 by the CMS and ATLAS collaborations. Although the standard model is now complete and in general describes excellently the physics phenomena, it suffers from several shortcomings. This issue makes us believe that the standard model is an effective theory at low energy of a more fundamental theory which is to be determined. Over years, many theories  were proposed and one, referred to as supersymmetry, became of a special interest due to its capability to address many of the standard model shortcomings. 

%Supersymmetry introduces a new partner to each standard model particle and therefore extensive searches for these particles have been performed by the CMS collaboration as well as other collaborations. One of these particles is the supersymmetric partner of the top quark, the stop, which is expected to have a mass around 1~TeV in natural supersymmetry and therefore be accessible at the LHC energies. No evidence for the stops was  found in Run~1, but the increase in luminosity as well as the center-of-mass energy in Run~2 allows us to probe the stop masses beyond the Run~1 exclusion. 

Cette thèse a commencé en octobre 2015, la même année que le début de la période de collecte de données Run~2 (2015-2018) au cours de laquelle l’énergie dans le centre de masse, la luminosité instantanée et la fréquence des collisions a augmenté par rapport à Run~1 (2008-2013). Par conséquent, les possibilités d'analyses physiques ont été largement étendues avec le prix d'une plus grande fluence du côté du détecteur, en particulier sur le tracker qui est le détecteur interne. Avec cette fluence croissante, le détecteur souffre d’une irradiation plus importante qui pourrait être à l’origine de problèmes de performance. De plus, le détecteur vieillit également avec l'irradiation, entraînant une modification de certaines de ses caractéristiques. Par conséquent, il est très important de surveiller de près le détecteur, afin de conserver ses performances stables et pas d'impact sur les analyses de physique.

La physique des particules est décrite par le modèle standard dont la dernière pièce, le boson de Higgs, a été découverte en 2012 par les collaborations CMS et ATLAS. Bien que le modèle standard soit maintenant complet et décrive en général très bien les phénomènes physiques, il souffre de plusieurs défauts. Ce problème nous fait croire que le modèle standard est une théorie efficace à basse énergie d’une théorie plus fondamentale à déterminer. Au fil des ans, de nombreuses théories ont été proposées et l'une d'entre elles, appelée supersymétrie, a suscité un intérêt particulier en raison de sa capacité à répondre à de nombreuses lacunes du modèle standard.

Supersymétrie introduit un nouveau partenaire pour chaque particule de modèle standard et, par conséquent, des recherches approfondies sur ces particules ont été effectuées par la collaboration CMS, ainsi que par d’autres collaborations. L'une de ces particules est le partenaire supersymétrique du quark top, le stop, qui devrait avoir une masse d'environ 1~TeV en supersymétrie naturelle et donc être accessible aux énergies du LHC. Aucune preuve des stops n'a été trouvée dans Run~1, mais l'augmentation de la luminosité ainsi que l'énergie dans le centre de masse dans Run~2 nous permettent de sonder les masses de stop au-delà de l'exclusion de Run~1.

\vspace*{1cm}

%The description of the Compact Muon Solenoid~(CMS) detector is given in the first chapter, Chapter~\ref{sec:detch}, together with a brief introduction of the Large Hadron Collider. This chapter focuses on the silicon strip tracker, whose deeper understanding is required for the following chapters. This chapter also presents the reconstruction of the physics objects corresponding to particles passing through the detector. In following Chapter~\ref{sec:HIPch}, I was interested in the highly ionizing particles in the silicon strip tracker. Already before the beginning of the Large Hadron Collider~(LHC) operation, the highly ionizing particles were studied and identified to cause dead-time in the front-end electronics leading to a hit inefficiency in the silicon strip tracker. In the beam tests of the tracker modules, it was found that by decreasing the inverter resistor value of the front-end electronics,  the dead-time can be shortened as shown in Table~\ref{tab:tableDeadtimes2} and therefore the design of the electronics was adjusted to mitigate this issue. Later no analysis with LHC collisions had been performed yet. So when an increase of the hit inefficiency in the strip tracker was observed in 2015-2016, a HIP was immediately identified as a possible explanation of it. I studied the HIP effect with collision events recorded in 2016 using a special data format in which no subtraction and suppression were applied at the back-end electronics level and therefore full information about all channels was available. They allowed me to  confirm that the HIP induces a large charge on few channels and drives the remaining channels belonging to the same front-end electronics, the APV chip, towards lower charges up to the point when the signal is too small to be converted into light what can be observed in Fig.~\ref{fig:figures/peakinmodule}. Reaching this zero light level also induces a decreased spread of the charges on all channels within one APV, when excluding the channels reading a large charge. With these data I was able to measure the rate of the HIP events, i.e. the number of events with a HIP interaction to all events, to be around $4 \times 10^{-3}$ for the first layer of Tracker Outer Barrel~(TOB). I also evaluated the dead-time for this layer to be up to around 250~ns. It is of the same order as the dead-times evaluated in  previous studies performed at the PSI beam test, namely a  mean and maximal dead-times for the TOB modules of 100~ns and 275~ns, respectively.  

La description du Compact Muon Solenoid~(CMS)  détecteur est donnée dans le premier chapitre, Chapitre~\ref{sec:detch}, avec une brève introduction du Large Hadron Collider~(LHC). Ce chapitre se concentre sur le tracker à bande de silicium, dont la compréhension approfondie est requise pour les chapitres suivants. Ce chapitre présente également la reconstruction des objets physiques correspondant aux particules traversant le détecteur. En suivant le Chapitre~\ref{sec:HIPch}, je me suis intéressée aux particules hautement ionisantes~(HIP) dans le tracker à bande de silicium. Déjà avant le début de l'opération Large Hadron Collider, les particules hautement ionisantes ont été étudiées et identifiées pour causer un temps mort dans l'électronique frontale, entraînant une inefficacité dans le tracker à bande de silicium. Dans les tests de faisceau des modules tracker, il a été constaté que, en diminuant la valeur de la résistance de résisteur de inverter de l’électronique frontale, le temps mort peut être diminué comme indiqué dans le Tableau~\ref{tab:tableDeadtimes2} et donc le désign de l'électronique a été ajustée pour atténuer ce problème. Plus tard, aucune analyse avec des collisions du LHC n'a encore été réalisée. Ainsi, lorsqu'une augmentation de l'inefficacité du hit dans le tracker a été observée en 2015-2016, un HIP a immédiatement été identifié comme une explication possible. J'ai étudiée l'effet HIP avec des événements de collision enregistrés en 2016 en utilisant un format de données spécial dans lequel aucune soustraction et suppression n'a été appliquée au niveau de l'électronique arrière-plan et donc des informations complètes sur tous les canaux étaient disponibles. Ils m'ont permis de confirmer que le HIP induit une charge importante sur quelques canaux et pilote les canaux restants appartenant à la même électronique frontale, la puce APV, vers des charges plus faibles jusqu'au moment où le signal est trop petit pour être converti en lumière ce qui peut être observé sur la Fig.~\ref{fig:figures/peakinmodule}. Atteindre ce niveau de lumière zéro induit également une réduction des différences entre charges des tous les canaux dans un APV, en excluant les canaux lisant une charge de HIP. Grâce à ces données, j'ai pu mesurer le taux d'événements HIP, c'est-à-dire le nombre d'événements ayant une interaction HIP  à  tous les événements, autour de 4 $\times 10^{-3} $ pour la première couche de Tracker Outer Barrel~(TOB). J'ai également évalué le temps mort pour cette couche à environ 250 ~ ns. Il est du même ordre que les temps morts évalués dans des études antérieures effectuées au test du faisceau PSI, à savoir un temps mort moyen et maximal pour les modules TOB de 100~ns et 275~ns, respectivement.

\begin{table}[h]
\begin{center}
\begin{tabular}{|l|l|l|}
\hline
Sensor type and $R_{inv}$~[$\Omega$] & $\Gamma_{mean}$~[ns]  & $\Gamma_{max}$~[ns] \\
\hline
\hline
TIB 100 $\Omega$ & 99.5 $\pm$ 12.0 & 200 $\pm$ 25 \\
TIB 50  $\Omega$ & 69.6 $\pm$ 9.4 & 250 $\pm$ 25 \\
TOB 100  $\Omega$ & 122.5 $\pm$ 12.6 & 275 $\pm$ 25 \\
TOB 50 $\Omega$  & 100.5 $\pm$ 3.6 & 275 $\pm$ 25 \\
\hline
TIB $\Gamma_{mean}(50~\Omega )/\Gamma_{mean}(100~\Omega)$ &  0.70 $\pm$ 0.13  & \\
TOB $\Gamma_{mean}(50~\Omega )/\Gamma_{mean}(100~\Omega)$ &  0.82 $\pm$ 0.09 & \\
\hline
\end{tabular}
%\caption[Table caption text]{The mean~($\Gamma_{mean}$) and maximum~($\Gamma_{max}$) dead-time of the APV chip induced by the HIP events for fully suppressed~($\sigma_{raw}<1$~ADC) baseline events. The dead-times were evaluated for two different module geometries~(TIB or TOB) as well as for two inverter resistor values~(100 or 50~$\Omega$). These results were obtained with PSI beam test data~\cite{Bainbridge:2004jc}. }
\caption[Table caption text]{La moyenne~ ($\Gamma_{mean} $) et le maximal temps mort~($\Gamma_{max}$) de la puce APV induit par les événements HIP pour les baselines qui sont entièrement supprimé~($\sigma_{raw} <1 $~ADC). Les temps morts ont été évalués pour deux géométries de module différentes~(TIB ou TOB) ainsi que pour deux valeurs de résistance d'inverter~(100 ou 50 ~ $\Omega$). Ces résultats ont été obtenus avec des données de test de faisceau PSI~\cite{Bainbridge: 2004jc}.}
\label{tab:tableDeadtimes2}
\end{center}
\end{table}


    \insertFigure{figures/peakinmodule}
                 {0.47}
                 {Exemple de distribution de raw digis~(rose), piédestal soustrait digis~(bleu), baseline soustrait digis~(rouge) et grappes~(vert) en fonction du numéro de bande dans un module. Le troisième APV dans le module montre un comportement induit par un événement HIP: une faible variation de charge pour les raw digis et un grand signal observé pour quelques canaux.}       % Width, in fraction of the whole page width
                 %{Example of distribution of raw digis~(pink), pedestal subtracted digis~(blue), baselines~(red) and clusters~(green) as a function of the strip number in one module. The third APV in the module shows a behavior induced by a HIP event: a low charge variation for the suppressed raw digis and a large observed signal for few channels.} 

%However the HIP effect was not responsible for the observed inefficiencies and after the largest source of inefficiency was found and fixed, a new set of data provided an opportunity to study the HIP effect in a greater detail disentangled of the previous issue. Because of the data-taking conditions, these new data permitted to perform a  measurement of the HIP probability in a cleaner environment. The HIP probability was computed per pp interaction, i.e. rescaled by the peak pileup of the fill, in order to have the possibility to apply this probability to any LHC fill. The HIP probability in the silicon strip tracker, shown in Fig.~\ref{fig:figures/probPerPU2}, was measured to be of the order of~$10^{-4}-10^{-3}$\% depending on the tracker layer/wheel/ring. The HIP probabilities, already evaluated in the past from simulations and using beam test data, were derived per particle and therefore they cannot easily be compared with the current measurement. Moreover during the beam test, a different particle composition and energy spectra were present compared to the CMS environment.

Cependant, l'effet HIP n'était pas responsable des inefficacités observées et, une fois que la plus grande source d'inefficacité a été trouvée et corrigée, un nouvel ensemble de données a permis d'étudier l'effet HIP de manière plus détaillée. En raison des conditions de prise de données, ces nouvelles données ont permis de mesurer la probabilité HIP dans un environnement plus propre. La probabilité HIP a été calculée par interaction pp, c'est-à-dire redimensionnée par le pileup de pointe, afin de pouvoir appliquer cette probabilité à tout remplissage LHC. La probabilité de HIP dans le tracker de bande de silicium, représentée sur la Fig.~\ref{fig:figures/probPerPU2}, a été mesurée de l'ordre de ~10$^{-4} -- 10^{-3}$\% selon sur la couche/roue/anneau. Les probabilités HIP, déjà évaluées par le passé à partir de simulations et utilisant des données de test de faisceau, ont été dérivées par particule et ne peuvent donc pas être facilement comparées à la mesure actuelle. De plus, lors du test de faisceau, une composition de particules et des spectres d'énergie différents étaient présents par rapport à l'environnement CMS.

    \insertTwoFigures{figures/probPerPU2}
                 {figures/probFinalLayerPU} % Filename = label
                 {figures/probFinalPURings} % Filename = label
                 {0.47}       % Width, in fraction of the whole page width
                 { Probabilité moyenne d'un événement HIP par pileup~(PU) pour les couches (gauche, droite) de TIB, TOB et roues (gauche) ou anneaux (droite) des partitions TID et TEC du tracker de silicium, calculées à partir du cycle de données 281604.}
                 %{The average probability of a HIP event per PU for layers (left, right) of TIB, TOB, and wheels (left) or rings (right) of the TID and TEC partitions of the silicon strip tracker, computed from the data run 281604.}

%The HIP probability itself cannot be decreased but we might investigate if the dead-time, and consequently the hit inefficiency, could be reduced. An option how to decrease the dead-time was identified and studied in the past~\cite{website:hitLoss}. This option is to maximize the pedestal subtracted data in order to decrease the chance that the channel charges are shifted beyond the measurable range. The pedestal is the mean strip activity when no particle is present and in the standard data-taking mode, not pedestals themselves are subtracted by the back-end electronics, but the pedestals minus a fixed offset, resulting in the positive mean channel charges after this subtraction. This offset can be further maximized, but it was found out that this option only increases the fake cluster multiplicity without improving the hit efficiency. Moreover this option reduces the dynamic range, resulting in larger charges to be truncated faster. However this test was performed before the fix of the largest source of inefficiency and it could be worth  to repeat it  now.

La probabilité HIP elle-même ne peut pas être réduite, mais nous pourrions rechercher si le temps mort et, par conséquent, l’inefficacité de hit, pourraient être réduits. Une option permettant de réduire le temps mort a été identifiée et étudiée dans le passé ~\cite{website: hitLoss}. Cette option permet de maximiser les données piédestal soustraites afin de réduire le risque que les charges du canal soient décalées au-delà de la plage mesurable. Le piédestal est l’activité de bande moyenne quand aucune particule n’est présente et dans le mode de prise de données standard, mais les piédestal eux-mêmes ne sont pas soustraits par l’électronique back-end, mais les piédestaux moins un décalage fixe. Ce décalage peut encore être optimisé, mais il a été constaté que cette option ne fait qu’accroître la multiplicité des clusters fausses  sans améliorer l’efficacité des hit. De plus, cette option réduit la plage dynamique, ce qui se traduit par des charges plus importantes à tronquer plus rapidement. Cependant, ce test a été effectué avant la résolution de la plus grande source d’inefficacité et il pourrait être utile de le répéter maintenant.

%A measured HIP probability of $10^{-4}-10^{-3}$\% means that in case of a dead-time of 250~ns obtained from the first data and bunch spacing of 25~ns, an APV chip would never be fully efficient if the pileup were of the order of $10^4-10^5$. The current pileup during the summer 2018 is of the order of 40 interactions per bunch crossing and therefore far from the fully inefficient chip scenario. The inefficiency resulting from the HIP was recently calculated, using the dead-time of 250~ns and the measured HIP probability, to be of the order of 0.1-1\%, for a pileup of 30 as in the representative run used for the 2018 hit efficiency measurement. This is of similar order as the measured 2018 hit inefficiency. Therefore the main source of the measured inefficiency now seems to come from the HIP effect. This inefficiency is not considered during the event simulation. 


Une probabilité HIP mesurée de $10^{-4}-10^{-3}$ signifie que dans le cas d'un temps mort de 250~ns obtenu à partir des premières données et de l'espacement des paquets des particules de 25~ns, une puce APV ne serait jamais pleinement efficace si le pileup était de l'ordre de $10^4 - 10^5$. Le pileup actuel au cours de l'été 2018 est de l'ordre de 40 interactions par événement et donc loin du scénario de puce totalement inefficace. L'inefficacité résultant du HIP a récemment été calculée, en utilisant le temps mort de 250~ns et la probabilité HIP mesurée, de l'ordre de 0.1-1\%, pour un pileup de 30 comme dans le run représentative utilisé pour le 2018 mesure de l'efficacité de hit. Cet ordre est similaire à l'inefficacité de 2018 mesurée. Par conséquent, la principale source de l’inefficacité mesurée semble désormais provenir de l’effet HIP. Cette inefficacité n'est pas prise en compte lors de la simulation d'événement.

%The second set of data also permitted to study the cluster charge and multiplicity, presented in Fig.~\ref{fig:figures/avClusterMultiplicitySecondT2}, in presence of a HIP event or right after. I found out that the fake cluster multiplicity is increased after the HIP event compared to a standard event not influenced by a HIP. However the HIP probability is low and therefore in average over the runs the fake cluster multiplicity resulting from the HIP is of the same order of magnitude as the fake cluster multiplicity when no HIP had occurred. I observed that fake clusters induced by the HIP originate from the baseline distortions. For the first layer of TOB, I also estimated a lower limit on the probability that a track is reconstructed with at least one fake cluster originating from a HIP event. This probability was computed to be 0.002\% and therefore is negligible. These fake clusters are not included in the simulation either, but as discussed, their impact is negligible and therefore it is reasonable to omit them from simulation.

Le deuxième ensemble de données a également permis d'étudier la charge et la multiplicité des grappes, présentées dans la Fig.~\ref{fig:figures/avClusterMultiplicitySecondT2}, en présence d'un événement HIP ou juste après. J'ai découverte que la multiplicité des clusters fausses est augmentée après l'événement HIP par rapport à un événement standard non influencé par un HIP. Cependant, la probabilité HIP est faible et, par conséquent, en moyenne pendant le run, la multiplicité de grappes fausses résultant du HIP est du même ordre de grandeur que la multiplicité de grappe fausses en l'absence de HIP. J'ai observé que les fausses grappes induites par le HIP proviennent des distorsions de baseline. Pour la première couche de TOB, j'ai également estimé une limite inférieure sur la probabilité qu'une piste soit reconstruite avec au moins un faux cluster provenant d'un événement HIP. Cette probabilité a été calculée pour être 0.002 \% et est donc négligeable. Ces fausses grappes ne sont pas non plus incluses dans la simulation, mais comme on l’a vu, leur impact est négligeable et il est donc raisonnable de les omettre de la simulation.


    \insertTwoFigures{figures/avClusterMultiplicitySecondT2}
                 {figures/avClusterMultiplicitySecond} % Filename = label
                 {figures/avClusterChargeSecond} % Filename = label
                 {0.45}       % Width, in fraction of the whole page width
                 {La multiplicité de grappe moyenne (gauche) et la charge de grappe moyenne (droite) en fonction du nombre de paquet croisement pour le run 281604. En triangles roses, la multiplicité de grappes moyenne de l’événement HIP et les événements plus tard tandis que les carrés bleus sont pour les événements non-HIP. }
                 %{ The average cluster multiplicity (left) and average cluster charge (right) as a function of the bunch crossing number for the run 281604. In pink triangles is the average cluster multiplicity of the HIP event and events later while the blue squares are for the non-HIP events.  } % Caption



%Unfortunately, because of the data-taking conditions of the second data-taking period, it was impossible to measure the dead-time in them. The first data provided a better opportunity to measure the dead-time, but with the  difficulty to precisely fix the time of the HIP occurrence. There are not any simple options of data-taking conditions suitable to  measure the dead-time in future, in this paragraph I propose two more complex ideas of  data-takings which could be considered. The first option would be to arrange data-taking in which events would be recorded every 25~ns during at least 15 bunch crossings not respecting any trigger rule, in order to have the possibility to track a given HIP in time. On top of it, it would be needed to reconsider the analysis strategy. For example, as we would not be interested in the HIP probability anymore, we could spot the HIP event by presence of a large peak, which appears only for few ns and therefore should better fix the time of the HIP event. But this approach could overload the system and therefore it would be needed to evaluate if such data-taking would be reasonable and how we could possibly avoid the overload. The second option would be to have a special fill and partially violate the trigger rules as well. In such fill we would need trains of two bunches and in each train we would need different time gap between the two bunches: in the first one 25~ns, the second one 50~ns, the third one 75~ns, etc. In these data we would have the time of HIP event fixed by the first bunch crossing in train and in the second one we could measure the hit efficiency. The dead-time would be the time gap between two bunch crossings in the same train in which the same hit efficiency is measured. In both these proposed data-takings, it would be needed to make sure, that the fake clusters from baseline distortions are not considered as real clusters by tracking and therefore that the fake clusters do not artificially increase the hit efficiency. 

Malheureusement, en raison des conditions de prise de données de la deuxième période de collecte de données, il était impossible de mesurer le temps mort. Les premières données ont fourni une meilleure occasion de mesurer le temps mort, mais avec la difficulté de déterminer précisément le temps de l'occurrence du HIP. Il n’existe pas d’options simples de conditions de prise de données permettant de mesurer le temps mort à l’avenir. Dans ce paragraphe, je propose deux idées plus complexes de prises de données qui pourraient être envisagées. La première option consisterait à organiser la prise de données dans laquelle les événements seraient enregistrés tous les 25~ns pendant au moins 15 croisements de paquet, sans respecter aucune règle de déclenchement, afin de pouvoir suivre un HIP donné dans le temps. De plus, il serait nécessaire de reconsidérer la stratégie d’analyse. Par exemple, comme nous ne serions plus intéressés par la probabilité HIP, nous pourrions repérer l’événement HIP en présence d’un grand pic, qui n’apparaît que pour quelques instants et devrait donc mieux déterminer le temps de l’événement HIP. Mais cette approche pourrait surcharger le système et il faudrait donc évaluer si une telle prise de données serait raisonnable et comment nous pourrions éventuellement éviter la surcharge. La deuxième option serait d'avoir un remplissage spécial et de violer partiellement les règles de déclenchement également. Dans un tel remplissage, nous aurions besoin de trains de deux paquets et dans chaque train, nous aurions besoin d'un intervalle de temps différent entre les deux paquets: dans le premier 25~ns, le second 50~ns, le troisième 75~ns, etc. Dans ces données, nous aurions le temps de l'événement HIP fixé par la première croisement en train et dans la seconde, nous pourrions mesurer l'efficacité de hit. Le temps mort serait l'intervalle de temps entre deux croisements de paquets dans le même train dans lequel la même efficacité de hit est mesurée. Dans ces deux recueils de données proposés, il serait nécessaire de s'assurer que les fausses grappes issues des distorsions de baseline ne sont pas considérées comme de bonnes grappes par traçage et que, par conséquent, les fausses grappes n'augmentent pas artificiellement l'efficacité.

%In the future upgrade of the silicon strip tracker for the High Luminosity LHC project (HL-LHC), we must pay attention to the HIP effect already at the design level. Ideally, it would be desired that the electronics is designed in a way that if a large charge is read by one channel, it does not affect the other channels belonging to the same chip as it is the case now. The HIP probability computed in this thesis is related to the module geometry and its distance from the interaction point and therefore cannot be easily extrapolated to the future detector. Assuming that the tracker would remain the same and for a pileup of around 200 interactions per bunch crossing, the HIP rate per chip could be expected to be of the order of $10^{-2}-10^{-1}$ \%. In the reality, the detector will be upgraded and therefore the HIP probability and induced dead-time will change depending on the design of the tracker modules and the readout electronics.

Dans la future mise à niveau du tracker à bande de silicium pour le projet High Luminosity LHC (HL-LHC), nous devons prêter attention à l’effet HIP dès la conception. Idéalement, il serait souhaitable que l’électronique soit conçue de telle manière que si une charge importante est lue par un canal, cela n’affecte pas les autres canaux appartenant à la même puce que c’est le cas actuellement. La probabilité HIP calculée dans cette thèse est liée à la géométrie du module et à sa distance par rapport au point d'interaction et ne peut donc pas être facilement extrapolée au futur détecteur. En supposant que le tracker resterait le même et pour un pileup d'environ 200 interactions par croisement de paquets, le taux de HIP par puce pourrait être de l'ordre de 10$^{-2}-10 ^{-1}$\% . En réalité, le détecteur sera mis à niveau et, par conséquent, la probabilité HIP et le temps mort induit changeront en fonction de la conception des modules de trajectographe et de l’électronique de lecture.

\vspace*{1cm}

%SIMU


The third chapter, Chapter~\ref{ch:simu} focuses on the  simulation of the CMS silicon strip tracker and the tracker conditions used in simulation in order to provide realistic results. These conditions change with the tracker operating conditions, e.g. the temperature, but also evolve with tracker ageing resulting from the radiation damage, for example the cross talk, i.e. the charge shared with the neighboring strips. As the majority of these conditions was never updated since the beginning of the Run~1, we decided therefore to revise the simulation and study the impact of the outdated conditions on the simulation of the clusters in the tracker. I found out that the change of conditions (except noise and gains) has a negligible impact on the cluster properties in many cases, but the change of the cross talk parameters largely impacts the cluster width and seed charge.

After having identified that the cross talk parameters cause large discrepancies in cluster shape distributions between data and simulation as shown in Fig.~\ref{fig:figures/seedwidthTOBXT2}, a cosmic data-taking period was arranged in absence of magnetic field, dedicated to re-measure the cross talk parameters. As for the HIP studies, no subtraction and suppression of channel charges were applied on the back-end electronics level. In these data I spotted that the cross talk parameters evolve as a function of the time when the particle arrives to a given module. Therefore the cross talk needs to be measured at the correct timing, i.e. as when a particle originating from the pp interaction would reach a given module. Due to the trigger conditions, there was only sufficient statistics to measure the cross talk parameters in the barrel. The newly measured parameters together with the current parameters used in simualtion are shown in Table~\ref{tab:measuredXtalk2}. I measured that the cross talk decreased for all barrel module geometries compared to the previous measurement. The sharing to the first neighboring strips in barrel decreased by around 18-27\%, the change being larger for geometries closer to the interaction point i.e. with larger fluence and consequently larger radiation damage. The sharing to the second neighboring strips diminished by around 24\%. To determine the cross talk parameters in disks and endcaps, I corrected the old cross talk parameters based on the change of the cross talk parameters in barrel and data to simulation comparisons of the mean cluster seed charge in collision events. The results of cross talk measurement in disks and endcaps are summarized in Table~\ref{tab:measuredXtalkTODTEC2}. Updating the cross talk parameters in simulation by newly measured values, we found out that the newly measured parameters improved largely description of cluster shape in data by simulation as shown on example in Fig.~\ref{fig:figures/widthTOBXTm2}.

    \insertTwoFigures{figures/seedwidthTOBXT2} % Filename = label %TDO continue here
                 {figures/clusterwidthTOBl1to4XT}
                 {figures/clusterseedchargeRescaledTOBl1to4XT} % Filename = label
                 {0.45}       % Width, in fraction of the whole page width
                 { Distribution of the on-track cluster width (left) and cluster seed charge (right) in data and simulation for the OB2 geometry and for different values of the cross talk $XT$. The first of the three XT numbers corresponds to the fraction of charge induced on the seed strip, the second and third number represent the fractions of charge induced on each first and second neighboring strips, respectively. The simulated distributions are rescaled to the number of clusters in data. The bottom plot represents the data to simulation ratios. }

\begin{table}[h]
\begin{center}
\begin{tabular}{|l|l|l|l|l|}
\hline
Geometry & Type & $x_{0}$ & $x_{1}$ & $x_{2}$ \\
\hline
\hline
IB1 & current measurement & $ 0.836 \pm 0.009 $ & $0.070 \pm 0.004 $ & $0.012 \pm 0.002 $ \\
IB1 & currently in simulation & $ 0.775 $ & $ 0.096 $ & $0.017 $  \\
\hline
IB2 &  current measurement & $0.862 \pm 0.008 $ & $0.059 \pm 0.003 $ & $0.010 \pm  0.002 $  \\
IB2 & currently in simulation &  $0.830 $ & $0.076 $ & $ 0.009$   \\
\hline
OB2 &  current measurement & $0.792 \pm 0.009 $ & $0.083 \pm 0.003 $ & $0.020 \pm 0.002$  \\
OB2 & currently in simulation &   $0.725 $ & $0.110 $ & $ 0.027 $  \\
\hline
OB1 &  current measurement &  $0.746 \pm 0.009 $ & $0.100 \pm 0.003 $ & $0.027 \pm 0.002 $  \\
OB1 & currently in simulation &  $0.687 $ & $0.122 $ & $ 0.034 $ \\
\hline
\end{tabular}
\caption[Table caption text]{The cross talk measured in 2018 CRUZET VR data and the cross talk values used currently in simulation for barrel geometries. }
\label{tab:measuredXtalk2}
\end{center}
\end{table}

\begin{table}[h]
\begin{center}
\begin{tabular}{|l|l|l|l|l|}
\hline
Geometry & Type & $x_{0}$ & $x_{1}$ & $x_{2}$ \\
\hline
\hline
W1a &  new values & 0.8571 & 0.0608 & 0.0106 \\
W1a &  currently in simulation & 0.786 & 0.093 & 0.014 \\
\hline
W2a &  new values & 0.8861 & 0.049 & 0.008 \\
W2a &  currently in simulation & 0.7964 & 0.0914 & 0.0104 \\
\hline
W3a &  new values & 0.8984 & 0.0494 & 0.0014 \\
W3a &  currently in simulation & 0.8164 & 0.09 & 0.0018 \\
\hline
W1b &  new values & 0.8827 & 0.0518 & 0.0068 \\
W1b &  currently in simulation & 0.822 & 0.08 & 0.009 \\
\hline
W2b &  new values & 0.8943 & 0.0483 & 0.0046 \\
W2b &  currently in simulation & 0.888 & 0.05 & 0.006 \\
\hline
W3b &  new values & 0.8611 & 0.0573 & 0.0121 \\
W3b &  currently in simulation & 0.848 & 0.06 & 0.016 \\
\hline
W4 &  new values & 0.8881 & 0.0544 & 0.0015 \\
W4 &  currently in simulation & 0.876 & 0.06 & 0.002 \\
\hline
W5 &  new values & 0.7997 & 0.077 & 0.0231 \\
W5 &  currently in simulation & 0.7566 & 0.0913 & 0.0304 \\
\hline
W6 &  new values & 0.8067 & 0.0769 & 0.0198 \\
W6 &  currently in simulation & 0.762 & 0.093 & 0.026 \\
\hline
W7 &  new values & 0.7883 & 0.0888 & 0.0171 \\
W7 &  currently in simulation & 0.7828 & 0.0862 & 0.0224 \\
\hline
\end{tabular}
\caption[Table caption text]{The updated cross talk for rings of TID and TEC. }
\label{tab:measuredXtalkTODTEC2}
\end{center}
\end{table}


    \insertTwoFigures{figures/widthTOBXTm2} % Filename = label %TDO continue here
                 {figures/clusterwidthTOBl1to4XTm}
                 {figures/clusterseedchargeRescaledTOBl1to4XTm} % Filename = label
                 {0.45}       % Width, in fraction of the whole page width
                 { Distribution of the on-track cluster width (left) and seed charge (right)  in data and simulation for the OB2 geometry, for the current (default) and newly measured (updated) cross talk parameters~($XT$).  The simulated distributions are rescaled to the number of clusters in data.  The bottom plots represent the data to simulation ratios. }


The cross talk parameters influence profoundly the cluster shape, but not the total cluster charge. The change of the cross talk might cause that the clusters which were slightly above the clustering threshold before do not have to pass it now and therefore are not reconstructed anymore and vice versa. In addition, with the change of the cross talk parameters, the cluster position and its resolution change in simulation, resulting in small changes in tracking. Therefore the effect of a change of the cross talk may propagate up to some of the physics analyses, e.g. in searches for appearing/disappearing tracks. The object discriminators, mainly the b-tagging discriminators, which are strongly dependent on the tracking, are also influenced by it. The impact on the other physics objects and analyses, not largely dependent on tracking, is expected to be negligible.

It was shown that the cross talk parameters evolve as a function of the fluence and therefore they should be remeasured and updated regularly. Also more frequent measurements of the cross talk parameters could help us to understand why the cross talk is decreasing with respect to the previous measurement. The literature is stating that with an increasing fluence, the inter-strip capacitance should increase and the inter-strip resistance decrease. Both these changes should lead to the increase of cross talk, what was not observed. However the inter-strip resistance and capacitance are not the only factors influencing the signal formation, but there is more complex network of capacitances and resistances which has its impact on the cluster shape.  Note that the cross talk was measured only in the deconvolution mode, so it should be updated for the peak mode as well. Nevertheless, the peak mode is not used in standard data-taking, but it could help us to investigate the decrease of the cross talk, by disentangling the effects of the deconvolution from other effects.

The cosmic data recorded in absence of magnetic field, which were used for the cross talk measurement, posed several constraints and difficulties. First, due to the trigger conditions there is insufficient statistics in the disks and endcaps. Even if the trigger would be re-designed, due to the $\eta$ distribution of the cosmics, the data-taking would have to be long in order to collect sufficient statistics, which is difficult to arrange in the tight CMS schedule. The second large issue is related to the tracker timing when it samples the collected signal. The tracker has no special timing configuration for the cosmics and therefore the collision timing is always used.  As the cross talk depends on timing, we need to use for the cross talk measurement only cosmics which arrive to the given module at the same time as a particle produced in the pp collision. Another ambiguity comes from the computation of  the particle arrival time, which extrapolates all particles to the interaction point, even though they do not have to pass through there. Both the pseudorapidity coverage and the timing issues would be solved if it is possible to arrange a new data-taking of non suppressed collision data without magnetic field. But such data-taking is not compatible with the CMS program and priorities.

After the update of the cross talk parameters and conditions such as gains and noise updated by other members of the tracker local reconstruction group, the description by the simulation of the in-time clusters in data was largely improved. It was identified that there are still several parameters which are outdated and could be  updated in future, however these parameters do not largely change the cluster description. In this thesis on one hand I found out that the simulation is simplified and sometimes not realistic, but on the other hand developing more sophisticated models would not largely improve the description of the clusters in data by simulation. I also identified that already some existing parts of the simulation have only a negligible impact, i.e. the diffusion, and it could be good to evaluate what parts of the simulation are really needed in order to reduce the time needed for the event simulation.

Several improvements could be still included in the simulation of the out-of-time clusters. First, the pulse shape has changed due to the change of the APV parameter and therefore the pulse shape should be updated in order to reduce the simulated charge by a correct fraction. Secondly, it was shown that the cross talk depends on timing. The cross talk parameters were derived for in-time clusters and are not correct for out-of-time clusters. Ideally, the cross talk parameters should be parameterized as a function of the particle arrival time to the module. The out-of-time clusters need also to be well simulated as these clusters might be used by the tracking algorithm to reconstruct the tracks of the particles.

Few other interesting studies could be performed with the taken cosmic data and here I discuss few examples. In the past it was observed that there is a left-right asymmetry in the charge sharing. This effect could be as well studied with these data. Moreover, it is possible to study the cross talk and cluster properties as a function of the position where the track intercepts the strip plane, close and far from the strip. In the past it was also observed that there is an evolution of the cluster seed charge as a function of the strip number within one APV chip, this study could be repeated with these data. All these studies could help to better understand the formation of the signal, the features of the sensors such as non-uniformities in the electric field, and also the changes in the sensors properties resulting from their irradiation. 


In the future tracker for the HL-LHC, the clustering will be largely different, only the binary information from the strip/macro-pixel will be sent, therefore the cluster seed charge and cluster charge will not be available and just the cluster width will be known. As the cluster width depends on cross talk, it will also be needed to measure the cross talk in the new tracker. Moreover it will be important to determine and monitor the cross talk in order to design and maintain the threshold for strip/macro-pixel charge to be correctly set to one or zero.



\vspace*{1cm}

After an introduction of both SM and SUSY in Chapter~\ref{sec:SUSYch}, the search for the stop is presented in Chapter~\ref{sec:stopch} in the single lepton final state with the data of Run~2. This last part of the thesis is discussing a search for physics beyond the standard model. Despite its capability to well describe the majority of observed physics phenomena, the standard model (SM) of particle physics suffers from several shortcomings such as the hierarchy problem or the absence of a dark matter candidate. Due to these shortcomings, theories beyond the SM were proposed and one of them, the supersymmetry, became the most promising one because of its ability to address large part of the SM issues. In this thesis I presented a search for the supersymmetric partner of the top quark, the stop, with the CMS Run~2 data. This search targets the stop pair production, with three different possibilities for the decays of the stops: the two stops decay each to a top quark and a neutralino, or the two stops decay to a bottom quark and a chargino, or each stop decays differently as a mixture of the two previous cases. In all cases, the targeted signal final states contain one lepton, jets and missing transverse energy. 

I was involved in three public releases of this search: one based on the data recorded in 2015 corresponding to an integrated luminosity $\int{\mathcal{L}}$ of 2.3~fb$^{-1}$~\cite{Sirunyan:2016jpr}, the second one based on the data recorded at the beginning of 2016 ( $\int{\mathcal{L}}=12.9$~fb$^{-1}$)~\cite{CMS:2016vew} and the last one corresponding to $\int{\mathcal{L}}= 35.9$~fb$^{-1}$ collected during the full 2016 pp data-taking period~\cite{Sirunyan:2017xse}. I was mainly involved in the last analysis, where I was responsible for the estimation of one of the SM backgrounds, in which a Z boson decays to two neutrinos. In this thesis I also exploited a technique for tagging of merged jets originating from a W boson and I showed that the gain of implementing such technique is growing with the integrated luminosity. 

The searches for the stop pair production in different final states were already performed with the Run~1 data. The analyses exclude stop masses in terms of simplified model spectra with stops decaying to  a top quark and a neutralino up to around 755~GeV for a neutralino mass below 200 GeV. With the increase of the center of mass energy and then also the integrated luminosity, it was soon possible to further probe the stop masses. As it can be seen in Fig.~\ref{fig:figures/resultPlot2}, no excess was found in the full 2016 data ($\int{\mathcal{L}}=35.9$~fb$^{-1}$) with respect to the SM background predictions and therefore exclusion limits were derived. This analysis excluded stop masses up to 1120 GeV for a massless neutralino in terms of simplified model spectra where both stops decay to a top quark and a neutralino as shown in Fig.~\ref{fig:figures/limitT2tt}. In the case where both stops decay to a bottom quark and a chargino or in the case of mixed decay, the stop masses were excluded up to 1000 GeV  and 980 GeV, respectively, presented in Figs.~\ref{fig:figures/limitT2bW2} and \ref{fig:figures/limitT2tb}.

    \insertFigure{figures/resultPlot2} % Filename = label
                 {0.75}       % Width, in fraction of the whole page width
                 { The presentation of the observed data (black dots), background estimates and expected signal for three different signal models in the 31 signal regions. The lost lepton background is shown in green, the $1 \ell$ background from $W+jets \to 1\ell$ (i.e. not from top) in yellow, the $Z \to \nu \bar{\nu}$ background in violet and the $t\bar{t} \to 1\ell$ contribution in red. The T2tt model ($\tilde{t}_{1} \to t \tilde{\chi}^{0}_{1}$ ) with a stop mass of 900~GeV and an LSP mass of 300~GeV is shown in blue dashed line. The T2tb ($\tilde{t}_{1} \to t \tilde{\chi}^{0}_{1} /\tilde{t}_{1} \to b \tilde{\chi}^{\pm}_{1}$) and T2bW ($\tilde{t}_{1} \to b \tilde{\chi}^{\pm}_{1}$) scenarios for a stop mass of 600~GeV and a neutralino mass of 300~GeV are shown in pink and green dashed lines, respectively. The red dashed line separates the signal regions of the nominal and optimized compressed analyses~\cite{Sirunyan:2017xse}. }

    \insertFigure{figures/limitT2tt}
                 {0.49}       % Width, in fraction of the whole page width
                 { The exclusion limits at 95\% CL on the T2tt model corresponding to an integrated luminosity of 35.9~fb$^{-1}$ ~\cite{Sirunyan:2017xse}. The interpretation is provided in the context of SMS, in the plane of the stop mass vs the LSP mass. The color code shows the 95\% CL upper limit on the cross section assuming a branching ratio of 100\% in $ \tilde{t}_{1} \to t  \tilde{\chi}^{0}_{1} $. The red and black contours represent the expected and observed exclusion limits, respectively. The blue and magenta lines in the left plot show the expected limits of $1 \ell$ and $0 \ell$ analyses, respectively.  The area below the black thick curve indicates the excluded region of the signal points.  }


    \insertFigure{figures/limitT2bW2} % Filename = label
                 {0.5}       % Width, in fraction of the whole page width
                 { The exclusion limits at 95\% CL on the T2bW model corresponding to an integrated luminosity of 35.9~fb$^{-1}$~\cite{Sirunyan:2017xse}. The interpretation is provided in the context of the SMS, in the plane of the stop mass vs LSP mass. The color code shows the 95\% CL upper limit on the cross section. The red and black contours represent the expected and observed exclusion limits, respectively. The area below the black thick curve indicates the excluded region of the signal points.  }


    \insertFigure{figures/limitT2tb}
                 {0.49}       % Width, in fraction of the whole page width
                 { The exclusion limits at 95\% CL on the T2tb model corresponding to an integrated luminosity 35.9~fb$^{-1}$ ~\cite{Sirunyan:2017xse}. The interpretation is provided in the context of SMS, in the plane of the stop mass vs the LSP mass. The color code shows the 95\% CL upper limit on the cross section assuming a branching ratio of 50\% in $ \tilde{t}_{1} \to t  \tilde{\chi}^{0}_{1} $ and 50\% in $ \tilde{t}_{1} \to b  \tilde{\chi}^{\pm}_{1} $. The red and black contours represent the expected and observed exclusion limits, respectively. The blue and magenta lines in the left plot show the expected limits of $1 \ell$ and $0 \ell$ analyses, respectively.  The area below the black thick curve indicates the excluded region of the signal points.  }



Despite the stop exclusion in the TeV range, there is still room for the natural supersymmetry and therefore the effort to search for stops is not diminishing. According to the CMS predictions, with an integrated luminosity of 3000~fb$^{-1}$ envisioned to be collected at the HL-LHC, it will be possible to probe the stop masses up to 2~TeV. An upgrade of the LHC to higher center of mass energies would permit us to go even further in the stop masses. 

Before migrating to future projects, there are several directions how to improve the presented stop analysis for its future update based on data with a larger integrated luminosity. On the side of the SM backgrounds it is desired to use the data-driven methods for the background estimations in order to minimize a dependence on the simulation and consequently reduce the systematic uncertainties. The data-driven estimates are based on control regions, these control regions should be defined very similarly as the signal regions to avoid uncertainties due to extrapolations in variables.  These extrapolations rely on the shapes of variables in simulation, and if not well modeled they largely increase the systematic uncertainties on the background estimates. On the side of the signal we discussed that the simplified event simulation is not reliable in the region of the 2D mass plane where $\Delta m(\mathrm{stop, neutralino}) \sim m(\mathrm{top})$ at low neutralino mass. This could be solved by using the full simulation of the detector in this particular kinematic region. Then also the analysis strategy can be reoptimized. It is possible to reoptimize the signal regions by finding optimal cuts but also trying to find better discriminating variables. Another option is to use more object taggers. Nowadays there is a variety of taggers targeting the W boson and top quark identification, not only targeted as a  single merged jet, but also as completely resolved jets or partially merged jets. We could also consider not to perform a cut and count analysis but move to machine learning techniques providing discriminant between signal and background we can cut on. 


Within the CMS collaboration a large variety of SUSY searches is performed. The stops are being sought in several different final states and the other predicted SUSY particles are targeted by dedicated searches as well. Many analyses search for SUSY within the context of the minimal supersymmetric standard model but there are also several searches which go beyond, e.g. searches allowing violation of the R-parity. In general the SUSY searches are interpreted in terms of simplified spectra. A lot of efforts are being done to reinterpret these searches in more realistic models. Another part of activities is the combination of searches to reach a better discovery/exclusion potential. The CMS experiment and in general the collider experiments do not provide the sole opportunity to search for supersymmetry. For example, supersymmetry can also be studied via direct dark matter detection experiments. 

In summary, there are still many options for natural SUSY in term of parameter space and also many options how to search for it. Collider experiments are one of these options, and with increasing luminosity, energy and optimization of analyses they are providing great opportunities to go beyond the current limits and further probe the uncovered phase-space.

%----------------------------------------------------------------------------------------------------------------------------------------------------------------------------------


\newpage

% ============================================================================
%   Bibliography
% ============================================================================

%\emptypage
%\emptypage

\singlespace

\addcontentsline{toc}{chapter}{Bibliography}
\renewcommand{\leftmark}{Bibliography}

%\begin{thebibliography}{2}

%    
% Not sure if that's the best way to manage the biblio, but it works.

\addReference{topPtReweighting}
{CMS Collaboration}
{Measurement of differential top-quark pair production cross sections in pp collisions at $\sqrt{s}$ = 7 TeV}
{\doi{10.1140/epjc/s10052-013-2339-4}{Eur.Phys.J.C 73 (2013)}, \pas{TOP-11-013}, \cds{1493228}, \arXiv{1211.2220}}

\addReference{SUS-13-024}
{CMS Collaboration}
{Search for top-squark pairs decaying into Higgs or Z bosons in pp collisions at sqrt(s) = 8 TeV}
{\doi{10.1016/j.physletb.2014.07.053}{Phys. Lett. B 736 (2014)}, \arXiv{1405.3886}}



%\end{thebibliography}

\bibliographystyle{unsrt}
%\bibliographystyle{amsalpha}

\bibliography{sample_y2}

\newpage
\thispagestyle{empty}

% ============================================================================
%   Summaries
% ============================================================================

\newgeometry{top=0.8cm, bottom=0.7cm, left=0.95cm, right=0.95cm, bindingoffset=0cm}

\vspace*{-0.5cm}
\begin{center}
    \textbf{\Large{Mark\'{e}ta Jansov\'{a}}}\\
\vspace*{0.2cm}
    \textbf{\Large{Search for the supersymmetric partner of the top quark and measurements of cluster properties in the silicon strip tracker of the CMS experiment at Run 2\vspace*{0.3cm}}}
\end{center}

\begin{textblock}{0}[0,0](0.5,0.32)
{
    \setlength{\fboxsep}{0.7pt}
    \setlength{\fboxrule}{1pt}
    \includegraphics[height=1.5cm]{figures/lunistra}
}
\end{textblock}
\begin{textblock}{0}[0,0](13.4,0.28)
{
    \setlength{\fboxsep}{0.7pt}
    \setlength{\fboxrule}{1pt}
    \includegraphics[width=1.5cm]{logo/CNRS}
}
\end{textblock}
    \vspace*{-0.7cm}
\small
\begin{framed}
    \vspace*{-0.4cm}
\textsc{Résumé} : Cette thèse présente trois études différentes basées sur les données de CMS du Run~2. Les deux premières sont des mesures des propriétés des amas dans le trajectographe  à pistes de silicium de CMS, liées respectivement aux particules hautement ionisantes (HIP) et au partage de charge entre les pistes voisines (également appelé diaphonie). Le dernier sujet abordé dans ce document est la recherche du partenaire supersymétrique du quark top, appelé stop.

Une augmentation de l'inefficacité de reconstruction des hits dans le trajectographe  à pistes de silicium de CMS a été observée au cours des années 2015 et 2016. Les particules hautement ionisantes ont été identifiées comme une cause possible de ces inefficacités. Cette thèse apporte des résultats qualitatifs et quantitatifs sur l'effet HIP et sa probabilité. Le HIP n'était pas la source la plus importante d’inefficacité et, une fois la source identifiée et corrigée, les nouvelles données révèlent qu’après cette correction, le HIP représente à présent la principale source d’inefficacité.

La seconde étude présentée porte sur les conditions utilisées dans la simulation du trajectographe par CMS afin de fournir des résultats réalistes. Ces conditions changent avec les conditions de fonctionnement du trajectographe et évoluent avec le vieillissement du trajectographe résultant des dommages causés par le rayonnement. Nous avons constaté que les paramètres de diaphonie obsolètes avaient une grande incidence sur la forme de l'amas. Dans cette thèse, les paramètres ont été réévalués et il a été confirmé que les nouveaux paramètres améliorent grandement l’accord des amas entre données et simulation.

La dernière partie décrit en profondeur la recherche de stop en utilisant les données  collectées en 2016 (correspondant à $ \int{\mathcal{L}} = 35.9 $~fb$^{-1} $) avec un lepton dans l'état final. Aucun excès n'a été observé par rapport aux prédictions  attendues par le modèle standard et les résultats ont été interprétés en terme de limites d'exclusion sur des modèles simplifiés.

    \textsc{Mot-clés} : physique des particules, LHC, CMS, supersymétrie, trajectographe à pistes de silicium, simulation, particules hautement ionisantes, diaphonie
\end{framed}
\vspace*{-0.7cm}
\begin{framed}
    \vspace*{-0.4cm}


\textsc{Abstract} : This thesis presents three different studies based on the CMS Run~2 data. The first two are measurements of the cluster properties in the CMS silicon strip tracker related respectively to the highly ionizing particles~(HIP) and the charge sharing among neighboring strips (also known as cross talk). The last topic discussed in this document is the search for the supersymmetric partner of the top quark, called the stop.

An increase in the hit inefficiency of the CMS silicon strip tracker was observed during the years 2015 and 2016. The highly ionizing particles were identified as a possible cause of these inefficiencies. This thesis brings qualitative and quantitative results on the HIP effect and its probability. The HIP was found not to be the largest source of inefficiency at that time and once the source was identified and fixed, the new data revealed that after this fix the HIP now represents the major source of the hit inefficiency.

The second study presented in this thesis focuses on the conditions plugged in CMS tracker simulation in order to provide realistic results. These conditions change with the tracker operating conditions and also evolve with tracker ageing resulting from the radiation damage. We identified that the outdated cross talk parameters largely impact the cluster width and seed charge. In this thesis the parameters were remeasured and it was confirmed that the new parameters largely improve the agreement of clusters between data and simulation.

The last part describes deeply the stop analysis using data recorded in 2016 (corresponding to $\int{\mathcal{L}}= 35.9$~fb$^{-1}$) with single lepton in the final state. No excess was observed in the full 2016 data ($\int{\mathcal{L}}=35.9$~fb$^{-1}$) with respect to the standard model background predictions and therefore exclusion limits in terms of simplified model spectra were derived.

    \textsc{Keywords}: particle physics, LHC, CMS, supersymmetry, CMS silicon strip tracker, simulation, highly ionizing particles, cross talk
\end{framed}
