\clearpage

\setcounter{secnumdepth}{4}
\chapterwithnum{The CMS experiment at the LHC}
\setcounter{secnumdepth}{5}

introductory word

\section{The Large Hadron Collider}

The Large Hadron Collider~(LHC) is a particle accelarator of cicumference of 27~km, which is a part of the CERN accelerator complex located near Geneva, Switzerland~\cite{CERN-Brochure-2017-002-Eng, Evans:2008zzb}. The LHC project was approved in 1994 and designed to provide mainly collisions proton bunches. Part of the LHC running time is also dedicated to collisions of heavy nuclei. Since 2008 the LHC is colliding bunches of protons, which contain around 100 bilion protons each and are arranged in the 2 beams of around 3000 bunches, which are accellarated and then collided against each other. The LHC operations is divided two wo eras, Run~I and Run~2. The Run~I started in 2008 and ended in 2013. During this era the center-of-mass energy of the collisions was 8~TeV and the beam structure allowwd collisions every 50~ns. The Run~II started in 2015 and is still ongoing. In the Run~II the collison center-of-mass energy incerased to 13~TeV and the time between collisions decreased to 25~ns.

At the beginning of the of the acceleration prcocess the protons are obtained by stripping the electrons from the hydrogen atoms. Protons are then pre-accelareted through a chain of accelerators: Linac2, PS Booster, Proton Synchrotorn~(PS) and Super Proton Synchrotron~(SPS). The proton beemas accelerated to 450~GeV are injected from SPS into the LHC~(both clockwise and anti-colckwise), where the beams are further shaped and accelerated with eight RF~(radiofrequency) cavities per beam. The LHC is equipled with around ten thousand magnets, mainly dipoles and quadrupoles, to bend and focus the beams. There is a vacuum in the LHC tube to avoid collisions of particles with gass and vacuum is also used to fo insultation of cryomagnets and helium distribution line.

The bunches of protons collide at four interaction points where the four main experiments where installed. The two general-purpose detectors, which were designed to cover wide range of physics are A Toroidal  LHC ApparatuS~(ATLAS)~\cite{Aad:2008zzm} and Compact Muon Solenoid~(CMS)~\cite{Chatrchyan:2008aa}. The other two main experiments are A Larege Ion Collider Experiment~(ALICE)~\cite{Aamodt:2008zz} which focuses on the analysis of heavy-ion collisions and the Large Hadron Collider beauty~(LHCb)~\cite{Alves:2008zz} specialized in the physics of b-quark. There are threee smaller experiment along the LHC, TOTEM~\cite{Anelli:2008zza} which is close to the CMS and its principal goal is to measure the total cross section of the proton at LHC.  The LHCf~\cite{Adriani:2008zz} which is close to the ATLAS studies particles from interactions which move very close to the proton beams. Finally the MoEDAL~\cite{Acharya:2014nyr} which is close to the LHCb focuses on the search of hypothetical paarticles, e.g. magnetic monopoles. The schematic view of the CERN accelerator complex with its main experiments can be seen in Fig.~\ref{fig:figures/CCC-v2016}.

    \insertFigure{figures/CCC-v2016} % Filename = label
                 {0.9}       % Width, in fraction of the whole page width
                 { The CERN accelerator complex~\cite{Mobs:2225847}. }

One of the main motivations for the LHC and its experiments was the search for the Higgs boson which was discovered at 2012. The other principal topic is search for new particles, motivated mainly by cosmological obervations and unification of known forces. Additionaly, the main focus of the  heavy-ion collisions is state called ``quark-gluon plasma'', which existed in the early Universe.
%- in the LEP tunnel
%- 50 - 175m underground

\newpage

\section{The Compact Muon Solenoid}

\subsection{CMS detector}

The CMS~\cite{Chatrchyan:2008aa, CMSproposal} is a multipurpose detector of a length of 28.7~m, a dimaeter 15~m and a weight 14~t,  which is located at the interaction point 5~\cite{Chatrchyan:2008aa}. The CMS was designed to have good muon identification and resolution, to have good charged-particle momentum and resolution as well as high efficiency in reconstruction of charged-particle tracks. Further it was required to ahve good electromagnetic energy, missing-transverse-energy~(MET) and dijet mass resolution. One of the important requriments was also high efficiency in offline tagging of tau particles and jets originating fro b-quarks. These confitions have to be fulfilled in the LHC environmnt, with bunch crossing every 25~ns, where every bunch crossing leads to about 20i inelastic interactions on top of the interaction of the interest resulting in around 1000 charged particles in the CMS detector every 25~ns. Because of the large radiation, the detectors and front-end electronics have to be radiation-hard.

To achieve given requirements, the CMS was built in layers around large solenoidal magnet. Inside the magnet from the intaraction points outwards, there is a pixel and silicon strip tracker, followed by the electromagnetic and hadronic calorimeter. Outside of the magnet teher is an steel return yoke with embedded muon chambers. The CMS detector is shown in Fig.~\ref{fig:fogures/cmsdetector}. In  the following sections, the layers of the CMS, starting with the innermost, are introduced in more details.
%-coverage up to |eta|<5


    \insertFigure{figures/cmsdetector} % Filename = label
                 {0.9}       % Width, in fraction of the whole page width
                 { A schematic layout of the CMS detector~\cite{website:CMSdet}. }

\subsubsection{Coordinate system and conventions}


The coordinate system used by CMS is sketched in Fig.~\ref{fig:figures/coordinates}. In this convention, the Cartesian coordinate system has a center at interaction point, with x-axis is pointing into the center of the LHC, y-axis is going upwards and z-axis is going anti-clockwise in the beam direction. The azimuthal angle $\Phi$ is defined in the x-y plane and is measured from the x-axis. The polar angle $\Theta$ is measured from the z-axis and is defined in x-z plane. Finally the $R$ is a radial coordinate in x-y plane. In this convention the pseudorapidity $\eta$ is defined as

    \insertFigure{figures/coordinates} % Filename = label
                 {0.6}       % Width, in fraction of the whole page width
                 { The CMS coordinate system with the three axes intecpeting at interaction point, x-axis pointing inside the LHC ring, y-axis going upwards and z-axis pointing anti-clockwise in the direction of the beam.~\cite{Pantaleo:2293435}. }

\eq{pseudorapidity}
{
    \eta =  -\ln [ \tan \left( \frac{\Theta}{2} \right) ].
}

The distance of two points can be measured with help of $\Delta R$ defined as

\eq{deltaR}
{
    \Delta R = \sqrt{ \Delta \Phi^2 + \Delta \eta^2}.
}


The transverse momentum $p_{T}$ can be computed from the x and y mometum components as

\eq{pseudorapidity}
{
    p_{T} =  \sqrt{p_{x}^2 + p_y^2 }.
}


\subsubsection{Solenoid}


The CMS magnet is superconducting solenoid providing magnetic field of 3.8~T. The magnet is surrounded from outside by the steel yoke which returns the magnetic flux of the solenoid~\cite{tdrMagnet}. Because of the pgysics reson, the radius of the magnet has to be small and therefore the available space between the magnet and interaction point is limited.

\subsubsection{Silicon tracker}

The silicon tracker~\cite{CMS:1997tlf, CMS:2000eqx} is the innermost subdetector of the CMS detector, which consist of layers of the silicon sensors. The CMS tracker can be divided into two parts~\cite{CMS:1997tlf}. The inner part is composed by pixel detectors, while the outer part by strip detectors. Its purposed is to reconstruct the tracks of charged particles and the primary and secondary vertices. The track is the reconstructed path of the particle passing trough the tracker. The primary vertices, vertex from the interaction of interest and PU interactions, are the positions of the interaction and can be determined using the tracks. The secondary vertices are the places where one particle decayed to other particles and are also found with help of tracks. 


As the tracker is in the magnetic field in which the cahrged particles bend, their momenta and charge sign can be measured as well. To perform so, the tracker has to have good spatial resolution and to be extremly radiation-hard due to large flux of incoming particles. Also the material has to be chosen carefuly, in order to avoid multiple scattering, nuclear interactions or bremsstrahlung.

The density of the particles decrease with distance from the interaction point and therefore the inner part of the tracker is made of silicon pixels, which are able to measure aprticle trajactories and its properties high particle density environment. In the outer part of the tracker the density is low enough to use silicon strip sensors, which are cheaper mainly because they require less readout channels. Overall, the tracker pseudorapidity coverage is  $\eta < 2.5$.

\textbf{Silicon strip tracker}

In the following chapters the studies of the silicon strip tracker and its simulation are presented and thus it is necessary to take a deeper look into its design. The silicon strip tracker is divided into four partitions. Each of the partition has layers of modules, which have either one side~(mono) or two sides~(stereo) of silicon sensors. The tracker inner barrel~(TIB) is the innermost barell layer with two layers of stereo modules suceeded with two layers of mono modules. The tracker outer barrel~(TOB) surrounds the TIB. In the innermost part, it has two layers of stereo modules, the remaining four layers are mono. On each side of the barrel, tracker inner disks~(TID) and tracker endcapes~(TEC) are located. There are three wheels of TID with three module rings and 9 wheels of TEC with four to seven rings of modules on each side. In each layer there is a mixture of mono and stereo modules, but each ring has either mono or stereo modules. The overall layout of the silicon strip tarcker can be seen in Fig.~\ref{fig:figures/cmsTracker}.

    \insertFigure{figures/cmsTracker}
                 {0.9}       
                 {A schema of the upper half of the CMS silicon strip tracker tracker and its partition layout. The modules in blue are stereo, while the mono ones are shown in black color~\cite{Chatrchyan:2014fea}. }

Each of the modules holds 320$\mathrm{\mu m}$ or 500$\mathrm{\mu m}$ thick silicon sensors with either 512 or 768 silicon strips, in case of mono modules, for stereo modules the number doubles. The strip lenght is between 8 and 25~cm and their width is 18~$\mathrm{\mu m}$. The distance between strips, called pitch, varies between 80 $\mu m$ and 200 $\mu m$. In barrels the strips are parallel to the z-axis, or are tilted by 100~mrad with respect to the z-axis. In the disks and endcapes, the strips are allingned to be parallel to $R$. The details about the module geometries for each laer of TIB and TOB and each ring of TID and TEC can be found in Tab.~\ref{tab:trackerGeometries} .

\begin{table}[h]
\begin{center}
\begin{tabular}{|l|l|l|l|l|}
\hline
Layer & Type  & Strips & Thickness~[$\mathrm{\mu m}$] & Pitch~[$\mathrm{\mu m}$]  \\
\hline
\hline
TIB L1 & stereo & 768 & 320 & 80  \\
TIB L2 & stereo & 768 & 320 & 80  \\
TIB L3 & mono & 512 & 320 & 120  \\
TIB L4 & mono & 512 & 320 & 120 \\
\hline
TOB L1 & stereo & 768/512 & 500 & 122/183 \\
TOB L2 & stereo & 768/512 & 500 & 122/183 \\
TOB L3 & mono & 512 & 500 & 183 \\
TOB L4 & mono & 512 & 500 & 183 \\
TOB L5 & mono & 768 & 500 & 122 \\
TOB L6 & mono & 768 & 500 & 122 \\
\hline
TID R1 & stereo & 768 & 320 & 81...112  \\
TID R2 & stereo & 768 & 320 & 113...143  \\
TID R3 & mono & 512 & 320 & 124...158  \\
\hline
TEC R1 & stereo & 768 & 320 & 81...112  \\
TEC R2 & stereo & 768 & 320 & 113...143  \\
TEC R3 & mono & 512 & 320 & 124...158  \\
TEC R4 & mono & 512 & 320 & 113...139  \\
TEC R5 & stereo & 768 & 500 & 126...156  \\
TEC R6 & mono & 512 & 500 & 163...205  \\
TEC R7 & mono & 512 & 500 & 140...172  \\
\hline
\end{tabular}
\caption[Table caption text]{The module type, number of strips, thickness and pitch for layers or rings of four silicon strip tracker partitions ~\cite{website:hephyPage}. }
\label{tab:trackerGeometries}
\end{center}
\end{table}


%-hit resolution

\subsubsection{Electormagnetic calorimeter}

The electromagnetic calorimeter (ECAL)~\cite{tdrECAL} is a layer following the silicon tracker. It is homogenous, fast, radiation resistant calorimenter with a good energy resolution which is composed of lead-tungstate ($\mathrm{PbWo_{4}}$) crystals and its purpose is to measure energy of electrons and photons. The ECAL consist of two parts, barrel~(EB) covering $|\eta|<1.479$ and endcaps~(EC) extending coverage up to $\eta =3$. A preshower is placed in front of the endcaps in order to separate highly energetic single photons from the photon originating from the decay of neutral pions.

The energy resolution of ECAL was determined to be

\eq{ECALresol}
{
 \frac{\sigma_{E}}{E} = \frac{0.028}{\sqrt{E}} \bigoplus \frac{0.12}{E} \bigoplus 0.003 ,
}

where $E$ is energy and $\sigma_{E}$ is energy resolution. The first term is stochastic part, it corresponds to e.g. fluctioations in number of particles. The second term accounts for noise and the third term covers mainly the non-uniformities, energy leakage and intercallibration errors.


\subsubsection{Hadron calorimeter}

The purpose of the hadron calorimeter~(HCAL)~\cite{tdrHCAL}, which is subdetector another subdetector of CMS, is to measure energy of strongly interacting particles.  The HCALi is a sampling colorimeter which has four parts, out of them two are located between the ECAL and magnet, these are HCAL barrrel~(HB) and endcaps~(HE). Both HE and HB have a brass absorber and their active material is made of plastic scintilator. The pseudorapidity coverage of HB is $|\eta|<1.3$, and of HE $1.3<|\eta|<3$, which is further extended up to $|\eta|=5.2$ by the third part called forward calorimeter~(HF). The HF, installed 11.2 meters far from interaction point on both sides, has is made of steel as an absorber and quartz fibers as an active volume. The technology of the HF is very radiation-hard as around third of particles produced in the final state hits the HF. Because of the available space between the ECAL and magnet was not large enough to build caloriemter with enough stopping power, the last part of the calorimeter, the outside calorimeter~(HO), was added after the magnet. The HO is covering region $|\eta|<1.3$ and stops particles escaping the barrel, for this reason it is someties also reffered as ``tail catcher''. The magnet material is used as an absorber for HO.

The hadron energy resolution from combination of ECAL and HCAL (barrel and endcapes)~\cite{Chatrchyan:2009ag} was determined to be


\eq{HCALresol}
{
 \frac{\sigma_{E}}{E} = \frac{0.847}{\sqrt{E}} \bigoplus 0.074 ,
}

where the terms have similar meaning as for ECAL.
%- for HF 1.98/sqrt(E)+0.09 -> higher because of high energy of jets in this region but as divided by energy, it is ok

\subsubsection{Muon chambers}

Bacause many interesting physics processes have signature with muons in the final state, good and precise measyrement of muons is one of teh main goals of the CMS. This comprises muon identification, momentum measurement and triggering. The good triggering and momentum measurement is achieved with help of the high magnetic field provided by the solenoid. The muon mesurement is provided by three gaseous subdetectors, the drift tube~(DT), cathode strip chamber~(CSC) and resistive plate chamber~(RPC) systems~\cite{tdrMuon}.

The DTs are located in the barrel region, they are partly integrated into the return yoke and are covering pseudorapidity of $|\eta|<1.2$. They are composed of plane cathodes with anode wires are in between. The endcaps are outside the return yoke and are composed by the CSCs. The CSCs coverage is $0.9<|\eta|<2.4$, which is partly overlapping with teh DTs. The CSCs also contain cathodes with anode wires in between, but one cathode of the pair is segmented into strips. The DTs and CSCs provide good triggering on uons independent of the rest of the CMS.

To ensure the triggering on the right bunch crossing, the compementary RPCs are present in both barrel and endcap regions. The RPCs are faster DTs and CSCs, but on the other hand they provide worse position resolution. The RPCs trigger is independent of the CSCs and DTs. They are composed of parallel plates of anodes and cathodes and redout strips. 

in the following sub-section the time measurement in DTs is introduced. The timing can be also measured with CSCs and ECAL in similar manner, but for the purposes of the thesis only the measurement in DTs is needed.

\textbf{Muon timing measurement in DTs}

The muon going through DTs deposits its energy by ionization of the gas. Created charge carriers drift towards the wires and are read at time $t_{read}$. If it is assumed that the muon was produced in time with collision and its speed is speed-of-light, the time ($t_{cell}$) when the muon arrives to the given drift cell can be calculated. Then with the knowledge of the drift velocity~$v_{drift}$, the difference between $t_{read}$ and $t_{cell}$ can be converted into distance as shown in Fig.~\ref{fig:figures/dtTiming}. In this figure the distance $ v_{drift} (t_{read} - t_{cell})$ is illustrated in violet arrow. The distance is then used to compute the hits where muon crossed the cells, shown in blue crosses. In case that the production time was assumed correctly, all hits should be on one line. If the particle was not produced ``in time'' with the assumed production time, the hits will be shifted~(red crosses) by and the line which connects them is not straight, but curly. The time distance between reconstructed hits and hits on the straight line is converted to the time~$\delta t$, which measures real arrival time of muon with respect to the expected arrival time, and thus can be used to compute the muon timing~\cite{Traczyk:1365029}.


    \insertFigure{figures/dtTiming} % Filename = label
                 {0.5}       % Width, in fraction of the whole page width
                 { A schema of few drift cells of DT. The wires are shown in black dots, the drift direction in purple arrows, the expected hits in blue crosses and the reconstructed hits in the red crosses. The distance between reconstructed and expected hits is coverted to time ans denoted as $\delta t$~\cite{Traczyk:1365029}. }

The timing measurements are combined to produce, among others, the following variables 

\begin{description}
\item [TimeAtIPInOut]
This variable corresponds to the ttime at which muon passed the interaction point, assunimng that particle moves at speed of light from IP outside the CMS. It is compuetd as a weighted avarage of $\delta t$ values, where for the DTs each weight weight is equal to N-2, with N being a number of hits in a segment to which the the hit belongs. The error on the time measurement is compued as

\eq{timingResol}
{
 \sigma^{2} = \frac{1}{N-1} \times \frac{1}{\sum{w_{i}}} \times \sum{(t_i-\bar{t})^2 w_{i} },
}

where wight $w_{i}$ is defined as $w_i = 1/\sigma_{i}^2$ with $\sigma_{i}$ being a single hit resolution, $\bar{t}$ is the weighted average of the $\delta t$ measurements.

\item[TimeAtIPOutIn]
The TimeAtIPOutIn is the muon time at interaction point assuming a muon moving from outside of the detector towerds the IP. During the calculation, each $\delta t$ measureemnts is increased by twice time-of-flight~(TOF) of the in-time muon from IP to the DT cell measuring $\delta t$. Then the proceedure continues as for the TimeAtIPInOut.

\item[Direction]
The direction variable provides a simple and rubust estimate if the muon moved from the IP out or opposite. The evaluation takes into account the errors on the TimeAtIPOutIn and TimeAtIPInOut variables and assumes that the correct time hypothesis has smaller error.

\item[Free inverse beta]
This variable is free $c/v$, where $c$ is the speed of light and $v$ speed of the muon. The word ``free''indicates that neither particle production time, direction nor velocity is assumed and are free parameters. It is obtained from the fit of muon time-of-flight measurements.

The timing resolution for DTs in Run~I was determined to be 7-9~ns~\cite{Traczyk:1365029}, compared to 2.3~ns obtained from simulation. But the measured resolution is expected to decrease as the detectro synchronization improves.

\end{description}

%-on top of it
 %CMS being synchronized "top-down" for cosmic runs, so that a cosmic muon going vertically down is always in-time (or "at the same time" to be more precise). So basically the DT system is timed for muons going straight down. (just like for collisions it's timed for muons going outwards from IP).In practical terms. The "muons" collection takes as timeAtIpInOut what is more or less the mean of the segment times for a muon. Because it's assuming that the muon is propagating in the same direction that the system is synchronized in. From my observations above it looks like this is the case, the system is synchronized for downward-going cosmics. - not the case in the paper


\subsubsection{Trigger and data acquisition}

As in the CMS the buches are colliding every 25~ns, thus with a rate of 40~MHz, a good, fast and reliable triggering system is required. The data size of one event is approximately 1~MB, therefore if there would not be any dedicated trigger, 40~TB of data per second would have to be stored, what is far beyond current technical capacities. In the CMS there are two level of triggers which provide physics motivated selection of the interesting events. The first one called Level-1~(L1), is hardware based and for its selection it uses information from muon chambers and calorimeters. The L1 is capable to decide within 3.4~$\mu s$ and its output rate is around 100~kHz. The rest of the event is read upon the decision of L1 and sent to the second level called High Level Trigger~(HLT), which is software based and provides further selection. The output rate from HLT is of orther of hundreds Hz.

\subsubsection{Luminosity and pile-up}

In the particle physcis experiments it is very imprtant to know an expected event rate during a time period. For this evaluation, the following formula can be used

\eq{nev}
{
 \frac{\mathrm{d}N}{\mathrm{d}t} = \sigma \times \mathcal{L},
}

where $\sigma$ is the cross section of process of interest and $\mathcal{L}$ is the instantenous luminostity given by

\eq{luminosity}
{
 \mathcal{L} = \frac{N_{p}^2 N_{b} f_{rev} \gamma}{4 \pi \epsilon_{n} \beta^{*}}F,
}

where the variables in numerator are $N_{p}$ which is the number of particle in one bunch, $N_{b}$ which is the number of bunches in one beam and $f_{rev}$ which is the revolution frequency and $\gamma$ is the relativistic gamma factor. In the denominator the $\epsilon_{n}$ is normalized transverse beam emmitance and $\beta_{*}$ is the beam amplitude function. Lastly, because of the beam crossing angle the reduction factor $F$ is introduced.

By integration of the instantenout luminosity over time, the integrated luminosity can be obtained:

\eq{intluminosity}
{
 L = \int{ \mathcal{L} \mathrm{d}t}.
}

The LHC was designed to deliver the instantenous luminosity of LHC is $1 \times 10^{34} cm^{-2}s^{-1}$. At the begining of Run~I the instantenous luminosity was lower than designed and was beaing increased during the years of LHC operation. Later, bacause of the smooth running, it was decided to go even beyond the designed luminosity, up to $1.58 \times 10^{34} cm^{-2}s^{-1}$~\cite{Pralavorio:2272474}. The integrated luminosity delivered to CMS over years 2010-2017 can be seen in Fig~\ref{fig:figures/cmsLumi}.

    \insertFigure{figures/cmslumi} % Filename = label
                 {0.6}       % Width, in fraction of the whole page width
                 { The delivered luminosity to the CMS detector for years 2010-2017~\cite{website:CMSlumi}. }


High luminosity is essential to study rare processes, but on the other hands it brings effect called pile-up~(PU)~\cite{Bayatian:2006nff}. Pile-up are particles which are not originating from the interaction of interest and can be divided into two categories. During one bunch crossing, not only the intearction of interest is produced, but more inelastic or diffractive interactions appear at the same time. This kind of pile-up is refefred as in-time pile-up. The second kin of pile-up is called out-of-time pile-up~(OOT PU) and originates from particles produced before or after the bunch crossing of interest. The OOT PU is cause for exaple by slow particles looping in the detector.

\newpage

\subsection{Event and object reconstruction at CMS}

-the data taken by CMS reconstructed and slimmed to be used by analyses
-describe data formats and reconstruction of the objects

-each aprticle leaves characteristic signature and with it can be reconstructed - describe transverse view

    \insertFigure{figures/CMStransverse} % Filename = label
                 {0.9}       % Width, in fraction of the whole page width
                 { Transverse view through the CMS detector~\cite{Sirunyan:2017ulk}. }

-details about data formats and objects in CMS in following sub-sections

\subsubsection{Data formats}

HLT -> data eneters in the RAW format to Tier-0 where data are reconstructed for monitoring and calibration purposes -> express reconstruction.
Prompt-RECO -> reconstruction of physics objects (At T0 or T1)
-then data are saved to tapes and moved to Tier-1s if there is an interest
-mini-AOD format~\cite{Petrucciani:2029414}, slimming in order to reduce size of datasets -> enough for majority of analyses
-other formats as well but tehse are important
%-primary datasets according to trigger path?!

\subsubsection{Particle-Flow algorithm}

-two options of object reconstruction -> recunstruction of whole event 
-> reconstruction of objects specific to one subdetector (only one subdetector info used )
-> connection of all information from subdetectors -> reconstruct each particle ->PF algo

- all subdetetor information conencted by using geomterical criteria
-> in general  first the energy deposits are clustered in ECAL, which are seeds to PF particle reconstruction

-stadalone muon -muon hits tehn matched to innter tracks in tracker
-electrons seeding in ECAL, cluster position defines where to find a track in tarcker

~\cite{Sirunyan:2017ulk}



\subsubsection{Jets and b-jets}

anti-kt clustering algorithm, which defines distance between objects~\cite{Cacciari:2008gp}
\eq{antikt}
{   
    d_{ij} = \mathrm{min}(p_{T}^{-2}(i), (p_{T}^{-2}(j)) \frac{(\eta_{i} -\eta_{j}^2)+ (\Phi_{i} -\Phi_{j}^2)}{R^2} =  \mathrm{min}(p_{T}^{-2}(i), (p_{T}^{-2}(j)) \frac{\Delta R_{ij}^{2}}{R^2}}.
}
where R(definititon?) is the radus parameter and in standard jet reconstruction its value is of 0.4 during the run~I. Merged jets can be reconstructed wit 0.8 or 1 -> jet size parameter

-antikt - joining(clustering) together objects with the smallest distance between each other and goes to larger distances (when ends?)

-antikt can cluster particles comming from PF -> PF jets, or clustering energy deposits in calorimeters (Calo jets).

-suppress PU  - Charged Hadron Subtraction~(CHS) used -> removing charged particles not oridinating from primary vertex from the clustering

-the energy mesurement can have some biases and the measured energyi does not have to be same as the particle energy thus needs to be corrected -> jet energy calibration.

-b-jets -> jet tagging techniques
-to know from which particle ejt originated
-heavy flavour tagging -> b-tagging~\cite{Sirunyan:2017ezt}
-heavy flavour jet tagging - heavy flavour hadron from the hadronization of b-quark presen in jets -> the lifetime of hadrons containing b-quark of order 1.5 ps displacement of few mm up to cm -> displaced tracks (wrt pripary vertex) from which secondary evrtex can eb reconstructed
-impact parameter - distance between PV and SV
-Combined Secondary Vertex algorithm~(CSVv2) based on CSV algorithm from Run~I
	-combines info of displaced tracks with info on the secondary vertex
-combined multivariate analysis~(cMVAv2) version of cMVA from run~I.
	-soft electrons or muons can be  present in the b-jet - soft leptons used in combination of other taggers to discriminate b-jet
-(DeepCSV) -> CSV algorithm using deepi machine learning (neural network)
-b-tagging scale factors - data to simulation factor, compensates for different b-tagging efficiency in data and MC.

\subsubsection{Leptons}

\textbf{Muon}
-muons interact in all layers of the CMS
%-two ways -outside in and inside out -> name form the sysem where reconstruction started - outside-in starts with muon chambers and then continues into tracker. in-out start in tracker and are extrapolated to muon chambers
-tracks (standalone muon) are created in the muon chambers from the hits ~combining CSC,DT,RPC info 
(inner tracks) from silicon tarcker only ; -> 
-first option: (tracker muons) built in-out using tracker info and matching with the muon chamber info (hits)
-second option: (global muons) combining inner tracks with standalone muon
~\cite{Chatrchyan:2012xi}
-leptons from ahrd scattering - they should be isolated - not much activity in proximity -> improved reconstruction of muons by using isolation criterium

\textbf{Electron}
Depositions in ECAL and tracker -> bremstralung and photon emission -> elmag shower in ecal
-> matching of energy deposits and tracks in tarcker~\cite{Khachatryan:2015iwa}


\textbf{Tau}
-short living, too short to see secondary vertex
-usually decays hadronically to combination of charged and neutral mesons plus tau neutrino
-hadronic taus reconstructed with hadron-plus-strips algorithm (HPS)-> can distinguish several decay signatures ~\ref{CMS:2016gvn}
	-looks into jet to reconsctuct neutral pion decaying to two gamma. The gammas convert to electron-positron pair. Neutral pions in most of teh hadronic decaying taus
         -te electrons/positrons from conversion have broadened signature in ECALi in azimuthal direction because of bending in mg field. This is taken into account by reconstructing the photons into so called ``strips'', which are objects reconstructed from photons and electrons. The reconstruction centers a strip to the position of the most energetic elmag particle within a jet. Then it searches for other elamg particle within given space window - if more energetic particle foundit is associated with the strip. Repeated up to the point that there are no other elmag particles in proximity. Then the strips passing certain momentum threshold are combined with charged hadrons to form the hadronically decaying tau.
	-each antikT jet can eb possibly a pizero comming from tau -> pizero decays to gammas which convert to e-e+. electron-positron pair is bended in mg field

\subsubsection{Missing transverse energy}


imbalance in thransverse momenta summing all particles enterinf the detector
negative of the vectorial sum of the transverse momenta of all PF particles.
\vec{ET} miss =
ETmiss magnitude
~\cite{CMS:2016ljj}

-presence of weekly interacting particles escaping detection -> imbalance of the momentum in plane perpendicular to the beam direction. Other particles must be measured precisely in order to be able to reconstruct the MET.
-very important for searches beyon teh standard model as they ofthen predictneutral weekly interactiong particles.

-because of the min energy thresholds in calorimeters, inefficiencies in tarcker, nonlinearity i  response for hadrons of calos, the MET can be not completely correct. -> the correction are applied to correct the MET for such inefficiencies (correction by correcting the pT of jets to the particle level using jet energy corrections -> propagation to the MET calcylation METcorr= MET -(pTcorr-pT) )
-> this is type-I correction %- it uses jet energy scale corrections for corrected jets above certain pT threshold (15GeV), which deposited less than 90\% of teir energy in ECAL.
-uncertainty on MET - decompose type of objects (jets, photons, electrons ..) and vary each object within its resolution uncertainty  

%photons?
