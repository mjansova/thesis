\clearpage

\setcounter{secnumdepth}{4}
\chapterwithnum{The CMS experiment at the LHC}
\setcounter{secnumdepth}{4}

introductory word

\section{The Large Hadron Collider}

The Large Hadron Collider~(LHC) is a particle accelarator of cicumference of 27~km, which is a part of the CERN accelerator complex located near Geneva, Switzerland~\cite{CERN-Brochure-2017-002-Eng, Evans:2008zzb}. The LHC project was approved in 1994 and designed to provide mainly collisions proton bunches. Part of the LHC running time is also dedicated to collisions of heavy nuclei. Since 2008 the LHC is colliding bunches of protons, which contain around 100 bilion protons each and are arranged in the 2 beams of around 3000 bunches, which are accellarated and then collided against each other. The LHC operations is divided two wo eras, Run~I and Run~2. The Run~I started in 2008 and ended in 2013. During this era the center-of-mass energy of the collisions was 8~TeV and the beam structure allowwd collisions every 50~ns. The Run~II started in 2015 and is still ongoing. In the Run~II the collison center-of-mass energy incerased to 13~TeV and the time between collisions decreased to 25~ns.

At the beginning of the of the acceleration prcocess the protons are obtained by stripping the electrons from the hydrogen atoms. Protons are then pre-accelareted through a chain of accelerators: Linac2, PS Booster, Proton Synchrotorn~(PS) and Super Proton Synchrotron~(SPS). The proton beemas accelerated to 450~GeV are injected from SPS into the LHC~(both clockwise and anti-colckwise), where the beams are further shaped and accelerated with eight RF~(radiofrequency) cavities per beam. The LHC is equipled with around ten thousand magnets, mainly dipoles and quadrupoles, to bend and focus the beams. There is a vacuum in the LHC tube to avoid collisions of particles with gass and vacuum is also used to fo insultation of cryomagnets and helium distribution line.

The bunches of protons collide at four interaction points where the four main experiments where installed. The two general-purpose detectors, which were designed to cover wide range of physics are A Toroidal  LHC ApparatuS~(ATLAS)~\cite{Aad:2008zzm} and Compact Muon Solenoid~(CMS)~\cite{Chatrchyan:2008aa}. The other two main experiments are A Larege Ion Collider Experiment~(ALICE)~\cite{Aamodt:2008zz} which focuses on the analysis of heavy-ion collisions and the Large Hadron Collider beauty~(LHCb)~\cite{Alves:2008zz} specialized in the physics of b-quark. There are threee smaller experiment along the LHC, TOTEM~\cite{Anelli:2008zza} which is close to the CMS and its principal goal is to measure the total cross section of the proton at LHC.  The LHCf~\cite{Adriani:2008zz} which is close to the ATLAS studies particles from interactions which move very close to the proton beams. Finally the MoEDAL~\cite{Acharya:2014nyr} which is close to the LHCb focuses on the search of hypothetical paarticles, e.g. magnetic monopoles. The schematic view of the CERN accelerator complex with its main experiments can be seen in Fig.~\ref{fig:figures/CCC-v2016}.

    \insertFigure{figures/CCC-v2016} % Filename = label
                 {0.9}       % Width, in fraction of the whole page width
                 { The CERN accelerator complex~\cite{Mobs:2225847}. }

One of the main motivations for the LHC and its experiments was the search for the Higgs boson which was discovered at 2012. The other principal topic is search for new particles, motivated mainly by cosmological obervations and unification of known forces. Additionaly, the main focus of the  heavy-ion collisions is state called ``quark-gluon plasma'', which existed in the early Universe.
%- in the LEP tunnel
%- 50 - 175m underground

\newpage

\section{The Compact Muon Solenoid}

\subsection{CMS detector}

The CMS~\cite{Chatrchyan:2008aa, CMSproposal} is a multipurpose detector of a length of 28.7~m, a dimaeter 15~m and a weight 14~t,  which is located at the interaction point 5~\cite{Chatrchyan:2008aa}. The CMS was designed to have good muon identification and resolution, to have good charged-particle momentum and resolution as well as high efficiency in reconstruction of charged-particle tracks. Further it was required to ahve good electromagnetic energy, missing-transverse-energy~(MET) and dijet mass resolution. One of the important requriments was also high efficiency in offline tagging of tau particles and jets originating fro b-quarks. These confitions have to be fulfilled in the LHC environmnt, with bunch crossing every 25~ns, where every bunch crossing leads to about 20i inelastic interactions on top of the interaction of the interest resulting in around 1000 charged particles in the CMS detector every 25~ns. Because of the large radiation, the detectors and front-end electronics have to be radiation-hard.

To achieve given requirements, the CMS was built in layers around large solenoidal magnet. Inside the magnet from the intaraction points outwards, there is a pixel and silicon strip tracker, followed by the electromagnetic and hadronic calorimeter. Outside of the magnet teher is an steel return yoke with embedded muon chambers. The CMS detector is shown in Fig.~\ref{fig:fogures/cmsdetector}. In  the following sections, the layers of the CMS, starting with the innermost, are introduced in more details.
%-coverage up to |eta|<5


    \insertFigure{figures/cmsdetector} % Filename = label
                 {0.9}       % Width, in fraction of the whole page width
                 { A schematic layout of the CMS detector~\cite{website:CMSdet}. }

\subsubsection{Coordinate system and conventions}


The coordinate system used by CMS is sketched in Fig.~\ref{fig:figures/coordinates}. In this convention, the Cartesian coordinate system has a center at interaction point, with x-axis is pointing into the center of the LHC, y-axis is going upwards and z-axis is going anti-clockwise in the beam direction. The azimuthal angle $\Phi$ is defined in the x-y plane and is measured from the x-axis. The polar angle $\Theta$ is measured from the z-axis and is defined in x-z plane. Finally the $R$ is a radial coordinate in x-y plane. In this convention the pseudorapidity $\eta$ is defined as

    \insertFigure{figures/coordinates} % Filename = label
                 {0.6}       % Width, in fraction of the whole page width
                 { The CMS coordinate system with the three axes intecpeting at interaction point, x-axis pointing inside the LHC ring, y-axis going upwards and z-axis pointing anti-clockwise in the direction of the beam.~\cite{Pantaleo:2293435}. }

\eq{pseudorapidity}
{
    \eta =  -\ln [ \tan \left( \frac{\Theta}{2} \right) ].
}

The distance of two points can be measured with help of $\Delta R$ defined as

\eq{deltaR}
{
    \Delta R = \sqrt{ \Delta \Phi^2 + \Delta \eta^2}.
}


The transverse momentum $p_{T}$ can be computed from the x and y mometum components as

\eq{pseudorapidity}
{
    p_{T} =  \sqrt{p_{x}^2 + p_y^2 }.
}


\subsubsection{Solenoid}


The CMS magnet is superconducting solenoid providing magnetic field of 3.8~T. The magnet is surrounded from outside by the steel yoke which returns the magnetic flux of the solenoid~\cite{tdrMagnet}. Because of the pgysics reson, the radius of the magnet has to be small and therefore the available space between the magnet and interaction point is limited.

\subsubsection{Silicon tracker}

The silicon strip is the innermost subdetector of the CMS detector, which can be divided into two parts. The inner part is composed by pixel detectors, while the outer part by strip detectors.

\subsubsection{Electormagnetic calorimeter}

The electromagnetic calorimeter (ECAL)~\cite{tdrECAL} is a layer following the silicon tracker. It is homogenous, fast, radiation resistant calorimenter with a good energy resolution which is composed of lead-tungstate ($\mathrm{PbWo_{4}}$) crystals and its purpose is to measure energy of electrons and photons. The ECAL consist of two parts, barrel~(EB) covering $|\eta|<1.479$ and endcaps~(EC) extending coverage up to $\eta =3$. A preshower is placed in front of the endcaps in order to separate highly energetic single photons from the photon originating from the decay of neutral pions.

The energy resolution of ECAL was determined to be

\eq{ECALresol}
{
 \frac{\sigma_{E}}{E} = \frac{0.028}{\sqrt{E}} \bigoplus \frac{0.12}{E} \bigoplus 0.003 ,
}

where $E$ is energy and $\sigma_{E}$ is energy resolution. The first term is stochastic part, it corresponds to e.g. fluctioations in number of particles. The second term accounts for noise and the third term covers mainly the non-uniformities, energy leakage and intercallibration errors.


\subsubsection{Hadron calorimeter}

The purpose of the hadron calorimeter~(HCAL)~\cite{tdrHCAL}, which is subdetector another subdetector of CMS, is to measure energy of strongly interacting particles.  The HCALi is a sampling colorimeter which has four parts, out of them two are located between the ECAL and magnet, these are HCAL barrrel~(HB) and endcaps~(HE). Both HE and HB have a brass absorber and their active material is made of plastic scintilator. The pseudorapidity coverage of HB is $|\eta|<1.3$, and of HE $1.3<|\eta|<3$, which is further extended up to $|\eta|=5.2$ by the third part called forward calorimeter~(HF). The HF, installed 11.2 meters far from interaction point on both sides, has is made of steel as an absorber and quartz fibers as an active volume. The technology of the HF is very radiation-hard as around third of particles produced in the final state hits the HF. Because of the available space between the ECAL and magnet was not large enough to build caloriemter with enough stopping power, the last part of the calorimeter, the outside calorimeter~(HO), was added after the magnet. The HO is covering region $|\eta|<1.3$ and stops particles escaping the barrel, for this reason it is someties also reffered as ``tail catcher''. The magnet material is used as an absorber for HO.

The hadron energy resolution from combination of ECAL and HCAL (barrel and endcapes)~\cite{Chatrchyan:2009ag} was determined to be


\eq{HCALresol}
{
 \frac{\sigma_{E}}{E} = \frac{0.847}{\sqrt{E}} \bigoplus 0.074 ,
}

where the terms have similar meaning as for ECAL.
%- for HF 1.98/sqrt(E)+0.09 -> higher because of high energy of jets in this region but as divided by energy, it is ok

\subsubsection{Muon chambers}

Bacause many interesting physics processes have signature with muons in the final state, good and precise measyrement of muons is one of teh main goals of the CMS. This comprises muon identification, momentum measurement and triggering. The good triggering and momentum measurement is achieved with help of the high magnetic field provided by the solenoid. The muon mesurement is provided by three gaseous subdetectors, the drift tube~(DT), cathode strip chamber~(CSC) and resistive plate chamber~(RPC) systems~\cite{tdrMuon}.

The DTs are located in the barrel region, they are partly integrated into the return yoke and are covering pseudorapidity of $|\eta|<1.2$. They are composed of plane cathodes with anode wires are in between. The endcaps are outside the return yoke and are composed by the CSCs. The CSCs coverage is $0.9<|\eta|<2.4$, which is partly overlapping with teh DTs. The CSCs also contain cathodes with anode wires in between, but one cathode of the pair is segmented into strips. The DTs and CSCs provide good triggering on uons independent of the rest of the CMS.

To ensure the triggering on the right bunch crossing, the compementary RPCs are present in both barrel and endcap regions. The RPCs are faster DTs and CSCs, but on the other hand they provide worse position resolution. The RPCs trigger is independent of the CSCs and DTs. They are composed of parallel plates of anodes and cathodes and redout strips. 

\subsubsubsection{Muon timing measurement in DTs}

- in DTs the local position of muon is determined from the drift time of electrons (ionized by muon) to the wires. The drift velocity is known and the time when the particle hits detector is know as well (particle produced at IP travelloing at c). The difference of the time when the detector is hit and the readout time can be converted into distance from the anode wire. What is unknown if th ewire is hit fro left or right, but this is solved by using more layers as layers have shifted chambers one against each other. (need to add Fig. 1). If particle arrives at different time than assumed during standard reconstruction, the trajectory will not be straight because of the shift of delta(t) -> wavy pattern. Deviation of hit from the straight line -> measure of timing offset delta(t), wich can be converted to time again.

Timing reconstruction -> input are dleta(t) measurements = measured arrival time to expected arrival time. 

-timeAtIPInOut

assumes particle going from inside of the CMS out at speed of light. time when muon passed the IP is computed from the weighted average of delta(t) measurements. the weight for DT is N-2, where N is number of hits in DT segment to which the hit belongs.  The error is caluculated as

\eq{timingResol}
{
 \sigma^{2} = \frac{1}{N-1} \times \frac{1}{\sum{w_{i}}} \times \sum{(t_i-\bar{t})^2 w_{i} },
 
}

where $wi = 1/\sigma_{i}^2$ (single hit resolution), t stredni is the delta(t) calculated from the  wighted average

-timeAtIPOutIn 


-on top of it
 CMS being synchronized "top-down" for cosmic runs, so that a cosmic muon going vertically down is always in-time (or "at the same time" to be more precise). So basically the DT system is timed for muons going straight down. (just like for collisions it's timed for muons going outwards from IP).In practical terms. The "muons" collection takes as timeAtIpInOut what is more or less the mean of the segment times for a muon. Because it's assuming that the muon is propagating in the same direction that the system is synchronized in. From my observations above it looks like this is the case, the system is synchronized for downward-going cosmics.



~\cite{Traczyk:1365029}



\subsubsection{Castor}
\subsubsection{Trigger and data acquisition}
\subsubsection{Luminosity and pile-up}


    \insertFigure{figures/cmslumi} % Filename = label
                 {0.6}       % Width, in fraction of the whole page width
                 { The delivered luminosity to the CMS detector for years 2010-2017~\cite{website:CMSlumi}. }

\newpage

\subsection{Event and object reconstruction at CMS}


\subsubsection{Data formats}
\subsubsection{Particle-Flow algorithm}
\subsubsection{Tracks and vertices}
\subsubsection{Jets and b-jets}
\subsubsection{Leptons}
\subsubsection{Missing transverse energy}
\subsubsection{Photons}

