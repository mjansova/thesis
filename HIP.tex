\chapterwithnum{Study of highly ionizing particles in the strip tracker}

\section{Tracking inefficiencies at beginning of RunII (or clustering, to be decided)}

\subsection{Observed inefficiencies in tracks reconstruction}
\subsection{Highly ionizing particles as possible explanation}
        (how the nuclear event looks like (inelastic, most energy from recoil))


\section{Strip tracker readout system}

\subsection{Overview}
       (give overview about the readout - how the signal goes)
\subsection{Silicon strip modules}
       (their description and energy loss mechanism)
%TODO check the pitch size

The CMS silicon sensors are formed by n-type bulk, which has on one side uniform n+ implant while on the other p+ strips are located. The implants are connected to reverse bias voltage to completly deplete the bulk of the sensor. The thickness of both p+ and n+ implants is small and negligible compared to the bulk, thus almost whole volume of sensor is depleted. Every p+ strip are connected by a wire bond to a read-out electronics.

%is true 4-6 sensors?, or one sensor with more or less strips?
The tracker modules are consisted by 4-6 sensors. The larger part of modules have one layer of sensors~(mono modules), the other holds two layers of sensors, which are attached back to back and with a strip inclination of $5.7^{\circ}$ against each other~(stereo modules). Thus the stereo modules are able to give 2-D information about the position where particle hitted the module~(hit position). The modules also differ by the pitch size between each strip which can vary from 80 $\mu$m up to 200 $\mu$m depending on the tracker layer and partition.

The particle crossing silicon sensor is leaving energy predominantly via electromagnetic interaction - by ionization of the silicon volume, the electron-hole pairs are produced along the path of a particle. The energy loss in the material can be described by the Bethe-Bloch formula~\cite{Groom:2000sm} as a function of $\beta\gamma = p/Mc$, where $p$ and $M$ is mometum and mass of the interacting particle. The Bethe-Bloch function has a minimum for $\beta\gamma \approx 3$. Majority of relativistc particles are having this minimal value of $\beta\gamma$ and thus they are called be Minimum Ionizing Particles~(MIP).

Under normal circumstances the created charge carriers, electrons and holes, would drift on opposite sides directly towards electrods (n+ and p+ implants). But as in the barrels case a perpendicular magnetic field is present, the charge carrier $q$ is deflected from the direction of electric field due to the Lorentz force

\eq{LorentzEquation}
{
    q(E+v \times B).
}

The charge collected at strips is read by APV25 chip which is glued to the module. As one APV chip is reading 128 strips, 4-6 chips are present at one module.

-charge collection
-cross talk
-analog signal read from all channels
\subsection{The APV25 readout chip}
\subsection{The Front End Driver}
-FED description
-pedestals,baselines
\subsection{Offline data treatement}
-tracking -connecting hiots to tracks
