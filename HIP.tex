\chapterwithnum{Study of highly ionizing particles in the strip tracker}

\section{Tracking inefficiencies at beginning of RunII (or clustering, to be decided)}

\subsection{Observed inefficiencies in tracks reconstruction}
TODO
%-TOBl1 affected the most
%TODO plots from my group presentation: /home/mjansova/Downloads/presentations/thesisDirectory/literature_HIP/my_old_presentations/groupPresentation_v2.pdf

\subsection{Highly ionizing particles as possible explanation}

In the CMS environment the main mechanism of energy loss of charged particles~(except of electrons) in silicon sensors is ionization. Beyond ionization the traversing particle energy can be deposited in sensor via elastic and inelastic nulear interaction with the siclicon nucleus. These interaction result in nuclear recoil and nuclear fragmentation in case of inelastic interaction. A sufficiently energetic recoiled nucleus can induce displacement of other nuclei in its proximity. All affected nucleai as well as the fragments undergo energy loss by ionization, resulting in very localized large energy depositions in the silicon volume. By simulation of elastic and inelastic interactions in the silicon it has been shown that elastic interactions do not lead to high energy depositions, while inelastic interactions can deposit energy up to $\sim$100MeV in tracker sensors, which is $\sim$1000 times larger energy deposit than by ionization only~\cite{Huhtinen:2002yda}. The readout electronics of tracker modules is not designed for such large energy deposits resulting in saturation of electronics and a deadtime in charge collection and thus loss of hits.

\section{Strip tracker readout system}

\subsection{Overview}

    \insertFigure{figures/dataFlow} % Filename = label
                 {0.6}       % Width, in fraction of the whole page width
                 {Overview of the readout chain taken from~\cite{Bainbridge:2004jc}.} % Caption

Particles passing through the modules of silicon strip tracker are loosing their energy mainly via ionization of silicon volume. The movement of the charge carriers toward electords is inducing current on the silicon strips, which is read by the APV25~\cite{French:2001xb} chips glued to the module. The analogue signal from these chips is sent via optical link to the front-end drivers~(FED)~\cite{Baird:2002wg} located at control room. In standard configuration, in FEDs the data are digitalized, processed and reduced. The graphical overview of this data flow is shown in figure~\ref{fig:figures/dataFlow}

\subsection{Silicon strip modules}
%TODO check the pitch size

The CMS silicon sensors are formed by n-type bulk, which has on one side uniform n+ implant while on the other p+ strips are located. The implants are connected to reverse bias voltage to completly deplete the bulk of the sensor. The thickness of both p+ and n+ implants is small and negligible compared to the bulk, thus almost whole volume of sensor is depleted. Every p+ strip are connected by a wire bond to a read-out electronics. The distance between strips is called a pitch.

%TODO is true 4-6 sensors?, or one sensor with more or less strips?
The tracker modules are consisted by 1-2 sensors, each sensor is having three or four times 128 p+ strips. The larger part of modules have one layer of sensors~(mono modules), the other holds two layers of sensors, which are attached back to back and with a strip inclination of $5.7^{\circ}$ against each other~(stereo modules). Thus the stereo modules are able to give 3-D information in global coordinates about the position where particle has hit the module~(hit position). The modules also differ by the pitch size between each strip which can vary from 80 $\mu$m up to 205 $\mu$m depending on the tracker layer and partition.

The particle crossing silicon sensor is leaving energy predominantly via electromagnetic interaction - by ionization of the silicon volume, the electron-hole pairs are produced along the path of a particle. The energy loss in the material can be described by the Bethe-Bloch formula~\cite{Groom:2000sm} as a function of $\beta\gamma = p/Mc$, where $p$ and $M$ is mometum and mass of the interacting particle. The Bethe-Bloch function has a minimum for $\beta\gamma \approx 3$. Majority of relativistc particles are having this minimal value of $\beta\gamma$ and thus they are called be Minimum Ionizing Particles~(MIP).


Signal starts to be induced on strips once electrons and holes begin to drift towards the electrodes - holes to strips and electrons to back-plane. The charge induced at electrodes can be calculated using Shockley Ramo theorem~\cite{doi:10.1063/1.1710367,Ramo:1939vr}. In framework of this teorem it can be shown~\cite{Bloch:2007zza} that charges drifting toward one strip induce charge on the neighboring strips as well, but the total charge on the neighboring strips will integrate to zero in case that at the end charge carriers are collected at one strip and are not trapped by other strips.

The number of strips reading signal left by a particle depends on charge sharing and electronics crosstalk. Charge sharing is srongly dependent on the particle trajectory position and can be caused by difussion, trajectory inclination or Lorentz angle. While electronics crosstalk is independent of particle trajectory and it is result of the strip coupling via inter-strip capacitance. 

The charge can be induced on more strips also because of inclination of trajectory. Then towards different strips the charge carriers from different parts of trajectory are drifting.

Under normal circumstances the created charge carriers, electrons and holes, would drift on opposite sides directly towards electrods (n+ and p+ implants). But as in the barrels case a perpendicular magnetic field is present, the charge carrier $q$ is deflected from the direction of electric field due to the Lorentz force

\eq{LorentzEquation}
{
    q(E+v \times B).
}

The angle between the electric field lines and drift direction in magnetic field is called Lorentz angle. This angle is independent of track inclination and can be compensated by tilting the modules.

The diffusion, caused by collisions of charge carriers with each other, can also modify the path of the charge carriers from the straight drift following the fieled lines. Hovewer this effect is very small and can result only in small amout of charge collected by neighboring strips just in case the particle has passed close the middle of the pitch.


Electronics crosstalk arises from the readout capacitive network -- strips are coupled via capacitance to backplaine and to neighboring strips via inter-strip capacitance. Because of the inter-strip capacitance the collected charge by one channel is shared with the neighboring channels. The coupling of neighboring strips depend strongly on the time~\cite{Bloch:2007zza} thus the coupling constants are different for peak and deconvolution mode, as in deconvolution mode there are 3 sampling times. This effect can be studied numerically via description of the capacitive network by SPICE simulations~\cite{Barberis:1993ph}.


%Chare sharing can be measured from eta function (response function) -eta function R/(L+R), two separate peaks at 0 and 1 if no charge sharing. Shoft because of electronic coupling. Width of the peak determined by the noise. Almost linear charge sharing outside of the peak (plateau) - amount of charge collected by a strip is inversly proportional to the distance of the impact point from that strip (linear charge sharing) - for perpendicular tracks negligible.


\subsection{The APV25 readout chip \label{sec:APV}}


    \insertFigure{figures/APVreadout} % Filename = label
                 {0.6}       % Width, in fraction of the whole page width
                 {The schema of the APV25 chip taken from~\cite{Friedl:2001kra}.} % Caption

The charge collected by each channel~(strip) is read by APV25 chip located at the module. The APV is front-end chip providing amplification and shaping of signal from each channel separately. To achive this all APV chips are equipped by preamplifier, CR-RC shaper, analog pipeline and deconvolution filter. The block diagram of APV chip is shown in figure~\ref{fig:figures/APVreadout} As one APV chip is reading signal from 128 strips, 4-6 chips are present at each module.

The amplified signal is sent to CR-RC shaper to convert strip signal into $\sim$50~ns long voltage pulses. The shaper is providing an output with frequecy of bunch crossings. In case of APV working in the ``peak mode'' the sampled signal at maximum of the pulse shape corresponding to given bunch crossing, is used directly, but usually the APV is operating in ``deconvolution mode'' to reduce the out-of-time pile-up. In the deconvolution mode the weighted sum of the shaper output from three consecutive bunch crossing is used instead.To have a possiblity to optimize the puls shape, the feedback resistors of both preamplifier and shaper as well as  bias current and voltage are fully programable. For the calibration and test of the chip the internal calibration circuit is present. This circuit enables to inject charge to channels prior to the preamplifier stage.

The puls heights and bunch crossing information for all 128 channels are extracted at the end of the analog pipeline upon the request of trigger. The average signal level from 128 channels can be adjusted within the dynamic range of the APV, in order to reduce the signals exceeding the APV range. The processed signal from two APV chips is multiplexed by APVMUX~\cite{Ball:2007zza} into a single line and converted by laser from an electrical to optical signal, which is sent via analog optical fibre to control room. At the control room the optical signal is recevied by the pin diod which is a part of FED.

-TODO write sth about inverter
%-tickmarck sent every 70 clock cycles when no data are qued for output -  used for synchronization betwwen frontend and backend electronics.

\subsection{The Front End Driver}



The FED is recieving data from 96 optical fibers, each sending information from 2 APVs. The data in form of optical signal are converted to electrical signal, they are reordered and synchronized. For each APV input the signals per channel are extracted and digitized into 10-bit range AD counts~(ADC). The output signal for given channel is reffered as digi. In the standard operation mode, the ``Zero Supression~(ZS)'', the pedestal subtraction followed by common mode noise~(CMN or baseline) subtraction is performed. Pedestal is mean strip activity for given strip when no particle is present, which is evaluated from special ``pedestal runs'' taken few times per year. After pedestal subtraction the CMN is the remining noise common to all channels~(electronics, power supplies origin) and calculated on event by event basis as a median over the 128 strips. After both subtractions the channels with values lower than zero are truncated to zero. For all the channels the signal-to-noise ratio~(S/N) is checked separately and if S/N of the channel is smaller than five or S/N of the channel and its neighbor is smaller than two, the ADC of these channels is set to zero~(zero suppressed). Morover the the ADC range is truncated to 8 bits. No change is applied for charges lower than 254 ADC, charges between 254 ADC and 1022 ADC, included are set to 254 ADC and charges aceeding 1022 ADC are stored as 255 ADC. Only information about strips with non-zero ADC values are sent to the CMS data acqusition system~(DAQ). By this procedure the available data are reduced by factor $\sim$60 in order not to overload DAQ system.

For testing purposes the FED is able to operate in others than ZS mode. In case of the ``Virgin Raw~(VR)'' data taking mode no subtractions and suppressions are applied and thus it is suitable mode e.g. for studies of the APV behavior.

In the ideal case when constant current is injected, the output signal should be constant. But is is not the case because of the random fluctuations called elctronics noise. The silicon strips sensors has two sources of electronics noise -- voltage and current sources. These two sources can be induced by either variations of velocity (thermal noise) or by fluctuating number of charge carriers (shot noise)~\cite{website:noise}. Usually the largest noise comes from amplification of signal. 

The noise can be correlated between channels - like in case of CMN, but also due to electronics coupling. Anti-correlation~\cite{Lutz:1987wd} of noise between neighboring channels originating from the inter-strip capacittance has been observed. As the total charge must be conserved, in case of upward fluctuation on one strip, the downward fluctuation must occure on neighbor strips leading to anti-correlation of noise between neighboring channels.

\subsection{Offline data treatement}

    \insertFigure{figures/event2layer4} % Filename = label
                 {0.6}       % Width, in fraction of the whole page width
                 {Example of raw and pedestal subtracted data, baselines and clusters for one of the tracker modules.} % Caption

Offline, the clustering procedure is applied on the ZS channels. The default clustering algorithm is called ``Three threshold algorithm'', posing threshold on seed strip~(strip with the largest signal), neighboring strips and cluster charge in terms of signal-to-noise ratio. Cluster seed must pass requirement S/N>3, adjacent stips can be added in case S/N>2. On top of those requirement the total cluster charge must be five times larger than total cluster noise $\sigma_{total}$ which is defined as

\eq{noiseEquation}
{
    \sigma_{cluster} = \sqrt{\sum_{i} \sigma_{i}^{2}},
}

where $\sigma_{i}$ is noise of channel $i$. Example of data from one module undergoing zero supression and clustering is hown in figure~\ref{fig:figures/event2layer4}.

%-gains %https://github.com/cms-sw/cmssw/blob/09c3fce6626f70fd04223e7dacebf0b485f73f54/RecoLocalTracker/SiStripClusterizer/src/ThreeThresholdAlgorithm.cc
During clustering the strip charge is corrected by two calibration -- tickmark gain~(G1) and particle gain~(G2). The tickmark gain is correcting the signal for the transmission losses, mainly for the losses caused by transmission of signal via $\sim$100m long optical fibers. The tickmark is a signal injected to the calibration circuit and from change of its height at readout output the gain per APV can be computed. The other purpose of tickmarks is synchronization of the APVs to central trigger.

The particle gain is coorrecting for differences at the sensor level. This gain is determined from the inoization of silicon sensitive volume per unit of lengt by particle traversing the sensor. The MPV value of the ionization per unit of lenght is then used to equlize the response of different sensors to the MIP charge. 

These calibrations need to be determined frequently as they change because of the aging of the detector orchange of operating conditions~(e.g. temperature).

The final position of hit is obtained charge-weighted positions of channels in cluster, corrected for Lorentz drift in case of TIB and TOB. Additional correction is applied due to the inefficiency of collection of charge deposited close to back-plane.

The reconstructed hits are used for tracks reconstruction~\cite{Chatrchyan:2014fea}, utilizing software reffered as Combinatorial Track Finder~(CTF), based on combinatorial Kalman filter~\cite{Fruhwirth:1987fm}. The CMS tracking is using iterative approach -- first the track the easiest to find are reconstructed (e.g. high $p_{T}$ ones ) and its hits are masked in order to reduce combinatorics for further iterations of tracks finding.


The track reconstruction is performed in 4 steps. First the \textit{track seed} is built from two or three 3-D hits. Then during the \textit{track finding} the track is propagated to neighboring layers of tracker, testing the compatibility of hit with the track by $\chi^{2}$ test. Once the track candidate is complete, the \textit{track fitting} is performed to obtain the best parameters of the track. The last stage is \textit{track quallity selection} when it is tested if the final fit resulted in good $\chi^{2}$, if there are enough layers with hit associated to the track and if the track is originating from the primary vertex.

-tracking cosmics?
-how is it with missing hits? -track is only lost if two consecutive hits are missing? Can differ but usually should be 1 missing hit per track maximum -  in note CMS-TRK-11-001
-define cluster width and cluster charge

\section{The impact of highly ionizing particles on APV25 chip}

%energy spectra of heavy fragments produced in silicon are insensitive to energy of incident particle and do not go further than 10MeV, energy loss for such fragments are of order$MeV\mum^{-1}$~thus the fragments can go up to 100 $\mum$ (compare with sensor thickness) - very localized depositions. The light particles from nuclear interactions can travel longer and also contribute significantly to the total energy depositied~\cite{Huhtinen:2000nk}.


\subsection{Studies prior to the LHC data taking~\label{sec:HIPinPast}}

The effect of Highly Ionizing particles has been studied in the past during beam tests -- PSI and CERN X5. Large energy depositions impact on electronics has also been studied by injection of charge to calibracton circuit of APV or by laser tests~\cite{Adam:2005pz}. In this section I will only describe results of the PSI beam test.

\subsubsection{Experimental setup}

The M1 beamline at PSI provided continuous beam of protons and pions with bunch crossing every 20~ns. For the APV studies the beam was tuned to pion momentum of 300~MeV to best mimic conditions at CMS environment. The tracking system was consicted of 12 layers of tracker modules (3xTIB,3xTEC,6xTOB), but for the study of HIP events only TOB modules were used. These TOB modules had 500~$\mu m$ thick sensors with strip pitch of 183~$\mu$m and were equipped by inverter resistors of either 50, 75 or 100 $\Omega$ . Special trigger burst and APV setting was used to trigger HIP event and then provide 29 more events every 25 ns -- resulting in data every 25~ns for 750~ns. All modules were operated in the peak mode and taken data were equivalent to the CMS VR data format. 


%-measured probability per pion per sensor plane ins lower than 10-3

%-sensors 320 or 500 mum

\subsubsection{Response of the APV25 chip to the HIP events}

As discussed above the highly ionizing particles deposit large energies up to equivalent of ~1000 MIPs and thus saturate the APV chips. The affected channels collect charge beyond the range of APV which is few tens of MIPs. The remaining channels from 128-channel APV are shifted towards the low part of the APV range and even exceed it. This behavior is result of cross talk effect on the APV chip worsened by the biasing scheme powering inverter and shaper which ensures stable CMN and during normal conditions~\cite{Bainbridge:2004jc}. The CMN distribution can be seen at figure~\ref{fig:figures/CMNandRMSrawPast} in the bottom plot. The CMN distribution is peaking around 0~ADC in the standard consitions. Smaller peak around $\sim$-100~ADC comes from suppressed baselines which are result of HIP events. 

Because of the shift of CMN the HIP events can be easilly recognised by posing requirement on value of the CMN, in case of this study the APV chip which exhibited CMN$\leq$-20 ADC during 750~ns trigger burst was tagged as containinga  HIP candidate. The response of the APV chip on the HIP event is shown in figure~\ref{fig:figures/thesisEvolution}. In the plot there are pedestal subtracted data from 6 consecutive TOB modules in 4 different times. In the top-left plot~($T_{event}$ = 300~ns) the first evidence for HIP event -- large signal peak and small shift towards low values of range of the other channels, can be observed in the second module from bottom, second APV. After 50~ns~($T_{event}$ = 350~ns) the channels of APV not collectiong signal are suppresed and thus the peak is fully revealed. The suppresion and large peak can be still observed at $T_{event}$ = 525~ns. At $T_{event}$ = 575~ns the channels start to recover to their initial position. At $T_{event}$ = 525~ns and $T_{event}$ = 300~ns the MIP signal the MIP particle was passing through all six layers of modules. The signal is observed in relative APVs of all layers except of the APV affected by HIP event. The time period, when APV is insensitive to MIP signal is reffered as deadtime.

In the this analysis the selected HIP event satisfies selection CMN$\leq$-20 and HIP cluster seed charge larger than 125~ADC.  In the discussed figure~\ref{fig:figures/thesisEvolution} HIP event is associated with event at $T_{event}$ = 350~ns, but as shown at the exaple, the signal from actual HIP interaction occured and was observed already 50~ns before the selected HIP event.

    \insertFigure{figures/CMNandRMSrawPast} % Filename = label
                 {0.6}       % Width, in fraction of the whole page width
                 {(top) RMS spread of raw data~($\sigma_{raw}$. (bottom) CMN distribution~\cite{Bainbridge:2004jc}.} % Caption


    \insertFigure{figures/thesisEvolution} % Filename = label
                 {0.6}       % Width, in fraction of the whole page width
                 {Example of the APV behavior as a response on the HIP event in time. Pedestal subtracted data of six layers of TOB modules in four timestamps are plotted~\cite{Bainbridge:2004jc}.} % Caption


%Fully suppressed baselines remain suppressed for around 250ns

\subsubsection{Deadtime induced by HIP events}

The HIP event is supressing all channels from the affected APV which are not collecting signal. Large HIP signals result in full suppression of the channels beyond the lower limit of the APV range. The APV with fully supppressed channels exhibit very small RMS spread~($\sigma_{raw}$) of the data before pedestal subtraction~(raw data), if excluding the channels reading signal. The RMS spread of raw data is shown in top aprt of figure~\ref{fig:figures/CMNandRMSrawPast} where the large peak population are normal operation with value around $\sim$1.6~ADC which is exhibiting spread of pedestals. The smaller peak population with RSM spread <1~ADC are fully supressed baselines not anymore sensitive to the pedestal spread and the tail is populated by data with distorted baselines usually originating from the HIP event. Distorted baselines are result of non-uniform supression and recovery of baseline in response on HIP event.

Based on these observations two event categories were defined. Events with ``fully suppressed'' baselines  satisfy $\sigma_{raw}< 1~ADC$. ``Partially suppressed'' baselines are required to to have $\sigma_{raw}\geq$1~ADC and CMN$\leq$-20~ADC.



    \insertFigure{figures/tableDeadtimes} % Filename = label
                 {0.6}       % Width, in fraction of the whole page width
                 {The mean~($\Gamma_{mean}$) and maximum~($\Gamma_{max}$) deadtime induced by HIP events for ``fully suppressed''~($\sigma_{raw}<1~ADC$) baselines. The deadtimes were evaluated for two module geometries as well two inverter resistor values~\cite{Bainbridge:2004jc}.} % Caption

The deadtime of the APV is evaluated in terms of hit efficinecy $\epsilon_{hit}$ for both APV influence by HIP event $\epsilon_{hit}^{HIP}$ and efficent APVS~(not influenced by HIP event) $\epsilon_{hit}^{good}$. The hit efficiency is defined as $\epsilon_{hit} = N_{hit}/N_{tracks}$, where $N_{hit}$ is number of clusters reconstructed in APV around the track intercetp point, and $N_track$ is number of reconstructed tracks traversing the APV. The deadtime is then time interval during which APV is not fully efficient. The averaged deadtimes for ``fully suppressed'' baselines APVS are shown in table~\ref{fig:tableDeadtimes} for both inverter resistor values of 50~$\Omega$ and 100~$\Omega$. The ``partially suppressed '' baselines APV exhibit much smaller deadtime compared to the previous case, typically in order of few tens of ns. In the table~\ref{table:deadtimesPast} it can be noticed, that the decreased resistor value isi significantly decreasing deadtime. Although decreasing resistor value has its disadvantage wich is enhancement of baseline distortions, which lead to reconstruction of ``fake'' clusters disscused in more details in \cite{Bainbridge:2004jc} and section~\ref{section:fakeClusters}.


%plots
%-str 81 fig 3.5

\subsection{Probability of HIP event.}

In the PSI beam test also the measurement of HIP probability was provided. The HIP probability~($P_{HIP}(CMN_{HIP}\leq CMN_{threshold})$) is defined as 

\eq{HIPprob}
{
P_{HIP}(CMN_{HIP}\leq CMN_{threshold}) = \frac{N_{HIP}(CMN_{HIP}\leq CMN_{threshold})}{N_path},
}

where $N_{HIP}(CMN_{HIP}\leq CMN_{threshold})$ is number of HIP events selected HIP events with CMN value~($CMN_{HIP}$) lower than threshold~$CMN_{threshold}$ and $N_{path}$ is number of tracks traversing the sensor.

The measurements were provided with pion beam of energy $300~MeV$, which is the most probable energy value of pions in CMS tracker. The measured probability for different modules and $CMN_{thteshold}=-20$ was found out to be of order $10^{-3}$. It was concluded that the HIP probability does not scale with beam intensity, but it scales with sensor thickness. Also the HIP probability was lowered by changing the inverter resistor value from $100~\Omega$ to  $50~\Omega$. Similar measurement was provided as well with proton beam of momentum $300~MeV/c$, which is not compatible with CMS consitions.


\section{Studies of the HIP events with CMS pp collision data}

\subsection{Strategy of the HIP studies}

As explained in the section~\ref{sec:APV} in the standard data-taking the Zero Suppresson mode is used. During the ZS procedure all negative channels after pedestal subtraction are truncated to zero, thus this mode is not suitable to study the HIP events, which are known for causing a drop of the CMN as described in section~\ref{sec:HIPinPast}. The solution is to use data taken in Virgin Raw mode, what comes with a cost of increased event size. In the VR data-taking no subtractions and supressions are done, and if needed the ZS can be done offline to be able to compute CMN and proceed with clustering and further data treatement. In the following analyses the CMN has been computed from all 128 raw digis after pedestal subtraction as a median over 128 strips. The rms spread~($\sigma_{rms}$) has been computed from the raw digis of 80\% of the 128 channels which have the lowest ADC value, to avoid the clusters. But the ZS and clustering has been perforemed as in standatrd data-taking (truncation to zero, truncation of digis to 8 bits) to mimic the standard data-taking output.

The strategy of further presented studies is to select a HIP event and study the influence of such events on electronics and clustering. As seen in previous studies in section~\ref{sec:HIPinPast}, the CMN recovery as well as observed deadtime is in order of O(100)~ns. Thus to be able to study evolution of CMN and cluster properties, consecutive events from window of few hunderds of ns are needed. However the probability of closely spaced events is very low without special trigger configuration. The possibilities for new trigger configuration are very limited because of increased size of events during the VR data-taking by factor $\sim$~35 compared to standard ZS data run. Moreover not to overload CMS data acquisition system following trigger criteria on number of triggers in number of bunch crossings~(bx) are  imposed~\cite{website:VRtrigger}:

\begin{itemize}
\item{No more than 1 trigger in 3 bx}
\item{No more than 2 triggers in 24 bx}
\item{No more than 3 triggers in 100 bx}
\item{No more than 4 triggers in 240 bx}
\end{itemize}

In order not to overload the acquistion system, the triggered events have to be spread over many datasets in a way that consecutive events are in different datasets. Because of this limitation, the sorting and ordering of events had to be performed before the study.

In the following sections of this chapter, for simplicity, are shown only plots for first layer of TOB, which has exhibited the largest drop in the hit efficiency. Although the fraction of HIP events for given APV defined as


\eq{HIPfrac}
{
f_{HIP} = \frac{N_{HIP}}{N_{all}},
}

where $N_{HIP}$ is number of selected HIP events, $N_{all}$ is number of all events, is computed for all partitions and layers of the tracker. (Dependence on the inst lumi)

%To be able to run the study in reasonable time and with reasonable amount of resources, the reduction of the data was done based on 

\subsection{Study of HIP events as possible cause of cluster inefficiencies}

\subsubsection{Motivation}

TODO

Inefficiencies - look in the data for possible explanation - HIP
look if there is a HIP effect at the CMS conditions, characterize it, look for its imapct and characterize its frequency.
VR studies: one of the studies which were done

\subsubsection{Experimental setup} 

This study is based on VR data from 12 of April 2016, CMS run 273162, LHC fill 4915, with only silicon strip tracker included in the run. This run was 23 minutes 33 seconds long with initial instantenous luminosity 1548$\times 10^{30} cm^{-2} sec^{-1}$, and end instantenous luminosity 1538$\times 10^{30} cm^{-2} sec^{-1}$. During this run all APVs were operated in deconvolution mode. The LHC delivered beams with 601 bunches each, 589 pairs of bunches collided at CMS. The avarage pile-up (interactions per bx) for this fill was 26. The beams were mainly composed of 72 bunches long trains, in which bunches were spaced in a way that they can be collided every 25~ns.

Closely spaced events were enriched by using special trigger configuration, which forced first bunch crossing in fixed train to be triggered and then wait for the next orbit for the same train. After this trigger two other bunch crossings in the same train were triggered randomly. The final number of triggered events as function of bunch crossing is shown in figure~\ref{fig:figure\triggerStudy1}. The shape of the trigger distribution is given by trigger rules as well as forced trigger on first bx in the train.

%-trigger rules: %https://indico.cern.ch/event/512685/contributions/2167961/attachments/1273330/1887985/virgin_raw_test_2016_ebutz.pdf

%TODO figure trigger

 \subsubsection{Methodology}
 
%HIP in module

In section~\ref{sec:HIPinPast} it has been observed that HIP event can be identified via low value of baseline. Applying this approach to the CMS data an event like the one shown in figure~\ref{fig:figure\peakinmodule} has been selected. In this example the typical effect of the HIP event on the chip can be seen - negative baseline, large signal on few channels and low rms spread of raw digis. But in many cases the large signal has not been observed as shown in figure~\ref{fig:figure\nopeakinmodule} contradictory to what has been observed in study in section~\ref{sec:HIPinPast}. This can be explained by different mode of the APV, in this case deconvolution compared to the peak mode which has been used for HIP studes at PSI. During studies of the behavior of the APV chip when injecting large charge to the calibration circuit~\cite{Bainbridge:2002bda} it has been observed that in case that APV is operating in the deconvolution mode, the large signals can be observed only for few ns and the affected channels are also driven to the low range of the APV. 

%TODO figure module with APV with saturated baseline and peak
%TODO figure module with APV with saturated baseline and without  peak

Reliable selection of the APVs influenced by the HIP event can be designed via presence of ``fully saturated'' baselines. Analyzing correlation of CMN and $\sigma_{raw}$ in figure~\ref{fog:figure/baselinevsRMSrawFirst} it is obvious that the standard events with nominal value of the baseline around 128~ADC are having $\sigma_{raw}$ of order of few units. The second large population with small value of baseline and $\sigma_{raw}$ is the one of fully saturated baselines, whichcan be connected with large energy deposits in the APV chip. Based on the knowledge of the correlation between CMN and $\sigma_{raw}$ the selection of HIP event has been chosen to be 

\eq{selection}
{
$CMN<-25~ADC \mathrm{and} \sigma_{raw}<2.5$.
}

%TODO figure vs RMS

It can be observed that there are many events with negative value of baseline, but large value of $\sigma_{raw}$. These events can originate from the baseline drop, baseline recovery and also large energy deposits, but not large enough to fully saturate baseline. In order not to mix different populations, these partially saturated events are not selected as HIP events. It has also been observed that the full saturation can last for several bunch crossings what will be shown in section~\ref{sec:limitationsSelection} and consequently more events in a row can be selected as HIP. In this case the first event of the possible are selected as a HIP. Also the HIP event does not have to be selected at the real HIP event has occured in the sensor as the saturation of baseline is consequence of the HIP event. Other possibilites how to select HIP event will be discussed in section~\ref{sec:limitationsSelection}.

The analysis of the APVs influenced by the HIP event has been performed statistically. When the HIP event has been selected it has been set to occure during bx=0, the bunch crossings of two remaining events in the same train has been set relatively to the HIP event, e.g. when other event has been triggered five bx after selected HIP, its bx has been set to 5. Then the average information per each bx has been plotted. The average baseline distribution as a function of bx, and thus time before and after selected HIP, is show in figure~\ref{fig:figure\baselineFirst}. When tracking the baseline evolution in time, going from negative values of bx to positive, long before a HIP has occured the baseline shows stable value around 128~ADC. Shortly before the seelcted HIP~(bx=0) it starts to drop as a consecquence of the large energy deposition in the sensor. At bx=0, by definition, the baseline is saturated. The baseline recovers to normality in $\sim$15~bx and slightly overshoots for the remaining duration of the train.

A similar distribution as a function of bx is shown for $\sigma_{raw}$ in figure~\ref{fig:figure\rmsFirst}. Long before the HIP event the $\sigma_{raw}$ is stable with value around 8. Right before the selected HIP the $\sigma_{raw}$ increases due to not uniform drop of the baseline. The $\sigma_{raw}$ recovers in around 10~bx up to which point part of baseline is fully saturated, but this population is mixed with distorted baselines which on the other hand have large $\sigma_{raw}$ and thus this part cannot be straighforwadly interpreted.


%iTODO plot baseline
%TODO plot rms
%The error bars of both plots


During the laser simulation of HIP events, during which a charge has been induced in sensor by laser, the recovery of baseline and signal in terms of S/N has been studied~\cite{Adam:2005pz}. It was shown in figure~\ref{fig:baselineAndSignalRecovery} that the signal recovers from the HIP event differently and in different speed than baseline. Thus to estimate the signal recovery and deadtime induced by HIP event it is necessary to study clusters.

    \insertFigure{figures/baselineAndSignalRecovery} % Filename = label
                 {0.6}       % Width, in fraction of the whole page width
                 {The recovery of the baseline and S/N (expressed as ratio of S/N to reference S/N) as a function of time. The evolution is shown for inverter resistor value of $50~\Omega$ and energy deposit of 25~MeV~\cite{Adam:2005pz}.} % Caption

\subsubsection{Results}

The average cluster multiplicity and the maximal cluster charge~(charge of the cluster with the largest cluster charge) per APV as a function of bx are shown in figures~\ref{fig:figure/avMultiplicityFirst}~and~\ref{fig:figure/maxChargeFirst}. In these distributions again the averaging over APVs is performed as well as aligning the selected HIP with bx=0. The avarege cluster multiplicity~\ref{fig:figure/avMultiplicityFirst} is stable for events long before the occurence of the HIP. Around 5 bunch crossings the multiplicity is growing as additional cluster belonging to the HIP deposit~(recoil or fragments) starts to appear. As a consequence of the HIP event the chip becomes inefficient in signal collection and already at bx=0, when the baselines are fully saturated, the cluster multiplicity is close to zero. The cluster multiplicity is recovered in the $\sim$10 bunch crossings. The average cluster multiplicity distribution for bx>10 is flat with a constant higher for BX<-10, what is in contradiction with expectaton. The maximal cluster charge per APV~\ref{fig:figure/maxChargeFirst} is exhibiting stable behaviour for bx<-20, then the increase in charge appear (TODO: why so early, is it effect of differet OOT?). The charge is maximal for the selected HIP and then drops and recoveres almost immediately after the selected HIP, but to lower level than before the HIP event, even though the same level as before the HIP event is expected. 

%The disagreements between distributions of bx$\ll$0 and bx$gg$0 for average cluster multiplicity and maximal cluster charge will be discussed in details in section~\ref{sec:}.

%TODO average cluster multiplicity
%TODO max cluster charge

The mismatch between cluster properties long before and long after the HIP event, as shown in figures~\ref{fig:figure/avMultiplicityFirst}~and~\ref{fig:figure/maxChargeFirst}, can be understood when analyzing the cluster charge distribution of all clusters for first event in the train in the figure~\ref{fig:figure/chargeFirstInTrain} and other events in the train~\ref{fig:figure/chargeFirstInTrain}. The cluster charge distribution for the first event in the train exhibits double peak structure of similar height with maxima around 100~ADC and 300~ADC, while in the cluster charge distribution of the other events, the height of the peak around 100~ADC is clearly dominating. The enhanced population of clusters around 100~ADC for the clusters from other than first bunch crossing in train is comming from out-of-time pile up, which is not present in the first bunch crossing. As in the train we have 3 events, first bunch crossing in the train and two others, one of them can be selected as a HIP event, but only first or second event can be set to bx<0 and on the other hand only second and third event can be set as bx>0. In consequence part of bx<0 of average cluster multiplicity and maximal cluster charge at figures~\ref{fig:figure/avMultiplicityFirst}~and~\ref{fig:figure/maxChargeFirst} is dominated by population with lower out-of-time pile-up than the part with bx>0. Thus as for bx<0 there is a lower out-of-time pile-up, the average cluster multiplicity is lower and maximal charge higher than for bx>0. To avoid not equal mixing of events with diferrent properties, the first event in the train has been removed from the distributions. The average cluster multiplicity without first bunch crossing in the train is shown in figure~\ref{fig:figure/avMultiplicityCleanedFirst} and maximal charge~\ref{fig:figure/maxChargeCleanedFirst}. In both distributions the removal of first bunch crossing lead to the significant equalization between the levels of bx$\ll$0 and bx$\gg$0 .

%TODO cluster charge for the first event in train
%TODO cluster charge distribution for other eventsin the train
%iTODO corrected cluster charge 
%TODO corrected cluster multiplicity

The average deadtime for the modules of the first layer of TOB can be estimated from figure~\ref{fig:figure/avMultiplicityCleanedFirst}. The deadtime is the time interval between the selected event and full recovery of average cluster multiplicity which appears to be $\sim$250~ns~(10~bx). The recovery of the cluster multiplicity does not imply the full recovery charge collection. The recovery of charge collection should be deduced from the recovery of the maximal clutster charge shown in figure~\ref{fig:figure/maxChargeCleanedFirst}, however the recovery seems to be also immediate and no obvious trend is observed. This effect is most probably caused by mixing real and fake clusters and could be eliminated by using on-track clusters only.

The average fraction of HIP events for first layer of TOB, defined in equation~\ref{eq:HIPfrac}, was estimated to be .... . The computed fraction is dependend on the run instantenous luminosity and thus it is not probability of the HIP event as defined in equation~\ref{eq:HIPprob}. To estimate the HIP probability, the average number of reconstructed tracks per event per APV must be known. The average fraction of HIP events is biased by the used trigger by which, due to the first trigger rule, it is not possible to trigger on second and third bunch crossings in the train. The HIP events occuring in the first bx of the train and causing full saturation of baseline only in second and/or third bx of the train are never selected.
 

\subsubsection{Limitations of the study}

Several limitations of the study were already discussed in the text above. The mentioned limitations are the different fraction of out-of-time pile-up in different events, which was solved by removing the events from first bunch crossing from distributions. No tracking and thus both real and fake clusters are used, on top of that their fraction can change as a result of HIP event. There is an empty window in the triggered events caused by first trigger rule as seen in figure~\ref{fig:figure\triggerStudy1}, which leads to underestimation of fraction of HIP events.

%TODO did I miss something from the limitations?

Large limitation of this study represents the ambiguity of selection of HIP events. In the figure~\ref{fig:figure/RMSrawVSbx} is shown the $\sigma_{raw}$ per APV as a function of bunch crossing, where bx=0 is selected HIP event. In the bottom left part of the plot it is visible that there is large population of APVs with $\sigma_{raw}$<2.5 for bx>0, which corresponds to very large energy depositions keeping baseline fully saturated fo several bx. Due to this uncertainty it is impossible to determine the time of the HIP interaction in sensor and all distributions shown above with selected HIP at bx=0 areeffectively smeared. Also because of the full baseline saturation druning more bx the fraction of HIP events, when using selection on fully saturated beaselines, is lower during the first few events in the train than for the rest of train.

The improvement in the HIP selection has also been targeted in this study by trying to employ selection on the clusters. First it should be possible to target large charge deposits by tagging the saturated clusters. In the figure~\ref{fig:figure/fractionOfSaturatedClusters} can be  seen the fraction of APVs with saturated cluster as a function of bx, where bx=0 is selected HIP by criterum~\ref{eq:selection}. The fraction of saturated clusters is significantly higher only for bx=0, which is already selected HIP event and moreover as discussed at figure~\ref{fig:figure/avMultiplicityCleanedFirst}, the avaerage cluster multiplicity is very low for bx=0, so a requirement for the saturated cluster would result only in large reduction of statistics. Another possibility is to study maximal cluster width per APV as function of bx, shown in figure~\ref{fig:figure/largestClusterWidth} , where again bx=0 is selected HIP by criterum~\ref{eq:selection}. In the distribution, no obvious trend is observed. 

%TODO RMS_raw vs BX
%TODO fractionOfSaturatedClusters
%TODO largest cluster width

%-OOT HIP

\subsubsection{Change of the APV configuration settings}
          (The major source of the tracking inefficiencies)

%TODO  plot of the distribution of largest cluster charge in train (PBXvsCHall)
\subsection{Study of HIP events after change in APV settings}

\subsubsection{Motivation}

\subsubsection{Experimental setup}
        (trigger and filling scheme + differences to other study (no OOT, physical event only in first BX) )
-run 281604
-7 consecutive events
-3x3 isolated bunches -> no OOT
-physical bunch crossing only in 1st vent in train
-> we know when HIP event happened
-kind of reload of previous analysis
\subsubsection{Methodology}
        (selection, Evolution of baseline, cluster charge and cluster multiplicity in APV the chips affected by HIP)
-baseline evolution -> compare
-selection: baseline < -5 ADC, RMS of raw digis < 2.5
         1> Fake clusters
- average cluster charge and multiplicity - fake clusters with large charge
-table of cluster charge, mult, etc
\subsubsection{Results}
      (Fraction of HIP events in the data)

\subsubsection{Limitations of the study}
         1> Impossibility to study deadtime
- deadtime not really possible due to filling scheme

\section{Conclusion}




REMARKS
-in my case the fraction of HIP is also effected by pile-up?!
-computation how fake clusters affects tracking
-what is the inverter resistor value first it was 100 but now changes to 50~\cite{Gennai:2003as}
-change all past to present perfect
