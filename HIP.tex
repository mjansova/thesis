\chapterwithnum{Study of highly ionizing particles in the strip tracker}

\section{Tracking inefficiencies at beginning of RunII (or clustering, to be decided)}

\subsection{Observed inefficiencies in tracks reconstruction}
\subsection{Highly ionizing particles as possible explanation}
        (how the nuclear event looks like (inelastic, most energy from recoil))


\section{Strip tracker readout system}

\subsection{Overview}

Particles passing through the modules of silicon strip tracker are loosing their energy mainly via ionization of silicon volume. The movement of the charge carriers toward electords is inducing current on the silicon strips, which is read by the APV25~\cite{French:2001xb} chips glued to the module. The analogue signal from these chips is sent via optical link to the front-end drivers~(FED)~\cite{Baird:2002wg} located at control room. In standard configuration, in FEDs the data are digitalized, processed and reduced. 

\subsection{Silicon strip modules}
%TODO check the pitch size

The CMS silicon sensors are formed by n-type bulk, which has on one side uniform n+ implant while on the other p+ strips are located. The implants are connected to reverse bias voltage to completly deplete the bulk of the sensor. The thickness of both p+ and n+ implants is small and negligible compared to the bulk, thus almost whole volume of sensor is depleted. Every p+ strip are connected by a wire bond to a read-out electronics. The distance between strips is called a pitch.

%TODO is true 4-6 sensors?, or one sensor with more or less strips?
The tracker modules are consisted by 1-2 sensors, each sensor is having three or four times 128 p+ strips. The larger part of modules have one layer of sensors~(mono modules), the other holds two layers of sensors, which are attached back to back and with a strip inclination of $5.7^{\circ}$ against each other~(stereo modules). Thus the stereo modules are able to give 3-D information in global coordinates about the position where particle hitted the module~(hit position). The modules also differ by the pitch size between each strip which can vary from 80 $\mu$m up to 205 $\mu$m depending on the tracker layer and partition.

The particle crossing silicon sensor is leaving energy predominantly via electromagnetic interaction - by ionization of the silicon volume, the electron-hole pairs are produced along the path of a particle. The energy loss in the material can be described by the Bethe-Bloch formula~\cite{Groom:2000sm} as a function of $\beta\gamma = p/Mc$, where $p$ and $M$ is mometum and mass of the interacting particle. The Bethe-Bloch function has a minimum for $\beta\gamma \approx 3$. Majority of relativistc particles are having this minimal value of $\beta\gamma$ and thus they are called be Minimum Ionizing Particles~(MIP).


Signal starts to be induced on strips once electrons and holes begin to drift towards the electrodes - holes to strips and electrons to back-plane. The charge induced at electrodes can be calculated using Shockley Ramo theorem~\cite{doi:10.1063/1.1710367,Ramo:1939vr}. In framework of this teorem it can be shown~\cite{Bloch:2007zza} that charges drifting toward one strip induce charge on the neighboring strips as well, but the total charge on the neighboring strips will integrate to zero in case that at the end charge carriers are collected at one strip and are not trapped by other strips.

The number of strips reading signal left by a particle depends on charge sharing and electronics crosstalk. Charge sharing is srongly dependent on the particle trajectory position and can be caused by difussion, trajectory inclination or Lorentz angle. While electronics crosstalk is independent of particle trajectory and it is result of the strip coupling via inter-strip capacitance. 

The charge can be induced on more strips also because of inclination of trajectory. Then towards different strips the charge carriers from different parts of trajectory are drifting.

Under normal circumstances the created charge carriers, electrons and holes, would drift on opposite sides directly towards electrods (n+ and p+ implants). But as in the barrels case a perpendicular magnetic field is present, the charge carrier $q$ is deflected from the direction of electric field due to the Lorentz force

\eq{LorentzEquation}
{
    q(E+v \times B).
}

The angle between the electric field lines and drift direction in magnetic field is called Lorentz angle. This angle is independent of track inclination and can be compensated by tilting the modules.

The diffusion, caused by collisions of charge carriers with each other, can also modify the path of the charge carriers from the straight drift following the filed lines. Hovewer this effect is very small and can result only in small amout of charge collected by neighboring strips just in case the particle was passing close the middle of the pitch.


Electronics crosstalk arises from the readout capacitive network -- strips are coupled via capacitance to backplaine and to neighboring strips via inter-strip capacitance. Because of the inter-strip capacitance the collected charge by one channel is shared with the neighboring channels. The coupling of neighboring strips depend strongly on the time~\cite{Bloch:2007zza} thus the coupling constants are different for peak and deconvolution mode, as in deconvolution mode there are 3 sampling times. This effect can be studied numerically via description of the capacitive network by SPICE simulations~\cite{Barberis:1993ph}.


%Chare sharing can be measured from eta function (response function) -eta function R/(L+R), two separate peaks at 0 and 1 if no charge sharing. Shoft because of electronic coupling. Width of the peak determined by the noise. Almost linear charge sharing outside of the peak (plateau) - amount of charge collected by a strip is inversly proportional to the distance of the impact point from that strip (linear charge sharing) - for perpendicular tracks negligible.


\subsection{The APV25 readout chip}

The charge collected by each channel~(strip) is read by APV25 chip located at the module. The APV is front-end chip providing amplification and shaping of signal from each channel separately. To achive this all APV chips are equipped by preamplifier, CR-RC shaper, analog pipeline and deconvolution filter. As one APV chip is reading signal from 128 strips, 4-6 chips are present at each module.

The amplified signal is sent to CR-RC shaper to convert strip signal into $\sim$50~ns long voltage pulses. The shaper is providing an output with frequecy of bunch crossings. In case of APV working in the ``peak mode'' the sampled signal at maximum of the pulse shape corresponding to given bunch crossing, is used directly, but usually the APV is operating in ``deconvolution mode'' to reduce the out-of-time pile-up. In the deconvolution mode the weighted sum of the shaper output from three consecutive bunch crossing is used instead.To have a possiblity to optimize the puls shape, the feedback resistors of both preamplifier and shaper as well as  bias current and voltage are fully programable. For the calibration and test of the chip the internal calibration circuit is present. This circuit enables to inject charge to channels prior to the preamplifier stage.

The puls heights and bunch crossing information for all 128 channels are extracted at the end of the analog pipeline upon the request of trigger. The average signal level from 128 channels can be adjusted within the dynamic range of the APV, in order to reduce the signals exceeding the APV range. The processed signal from two APV chips is multiplexed by APVMUX~\cite{Ball:2007zza} into a single line and converted by laser from an electrical to optical signal, which is sent via analog optical fibre to control room. At the control room the optical signal is recevied by the pin diod which is a part of FED.

-write sth about inverter
%-tickmarck sent every 70 clock cycles when no data are qued for output -  used for synchronization betwwen frontend and backend electronics.

\subsection{The Front End Driver}

The FED is recieving data from 96 optical fibers, each sending information from 2 APVs. The data in form of optical signal are converted to electrical signal, they are reordered and synchronized. For each APV input the signals per channel are extracted and digitized into 10-bit range AD counts~(ADC). In the standard operation mode, the ``Zero Supression~(ZS)'', the pedestal subtraction followed by common mode noise~(CMN) subtraction is performed. Pedestal is mean strip activity for given strip when no particle is present, which is evaluated from special ``pedestal runs'' taken few times per year. After pedestal subtraction the CMN is the remining noise common to all channels~(electronics, power supplies origin) and calculated on event by event basis as a median over the 128 strips. After both subtractions the channels with values lower than zero are truncated to zero. For all the channels the signal-to-noise ratio~(S/N) is checked separately and if S/N of the channel is smaller than five or S/N of the channel and its neighbor is smaller than two, the ADC of these channels is set to zero~(zero suppressed). Morover the the ADC range id truncated to 8 bits. No change is applied for charges lower than 254 ADC, charges between 254 ADC and 1022 ADC, included are set to 254 ADC and charges aceeding 1022 ADC are stored as 255 ADC. Only information about strips with non-zero ADC values are sent to the CMS data acqusition system~(DAQ). By this procedure the available data are reduced by factor $\sim$60 in order not to overload DAQ system.

For testing purposes the FED is able to operate in others than ZS mode. In case of the ``Virgin Raw~(VR)'' data taking mode no subtractions and suppressions are applied and thus it is suitable mode e.g. for studies of the APV behavior.

In the ideal case when constant current is injected, the output signal should be constant. But is is not the case because of the random fluctuations called elctronics noise. The silicon strips sensors has two sources of electronics noise -- voltage and current sources. These two sources can be induced by either variations of velocity (thermal noise) or by fluctuating number of charge carriers (shot noise)~\cite{website:noise}. Usually the largest noise comes from amplification of signal. 

The noise can be correlated between channels - like in case of CMN, but also due to electronics coupling. Anti-correlation~\cite{Lutz:1987wd} of noise between neighboring channels originating from the inter-strip capacittance was observed. As the total charge must be conserved, in case of upward fluctuation on one strip, the downward fluctuation must occure on neighbor strips leading to anti-correlation of noise between neighboring channels.

\subsection{Offline data treatement}

Offline, the clustering procedure is applied on the ZS channels. The default clustering algorithm is called ``Three threshold algorithm'', posing threshold on seed strip~(strip with the largest signal), neighboring strips and cluster charge in terms of signal-to-noise ratio. Cluster seed must pass requirement S/N>3, adjacent stips can be added in case S/N>2. On top of those requirement the total cluster charge must be five times larger than total cluster noise $\sigma_{total}$ which is defined as

\eq{noiseEquation}
{
    \sigma_{cluster} = \sqrt{\sum_{i} \sigma_{i}^{2}},
}

where $\sigma_{i}$ is noise of channel $i$.

%-gains %https://github.com/cms-sw/cmssw/blob/09c3fce6626f70fd04223e7dacebf0b485f73f54/RecoLocalTracker/SiStripClusterizer/src/ThreeThresholdAlgorithm.cc
During clustering the strip charge is corrected by two calibration -- tickmark gain~(G1) and particle gain~(G2). The tickmark gain is correcting the signal for the transmission losses, mainly for the losses caused by transmission of signal via $\sim$100m long optical fibers. The tickmark is a signal injected to the calibration circuit and from change of its height at readout output the gain per APV can be computed. The other purpose of tickmarks is synchronization of the APVs to central trigger.

The particle gain is coorrecting for differences at the sensor level. This gain is determined from the inoization of silicon sensitive volume per unit of lengt by particle traversing the sensor. The MPV value of the ionization per unit of lenght is then used to equlize the response of different sensors to the MIP charge. 

These calibrations need to be determined frequently as they change because of the aging of the detector orchange of operating conditions~(e.g. temperature).


The final position of hit is obtained charge-weighted positions of channels in cluster, corrected for Lorentz drift in case of TIB and TOB. Additional correction is applied due to the inefficiency of collection of charge deposited close to back-plane.

The reconstructed hits are used for tracks reconstruction~\cite{Chatrchyan:2014fea}, utilizing software reffered as Combinatorial Track Finder~(CTF), based on combinatorial Kalman filter~\cite{Fruhwirth:1987fm}. The CMS tracking is using iterative approach -- first the track the easiest to find are reconstructed (e.g. high $p_{T}$ ones ) and its hits are masked in order to reduce combinatorics for further iterations of tracks finding.


The track reconstruction is performed in 4 steps. First the \textit{track seed} is built from two or three 3-D hits. Then during the \textit{track finding} the track is propagated to neighboring layers of tracker, testing the compatibility of hit with the track by $\chi^{2}$ test. Once the track candidate is complete, the \textit{track fitting} is performed to obtain the best parameters of the track. The last stage is \textit{track quallity selection} when it is tested if the final fit resulted in good $\chi^{2}$, if there are enough layers with hit associated to the track and if the track is originating from the primary vertex.

-tracking cosmics?
-how is it with missing hits? -track is only lost if two consecutive hits are missing? Can differ but usually should be 1 missing hit per track maximum -  in note CMS-TRK-11-001


\section{The impact of highly ionizing particles on APV25 chip}

\subsection{MIP vs HIP events (try to improve this title)}

       (what is the difference; how they look like; scale of deposit)


HIP energy deposition as several hunderds of MIP
inelastic interactions between hadrons and nuclei of silicon -> HIP recoiled nuclei (and fragments). HIP probability O(10-3) per incident hadron
saturation of APVs - deadtime
large signals for few channels? and shift(depression) of the other channels -> negative CMN - other APV chips not affected

many studies done  - Beam tests: PSI and CERN X5~\cite{Adam:2005pz}
injection of charge to calibration circuit, laser geberated charge test

signal truncation to 8 bit, plus pedestal and baseline subtraction - HIP deposition cannot be seen + saturation of the dynamic range itself

energy spectra of heavy fragments produced in silicon are insensitive to energy of incident particle and do not go further than 10MeV, energy loss for such fragments are of order$MeV\mum^{-1}$~thus the fragments can go up to 100 $\mum$ (compare with sensor thickness) - very localized depositions. The light particles from nuclear interactions can travel longer and also contribute significantly to the total energy depositied~\cite{Huhtinen:2000nk}.

From the simulations of nuclear interactions in TOb sensors - energy depostions up to $\sim$100MeV predicted~\cite{Adam:2005pz} -> as fragments can deposit only 10 MeV several particles must be involved in very high depositions

200MeV~400 MIPs

10MeV contributions most probable - they already saturate amplifying stages - instensitive to further signal -> deadtime for several hunderds of ns

suppresion of other channels of APV - crosstalk efect worsened by biasing scheme powering inverter and shaper. Which on the other hand removes external sources of CMN and -> stable baseline during normal MIPs~\cite{Bainbridge:2004jc} - shift of opposite plarity to signal, usually non uniform depression - baseline slopes.

HIP depositions can differ a lot in magnitude, thus the simulation is very diffiult to acheive


dead time+recovery time -> when chip is inefficient; only part of the signal is collected

elastic interactions lkower depositions - rare to induce deadtime



\subsection{Studies prior to the LHC data taking}

\subsubsection{Experimental setup}

PSI -  proton and pion beam of both charges with beam momentum 30MeV/c. 12 layers of detector modules (3xTIB,3xTEC,6xTOB).  Triggers - coincidence from photo-multipliers. Aditional trigger for HIP studies.
(Many options actually) Trigger burst, to study recovery after HIPi - data every 25 ns for 750ns.

Identification of HIP on basis of CMN shift - HIP probability from nr of events which induce CMN shift to the total nr of particles in the sensor - nr of tracks going through (Phip(CMNcut) = Nhip(CMN<CMNcut)/Ntrack)-> for this analysis CMN thershold =-40

CMN level expressed as measured CMN divided by lovest possible CMN of APV (=CMR) -> independence of bias setting of APV

-probability of HIP increases with thickness of sensor

measured probability per pion per sensor plane ins lower than 10-3

-looks liek only peak mode

-90 ADC corresponds to energy depositions of 6 MeV - not sure. Around 10MeV is needed to induce significant deadtime (and induce fully depressed baselines!!)

for the HIP analysis only tob modules used

Efficent APV definition - reconstructed cluster close to the former HIP cluster
In efficient case -CMN returned to normallity in 300 ns
In inefficient case - baseline fully supressed for up to 200ns, they remain inefficient for 700ns

Studies in CMS environment - the protons can release 10MeV via ionization what is 10 times more probable than through nuclear interaction
\subsubsection{Response of the APV25 chip to the HIP events}
\subsubsection{Deadtime induced by HIP events}

from the results, baseline can be fully depressed for 300ns and then there is an overshoot. Also the measurements suggest that the period when the baseline is fully suppressed is indicative of deadtime experienced by the chip.

because of CMN is computed as median , distorted baselione can induce lost of clusters or can induce fake clusters.

If CM<-90 which is likely to induce deadtiem: Deadtime 345+23ns(100ohm),210+15ns(50ohm)

for  320mum thick sensor the 300 MeV pion can iduce HIP with probability 3-10x10-4 -> in agreement with simulation. No significant dependence of recovery on sesnsor thickness.
REMARKS
-in my case the fraction of HIP is also effected by pile-up
-computation how fake clusters affects tracking
-what is the inverter resistor value first it was 100 but now changes to 50~\cite{Gennai:2003as}
