\chapterwithnum{Study of highly ionizing particles in the strip tracker}

\section{Tracking inefficiencies at beginning of RunII (or clustering, to be decided)}

\subsection{Observed inefficiencies in tracks reconstruction}
\subsection{Highly ionizing particles as possible explanation}
        (how the nuclear event looks like (inelastic, most energy from recoil))


\section{Strip tracker readout system}

\subsection{Overview}
       (give overview about the readout - how the signal goes)

particle going through, making e-holes pairs. Charge induced on strips is read by APV chips. output of two APV chips is multiplexed by APVMUX, converted to optical signal and read by the FED which digitizes data  and proceed with data processing and reduction - in order to reduce the ammount of data sent to DAQ.
APV25~\cite{French:2001xb}
FED~\cite{Baird:2002wg}
\subsection{Silicon strip modules}
%TODO check the pitch size

The CMS silicon sensors are formed by n-type bulk, which has on one side uniform n+ implant while on the other p+ strips are located. The implants are connected to reverse bias voltage to completly deplete the bulk of the sensor. The thickness of both p+ and n+ implants is small and negligible compared to the bulk, thus almost whole volume of sensor is depleted. Every p+ strip are connected by a wire bond to a read-out electronics.

%TODO is true 4-6 sensors?, or one sensor with more or less strips?
The tracker modules are consisted by 1-2 sensors, each sensor is having three or four times 128 p+ strips. The larger part of modules have one layer of sensors~(mono modules), the other holds two layers of sensors, which are attached back to back and with a strip inclination of $5.7^{\circ}$ against each other~(stereo modules). Thus the stereo modules are able to give 3-D information in global coordinates about the position where particle hitted the module~(hit position). The modules also differ by the pitch size between each strip which can vary from 80 $\mu$m up to 205 $\mu$m depending on the tracker layer and partition.

The particle crossing silicon sensor is leaving energy predominantly via electromagnetic interaction - by ionization of the silicon volume, the electron-hole pairs are produced along the path of a particle. The energy loss in the material can be described by the Bethe-Bloch formula~\cite{Groom:2000sm} as a function of $\beta\gamma = p/Mc$, where $p$ and $M$ is mometum and mass of the interacting particle. The Bethe-Bloch function has a minimum for $\beta\gamma \approx 3$. Majority of relativistc particles are having this minimal value of $\beta\gamma$ and thus they are called be Minimum Ionizing Particles~(MIP).

Under normal circumstances the created charge carriers, electrons and holes, would drift on opposite sides directly towards electrods (n+ and p+ implants). But as in the barrels case a perpendicular magnetic field is present, the charge carrier $q$ is deflected from the direction of electric field due to the Lorentz force

\eq{LorentzEquation}
{
    q(E+v \times B).
}


-cross talk
signal on strips is already induced once electrons and holes are separated and not at the time they reach the strips. The charge induced at electrodes can be calculated using shockley Ramo theorem~\cite{doi:10.1063/1.1710367,Ramo:1939vr}. Using the teorem for the TOB modules it can be shown~\cite{Bloch:2007zza} that charges drifting toward one strip induce charge on the two neighboring, but the net signal induced (total charge) for neighboring electrodes will be zero in case that all charges are collected at the seed and not trapped by other strips.

-noise
electronic noise - if constant current - in measurement we observe fluctuations. Noise sources - voltage sources or current sources. Noise: variations of velocity(termal noise) or fluctuating number of charge carriers (shot noise). The biggest noise comes from amplifying. There can be correlation between channels - positive correlation among strips of group.

-cmn - originating from power suplies or electronics pickup - affects many channels simuntaneously.

\subsection{The APV25 readout chip}

The charge collected by each channel~(strip) is read by APV25 chip located at the module. The APV is front-end chip providing amplification and shaping of signal from each channel separately. To achive this all APV chips are equipped by preamplifier, CR-RC shaper, analog pipeline and deconvolution filter. As one APV chip is reading signal from 128 strips, 4-6 chips are present at each module.

The amplified signal is sent to CR-RC shaper to convert strip signal into $\sim$50~ns long voltage pulses. The shaper is providing an output with frequecy of bunch crossings. In case of APV working in the ``peak mode'' the sampled signal at maximum of the pulse shape, corresponding to given bunch crossing is used directly, but usually the APV is operating in ``deconvolution mode'' to reduce the out-of-time pile-up. In the deconvolution mode the weighted sum of the shaper output from three consecutive bunch crossing is used instead.To have a possiblity to optimize the puls shape, the feedback resistors of both preamplifier and shaper as well as  bias current and voltage are fully programable. For the calibration and test of the chip the internal calibration circuit is present. This circuit enables to inject charge to channels prior to the preamplifier stage.

The puls heights and bunch crossing information for all 128 channels are extracted at the end of the analog pipeline upon the request of trigger. The average signal level from 128 channels can be adjusted within the dynamic range of the APV, in order to reduce the signals exceeding the APV range. The processed signal from two APV chips is multiplexed by APVMUX~\cite{Ball:2007zza} into a single line and converted by laser from an electrical to optical signal, which is sent via analog optical fibre to control room. At the control room the optical signal is recevied by the pin diod which is a part of FED.

-tickmarck sent every 70 clock cycles when no data are qued for output -  used for synchronization betwwen frontend and backend electronics.

\subsection{The Front End Driver}

The FED is recieving data from 96 optical fibers, each sending information from 2 APVs. The data in form of optical signal are converted to electrical signal, they are reordered and synchronized. For each APV input the signals per channel are extracted and digitized into 10-bit range AD counts~(ADC). In the standard operation mode, the ``Zero Supression~(ZS)'', the pedestal subtraction followed by common mode noise~(CMN) subtraction is performed. Pedestal is mean strip activity for given strip when no particle is present, which is evaluated from special ``pedestal runs'' taken few times per year. After pedestal subtraction the CMN is remining noise, which is common to all channels and calculated on event by event basis as a median over the 128 strips. After both subtractions the channels with values lower than zero are truncated to zero. For all the channels the signal-to-noise ratio~(S/N) is checked separately and if S/N of the channel is smaller than five or S/N of the channel and its neighbor is smaller than two, the ADC of these channels is set to zero~(zero suppressed). Morover the the ADC range id truncated to 8 bits. No change is applied for charges lower than 254 ADC, charges between 254 ADC and 1022 ADC, included are set to 254 ADC and charges aceeding 1022 ADC are stored as 255 ADC. Only information about strips with non-zero ADC values are sent to the CMS data acqusition system~(DAQ). By this procedure the available data are reduced by factor $\sim$60 in order not to overload DAQ system.


For testing purposes the FED is able to operate in others than ZS mode. In case of the ``Virgin Raw~(VR)'' data taking mode no subtractions and suppressions are applied and thus it is suitable mode e.g. for studies of the APV behavior.

\subsection{Offline data treatement}

Offline, the clustering procedure is applied on the ZS channels. The default clustering algorithm is called ``Three threshold algorithm'', posing threshold on seed strip, neighboring strips and cluster charge in terms of signal-to-noise ratio. Cluster seed must pass requirement S/N>3, adjacent stips can be added in case S/N>2. On top of those requirement the total cluster charge must be five times larger than total cluster noise $\sigma_{total}$ which is defined as

\eq{noiseEquation}
{
    \sigma_{cluster} = \sqrt{\sum_{i} \sigma_{i}^{2}},
}

where $\sigma_{i}$ is noise of channel $i$.

The final position of hit is obtained charge-weighted positions of channels in cluster, corrected for Lorentz drift in case of TIB and TOB. Additional correction is applied due to the inefficiency of collection of charge deposited close to back-plane.

The reconstructed hits are used for tracks reconstruction~\cite{Chatrchyan:2014fea}, utilizing software reffered as Combinatorial Track Finder~(CTF), based on combinatorial Kalman filter~\cite{Fruhwirth:1987fm}. The CMS tracking is using iterative approach -- first the track the easiest to find are reconstructed (e.g. high $p_{T}$ ones ) and its hits are masked in order to reduce combinatorics for further iterations of tracks finding.


The track reconstruction is performed in 4 steps. First the \textit{track seed} is built from two or three 3-D hits. Then during the \textit{track finding} the track is propagated to neighboring layers of tracker, testing the compatibility of hit with the track by $\chi^{2}$ test. Once the track candidate is complete, the \textit{track fitting} is performed to obtain the best parameters of the track. The last stage is \textit{track quallity selection} when it is tested if the final fit resulted in good $\chi^{2}$, if there are enough layers with hit associated to the track and if the track is originating from the primary vertex.



FURTHER
-APV -dynamic range of 8 MIPs.
-where gains?
