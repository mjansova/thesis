\chapternonum{Introduction}

%The knowledge about particle physics brought us many inventions widely used and appreciated in the society. The particle physics applications are now used in many fields, from medicine to the energy industry. To further understand and broaden our knowledge about the elementary particles and interactions, the Large Hadron Collider was built at CERN. The Large Hadron Collider is providing particle collisions in four detectors, one of them being the Compact Muon Solenoid~(CMS). In the detector the colliding particles interact, leading to the creation of new particles.  The created particles are traversing the detector and leaving there the energy deposits along their path. To be able to reliably reconstruct the picture about what happened in the detector, precise measurement of these deposits is vital.

This thesis started in October 2015, the same year as the beginning of the Run~2 data-taking period (2015-2018) during which  the center-of-mass energy, the instantaneous luminosity and the bunch crossing frequency was increased compared to the Run~1 (2008-2013). Consequently, the possibilities for physics analyses were largely extended with a price of a larger fluence on the detector side, especially on the tracker which is the inner subdetector. With this increasing fluence, the detector suffers from a larger irradiation which could be a cause of performance issues. Moreover, the detector is also ageing with irradiation, leading to a change of some of its characteristics. Therefore it is very important to monitor closely the detector, to keep its stable performance and consequently the potential physics reach.

Particle physics is described by the standard model whose  last piece, the Higgs boson, was discovered in 2012 by the CMS and ATLAS collaborations. Although the standard model is now complete and in general describes excellently the physics phenomena, it suffers from several shortcomings. This issue makes us believe that the standard model is an effective theory at low energy of a more fundamental theory which is to be determined. Over years, many theories  were proposed and one, referred to as supersymmetry, became of a special interest due to its capability to address many of the standard model shortcomings. 

Supersymmetry introduces a new partner to each standard model particle and therefore extensive searches for these particles have been performed by the CMS collaboration as well as other collaborations. One of these particles is the supersymmetric partner of the top quark, the stop, which is expected to have a mass around 1~TeV in natural supersymmetry and therefore be accessible at the LHC energies. No evidence for the stops was  found in Run~1, but the increase in luminosity as well as the center-of-mass energy in Run~2 allows us to probe the stop masses beyond the Run~1 exclusion. 

The description of the Compact Muon Solenoid~(CMS) detector is given in the first chapter, Chapter~\ref{sec:detch}, together with a brief introduction of the Large Hadron Collider. This chapter focuses on the silicon strip tracker, whose deeper understanding is required for the following chapters. This chapter also presents the reconstruction of the physics objects corresponding to particles passing through the detector.

In Chapter~\ref{sec:HIPch} a study of Highly Ionizing Particles (HIP)  in the strip tracker is presented, motivated by the observation of the inefficiencies in tracking during years 2015 and 2016. These HIPs are induced by inelastic nuclear interactions in the sensitive volume of the tracker. The front-end electronics of the strip tracker is not designed to cope with the large HIP energy deposits, therefore a HIP interaction saturates the electronics and induces a dead-time. This chapter presents two studies I performed during my PhD on the HIPs. They lead to the first  measurement, with the CMS data, of the probability that a HIP occurs in the tracker.  

The third chapter, Chapter~\ref{ch:simu}, focuses on the simulation of the CMS strip tracker. In order to be able to compare the results of the experiment with the theoretical expectations, the interactions of generated particles with the detector must be simulated. This chapter describes how the simulation in the strip tracker is implemented and how it can be improved. It is shown that several tracker properties and conditions are taken as parameters in the simulation. With the ageing of the detector these properties evolve and therefore need to be remeasured and updated in the simulation. In this chapter, I identify which parameters need to be reevaluated. In this context, a measurement of the cross talk, resulting from the capacitive coupling between neighboring strips, is presented using cosmic data. The impact of the newly measured cross talk parameters on the simulation is presented as well.

After an introduction of both SM and SUSY in Chapter~\ref{sec:SUSYch}, the search for the stop is presented in Chapter~\ref{sec:stopch} in the single lepton final state with the data of Run~2 recorded in 2016 corresponding to an integrated luminosity of 35.9~fb$^{-1}$. In this analysis I was mainly responsible for the estimation of one of the SM backgrounds entering into the targeted signal regions. I also tried several options to improve the analysis performance (i.e. W-tagging). 

