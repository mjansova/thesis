\chapternonum{Introduction}

%The knowledge about particle physics brought us many inventions widely used and appreciated in the society. The particle physics applications are now used in many fields, from medicine to the energy industry. To further understand and broaden our knowledge about the elementary particles and interactions, the Large Hadron Collider was built at CERN. The Large Hadron Collider is providing particle collisions in four detectors, one of them being the Compact Muon Solenoid~(CMS). In the detector the colliding particles interact, leading to the creation of new particles.  The created particles are traversing the detector and leaving there the energy deposits along their path. To be able to reliably reconstruct the picture about what happened in the detector, precise measurement of these deposits is vital.

This thesis has started in the same year as Run~2. In Run~2, the center-of-mass energy, instantaneous luminosity, and the bunch crossing frequency has increased compared to the Run~1 and consequently the fluence received by the detector has increased. With increasing fluence, the detector suffers from larger irradiation which could be a cause of issues in its performance. Moreover, the detector is also ageing with fluence, leading to a change of some of its characteristic. Therefore it is very important to watch closely the detector and avoid the increased fluence to affect the physics analyses.

The description of the Compact Muon Solenoid~(CMS) detector is given in the first Chapter~\ref{sec:detch} together with a brief introduction of the the Large Hadron Collider. This chapter is giving detailed information about the silicon strip tracker, which deeper understanding is required for the following chapters. This chapter also presents the reconstruction of the physics objects corresponding to particles passing through the detector.

In Chapter~\ref{sec:HIPch} I present the CMS strip tracker in even larger detail and in particular the inefficiencies in tracking observed during years 2015 and 2016, resulting from the increased fluence. Because of these inefficiencies, many hits in the detector were not reconstructed, and therefore it was essential to understand the source of them. This effect was believed to be caused by the inelastic nuclear interactions in the sensitive volume of the tracker, leading to creation of the highly ionizing particles (HIP). The front-end electronics of the strip tracker is not designed to cope with the large HIP energy deposits, therefore the HIP interaction saturates the electronics and induces the dead-time. This chapter presents two studies, I performed during my PhD, on the  nuclear interactions, which result in a measurement of the probability that such interaction occurs in the tracker. This measurement is the first measurement of the HIP probability with the CMS data. 

The third chapter, Chapter~\ref{ch:simu}, focuses on the simulation of the CMS silicon tracker. In order to be able to compare the results of the experiment with the theoretical expectations, the interactions of generated particles with the detector must be simulated. This chapter describes how the simulation in the strip tracker is implemented and how it can be improved. I show, that several tracker properties are taken as parameters in the simulation. But with aging of the detector these properties evolve and therefore they need to be remeasured and updated in the simulation. In this chapter I identify, which parameters need to be reevaluated. At the end I present results on the measurement in cosmic data of the cross talk parameters,  which were found to be strongly influenced by the detector ageing. I also show how introducing the newly measured cross talk parameters into simulation improve the description of data by simulation at the hit level.

Particle physics is described by the standard model, which last piece, the Higgs boson, was discovered in 2012 by the CMS and ATLAS collaborations. Although the standard model is now complete and in general describes excellently the physics phenomena, it suffers from several shortcomings. This issue makes us believe that the standard model is an effective theory, which is a part of a bigger theory. Over years, many theories  were proposed and one, referred to as supersymmetry, became of a special interest due to its capability to address many of the standard model shortcomings. Both standard model and supersymmetry are introduced in Chapter~\ref{sec:SUSYch}.

Supersymmetry introduces a new partner to each standard model particle and therefore extensive searches for these particles are performed by the CMS collaboration and other collaborations as well. One of these particles is the supersymmetric partner of the top quark, the stop, which is expected to have a mass around 1~TeV in natural supersymmetry and therefore be accessible at the LHC energies. No evidence for the stops was  found in Run~1, but the increase in luminosity as well as the center-of-mass energy in Run~2 allows us to probe the stop masses beyond the Run~1 exclusion. The last chapter before conclusion, Chapter~\ref{sec:stopch},  presents the search for this supersymmetric partner of the top quark in the single lepton final state with the data of Run~2 recorded in 2016 corresponding to an integrated luminosity of 35.9~fb$^{-1}$. In this analysis I was mainly responsible for estimation of one of the SM backgrounds entering into the targeted signal regions. 

