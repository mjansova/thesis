\chapternonum{Introduction}

The knowledge about particle physics brought us many inventions widely used and appreciated in the society. The particle physics applications are now used in many fields, from medicine to the energy industry. To further understand and broaden our knowledge about the elementary particles and interactions, the Large Hadron Collider was built at CERN. The Large Hadron Collider is providing particle collisions in four detectors, one of them being the Compact Muon Solenoid~(CMS). In the detector the colliding particles interact, leading to the creation of new particles.  The created particles are traversing the detector and leaving there the energy deposits along their path. To be able to reliably reconstruct the picture about what happened in the detector, precise measurement of these deposits is vital.

The first Chapter~\ref{sec:detch} briefly introduces the Large Hadron Collider and Compact Muon Solenoid, giving more detailed information about areas which are further developed in the following chapters. This chapter also discusses the reconstruction of the physics objects seen in the detector.

In Chapter~\ref{sec:HIPch} we discuss the CMS strip tracker and observed inefficiencies during years 2015 and 2016, in the reconstruction of particle paths due to its inability to measure the energy deposits. This effect was believed to be caused by the inelastic nuclear interactions in the sensitive volume of the tracker, which lead to the saturation of the tracker electronics. This chapter presents results on the probability of the nuclear interactions and effects resulting from such interactions. 

The third Chapter~\ref{ch:simu} focuses as well on the CMS silicon tracker, but in this case on its simulation. In order to be able to compare the results of the experiment with the theoretical expectations, the simulation of the interactions of particles generated according to the theoretical model with the detector must be simulated. This chapter describes how the simulation in the strip tracker is implemented and how it can be improved. At the end we present results on measurement of the tracker properties which are taken as a parameters in the simulation. We show how introducing these parameters into simulation improve the description of data by simulation at the level of the energy deposits.

The particle physics is described by the standard model, which last piece, the Higgs boson, was discovered in 2012 by the CMS and ATLAS collaborations. Although the standard model is now complete and in general well describing the physics phenomena, it suffers from several shortcomings. This issue makes us believe that the standard model is an effective theory, which is a part of a bigger theory. Over years, many theories  were proposed and one, referred to as supersymmetry, became of a special interest due to its capability to address many of the standard model shortcomings. Both standard model and supersymmetry are discussed in Chapter~\ref{sec:SUSYch}.

The supersymmetry introduces a new partner to each standard model particle and therefore extensive searches for these particles are performed by the CMS and other collaborations. One of these particles is a supersymmetric partner of the top quark which is expected to have a mass around 1~TeV and therefore be accessible at the LHC energies. The last Chapter~\ref{sec:stopch} before conclusion presents search for this supersymmetric partner of the top quark in the single lepton final state.


%particle physics studies the 
% particles building blocks of the matter around us
%to understand this matter and interaction between the particles
%-> LHC and experiments 

%to be able to study these particles we need well performing detector
%detector - cahpter one
%HIP - discusssed in the second chapterr
%then 

%stop chapter 3

%with the discovery of the Higgs boson, the standard model became complete, but still many puzzles persisted
%SM effective theory weell describing our world
%SUSY to target the SM problems chapter 4
% chapter 5
%stop expected to eb relatively light
%touchinh anturality bound, but there are still many corners unsearched
